% Options for packages loaded elsewhere
\PassOptionsToPackage{unicode}{hyperref}
\PassOptionsToPackage{hyphens}{url}
%

\documentclass[a4paper,14pt,russian]{extreport}

\usepackage{cmap} % для кодировки шрифтов в pdf
\usepackage[T1,T2A]{fontenc}
\usepackage[utf8]{inputenc}
\usepackage[russian]{babel}
\usepackage{pscyr}
\usepackage{amsmath,amssymb}
\usepackage{lmodern}
\usepackage{iftex}
\usepackage[T1]{fontenc}
\usepackage{textcomp} % provide euro and other symbols

\usepackage{graphicx} % для вставки картинок
\usepackage{amssymb,amsfonts,amsmath,amsthm} % математические дополнения от АМС
\usepackage{indentfirst} % отделять первую строку раздела абзацным отступом тоже
\usepackage[usenames,dvipsnames]{color} % названия цветов
\usepackage{makecell}
\usepackage{multirow} % улучшенное форматирование таблиц
\usepackage{ulem} % подчеркивания

\linespread{1.3} % полуторный интервал
\renewcommand{\rmdefault}{ftm} % Times New Roman
\frenchspacing

\usepackage{fancyhdr}
\pagestyle{fancy}
\fancyhf{}
\fancyhead[R]{\thepage}
\fancyheadoffset{0mm}
\fancyfootoffset{0mm}
\setlength{\headheight}{17pt}
\renewcommand{\headrulewidth}{0pt}
\renewcommand{\footrulewidth}{0pt}
\fancypagestyle{plain}{
    \fancyhf{}
    \rhead{\thepage}}
\setcounter{page}{1} % начать нумерацию страниц с №1

\usepackage{titlesec}

\titleformat{\chapter}[display]
    {\filcenter}
    {\MakeUppercase{\chaptertitlename} \thechapter}
    {8pt}
    {\bfseries}{}

\titleformat{\section}
    {\normalsize\bfseries}
    {\thesection}
    {1em}{}

\titleformat{\subsection}
    {\normalsize\bfseries}
    {\thesubsection}
    {1em}{}

% Настройка вертикальных и горизонтальных отступов
\titlespacing*{\chapter}{0pt}{-30pt}{8pt}
\titlespacing*{\section}{\parindent}{*4}{*4}
\titlespacing*{\subsection}{\parindent}{*4}{*4}

\usepackage{geometry}
\geometry{left=3cm}
\geometry{right=1.5cm}
\geometry{top=2.4cm}
\geometry{bottom=2.4cm}

\usepackage{enumitem}
\makeatletter
    \AddEnumerateCounter{\asbuk}{\@asbuk}{м)}
\makeatother
\setlist{nolistsep}
\renewcommand{\labelitemi}{-}
\renewcommand{\labelenumi}{\asbuk{enumi})}
\renewcommand{\labelenumii}{\arabic{enumii})}

\usepackage{tocloft}
\addto\captionsrussian{\def\chaptername{Раздел}}
\addto\captionsrussian{\def\sectionname{Глава}}
\cftsetindents{sec}{1em}{1em}
\cftsetindents{subsec}{2em}{2em}
\renewcommand{\thechapter}{\Roman{chapter}}
\renewcommand{\thesection}{\arabic{section}}
\renewcommand{\cftchapfont}{\normalsize\bfseries {\chaptername} }
\renewcommand{\cftsecfont}{\normalsize\bfseries {\sectionname} }
\renewcommand{\cfttoctitlefont}{\hspace{0.38\textwidth} \bfseries}
\renewcommand{\cftbeforetoctitleskip}{-1em}
\renewcommand{\cftaftertoctitle}{\mbox{}\hfill \\ \mbox{}\hfill{\footnotesize Стр.}\vspace{-2.5em}}
\renewcommand{\cftbeforechapskip}{1em}
\renewcommand{\cftparskip}{-1mm}
\renewcommand{\cftdotsep}{1}
\setcounter{tocdepth}{2}

\author{В редакции проф. С.М. Халина}
\date{Тюмень--2020}
\title{Основы философии диалектического материализма}

\begin{document}\sloppy

\maketitle

\newpage

УДК 101 (075.8)

ББК 87я73

Основы философии диалектическогор материализма : Учебник. / В
редакции д.филос.н, проф. С.М. Халина. --- Тюмень: Тюменский гос. ун-т,
2020. --- 338 с.

Настоящая работа представляет собой учебник по основам философии
диалектического материализма, включая её социально-философский компонент
--- исторический материализм, созданный в результате переработки
советских учебников 70--80-х годов ХХ в.

Учебник рассчитан на студентов высших учебных заведений, на всех тех,
кто самостоятельно изучает философию, проявляет интерес к философии
вообще и к философии диалектического материализма в частности.

\textsuperscript{\textcopyright} Халин Сергей Михайлович, 2020.

ISBN \_\_\_\_\_\_\_\_\_\_\_\_\_\_\_\_\_\_

\newpage

\tableofcontents

\newpage


УДК 101 (075.8)

ББК 87я73

Основы философии диалектическогор материализма : Учебник. / В редакции д.филос.н, проф. С.М. Халина. --- Тюмень: Тюменский гос. ун-т, 2020. --- 338 с.

Настоящая работа представляет собой учебник по основам философии диалектического материализма, включая её социально-философский компонент --- исторический материализм, созданный в результате переработки советских учебников 70--80-х годов ХХ в.

Учебник рассчитан на студентов высших учебных заведений, на всех тех, кто самостоятельно изучает философию, проявляет интерес к философии вообще и к философии диалектического материализма в частности.

\textsuperscript{\textcopyright} Халин Сергей Михайлович, 2020.

ISBN \_\_\_\_\_\_\_\_\_\_\_\_\_\_\_\_\_\_

\newpage

\tableofcontents

\newpage

\chapter{Предисловие редактора}

Мы живём во времена \emph{бурного прогресса} науки и техники. Глубокие изменения в общественной жизни предъявляют всё больше требований к убеждениям, философской культуре, научному мышлению людей.

Диалектический материализм возник более ста пятидесяти лет назад. Эта философская традиция была заложена \emph{К. Марксом} и \emph{Ф. Энгельсом}. Ряд вопросов был разработан \emph{Г.В. Плехановым}, \emph{В.И. Лениным}, их последователями.

Как \emph{творческая традиция} философия диалектического материализма непрерывно развивается на основе обобщения всемирно-исторического опыта, достижений естествознания и общественных наук.

В пособии, наряду с освещением основных вопросов диалектического материализма, позитивным изложением его важнейших идей, анализируется положения \emph{некоторых направлений} западной философии ХХв.

При подготовке пособия была сделана попытка учесть необходимость преодолеть многочисленные \emph{негативные последствия} для развития диалектического материализма. Это касается неоправданной политизации, идеологизации, односторонней праксиологизации положений данной философии, связанные, прежде всего, со сталинским периодом в истории Советского Союза, во времена которого философия диалектического материализма была фактически возведена в ранг \emph{государственной идеологии}, что, как правило, негативно сказывается на самой философии.

Выражаю признательность своим студентам Института биологии Тюменского госуниверситета, проделавшим большую техническую работу с текстом, особенно магистранту группы 26Б191 Анастасии Пелымской.

\chapter{Диалектико-материалистическая философия: введение. Основные законы и категории диалектики}

\section{Философия, её предмет и место среди других видов познания}

В наше время, в век расцвета научной мысли, можно услышать \emph{голоса, оспаривающие право на существование философии} как особой отрасли, вида научного знания и познания. Эти трактовки философии утверждают, что некогда, в античном мире, она \emph{была наукой наук}, но затем от неё в ходе исторического развития отпочковывались одна за другой специальные отрасли научного знания --- астрономия, физика, химия, биология, история, социология, логика и т.д.

В этих условиях философия оказалась якобы в положении шекспировского \emph{короля Лира}, который под старость раздал дочерям своё царство, а они выгнали его, как нищего, на улицу. Но такой взгляд в отношении научной философии неправилен. Размежевание между философией и специальными, частными науками, несомненно, способствовало формированию специфического \emph{\emph{предмета философского исследования}.}

С другой стороны, развитие специальных наук способствовало \emph{вычленению общих} для всех этих наук мировоззренческих и методологических проблем, которые не могут получить своего разрешения в рамках специальной области исследования.

\emph{Что} составляет сущность вселенной? \emph{В каком отношении} друг к другу находятся сознание и внешний мир, духовное и материальное, идеальное и реальное? \emph{Что такое} человек и каково его место в мире? \emph{Способен ли} он познавать и преобразовывать мир, и если да, то каким образом?

Эти и многие другие вопросы глубоко волнуют всех мыслящих людей. И издавна существует \emph{неистребимая потребность} найти ответы на эти вопросы, составляющие содержание философии.

Философия есть специфическое по своему содержанию и форме мировоззрение, познание, ориентированное на всеобщность, которое теоретически обосновывает свои принципы и выводы. Этим научная философия отличается от ненаучного религиозного мировоззрения, которое основывается \emph{на вере} в сверхъестественное и отражает действительность в эмоционально-фантастической форме.

Философское мировоззрение \emph{есть система} наиболее общих теоретических взглядов на мир, т.е. на природу, общество, человека.

Философия ставит своей \emph{целью} разработать, обосновать основные принципы научной, нравственной, эстетической, а также социально-политической ориентации людей.

У всякого человека складывается какое-либо воззрение на окружающий мир, но оно нередко состоит \emph{из обрывков} различных противоречивых представлений, теоретически не осмыслено, не обосновано.

Серьёзная же, научная философия представляет собой не просто сумму, а \emph{систему идей}, взглядов и представлений о природе, обществе, человеке и его месте в мире.

Научное философское мировоззрение не просто провозглашает свои принципы и пытается внушать их людям, а \emph{доказывает}, логически выводит их.

Конечно, \emph{далеко не всякое} теоретически обосновываемое мировоззрение носит научный характер. По своему содержанию философское мировоззрение может быть и научным, и ненаучным или даже антинаучным.

\emph{Научным} может считаться лишь такое мировоззрение, которое основывает свои выводы на данных современной ему науки, пользуется научным методом мышления и не оставляет места различного рода антинаучным, мистическим, религиозным взглядам и предрассудкам.

Само собой разумеется, что научность должна рассматриваться \emph{исторически}. Например, мировоззрение \emph{французских материалистов XVIII в}. было научным. В их взглядах наряду с исторически преходящим было и \emph{непреходящее} содержание, которое унаследовано современным материализмом.

Научные идеи, положения содержались и в великих идеалистических философских системах (например, у \emph{Декарта, Лейбница, Канта, Фихте, Гегеля} и др.) в той мере, в какой в них верно были схвачены реальные отношения и связи.

\emph{Диалектико-материалистическая философия, включая её социально-философский раздел --- исторический материализм, это научное философское мировоззрение, которое основывается на достижениях современной науки и практики, постоянно развивается и обогащается вместе с их прогрессом}.

Такое развитие, конечно, не может происходить автоматически, само собой, а для чего нужна постоянная \emph{критическая работа} самих представителей диалектического материализма, подобно тому, как критически переосмысливает постоянно свои положения любая уважающая себя наука.

Чтобы глубже понять предмет и значение диалектического материализма, его отличие от предшествующей философской мысли, следует более обстоятельно остановиться на характеристике философии как \emph{особой формы познания}.

\subsection{Развитие понятия о предмете философии}

Предмет философии \emph{исторически изменялся в тесной связи} с развитием всех сторон духовной жизни общества, с развитием науки и самой философской мысли.

Слово «\emph{философия}» древнегреческого происхождения. Оно образовано из двух слов: \emph{phileo} -- люблю и \emph{sophia} -- мудрость. В буквальном смысле \emph{философия --- это любовь} к мудрости или, как раньше говорили на Руси, «\emph{любомудрие}».

Сохранилось предание о том, что древнегреческий математик \emph{Пифагор} был первым человеком, который назвал себя «\emph{философом}», указав при этом, что человеку не следует переоценивать своих возможностей в достижении мудрости, кроме того, любовь к мудрости, стремление к ней \emph{приличествует} каждому разумному существу.

Однако разъяснить происхождение какого-либо термина \emph{ещё не значит} раскрыть суть того научного понятия, которое этот термин выражает.

Философия зародилась \emph{на заре цивилизации} в \emph{Древней Индии}, \emph{Китае}, \emph{Египте.} Своей классической формы она впервые достигла в \emph{Древней Греции}.

Древнейшая форма мировоззрения, непосредственно предшествовавшая философии --- это \emph{мифология}, то есть фантастическое отражение действительности, возникавшее в сознании первобытного человека, который одушевлял окружающий его мир.

\emph{В мифологии} с её верой в фантастических духов, богов значительное место занимали вопросы о происхождении и сущности мира.

Философия формировалась \emph{через преодоление} религиозно-мифологического сознания, как попытка рационального объяснения мира.

Зарождение философии исторически совпадает с возникновением \emph{зачатков протонаучного знания}, с формированием потребности в теоретическом исследовании. Философия, собственно, и сложилась как \emph{первая историческая форма} теоретического знания.

Первоначально философия отвечала на вопросы, которые \emph{были уже поставлены} религиозно-мифологическим мировоззрением. Однако способ решения вопросов у философии уже \emph{был иным}, он основывался на теоретическом, согласующемся с логикой и практикой анализе этих вопросов.

Первые мыслители античного мира стремились главным образом понять происхождение многообразных природных явлений (\emph{Фалес, Анаксимен, Анаксимандр, Парменид, Гераклит} и \emph{др.}).

\emph{Натурфилософия} (философское учение о природе) была первой исторической формой философского мышления.

По мере накопления конкретных знаний, выработки специальных приёмов осмысления предметов \emph{началось размежевание} между отдельными областями теоретического, а также прикладного знания. Уже в античную эпоху имеет место \emph{дифференциация} нерасчленённого знания, выделение математики, медицины, астрономии и т.д.

Однако наряду с ограничением круга проблем, которыми занималась философия, происходило также развитие, углубление, обогащение \emph{собственно философских представлений}, возникали различные философские теории и направления. Формировались такие философские дисциплины, как \emph{онтология} --- \emph{учение о бытии}, или о сущности всего существующего; \emph{гносеология} --- \emph{теория познания}; \emph{логика} --- наука о формах правильного, то есть связного, последовательного, доказательного мышления; \emph{философия истории}, \emph{этика}, \emph{эстетика}.

Начиная с эпохи \emph{Возрождения} и \emph{особенно с XVII--XVIII вв}. процесс размежевания между философией и бурно развивающимися специальными науками совершается всё более ускоренными темпами. Механика, физика, а затем химия, биология, юриспруденция, политическая экономия \emph{становятся самостоятельными} отраслями научного знания и познания.

Это \emph{прогрессирующее разделение} труда в сфере научного познания качественно изменяет роль и место философии в системе наук, её взаимоотношения с частными науками. Философия \emph{уже не занимается} решением специальных проблем механики, физики, астрономии, химии, биологии, права, истории и т.д. Однако в её сферу входит \emph{исследование общенаучных}, мировоззренческих вопросов, которые имеют место в частных науках, но не могут быть решены в их рамках, с помощью свойственных только им специальных методов.

Из истории известно, что взаимоотношения между философией и частными науками носили весьма \emph{сложный, противоречивый} характер.

Некоторые философы создавали \emph{энциклопедические философские системы} с целью противопоставить естествознанию философию природы, истории как науке --- философию истории, правоведению --- философию права. Эти мыслители обычно полагали, что философия способна выходить за пределы опыта, давать «\emph{сверхопытное}» знание.

Такого рода \emph{иллюзии} были опровергнуты развитием специальных наук, которое доказало, что физические проблемы может решить лишь физика, химические --- химия и т.д.

В то же время в ряде философских учений наблюдалась и противоположная тенденция --- стремление \emph{превратить философию} в частную, специальную дисциплину, отказаться от рассмотрения наиболее общих, мировоззренческих проблем.

Успехи специальных наук, в особенности математики и механики, \emph{побуждали философов} изучать те методы, при помощи которых были достигнуты эти успехи, с тем, чтобы выяснить возможность их применения в философии.

Однако развитие специальных наук показало, что \emph{существуют проблемы}, которыми занимаются не только эти науки, но и философия. Такие проблемы, естественно, могут быть решены \emph{лишь совместными усилиями} философии и частных наук.

Существуют и специфически философские проблемы, которые может решить лишь философия, однако в том случае, если она опирается на всю совокупность научных данных и достижений общественной практики.

\emph{Философия --- это осмысление, исследование действительности в целом или любой её части с привлечением максимально доступного на данный момент конкретного опыта, прежде всего в виде конкретно-научного знания, и на основе знания о всеобщих признаках (категории) и принципах (всеобщие законы).}

\subsection{Основной вопрос философии}

Как ни многообразны философские учения, все они, явно или неявно, имеют в качестве своего отправного пункта, теоретического ключевого камня вопрос об \emph{отношении сознания к бытию}, \emph{духовного к материальному.}

Коренной, \emph{основной вопрос} философии вытекает из фундаментальных фактов нашей жизни.

Существуют материальные, например, физические или химические и т.д. явления, \emph{но существуют} также духовные, психические явления --- сознание, мышление и т.п.

\emph{Разграничение} сознания и внешнего мира является необходимым условием существования сознания и всей человеческой деятельности: каждый человек \emph{выделяет себя} из всего, что его окружает, и отличает себя от всего другого.

Какое бы явление мы ни рассматривали, его \emph{всегда можно отнести} к сфере духовного, субъективного или же материального, объективного.

Однако при всех различиях объективного и субъективного между ними есть и \emph{определённая связь}, которая при ближайшем рассмотрении оказывается отношением зависимости.

Поэтому \emph{возникает вопрос}: что от чего зависит, что является причиной, а что следствием. \emph{Или,} выражаясь в более общей форме: что и в каком смысле считать первичным и что вторичным --- объективное или субъективное, материальное или духовное, объект или субъект?

Из прежних мыслителей ближе всех к правильному пониманию смысла и значения основного философского вопроса подошёл немецкий философ-материалист \emph{Л. Фейербах}.

Подвергая критике религиозное учение о сотворении мира сверхъестественной силой, духом, богом, Фейербах \emph{противопоставил} ему противоположное воззрение: духовное возникает из материального.

Последовательно научное решение основного вопроса философии \emph{предполагает} последовательно объективный, диалектический подход, попыткой которого и стала философия диалектического материализма.

Эта философия \emph{не ограничивается} рассмотрением сознания как свойства высокоорганизованной материи, но и характеризует сознание, общественное сознание, как отражение бытия, общественного бытия, материальных природных и общественных процессов,

Все философы, в зависимости от решения \emph{основного вопроса философии (ОВФ)}, разбиваются на \emph{две} большие основные группы, идеалистов и материалистов.

Для \emph{идеалистов} дух существует \emph{прежде} природы.

Для \emph{материалистов} --- духовное возникает после материального, после природы, или, точнее, \emph{внутри} самой природы.

Все многообразные философские направления и течения любой эпохи, в конечном счёте, примыкают либо к \emph{материализму}, либо к \emph{идеализму}, именно поэтому также вопрос об отношении духовного к материальному является основным философским вопросом.

С признанием первичности материи или сознания связано и решение \emph{вопроса о существовании} закономерностей природы, а также социальных закономерностей. Эти закономерности, как это вытекает из данных всех наук и общественной практики, не зависят от человеческого произвола, они существуют вне и независимо от нашего и какого-либо иного сознания.

Признание закономерностей природы и общества предполагает признание того, что \emph{мир существует независимо} от сознания человека, вне его сознания.

Это и есть точка зрения материализма, в т.ч. диалектического.

\emph{Иначе} в идеализме. \emph{Одни} из его представителей считают мир со всеми его закономерностями воплощением сверхприродного мирового духа, \emph{другие}, исходя из признания первичности духовного по отношению к материальному, утверждают, что человек непосредственно имеет дело лишь с явлениями собственного сознания и не в праве допускать существования чего-либо, находящегося вне сознания.

\emph{Отрицая} существование объективного мира, и считая предметы комбинациями ощущений и идей, они \emph{отрицают тем самым} и объективную закономерность явлений. Согласно подобному мнению, законы природы и общества, причины явлений и процессов, открываемые наукой, \emph{выражают только} ту последовательность явлений, которая имеет место в нашем сознании.

Из того или иного решения ОВФ вытекают определённые \emph{социальные выводы}: отношение человека к действительности, понимание исторических событий, нравственных принципов и т.д.

Если, например, признать сознание, дух первичным, определяющим, то источник социальных проблем \emph{придётся искать} не в характере материальной жизни людей, не в экономическом строе общества, не в его социальной структуре, а в сознании людей, их заблуждениях, их пороках \emph{и только}.

Такой взгляд \emph{не даёт возможности} определить главные пути преобразования общественной жизни, включая радикальные новшества.

Некоторые современные философы пытаются доказать, что ОВФ вообще не существует, что это якобы \emph{мнимая, надуманная} проблема. Кое-кто из них считает, что само разграничение духовного и материального является \emph{условным} и чуть ли не чисто словесным.

Так, по мысли \emph{Б. Рассела}, то, что называют словами «\emph{материя}» и «\emph{дух}», представляет собой лишь логические построения.

Однако все попытки устранить ОВФ оказываются несостоятельными, ибо \emph{нельзя игнорировать} отличие мышления от предмета мышления, ощущения от того, что ощущается, т.е. воспринимается зрением, слухом и т.д.

ОВФ имеет \emph{две} стороны.

Первая --- это \emph{вопрос о сущности}, \emph{природе мира}.

Вторая --- \emph{вопрос о познаваемости} \emph{мира}.

Рассмотрим сначала первую сторону ОВФ.

\emph{Идеализм}, как уже говорилось, исходит из положения: материальное есть продукт духовного. \emph{Материализм}, напротив, принимает в качестве отправного пункта положение: духовное есть продукт материального.

Оба эти подхода носят \emph{монистический} характер, т.е. исходят из одного определённого принципа: за первичное, первоначальное, определяющее принимается в одном случае материальное, в другом --- идеальное, духовное.

Есть, однако, и такие философы, которые исходят из \emph{двух} начал. Они полагают, что духовное не зависит от материального, а материальное --- от духовного.

Подобные философские теории называют \emph{дуалистическими}. В конечном счёте, они обычно тяготеют к идеализму.

Некоторые философы пытаются соединять отдельные положения идеализма с отдельными положениями материализма и наоборот. Такая философская позиция получила название \emph{эклектизма}.

Как материализм, так и идеализм прошли долгий путь развития и имели немало разновидностей.

Первой исторической формой материализма были элементы материализма \emph{ещё в рабовладельческом} обществе. Это стихийный, \emph{наивный материализм}, получивший свое выражение в \emph{Древней Индии} (философская школа \emph{чарваков}), а в наиболее развитой для своего времени форме --- в \emph{Древней Греции} (прежде всего атомистическое учение \emph{Демокрита} и \emph{Эпикура}).

При переходе к \emph{Новому времени} происходит противопоставление религиозно-идеалистическому мировоззрению материалистического миропонимания, которое получило своё наиболее глубокое выражение в трудах английских философов \emph{Ф. Бэкона} и \emph{Т. Гоббса}, нидерландского мыслителя \emph{Б. Спинозы} (XVII в.), в сочинениях французских материалистов XVIII в. \emph{Ж. Ламетри}, \emph{П. Гольбаха}, \emph{К. Гельвеция}, \emph{Д. Дидро}.

В XIX в. эта форма материализма нашла своё развитие в работах \emph{Л. Фейербаха}.

В ряде отношений можно считать, что \emph{высшей формой современного материализма является философия диалектического материализма, включая его социально-философский компонент} --- \emph{исторический материализм}.

Среди разновидностей идеализма следует, прежде всего, назвать \emph{объективный идеализм} (\emph{Платон, Гегель} и др.), согласно которому духовное существует \emph{вне и независимо} от сознания людей, независимо от материи, природы или до неё, как некий «мировой разум», «мировая воля», «бессознательный мировой дух», который якобы определяет все мировые процессы.

В отличие от объективного идеализма \emph{субъективный идеализм} (\emph{Дж. Беркли}, \emph{Э. Мах}, \emph{Р. Авенариус} и др.) считает, что предметы, которые мы видим, осязаем, обоняем, не существуют независимо от наших чувственных восприятий и выступают как комбинации наших ощущений, представлений.

Нетрудно понять, что субъективный идеалист, если он последовательно проводит свой принцип, должен прийти \emph{к абсурдному выводу}: всё существующее, в том числе и другие люди, не более чем «мои ощущения». Из этого следует, что существую \emph{лишь я один}.

Эта крайняя субъективно-идеалистическая концепция называется \emph{солипсизмом}.

Чаще всего субъективные идеалисты стремятся избежать солипсистских выводов, доказывая тем самым несостоятельность своего исходного положения, совершают логическую ошибку «\emph{самоубийственный аргумент}».

Так, английский субъективный идеалист \emph{Дж. Беркли}, утверждавший, что \emph{существовать} --- \emph{значит быть воспринимаемым}, тем не менее пытался доказать, что за пределами ощущений существует бог (христианский) и наши ощущения являются теми метками, знаками, посредством которых он сообщает нам свою волю.

Развитие наук опровергает положение относительно существования \emph{сверхприродной} духовной первоосновы мира.

Материалисты, опираясь на развитое, научное знание своего времени, рассматривают духовное \emph{как продукт} материального.

Однако, решение ОВФ в философии диалектического материализма, являясь дальнейшим развитием этой в целом правильной точки зрения, отличается своим \emph{последовательно} \emph{диалектическим} характером: духовное есть продукт \emph{развития} материи, свойство высокоорганизованной материи.

Духовное существует \emph{не всегда, не везде}, что оно возникает лишь на определённой ступени развития материи, её качественных превращений, что само оно, духовное, \emph{не остаётся неизменным}, а изменяется, развивается.

\emph{Вторую} сторону ОВФ составляет, как сказано выше, проблема \emph{познаваемости} мира.

Все последовательные представители философского материализма \emph{отстаивают} и обосновывают принцип познаваемости мира. Они рассматривают знания, понятия, идеи, как \emph{отражение} объективной реальности.

Философская позиция, отрицающая возможность достижения объективного знания, получила название \emph{агностицизма} (от греческого «\emph{а}» -- не, \emph{gnosis} -- знание).

Агностицизм в ходе развития истории философии иногда выступал как \emph{«стыдливая» форма материализма}.

Так, некоторые естествоиспытатели XIX в., например, \emph{Т. Гексли} в Англии, не решались под влиянием определённых социальных убеждений открыто признавать себя материалистами, прикрываясь одеждой именно агностицизма.

Что касается иных философских идеалистов, то \emph{одни} из них стояли на позиции познаваемости мира (например, \emph{Гегель}, рассматривавший, однако, познание не как отражение объективной реальности, а как постижение мировым духом самого себя).

\emph{Другие} идеалисты утверждали, что в познании мы имеем дело лишь с нашими ощущениями, восприятиями и \emph{не можем выйти} за пределы познающего субъекта (субъективные идеалисты \emph{Беркли}, \emph{Мах}, \emph{Авенариус} и \emph{др}.).

\emph{Третьи} --- принципиально отвергали возможность познания всего, что существует вне и независимо от человеческого сознания (\emph{И. Кант}, \emph{Ф. Ницше} и \emph{др}.).

\emph{Философы-агностики} нередко пытаются занять \emph{промежуточную} позицию между материализмом и идеализмом, но постоянно натыкаются на идеалистическое отрицание внешнего мира и объективного содержания человеческих представлений, понятий.

Для \emph{современного идеализма} характерно то, что в отличие от \emph{классического идеализма} (Платон, Гегель, Беркли, Мах), он в лице большинства своих представителей стоит на позициях \emph{агностицизма}.

Понимание смысла и значения ОВФ позволяет \emph{более последовательно} рассматривать всё многообразие философских учений, течений, направлений, сменяющих друг друга на протяжении многих веков.

Отношения материализма и идеализма тесно связаны с отношениями \emph{науки} и \emph{религии}.

Будучи противоположным идеализму и религии, \emph{материализм отрицает} веру в бога, в сверхъестественное в качестве средства разрешения реальных проблем человеческого бытия, он неотделим от \emph{атеизма}, что, конечно, не оправдывает в глазах атеистов тех \emph{бесчинств}, которые совершались в отношении религии, верующих, представителей духовенства, материальной инфраструктуры религии в известные периоды советской эпохи.

Идеализм исторически \emph{теснейшим образом} связан с религией, являясь её прямым или косвенным теоретическим обоснованием.

«Мировой разум» объективных идеалистов --- это, по существу, \emph{философский эквивалент} бога.

Однако было бы неправильным отождествлять идеализм и религию, поскольку идеализм представляет собой систему \emph{теоретических} заблуждений, сложившихся в ходе противоречивого развития сознания и познания.

Идеалистическая философия, как и материалистическая, имеет свои, как принято говорить, \emph{социальные} и \emph{гносеологические корни}, \emph{причины} возникновения и изменения.

\emph{Гносеологические корни идеализма}, как и всякой иной ошибочной позиции в познании, которые, кстати, вполне могут встречаться и у материалистов, состоят в \emph{одностороннем}, неполном подходе к познанию, в \emph{преувеличении} или даже \emph{абсолютизации} какой-либо одной из сторон, граней сложного, многостороннего, внутренне противоречивого процесса познания.

Но идеализм и даже любая религия \emph{не бессмыслица}, а \emph{искажённое} отражение действительности, связанное с некоторыми особенностями и противоречиями процесса познания, тесно связанного, в свою очередь, со всеми другими сторонами жизни человека и общества.

Противоречия познавательной деятельности \emph{многообразны}: это противоречия между мышлением, сознанием в целом и практической деятельностью людей.

Гносеологические корни идеализма и заключаются в том, что какая-либо сторона познания или какое-то отдельное положение настолько преувеличиваются, абсолютизируются, что \emph{теряют свою жизненность}, истинность, становятся законченным \emph{заблуждением}.

Так, некоторые идеалисты, всемерно \emph{подчёркивая} активный характер мышления, приходя к заключению, что мышление обладает \emph{независимой} от материи творческой силой.

\emph{Субъективные идеалисты}, исходя из того, что мы узнаём о свойствах вещей посредством чувственных восприятий, делают вывод о том, что нам фактически \emph{известны лишь ощущения} и эти ощущения составляют единственно известный нам предмет познания.

Для того, чтобы возможность возникновения идеализма превратилась в реальность, чтобы отдельные заблуждения в познании стали философской системой, необходимы \emph{общественные условия} определённого качества. Это происходит тогда, когда указанные заблуждения отвечают запросам определённых социальных групп и поддерживаются ими.

\emph{Социальные условиями} возникновения идеализма являются процессы возникновения \emph{противоположности} между физическим и умственным трудом.

Интеллектуальная деятельность, отделившись от физического труда, приобрела \emph{относительно самостоятельный} характер, некую автономию, и стала в значительной мере \emph{привилегией} материально обеспеченных групп общества.

На ранних этапах представители этих групп \emph{пренебрежительно} относились к физическому труду, одновременно впадая в \emph{иллюзию}, будто именно умственная деятельность является всецело определяющей в существовании и развитии общества.

Представители этих групп были заинтересованы и в том, чтобы развитие познания \emph{не нарушало} традиционных форм общественной жизни, в частности религиозных.

В то же время, философия и религия --- это \emph{качественно различные} формы общественного сознания.

Религия в своих доводах, особенно на ранних этапах развития даже её мировых видов, опирается на \emph{слепую веру}, а философия, даже идеалистическая, обращается \emph{к разуму}, стремится логически обосновывать свои положения.

\subsection{Диалектика и метафизика}

Если вопрос об отношении мышления к бытию --- первый и основной вопрос философии, то \emph{вторым}, так сказать, важнейшим её вопросом является вопрос о том, пребывает ли мир \emph{в неизменном} состоянии или же, напротив, он \emph{постоянно изменяется}, развивается.

Сторонники первой точки зрения называются \emph{метафизиками}, сторонники второй --- \emph{диалектиками.}

Слово «\emph{диалектика}» происходит от древнегреческого --- \emph{dialetomai}, что означает «\emph{вести беседу}, \emph{полемику (спор)}». В древности под диалектикой так и понимали способ раскрытия истины в спорах, в борьбе мнений путём обнаружения противоречий в мыслях собеседника.

Диалектика рассматривает вещи, их свойства и отношения, а также их умственные отражения, понятия \emph{во взаимной связи}, в движении, то есть в возникновении, противоречивом развитии и исчезновении.

\emph{Незнание диалектики} --- \emph{серьёзная слабость} соответствующих философских позиций.

Понятно, что такое обстоятельство \emph{затрудняло и затрудняет} последовательное проведение объективного взгляда на мир, в особенности на общественные процессы.

В частности, \emph{додиалектико-материалистические} формы материализма отличались идеалистическими подходами в понимании общества.

Диалектика --- это учение о развитии в его полном, глубоком, свободном от односторонности виде, учение об относительности человеческого знания, дающего нам отражение \emph{вечно развивающегося} материального мира.

Осознанное применение диалектики \emph{даёт возможность} правильно пользоваться понятиями, учитывать взаимосвязь явлений, их противоречивость, изменчивость, возможность превращения, перехода друг в друга.

Диалектика даёт возможность обоснованного \emph{предвидения} будущих событий, находить эффективные пути их учёта.

Противоположный диалектике метод называется \emph{метафизическим}.

«\emph{Метафизика}» --- слово греческого происхождения. Оно произошло от выражения «\emph{meta ta phisika}», по-русски это значит --- «\emph{то, что идёт после физики}» (науки о природе).

\emph{В литературе} метафизикой часто называли и называют философию, претендующую на \emph{познание} «\emph{сверхъестественного}» бытия, неизменной сущности всего существующего, потустороннего и т.п.

В диалектическом материализме термин «\emph{метафизика}» используется в значении, \emph{противоположном} значению термина «\emph{диалектика}» --- т.е. развитию, качественным изменениям.

Метафизические философы рассматривают предметы и явления \emph{в их обособленности} друг от друга, \emph{как неизменные} в своей сущности, лишённые внутренних противоречий.

Метафизика исходит из \emph{относительной устойчивости}, определённости предметов, явлений, но при этом недооценивает их изменение, развитие.

Метафизический способ мышления \emph{отрицает} объективное существование противоречий, утверждает, что противоречия свойственны \emph{лишь мышлению}, да и то лишь постольку, поскольку оно впадает в заблуждения.

Ранние формы материализма, например, в \emph{античной Греции}, были органически связаны с \emph{наивной диалектикой}, а в дальнейшем материализм, в особенности в \emph{Новое время}, под влиянием ограниченного современного ему естествознания, \emph{приобрёл}, очевидно, метафизический характер, что было существенной слабостью перед лицом диалектически мыслящих идеалистов, среди которых особенно значительная роль принадлежала \emph{Гегелю}.

В истории диалектики можно выделить следующие основные \emph{этапы} (и соответственно её формы):

\begin{itemize}
\item стихийная, наивная диалектика древних мыслителей;
\item диалектика эпохи \emph{Возрождения};
\item диалектика \emph{классической немецкой философии} (И. Кант, Й. Фихте, Ф. Шеллинг, Г. Гегель);
\item наконец, \emph{диалектико-материалистическая философия} как таковая.
\end{itemize}

\emph{Итак, метафизика --- это одна из двух основных исторических философских теоретико-методологических традиций, согласно которой в мире существуют некие вечные и неизменные начала, а существование всякой вещи представляет собой результат их сочетания.}

\emph{Диалектика --- это одна из двух основных исторических философских теоретико-методологических традиций, противоположная метафизике, согласно которой в мире нет никаких вечных и неизменных начал, а существование всякой вещи представляет собой необратимые качественные изменения.}

\subsection{Предмет диалектико-материалистической философии и её отношение к другим наукам}

Диалектико-материалистическая философия --- диалектический материализм \emph{опирается на весь фундамент} современной ему науки и наиболее развитых форм общественной практики.

Представители иных философских традиций \emph{нередко противопоставляют} философию и другие науки, полагая, что философия не может, а в сущности, и не должна быть наукой.

«Философия, как я буду понимать это слово, --- писал английский философ \emph{Б. Рассел}, --- является чем-то промежуточным между теологией и наукой». (\emph{Б. Рассел. История западной философии}. М., 1959, с. 7).

Эта характеристика \emph{вполне применима} ко многим подходам, в современной идеалистической философии, тесно связанной с религией.

Но кроме такого рода философствования существует философия, \emph{чётко ориентированная} на научные подходы, критерии, идеалы, требования.

К этому \emph{стремится} философия диалектического материализма.

Каждая наука, каждый научный вид познания исследует \emph{определённые} закономерности: механические, физические, химические, биологические, экономические, социальные и т.д.

Однако ни одна специальная наука, вид или род научного познания \emph{не пытается} изучать закономерности, общие и для явлений природы, и для развития общества, и для человеческого мышления.

Эти всеобщие закономерности развития \emph{с самого начала} составили предмет диалектико-материалистической философии.

И лишь гораздо позднее, уже в ХХ веке возникает т.н. \emph{общая теория систем} и некоторые другие концепции, пытающиеся также анализировать всеобщие закономерности.

Можно сказать, что диалектико-материалистическая философия есть \emph{попытка} научного изучения наиболее общих законов движения, развития природы, общества и мышления.

Изучение законов и категорий всеобщего диалектического процесса составляет \emph{основное содержание} диалектико-материалистической философии, направленное на развитие общей методологии научного познания мира.

Эта философия принимает \emph{специфическую форму} в каждой специальной науке, с учётом соотношения в них элементов стихийного и сознательного применения диалектических понятий.

\emph{Все науки пользуются} определёнными общими понятиями, или категориями, такими, как например, «\emph{причинность}», «\emph{необходимость}», «\emph{закон}», «\emph{форма}», «\emph{содержание}», «\emph{система}», «\emph{структура}», «\emph{элемент}» и т.д. В рамках специальных наук эти понятия, естественно, \emph{не могут} получить всесторонней разработки, всестороннего развития.

Так, \emph{химия} изучает закономерности химических превращений, \emph{биология} --- закономерности живого.

Лишь философия, философский уровень исследования может и \emph{должен охватить} закон как универсальную связь явлений, форму всеобщности взаимодействий и связей, качественные разновидности которой бесконечно разнообразны на уровне конкретных проявлений.

В специальных науках \emph{мы встречаемся} и с такими понятиями, содержание которых ограничено данной областью исследования.

Таковы, например, основные понятия экономической науки --- «\emph{товар}», «\emph{деньги}», «\emph{капитал}» и т.п.

В отличие от понятий частных наук, \emph{философские категории --- это наиболее общие понятия, которые прямо или косвенно применяются во всех науках.}

Философские категории выражают \emph{наиболее общие связи} между явлениями, представляют собой \emph{ступени} познания мира, обобщают исторический опыт изучения мира, служат инструментами мышления.

Разумеется, изучением философских категорий \emph{нельзя заменить} изучение конкретных вопросов.

Научная философия должна служить средством познания самых разных областей действительности, но она \emph{не должна подменять} и не может подменять частные науки, конкретные подходы, конкретные средства.

Научная философия должна \emph{давать образцы} общенаучной методологии.

У многих современных философов, людей, причисляющих себя к ним, можно видеть \emph{разрыв} между наукой о мышлении (\emph{логикой}) и теорией познания (\emph{гносеологией}), а также противопоставление последних учению о бытии (онтологии).

Диалектико-материалистическая философия \emph{отвергает} подобное метафизическое противопоставление, \emph{обосновывает} принцип единства диалектических, логических и теоретикопознавательных средств и подходов.

Неотъемлемой составной частью диалектико-материалистической философии является её социально-философский блок --- \emph{исторический материализм}.

Диалектико-материалистическая философия имеет \emph{сложную структуру}.

В настоящее время диалектический материализм представляет собой \emph{систему философских дисциплин}, целостное мировоззрение, которое вместе с тем выступает как теория познания, логика, общесоциологическая теория.

Опыт истории показывает, что философия только тогда была действенной, когда она опиралась \emph{на всю совокупность} человеческих знаний.

Специальные науки и философия всегда с пользой \emph{учились и учатся} друг у друга. Многие идеи, лежащие в основе современной науки, \emph{впервые} были выдвинуты философами.

Достаточно указать на гениальные мысли \emph{Левкиппа} и \emph{Демокрита} об \emph{атомном} строении космоса.

Можно напомнить понятие \emph{рефлекса}, впервые предложенное \emph{Р. Декартом}, так же как и сформулированный им \emph{принцип сохранения} движения (постоянство произведения массы на скорость).

Мысль о существовании \emph{молекул} как сложных частиц, состоящих из атомов, была впервые высказана французским философом \emph{П. Гассенди}, а также \emph{М.В. Ломоносовым}, и лишь позднее строго доказана \emph{А. Эйнштейном}.

\emph{Философы сформулировали} идею развития и всеобщей связи явлений мира, принцип материального единства мира, принцип неисчерпаемости материи, составляющий фундаментальную идею всего современного естествознания.

Прогресс науки вместе с тем существенно \emph{обогащал} философию. Тот же материализм изменял свою форму \emph{с каждым} новым эпохальным открытием естествознания.

Представители одного из самых распространённых течений в западной философии ХХ в., а именно \emph{неопозитивизма} утверждали, что наука не нуждается в какой-либо философии. \emph{Г. Рейхенбах}, например, заявил, что естествознание даёт ответы на вопросы, которые безуспешно пыталась решать философия.

Что же касается собственно философских вопросов, которыми не занимается естествознание, то они, по мнению Рейхенбаха, представляют собой \emph{мнимые проблемы}, псевдопроблемы, т.е. они лишены научного смысла.

Следует отметить, что такой подход к вопросу о соотношении философии и естествознания в настоящее время \emph{осуждён} многими последователями неопозитивизма, не говоря уже о т.н. \emph{постпозитивистах}, так как он оказался принципиально неприемлемым для естествознания, которое само ставит и пытается решать философские проблемы.

Впрочем, постпозитивисты так оперируют материалом истории философии, что у них \emph{философия также исчезает}, но существенно иным способом --- способом весьма \emph{неконструктивной деконструкции}.

Современное естествознание, научное познание в целом испытывают огромное \emph{влияние интегрирующих тенденций}, находятся в поисках новых обобщающих теорий. \emph{Таких, например}, как общая теория элементарных частиц, общая теория развития растительного и животного мира, общая теория систем, общая теория управления и др.

Обобщения такого высокого уровня \emph{могут быть сделаны} лишь при наличии солидной философской культуры.

В различных областях научного познания постоянно, и чем дальше, тем всё больше, возникает \emph{внутренняя потребность} в рассмотрении логического аппарата познавательной деятельности, характера теории и способов её построения, анализа соотношения эмпирического и теоретического познания, исходных понятий науки и методов постижения истины.

Всё это является \emph{задачей}, в том числе философского исследования.

Учёный, не прошедший школы философского мышления нередко \emph{впадает в грубые ошибки} мировоззренческого и методологического характера, да и предметного тоже, при трактовке полученных им самим новых фактов.

Наиболее известные естествоиспытатели нашего времени \emph{постоянно подчёркивают} громадное ориентирующее значение философского мировоззрения в научном исследовании.

\emph{Макс Планк} говорил, что мировоззрение исследователя \emph{будет всегда} определять направление его работы.

\emph{Луи де Бройль} указывает, что разобщение между наукой и философией, имевшее место в ХIХ в., \emph{принесло вред}, как философии, так и естествознанию.

\emph{Макс Борн} категорически заявлял, что физика тогда лишь жизнеспособна, \emph{когда она осознаёт} философское значение своих методов и результатов.

Как мировоззрение и философия диалектический материализм \emph{помогает понять} закономерную связь развития естествознания с конкретно-историческими условиями, \emph{глубже осмыслить} социальную значимость и общую перспективу научных открытий и их технических приложений.

Возрастает роль \emph{не только} естествознания и технических областей знания, \emph{но и} социально-гуманитарных наук.

Философия диалектического материализма рассматривает социальный прогресс, происходящие в современном обществе изменения под тем углом зрения, как они \emph{связаны с развитием} человеческой личности.

Одним из важнейших её принципов является последовательный \emph{гуманизм}, который, правда, сильно извращался в годы сталинских репрессий, переселения народов, псевдоколлективизации.

\subsection{Что понимается в диалектическом материализме под «партийностью» философии}

Философское мировоззрение носит заметный \emph{социально-групповой характер}, кстати, далеко не всегда совпадающий с традиционным классовым делением. И только в этом смысле оно «\emph{партийно}».

Но что означает «\emph{партийность}» философского мировоззрения? --- Это, прежде всего, \emph{ориентация} на одно из главных философских направлений --- материализм или идеализм.

При этом философия диалектического материализма \emph{открыто признаёт} свою приверженность материалистическому принципу философствования, пытается последовательно проводить его в отношении любых проблем.

Некоторые философы говорят, что такой своей определённостью диалектико-материалистическая философия \emph{только отталкивает} тем самым от себя значительную часть философов и представителей других общественных наук, особенно на Западе. Говорят, что она \emph{даже отказывается} от всего ценного, что имеется в других философских традициях, подходах, исследованиях, опирающихся на них.

Сразу нужно сказать, что всех этих людей отталкивает \emph{не столько сам по себе} диалектический материализм как философская теория, \emph{сколько тот образ} этой философии, который поневоле возникал и до сих пор возникает у многих людей из-за неумеренной \emph{идеологизации} и \emph{примитивизации} её в известное время в бывшем Советском Союзе. Тем более, что при этом утверждалось, что диалектический материализм --- основа всех наших достижений и «\emph{достижений}».

Достижения в кавычках, при этом, \emph{зачастую} брали верх.

Однако, по меньшей мере \emph{странно слышать} рассуждения об упрощении, якобы допускаемых самим делением на материализм и идеализм. Ведь такое разделение возникло \emph{задолго} до появления диалектического материализма.

\emph{Сами философы} ещё с древнейших времён разделились на две эти группы, и различие сохраняет свою силу до сих пор. Это реальный факт истории философии.

\emph{Ленин} лишь предложил названия этих групп философов: «\emph{линия Платона}» --- для идеалистов и «\emph{линия Демокрита}» --- для материалистов.

В свою очередь это \emph{связано с борьбой} различных социальных групп за свои интересы на всех этапах развития общества.

Идеализм является философско-мировоззренческим \emph{выражением интересов} вполне определённых социальных групп.

Впрочем, \emph{как и материализм}, к которому прибегают, кстати, представители практически всех социальных групп, \emph{поскольку у всех} у них есть объективные интересы, которые далеко не всегда согласуются с идеалистическим мировоззрением.

\emph{В конечном счёте}, объективный диалектико-материалистический подход к действительности отвечает базовым человеческим интересам \emph{всех} людей.

\emph{Нельзя согласиться} с теми философами, которые утверждают, что теория, философская, в том числе, возвышается над практическими интересами людей и представляет собой \emph{знание ради знания}.

Попытки поиска некоей «\emph{третьей линии}» в философии суть не что иное, как \emph{попытка придать} философии такой вид, который бы преодолевал все негативные последствия реально существующих практических подходов.

Но \emph{лучшим выходом} здесь может быть только изменение самих этих практических подходов, \emph{а не отказ} от той или иной теории, тем более, если она доказала свою практическую же ценность.

Сама идея всяческой нейтральности и в этом смысле беспартийности философии \emph{попахивает} нередко откровенным лицемерием.

Нет сомнения в том, что именно ориентированная на научную объективность философия \emph{отвечает коренным интересам} современного человечества, «\emph{обречённого}» на дальнейшее прогрессивное развитие (не путать с «\emph{обречённостью}» \emph{экзистенциалисто}в, о которой пойдёт речь в самом конце пособия).

Что касается заявлений «\emph{последовательной и непримиримой борьбы с враждебными делу социализма теориями и взглядами}», то их также \emph{нужно понимать в контексте} объективного подхода к оценке интересов различных социальных групп, а не просто отказываться от этого подхода.

\section{Возникновение и развитие диалектико-материалистической философии}

\subsection{Теоретические источники диалектико-материалистической философии}

Возникновение диалектико-материалистической философии можно рассматривать как \emph{качественный скачок} в развитии мировой философии.

Теоретическими источниками диалектического материализма являются работы представителей, \emph{прежде всего}, немецкой классической философии, хотя и далеко не только они.

Следует высоко оценить \emph{завоевания прежнего} философского материализма:

\begin{itemize}
\item стремление объяснить мир \emph{из него самого}, не прибегая к допущению сверхъестественных сил;
\item учение о природе, материи и её \emph{самодвижении};
\item взгляд на познание как на \emph{специфическое отражение} окружающей человека действительности;
\item стремление объяснить историю человечества \emph{естественными}, т.е. эмпирически констатируемыми факторами и т.д.
\end{itemize}

Вместе с тем следует видеть историческую \emph{ограниченность} этого материализма в философии.

Материализм, предшествующий диалектическому, носил по преимуществу \emph{механистический} характер, т.е. пытался объяснять всё многообразие явлений природы и общества законами лишь механического движения.

Механистическое мировоззрение было \emph{вполне прогрессивным в XVII и XVIII вв}., когда наибольшее развитие среди других наук получила механика.

Однако к середине XIX в. оно стало уже совершенно недостаточным, \emph{особенно в объяснении} биологических, психических и социальных процессов.

Тот материализм \emph{был преимущественно} метафизическим материализмом, т.е. рассматривал природу и общество как, в сущности, неизменные.

\emph{Это не значит}, конечно, что его представители отрицали движение материи, не признавали отдельных фактов изменения и развития. Некоторые из них, как это \emph{часто бывает в науке}, даже высказывали глубокие догадки об изменениях, совершающихся в неорганической природе, о возникновении одних видов живых существ из других.

Однако в целом для этого материализма было характерно непонимание \emph{всеобщности и существенности развития}, истолкование этого процесса лишь как увеличения или уменьшения уже существующего.

Соответственно этому и \emph{движение понималось} главным образом как простое перемещение в пространстве и времени, как вечное повторение, круговорот явлений природы.

Разумеется, метафизиками тогда были \emph{не только} материалисты, но в подавляющем большинстве и идеалисты.

Ещё один, \emph{третий}, недостаток старого материализма заключается в том, что он, ограничиваясь материалистическим пониманием природы, \emph{не был в состоянии} материалистически объяснить общественную жизнь.

Представители этого материализма \emph{выступали, правда, против} религиозного истолкования истории.

Они доказывали, что в общественной жизни действуют не сверхъестественные силы, \emph{а естественные}. Но источник движения общества усматривали \emph{исключительно в духовных}, идеальных факторах:

\begin{itemize}
\item \emph{в сознательной деятельности} исторических личностей --- королей, правителей, в чувствах, страстях людей, например, в честолюбии полководцев;
\item \emph{в эгоизме}, любви, ненависти, в новых идеях философов, политиков и т.д.
\end{itemize}

Все эти идеальные побудительные мотивы деятельности людей \emph{реально существуют и оказывают}, конечно, своё влияние на ход событий в истории.

Но \emph{они не видели} объективной, материальной обусловленности духовных побудительных мотивов деятельности людей, специфического, \emph{т.е. отличного от природной} (например, географической) основы человеческой жизни, материального базиса общественной жизни в виде, прежде всего, социально-экономических отношений людей.

Механистичность и метафизичность \emph{материализма XVII -- XVIII вв}. были подвергнуты критике уже классиками немецкой идеалистической философии конца XVIII -- начала XIX в., в особенности \emph{Гегелем}.

\emph{Диалектика Гегеля} представляла собой наиболее полную для своего времени теорию развития, \emph{хотя и разработанную} с ложных, идеалистических позиций.

\emph{Именно Гегель} первый дал всеобъемлющее и сознательное изображение всеобщих диалектических форм движения.

«\emph{Рациональным зерном}» диалектики Гегеля была \emph{идея всеобщности}, существенности и необходимости \emph{развития}, понимаемого как качественные изменения, осуществляющегося путём выявления и \emph{преодоления} внутренних противоречий, \emph{взаимопревращения} противоположностей, \emph{скачкообразного перехода} количественных изменений в качественные, отрицания старого новым.

\emph{Основное положение} гегелевской философии о постоянно совершающемся в мире процессе развития логически приводило к тому, что \emph{и борьба в общественной жизни} коренится в универсальном законе вечного изменения, развития и потому является разумной и необходимой.

Но \emph{сам Гегель} считал природу и общество воплощением духовной, божественной сущности --- «\emph{абсолютной идеи}».

Гегель \emph{не признавал} развития материи, природы, которые представлялись ему внешним проявлением этой идеи.

Напротив, \emph{Людвиг Фейербах} в своей материалистической философии противопоставил идеализму Гегеля свой антропологический материализм, согласно которому мышление есть не божественная сущность, а \emph{естественная человеческая способность}, неотделимая от мозга, телесной организации человека, неразрывно связанная с чувственным отражением внешнего материального мира.

Человек рассматривался Фейербахом как \emph{высшее выражение природы}.

Именно через человека \emph{природа ощущает}, воспринимает, познаёт себя.

Подчёркивая единство человека и природы, \emph{Фейербах} вместе с тем стремился раскрыть \emph{отличие человека} от других живых существ.

Неотъемлемым свойством человеческих индивидов Фейербах считал \emph{общительность}, тяготение их друг к другу.

Но Фейербах \emph{не дошёл} до понимания сущности человеческого общества и законов его развития, сводя общение между индивидами к любви, к духовной общности.

Фейербах \emph{недооценил} диалектику Гегеля, не понял, что она может и должна быть освобождена от идеализма и разработана на материалистической основе.

\emph{Философия Фейербаха} содержала в себе некоторые зародыши материалистического понимания общественных явлений, \emph{в особенности} религии, критика которой составляет, пожалуй, важнейшее содержание его учения.

Но в отличие от материалистов XVII -- XVIII вв. \emph{Фейербах} не сводил причины возникновения и существования религии к невежеству и обману, стремился показать, как в религиозных образах \emph{выражается жизнь людей}: их страдания, стремление к счастью, зависимость от природы и друг от друга.

В философии Фейербаха атеизм сочетается с \emph{попыткой рационально истолковать} религиозные догматы, с признанием необходимости гуманистической веры в человека, \emph{будто бы в чём-то} родственной религии.

Материализм Фейербаха \emph{наметил}, правда, лишь в самой общей форме, пути дальнейшего развития философского материализма.

\subsection{Диалектико-материалистическая философия и великие естественнонаучные открытия середины xix в}

Наиболее значительными \emph{достижениями} естествознания 30-50-х годов XIX в. были:

\begin{itemize}
\item открытие \emph{закона превращения} энергии,
\item открытие \emph{клеточной структуры} живых организмов,
\item создание \emph{эволюционного учения} --- дарвинизма.
\end{itemize}

В начале 40-х годов XIX в. немецкий врач \emph{Ю. Майер} открыл закон сохранения и превращения энергии, согласно которому \emph{определённое} \emph{количество движения} в одной из его форм (химической, тепловой и т.д.) превращается в равное ему количество движения \emph{в какой-либо другой} форме.

Этот закон сохранения теоретически и экспериментально обосновали \emph{Г. Гельмгольц} и \emph{М. Фарадей}, а \emph{Дж. Джоуль} и \emph{Э. Ленц} установили механический эквивалент теплоты, т.е. подсчитали, какое количество механической энергии даёт единица тепловой энергии.

\emph{Было доказано}, что механическое перемещение, теплота, электричество и другие состояния материи --- это качественно отличные друг от друга \emph{формы её движения}, которое не возникает и не уничтожается, но постоянно трансформируется.

Отсюда следовал \emph{вывод}, что движение материи \emph{не сводимо} к её перемещению в пространстве и во времени, т.е. к механическому движению.

Формы движения материи закономерно \emph{превращаются} друг в друга.

Это и есть \emph{качественное} изменение материи, следствием которого является её развитие.

Если представители прежнего материализма утверждали, что \emph{движение не привносится} в природу извне, что оно есть способ существования материи, то \emph{теперь стало возможным} естественнонаучное доказательство этого философского положения и диалектическое понимание единства движения и материи.

Правда, ни \emph{Майер}, ни другие естествоиспытатели \emph{не сделали, да и не могли} сделать философских выводов из закона превращения энергии.

Выводы \emph{впервые} были сформулированы одним из основоположников диалектического материализма --- \emph{Фридрихом Энгельсом}.

Не менее выдающимся достижением естествознания было открытие \emph{клеточной структуры} живого.

Это открытие \emph{вплотную} подводило к диалектико-материалисти-ческому пониманию органической природы.

Существование клеток \emph{стало известным ещё} в XVII -- XVIII вв., поскольку отдельные клетки и группы клеток \emph{постоянно обнаруживались} при микроскопическом исследовании тканей живых организмов.

Но лишь в XIX в. \emph{был непосредственно поставлен} вопрос о физиологической роли клетки, о том, что клетки являются анатомическими единицами животных и растительных тканей.

Немецкие биологи \emph{М. Шлейден} и \emph{Т. Шванн} разработали \emph{в 1838-1839 гг}. клеточную теорию.

\emph{Шванн}, в частности, установил, что животные и растительные клетки в основном имеют \emph{одинаковую структуру} и выполняют одну и ту же физиологическую функцию.

Возникновение и развитие организма происходит \emph{путём размножения клеток}, их непрерывного обновления --- возникновения и отмирания.

Клеточная теория \emph{обосновывала} внутреннее единство всех живых существ и \emph{косвенно указывала} на единство их происхождения.

Диалектико-материалистическая интерпретация клеточной теории была осуществлена впервые \emph{Ф. Энгельсом} в работах «\emph{Анти-Дюринг}» и «\emph{Диалектика природы}».

Эволюционная теория \emph{Чарльза Дарвина} --- \emph{третье} великое открытие естествознания середины XIX в.

\emph{Эволюционная теория Дарвина} положила конец воззрению на виды животных и растений как на ничем не связанные, случайные, «\emph{богом созданные}», неизменные и заложил тем самым \emph{основы теоретической биологии}, которая до Дарвина была в основном эмпирической наукой.

\emph{Дарвин доказал} изменяемость видов растений и животных, единство их происхождения.

Известно, что эволюционные идеи в общем виде высказывались \emph{задолго до Дарвина} философами и естествоиспытателями.

Положение о возможности трансформации видов выдвигал, например, французский материалист XVIII в. \emph{Дени Дидро}.

В отличие от своих предшественников, \emph{Дарвин не ограничивался догадками}: на основе громадного эмпирического материала он сформулировал ряд закономерностей видообразования.

При этом он рассматривал человека \emph{как высшее звено} в общей цепи развития животного мира, \emph{опровергая} тем самым, желая того или нет, христианские догматы и ненаучные представления о природе человека, распространённые в тогдашнем естествознании.

\emph{Хотя}, конечно, Дарвина нельзя считать сознательным сторонником философии диалектического материализма.

Названые открытия \emph{активно способствовали}, просто требовали формирования новой философии, каковой и стала диалектико-материалистическая философия.

\subsection{Создание диалектического материализма --- качественный скачок в развитии философии}

Диалектико-материалистическая философия возникла как прямое и непосредственное \emph{продолжение} достижений предшествующей мировой философской мысли.

Основоположники диалектико-материалистической философии --- \emph{Карл Маркс} и \emph{Фридрих Энгельс} --- вначале своей теоретико-философской деятельности были объективными идеалистами, сторонниками т.н. \emph{младогегельянского} крыла гегелевской философской школы.

\emph{На отход Маркса} и \emph{Энгельса} от гегельянства и создание принципиально иной философской традиции огромное влияние оказала их \emph{общественно-политическая позиция} --- позиция революционного демократизма.

Уже в ранних работах о Шеллинге \emph{Ф. Энгельс} отмечает \emph{противоречие} между диалектическим \emph{методом} Гегеля, требовавшим подходить к действительности как к постоянно меняющейся, и его \emph{консервативной системой}, провозглашавшей неизбежность завершения мировой истории на той ступени общественного развития, которой в основном уже достигла Западная Европа.

Работая, поначалу, \emph{независимо} друг от друга, Маркс и Энгельс пришли в основном к единым философским взглядам. \emph{С 1844 г. начинается дружба} и философский творческий союз Маркса и Энгельса.

Некоторые исследователи творчества Маркса и Энгельса изображают диалектико-материалистическую философию \emph{как соединение} диалектического (но идеалистического) \emph{метода Гегеля} и материалистической (но метафизической) \emph{теории Фейербаха}.

Это, конечно, явное \emph{упрощение}. \emph{Механически соединить} идеализм и материализм, диалектику и метафизику принципиально невозможно. Нужна была переработка, как положений старого материализма, так и положений идеалистической диалектики Гегеля, наполнение их реальным содержанием, почерпнутым из наук о природе и обществе.

Было бы поверхностным рассматривать материалистическую диалектику \emph{лишь как метод}, а философский материализм --- \emph{лишь как теорию,} применявшую этот метод для исследования.

Материалистическая диалектика не только метод, но и теория, а именно \emph{теория развития}, учение о наиболее общих законах развития природы, общества и познания.

Философский же материализм есть не только теория, \emph{но и материалистический метод}, определённый подход к исследованию явлений.

В диалектико-материалистической философии материализм и диалектика представляют собой \emph{не независимые} друг от друга части, а \emph{единое целое}, ибо сама действительность в своей основе одновременно материальна и диалектична.

Важнейшей стороной этой философии стало \emph{создание принципиально иной} социальной философии, а именно, распространение материалистического подхода на понимание общественной жизни.

Представители прежнего материализма \emph{не могли вычленить} удовлетворительно объективной, материальной основы общественной жизни, тех обстоятельств, которые лежат в основе развития всех форм общественной жизни, а именно, --- \emph{объективных, материальных отношений}, присущих обществу, начиная со сферы материального производства, и кончая самыми духовно насыщенными сферами.

\emph{Суть материалистического понимания общественной жизни} заключается в установлении \emph{внутренней связи} всех многообразных форм жизни людей, при их зависимости от порождающих, поддерживающих их материальных оснований.

Выяснение роли \emph{предметной трудовой деятельности} в истории человечества составляет \emph{исходный пункт} социальной философии диалектического материализма --- исторического материализма.

Метафизические материалисты правильно подчёркивали, что \emph{люди сами делают свою историю}, но они не смогли последовательно объективно, научно обосновать это положение, \emph{становились зачастую} на путь субъективистского истолкования исторических событий, как якобы вызванных только волей людей, в особенности выдающихся личностей.

Диалектико-материалистическая философия, её социально-философский раздел --- \emph{исторический материализм} идут по пути преодоления противопоставления философского знания содержанию специальных, частных наук, отказываясь в то же время от статуса «\emph{науки наук}».

Научная философия может быть \emph{не} «\emph{наукой наук}», свысока смотрящей на конкретные науки, а научным мировоззрением, \emph{знанием, обобщающим результаты всех этих наук}, а также всего практического опыта людей, и раскрывающим наиболее общие закономерности развития природы, общества и познания.

Диалектический материализм \emph{отказывается} от положения об абсолютности, завершенности, неизменности философского знания, якобы не требующего доказательства и дальнейшего развития. Он полностью переходит на позиции научного подхода, который всегда \emph{открыт для новых} выводов, постоянно развивается, \emph{отказывается от устаревших} положений.

Диалектико-материалистическая философия \emph{стремится широко и последовательно применять} принятые в науке методы исследования, в том числе гипотезы, постулаты, факты, их анализ, исследование вероятностного характера тех или иных процессов и т.д.

Диалектико-материалистическая философия, как научная философия, \emph{не есть откровение}, возвещённое неким гением.

Как и всякая наука, она разрабатывается \emph{сообществом учёных}, исследователей.

Как и всякая система научных знаний, диалектический материализм оценивает свои положения лишь \emph{как приблизительное} отражение действительности, как единство абсолютного и относительного аспектов наших знаний.

Философия диалектического материализма \emph{исходит из} необходимости преодоления противопоставления философии, науки вообще, практической

Философия \emph{не существует в абстрактной стихии} чистого мышления, равно как не существует никакого «\emph{чистого}», независимого от действительности мышления вообще.

Научное, в т.ч. философское объяснение \emph{должно служить} теоретическим обоснованием чего-либо в действительности.

Ряд интересных и важных моментов развития диалектико-материалистической философии \emph{в самом конце XIX -- начале XX в.} связан с теоретическими работами \emph{В.И. Ленина}, в частности, представленных в его, не побоимся этого слова, \emph{трактате} «\emph{Материализм и эмпириокритицизм»}, особенно в связи с проявлением качественно новых уровней в развитии современной науки, естествознания в первую очередь.

\section{Всеобщие диалектические законы развития}

\emph{Наиболее полное} на сегодняшний день учение о развитии --- диалектика --- составляет одну из основ диалектико-материалистической философии, её коренное теоретическое обоснование.

Всеобщие \emph{законы и категории диалектики} (о категориях речь пойдет специально в следующей главе) фиксируют существенные черты, признаки любого развивающегося явления, к какой бы области действительности оно ни относилось.

\subsection{Материалистическая диалектика как исследование всеобщей связи и развития}

Современное научное мировоззрение опирается на \emph{принцип движения}, изменения, развития как всеобщий фундаментальный принцип бытия и познания.

Этот принцип прокладывает себе дорогу на протяжении всей истории человеческой мысли \emph{в борьбе} с различными метафизическими воззрениями.

Без ложной скромности можно сказать, что \emph{огромную роль} в утверждении понятия о развитии и в разработке научной теории развития сыграла философия.

Уже \emph{задолго} до того, как конкретные науки о природе и обществе сумели подойти к исследуемым ими вопросам с позиций развития, \emph{философия выдвинула} положение о развитии как существеннейшем принципе бытия.

Так, многие \emph{древнегреческие философы} рассматривали весь мир и каждый отдельный предмет как \emph{результат процесса} становления.

Правда, их диалектика была \emph{наивной}. Но уже сама постановка вопроса о развитии как всеобщем законе всего существующего, сущего оставила \emph{глубокий след} в истории познания мира.

В дальнейшем, опираясь на конкретные области знания, философия разрабатывает \emph{всё более глубокие} представления о сущности развития.

В течение ряда веков, при этом, \emph{господствующим} мировоззрением была метафизика как учение о неизменности и постоянстве вещей и их свойств.

Только начиная \emph{примерно с конца XVIII в}. в науку и философию стали снова \emph{проникать} идеи развития, изменения, но уже основанные на гораздо более глубоком понимании природы.

Материалистическая диалектика \emph{возникла в результате} обобщения достижений науки, а также исторического опыта человечества, доказывавшего, что общественная жизнь и человеческое сознание, как и природа, находятся в состоянии постоянного изменения, развития.

В соответствии с этим \emph{диалектика определяется} в диалектико-материалистической философии \emph{как наука о всеобщих законах движения и развития} природы, человеческого общества и мышления, как учение о развитии в его наиболее полном, глубоком и свободном от односторонности виде, учение об относительности человеческого знания, дающего нам отражение вечно развивающейся материи.

Понятие развития нельзя осознать без понятий \emph{связи и взаимозависимости}, взаимодействия явлений. \emph{Вне связи и взаимодействия} между различными объектами, а также различными сторонами и элементами внутри каждого объекта невозможно было бы никакое движение.

Для того чтобы \emph{правильно понять} любое явление, нужно рассматривать его в связи с другими явлениями, знать его происхождение и дальнейшее развитие.

Связь между предметами носит различный характер: одни из них \emph{непосредственно} связаны друг с другом, другие --- через ряд \emph{опосредующих звеньев}, но всегда эта связь выступает как взаимозависимость, взаимодействие.

Любые системы в мире образуются \emph{лишь в результате взаимодействия} между составляющими их элементами.

Точно так же \emph{и все свойства} тел возникают на основе взаимодействия, движения и проявляются через них.

\emph{Взаимодействие универсально}: оно включает в себя всевозможные изменения свойств и состояний предметов, все типы связей между ними.

Понятие «\emph{взаимодействие}» является одной из важнейших философских категорий.

\emph{В мире нет} абсолютно изолированных явлений, каждое из них обусловлено какими-либо другими явлениями.

Конечно, в процессе познания мы, чтобы исследовать тот или иной предмет, \emph{выделяем его} на первых порах из всеобщей связи, в теории, конечно. Но рано или поздно логика исследования требует \emph{восстановления этой связи}, иначе невозможно получить истинный образ, знание предметов.

Каждое явление и весь мир в целом представляют собой \emph{сложную систему}, существеннейшей стороной которой является \emph{связь} и взаимодействие причин и следствий.

\emph{Благодаря} этой связи одни явления и процессы порождают другие, совершается \emph{переход} одних форм движения в другие --- осуществляется вечное движение и развитие.

Мир представляет собой \emph{не хаотическое} и случайное нагромождение предметов, событий, процессов, \emph{а закономерное} целое, в котором господствуют объективные законы, существующие независимо от сознания и воли людей\textsubscript{.}

\emph{Всеобщая, универсальная связь}, взаимодействие явлений и процессов должны найти свое отражение во взаимосвязи человеческих понятий. Только в этом случае человек \emph{может познать} мир в его единстве и движении.

Научное понятие или \emph{система понятий}, образуемые человеком в процессе познания, есть не что иное, как \emph{отражение} внутренней связи явлений, процессов между ними.

Наука всегда так или иначе \emph{стремилась} к раскрытию связей явлений. Но раньше исследование отдельных явлений \emph{как связного единого целого} не занимало такого места в науке, как в наше время.

Анализ явлений и процессов \emph{как систем}, т.е. как целостностей, элементы и части которых находятся в определённых внутренних и внешних связях и взаимодействиях, --- одна из \emph{характерных особенностей} современной науки.

Задача, цель науки заключается прежде всего в том, чтобы понять природу и общество как закономерный процесс движения и развития, как такой, который обусловливается и направляется объективными законами.

Но что такое сам закон? \emph{Закон --- это внутренняя связь и взаимообусловленность явлений}. Не всякая связь явлений и процессов есть закон, закономерность.

Для закона характерна именно \emph{необходимая,} \emph{существенная, устойчивая, повторяющаяся, внутренне присущая явлениям связь и взаимная обусловленность}.

Связь может быть и \emph{внешней}, несущественной, возникающей в силу случайного стечения и переплетения обстоятельств. Такая связь накладывает свой отпечаток на развитие, \emph{но не определяет} его.

Закон есть \emph{выражение необходимости}, т.е. такой связи, которая определяет при наличии известных условий характер развития.

Такова, например, \emph{связь между} экономическим строем общества и другими социальными явлениями (государством, формами общественного сознания и т.д.).

Изменение экономического строя \emph{вызывает с необходимостью} закономерные изменения других сторон общественной жизни.

\emph{Закон есть форма всеобщности.}

Благодаря познанию законов мы схватываем в понятиях сложный и многообразный \emph{мир в его единстве}, цельности.

Опираясь на знание законов природы и общества, люди действуют осознанно, предвидят наступление тех или иных событий, преобразуют предметы природы и их свойства в своих интересах, целенаправленно изменяют социальные условия своей жизни.

Материалистическая диалектика не занимается \emph{придумыванием}, искусственным изобретением связей, законов. Она ставит перед наукой \emph{задачу} открывать законы в самом объективном мире.

Рассмотрим теперь \emph{основные типы} объективных законов. Их можно разбить на \emph{три} большие группы:

\begin{enumerate}
\item \emph{частные}, выражающие отношения между специфическими свойствами объектов либо же между процессами в рамках той или иной формы движения;
\item \emph{общие} для больших совокупностей объектов и явлений;
\item \emph{всеобщие}, или универсальные.
\end{enumerate}

\emph{Первые} проявляются в определённых конкретных условиях и имеют весьма ограниченную сферу действия.

Во \emph{вторую группу} входят законы, выражающие связь между сравнительно общими (\emph{но не всеобщими}) свойствами многих качественно разнородных материальных объектов, между часто повторяющимися явлениями.

\emph{Сюда относятся}, например, законы сохранения массы, энергии, заряда, количества движения в физике, закон естественного отбора в биологии и т.п.

Законы \emph{третьей группы} выражают универсальные диалектические отношения \emph{между всеми}, \emph{любыми} существующими явлениями, их свойствами, тенденциями изменения.

Наряду с качественным многообразием материи присуще и определённое \emph{внутреннее единство}, проявляющееся во всеобщей связи и обусловленности всех явлений, в историческом развитии и превращении одних форм материи в другие.

Это единство и выражается \emph{во всеобщих} универсальных законах.

\emph{Как философская наука диалектика имеет своим предметом всеобщие законы}.

Законы диалектики имеют \emph{всеобщее} познавательное, методологическое значение и, следовательно, диалектика --- это метод не одной какой-либо области знания, а \emph{всеобщий метод познавательной деятельности людей}.

Важно учесть, что диалектика --- это не «\emph{универсальная отмычка}», с помощью которой можно открыть любую научную тайну.

Познавательное значение диалектики состоит в том, что она указывает \emph{правильный подход}, истинный угол зрения на действительность, но реализовать этот подход \emph{можно лишь} при конкретном исследовании соответствующих явлений во всех их многообразии.

Всеобщие законы развития разрабатываются диалектикой как \emph{законы бытия} и \emph{законы познания}, которые по своей сущности, по своему содержанию едины, совпадают. Вне такого единства невозможно никакое истинное познание, мышление.

Диалектика есть \emph{не только учение} о законах развития бытия, но и \emph{содержит все} необходимые основания для построения соответствующей теории познания, логики, т.е. учения о формах и законах мышления, познания.

Обладая объективным содержанием, законы диалектики являются вместе с тем (вместе с категориями диалектики) \emph{ступеньками познания}, его совершенствования, развития; \emph{логическими формами} отражения реальной действительности.

Перейдём к рассмотрению так называемых основных законов диалектики.

\subsection{Закон перехода количественных изменений в качественные и обратно}

\emph{Центр тяжести} в диалектическом подходе к развитию состоит не просто в утверждении того, что всё развивается, а в последовательно объективном раскрытии механизма этого развития.

Сегодня всё ещё существуют \emph{различные взгляды} на принцип развития.

Можно, однако, утверждать, что существует \emph{две} наиболее существенные концепции развития, --- \emph{диалектическая} и \emph{метафизическая}.

\emph{Метафизическая} трактует развитие как простое уменьшение и увеличение, как повторение. В ней остаётся в тени \emph{самодвижение}, его \emph{двигательная сила}.

Отличительная черта \emph{диалектической концепции} развития состоит в понимании его не как простого количественного роста существующего, а как процесса исчезновения, \emph{распада старого и возникновения нового}.

Этот процесс и получает свое отражение и \emph{обоснование в законе} перехода количественных изменений в качественные и обратно.

Но чтобы разобраться в этом законе, необходимо ознакомиться с рядом всеобщих понятий, диалектических категорий --- таких, как \emph{свойство}, \emph{качество}, \emph{количество}, \emph{мера}.

Как мы знаем, \emph{нас окружают} вещи, предметы, находящиеся в определённых связях и отношениях друг с другом. Познание их \emph{начинается} с каких-то внешних и непосредственных их проявлений, которые возникают лишь в процессе взаимодействия с другими вещами.

\emph{Вне отношений} и взаимодействий одной вещи с другими ничего о них знать невозможно.

Благодаря взаимодействию вещей проявляются их \emph{свойства}, которые и познаёт человек, а через них и сами вещи.

\emph{Металл}, например, имеет такие свойства, как плотность, сжимаемость, температура плавления, тепло- и электропроводность и др.

Из этого можно было бы заключить, что \emph{вещь есть не что иное}, как совокупность тех или иных свойств, что, следовательно, познать вещь --- значит \emph{установить}, из каких свойсте она «состоит».

Но как ни важны свойства вещи для её характеристики, она \emph{не сводится} к ним. Отдельные свойства могут \emph{изменяться или даже исчезать} без того, чтобы вещь перестала быть самой собой. Следовательно, сами свойства вещи представляют собой проявление \emph{чего-то более существенного}, что характеризует вещь.

Это более существенное есть \emph{качество} вещи.

Качество характеризует вещь именно как эту, \emph{а не иную} вещь.

Благодаря такой характеристике одни вещи отличаются от других, и вследствие этого \emph{образуется то качественное} разнообразие действительности, которое нас так поражает.

Вещь, \emph{теряя свое качество}, перестает быть тем, что она есть.

Качество есть \emph{нечто большее}, чем простая совокупность даже существенных свойств, ибо оно выражает единство, целостность вещи, её относительную устойчивость, тождественность с самой собой.

Качество тесно связано со \emph{структурностью} вещи, т.е. с определённой формой организации составляющих её элементов, свойств, вследствие чего оно есть не просто совокупность последних, а их единство и целостность.

Структурность вещи позволяет \emph{понять, почему} изменение или даже утрата тех или иных свойств вещи не ведёт непосредственно к изменению её качества.

До тех пор, пока не изменится сама структура связи между элементами вещи, она \emph{не перестаёт} быть самой собой.

Уже в самом определении качества \emph{мы сразу сталкиваемся} с диалектикой предмета, вещи. Ведь когда мы определяем вещь в её качественном своеобразии, то мы её соотносим с другими вещами и, следовательно, \emph{устанавливаем границы} её бытия. За этими границами она уже не то, чем была, она уже нечто иное.

Можно сказать, что качество вещи \emph{тождественно} с её конечностью, ограниченностью.

Качественная определённость какого-либо рода предметов означает их \emph{одинаковост}ь, подобие. Конечно, они отличаются по некоторым своим свойствам, но по своему качеству они тождественны.

Но будучи тождественными по своему качеству, \emph{они различаются} друг от друга количественно. Их может быть много или мало, они могут отличаться друг от друга по объему, величине и т.д.

Иначе говоря, тождественность, однородность предметов по качеству есть предпосылка для понимания другой их стороны --- \emph{количественной}.

\emph{Гегель говорил}, что количество есть «\emph{снятое качество}», т.е. анализ вещей в качественном отношении неизбежно приводит нас к категории количества.

Это естественно, так как \emph{нет, и не может быть} отдельно качества от количества, вещь одновременно обладает и количественной и качественной характеристиками.

\emph{Лишь в понятиях}, в целях познания мы искусственно отделяем одно от другого, но отделяем для того, чтобы затем установить их связь.

Категория \emph{количества} предполагает абстрагирование, т.е. отвлечение от качественного многообразия вещей.

Общая закономерность познания такова, что \emph{сначала} исследуются \emph{качественные различия} вещей, а \emph{затем их количественные} закономерности. Последние позволяют глубже познать сущность вещей.

Например, как известно из истории науки, она долго не могла понять причину качественного \emph{различия цветов} --- красного, зеленого, фиолетового и др. Это удалось выяснить только тогда, когда было установлено, что различие цветов определяется количественно различной \emph{длиной} электромагнитных волн.

Из сказанного можно видеть, что \emph{количество есть выражение однородности вещей, их подобия, сходства}, вследствие чего они могут подвергаться операции, действию, процессу увеличения или уменьшения, разделения или объединения и т.п.

Количество \emph{находит своё воплощение} в величине, числе, объеме, в степени и интенсивности развития тех или иных сторон объекта, в темпах протекания процессов, в пространственно-временных свойствах явлений. Чем более сложны явления, тем сложнее и их количественные параметры, тем труднее они поддаются точному количественному анализу.

Существенное отличие количества от качества заключается в том, что \emph{можно изменить} некоторые количественные свойства без того, чтобы вещь претерпела какие-либо значительные перемены.

\emph{Скажем}, размер вещи может быть большим или меньшим. Это на ней как на объекте определённого качества не сказывается.

\emph{Или можно} повышать температуру металла на десятки и даже сотни градусов, но он не плавится, т.е. не изменяет до поры до времени свего агрегатного состояния.

Это значит, что количественная характеристика \emph{не столь тесно связана} с состоянием вещи, как качественная. Поэтому при анализе количественных отношений можно в каких-то границах \emph{отвлечься от качества} предметов.

На этой особенности количества \emph{основана} широкая применимость количественных, математических методов во многих науках, исследующих качественно различные объекты.

Однако количественные изменения \emph{находятся во внешних} отношениях к вещи, её качеству лишь в определённых для каждой вещи пределах. Иногда даже \emph{малейший выход} за эти пределы, границы влечёт за собой коренное качественное изменение вещи (ситуация «\emph{последней капли}», см. \emph{синергетика}).

Конечно, \emph{любые} количественные изменения суть изменения, оказывавшие своё влияние на состояние вещи, её свойства. (При нагревании того же металла растёт его электрическое сопротивление).

Но только количественные изменения, достигающие \emph{известного уровня}, предела, связаны с коренными качественными изменениями предметов.

Зависимость качества от количества можно проследить на хорошо известном уже школьникам примере качественного \emph{многообразия атомов}.

Каждый вид атомов определяется \emph{числом протонов}, заключённых в ядрах, или, как говорят, порядковым номером в периодической системе элементов. Одним протоном больше или меньше --- и атом становится качественно иным.

Таким образом, качество вещи \emph{неразрывно связано} с определённым количеством.

Связь и взаимозависимость качества и количества называется \emph{мерой} вещи.

Категория меры выражает взаимоотношение между этими сторонами объекта, когда качество его \emph{основано на определённом количестве}, а последнее есть количество определённого качества.

Именно изменения этих взаимоотношений, \emph{изменения меры} объясняют тот механизм, в силу которого развитие следует понимать \emph{не как движение} в каких-то постоянных и неизменных рамках, \emph{а как смену} старого новым, как вечный и безостановочный процесс обновления существующего.

На какой-то ступени количественные изменения \emph{достигнут такого уровня}, когда былая гармония, устойчивость качества и количества превратится в дисгармонию, неустойчивость. И тогда старое качественное состояние \emph{уступает место} новому, словом, происходит то обновление существующего, которое и составляет сущность развития как диалектического процесса.

Переход количественных изменений в качественные сопровождается и \emph{обратным процессом}: новое качество порождает новые количественные изменения.

Это естественно, так как новое качество \emph{закономерно связано} с другими количественными параметрами.

Количественные изменения совершаются \emph{непрерывно} и \emph{постепенно}

Качественные изменения происходят в виде \emph{перерыва непрерывности}, постепенности.

Развитие, \emph{будучи единством} количественных и качественных изменений, есть вместе с тем \emph{единство непрерывности и прерывности.}

Если отрицать развитие как единство той и другой формы, то тогда нужно \emph{принять одно из д}вух возможных, но одинаково сомнительных представлений о мире:

--- \emph{либо считать}, что всё богатство мира, многообразные проявления неорганической и органической природы, многоликий мир растений и животных, человек существовали всегда и \emph{изменялись лишь количественно},

--- \emph{либо} полагать, что всё это каким-то чудом \emph{возникло сразу}, внезапно.

В истории познания, в т.ч. в истории науки и философии имели место оба эти представления, но \emph{они опровергнуты} всем ходом познания и исторической практики.

Всякое качественное изменение происходит \emph{скачкообразно}, \emph{в виде скачка,} завершая какой-либо процесс, скачок означает момент качественного изменения предмета, перелом, критическую стадию в развитии. Он как бы \emph{завязывает} новый узел в общей нити развития.

Скачок есть \emph{форма развития}, протекающая значительно быстрее, чем форма непрерывного развития. Это \emph{период} наиболее интенсивного развития, когда старое, прежнее качество преобразуется, уступая место для новых ступеней развития.

Так, \emph{познавательные революции} дают огромный толчок развитию экспериментальной техники и теоретических средств познания. В основе же самих таких революций лежат \emph{новые крупные открытия}.

Благодаря тому, что развитие осуществляется как единство непрерывности и скачкообразности, при этом одна мера уступает место другой, или превращается в другую меру, развитие можно представить в виде «\emph{узловой линии мер}» (\emph{Гегель}).

\emph{Наука} даёт всё больше доказательств в пользу воззрения на объекты и их развитие \emph{как на единство} непрерывности и прерывности.

Качественно различные формы движения материи --- механическая, физическая, химическая и др. --- рассматриваются наукой, как «\emph{узловые точки}» в процессе постепенной \emph{дифференциации} материи.

Такими же «\emph{перерывами непрерывности}» являются \emph{дискретные} (=прерывные) состояния материи на различных её структурных уровнях (элементарные частицы, ядра, атомы, молекулы и т.д.).

Эволюционные (\emph{=постепенные}) и революционные (\emph{=скачкообразные}) формы в их единстве составляют закон, закономерность и общественного развития.

В различных сферах действительности в зависимости от специфических условий количественные изменения переходят в качественные \emph{по-разному}.

Отдельные науки исследуют \emph{конкретные формы} этого перехода, скачков из одних состояний в другие.

Философия, диалектико-материалистическая философия \emph{помогают} \emph{разобраться} во всём этом разнообразии форм и способов перехода, \emph{выделить} некоторые наиболее типичные из них, \emph{не претендуя}, однако, на то, что они дают исчерпывающую картину, ибо жизнь всегда богаче самой сложной теории.

Типичными и наиболее общими \emph{формами скачков}, качественных переходов можно считать:

\begin{enumerate}
\item форму \emph{сравнительно быстрого} и резкого превращения одного качества в другое, когда объект как целостная система со свойственной ей структурой сразу, \emph{как бы одним приёмом} или серией приёмов претерпевает коренное качественное изменение, и
\item форму \emph{постепенного} качественного перехода, когда объект изменяется \emph{не сразу и не целиком}, а отдельными своими сторонами, элементами, путём постепенного накопления качественных изменений и только в итоге таких изменений переходит из одного состояния в другое.
\end{enumerate}

\emph{От чего} же зависит и \emph{чем определяется} форма скачка, \emph{почему} он протекает то в одной, то в другой форме?

Ответ на это вопрос следует искать прежде всего в особенностях \emph{самих объектов развития}.

Так, природа даёт нам множество примеров, когда скачки и переходы из одного качественного состояния в другое совершается \emph{в форме быстрых} изменений.

\emph{Таковы превращения} элементарных частиц, химических элементов, химических соединений, освобождение атомной энергии в виде атомных взрывов и т.д.

В то же время в природе \emph{существуют такие объекты}, качественные превращения которых в другие, более сложные, связаны с очень \emph{длительными процессами} и могут происходить, как правило, лишь постепенно.

Таковы, например, качественные превращения одних видов животных в другие.

Обычно два явно качественно различных состояния, полюса в такого рода превышениях связаны между собой массой промежуточных форм.

Как бы постепенно, однако, ни протекал процесс качественного превращения, переход в новое состояние \emph{есть скачок}.

Этим постепенные качественные изменения \emph{отличаются} от постепенных количественных изменений.

Количественные изменения, меняя те или иные отдельные свойства вещей, \emph{не затрагивают} до определённого момента их качества.

Постепенность качественных превращений было бы \emph{неправильно понимать} так, будто качественные изменения, возникнув, затем просто количественно накапливаются, пока не будет вытеснено прежнее качество в целом.

В действительности процесс этот \emph{сложнее} и многограннее.

Это \emph{не просто} арифметическое суммирование элементов нового качества, а \emph{путь постоянного совершенствования}, постоянных, иногда незаметных, качественных изменений, путь, предполагающий глубокие структурные изменения в прежнем качестве, ряд промежуточных ступеней и этапов восхождения к конечному результату, т.е. к полному завершению скачка.

\emph{Формы скачка находятся в зависимости не только от природы объекта, но и от условий, в которых этот объект находится.}

Так, в условиях естественной радиактивности распад некоторых веществ, например, урана, происходит \emph{чрезвычайно медленно}: период полураспада равен миллиардам лет. Но тот же процесс распада \emph{при взрыве атомной} бомбы по причине цепной реакции происходит мгновенно.

Исторический \emph{опыт показывает}, что и в общественном развитии имеют место рассмотренные выше формы качественных превращений, скачков.

Вcё сказанное позволяет сделать \emph{общий вывод} о сущности и значении закона перехода количественных изменений в качественные и обратно.

\emph{Этот закон есть такая взаимосвязь и взаимодействие количественной и качественной сторон предмета, в \textsc{силу} которых мелкие, вначале незаметные количественные изменения, постепенно накапливаются и рано или поздно нарушают меру предмета, и вызывают коренные качественные изменения, протекающие в виде скачков, и осуществляются в зависимости от природы объект\textsc{ов и} условий их развития в разнообразных формах.}

Знание закона имеет \emph{большое значение} для понимания развития.

Этот закон ориентирует на то, чтобы рассматривать и \emph{изучать явления как единство} качественной и количественной сторон, видеть сложные взаимосвязи и взаимодействия этих сторон, изменения отношений между ними.

\subsection{Закон единства и борьбы противоположностей}

Вследствие происходящих в предмете изменений он \emph{никогда не бывает} равным самому себе и выступает как внуртенне противоречивый.

Противоречие между качеством и количеством --- только одно из проявлений общего закона, согласно которому всем вещам и процессам присуща \emph{внутренняя противоречивость}, и именно это составляет источник и двигательную силу их развития.

Многие прежние и современные мыслители \emph{отрицают} диалектически противоречивую сущность явлений. Они при этом нередко относят противоречия лишь к \emph{мысли}, объективные же \emph{вещи} у них оказываются свободными от всяких противоречий.

Противоречия мысли, мышления, или, как их часто называют, «\emph{логические противоречия}», бесспорно имеют место, они --- следствие логической непоследовательности, логических ошибок. Такие противоречия недопустимы. Появление таких противоречий в научных теориях \emph{свидетельствуют} об их неистинности, недоработанности.

Вместе с тем нередко за подобными противоречиями мысли скрываются объективные \emph{противоречия самих явлений}, которые ещё недостаточно осознаны.

Именно \emph{против признания} объективных противоречий, противоречий в самих вещах нередко выступают критики диалектики.

В мире \emph{нет вещей}, явлений, которые были бы абсолютно тождественны не только другим вещам, но и самим себе.

Когда мы говорим о сходстве, тождественности каких-либо объектов, сравнивая их между собой, то сама их одинаковость предполагает, что они \emph{в чём-то различны}, неодинаковы, иначе теряет смысл само их сравнение.

Даже простое внешнее сопоставление двух вещей (или двух разных состояний одной и той же вещи) вскрывает \emph{единство тождества и различия}: каждая вещь одновременно и тождественна другой, и отличная от другой.

Тождество не есть абстрактное, одностороннее, а \emph{конкретное}, иначе говоря, содержащее в себе момент различия, тождество.

Различие, содержащееся в предмете, выступает не только как различие по отношению к другому предмету, но и как \emph{различие по отношению к самому себе}, т.е. сам данный предмет независимо от того, сравниваем мы его или не сравниваем с другим предметом, \emph{содержит в себе} различие.

Например, живое существо есть \emph{единство тождества и различия} не только по той причине, что оно и подобно и не подобно другим живым существам, но и потому, что, осуществляя процесс своей жизни, \emph{оно себя отрицае}т, или попросту говоря, идёт навстречу своему иному качеству, наконец, к своему концу, смерти.

Когда диалектический подход утверждает, что \emph{предмет одновременно существует и не существует}, содержит в себе своё собственное небытие, то это нужно понимать только в одном смысле: предмет есть единство устойчивости и изменчивости, положительного, наличного и отрицательного, отсутствующего, отмирающего и нарождающегося и т.д.

А это как раз и означает, что каждый предмет, \emph{каждое явление есть единство противоположностей}.

Смысл этого важного понятия заключается прежде всего в том, что всем объектам свойственны \emph{внутренние различные} и даже противоположные стороны, тенденции.

Внутренние противоположности --- \emph{неотъемлемое свойство} структуры всякого объекта и процесса.

Каждому объекту \emph{свойственны свои} специфические противоречия, которые исследуются конкретным анализом.

Но одним признанием внутренней противоречивости явлений \emph{не исчерпывается} понятие единства противоположностей. Очень \emph{важно учесть} характер связи и взаимодействия между противоположностями, их структуру.

Связь эта такова, что каждая из сторон единого целого \emph{существует лишь} \emph{постольку, поскольку} существует другая, противоположная ей сторона.

Раздвоеннность, \emph{множественность} объекта не означает, внешнего отношения между его противоположностями.

Взаимополагание, взаимообусловленность, взаимопроникновение противоположных сторон, свойств, тенденций развития целого --- \emph{существеннейшая черта} всякого единства противоположностей.

Но взаимная обусловленность противоположностей \emph{есть лишь одна из} особенностей диалектического противоречия. Другой неотъемлемой его стороной является их \emph{взаимное отрицание}.

Именно потому, что стороны единого целого суть противоположности, они находятся в состоянии не только взаимосвязи, \emph{но и взаимоисключения}, взаимоотталкивания.

Этот момент находит своё выражение в понятии \emph{борьбы} противоположностей.

Понятие \emph{борьбы противоположностей} в обобщенном виде фиксирует самые различные и многообразные формы взаимного отрицания и исключения противоположностей.

В ряде случаев, особенно в общественной жизни, отчасти в органической природе, это взаимоисключение противоположностей имеет характер, точно выражаемый термином «\emph{борьба}».

\emph{Применительно к неживой} природе «\emph{борьба}» противоположностей имеет преимущественно характер \emph{действия и противодействия}, притяжения и отталкивания и т.п.

Но каковы бы ни были конкретные формы этой борьбы, главное состоит в том, что диалектическое противоречие включает в себя \emph{момент взаимоотрицани}я противоположностей, притом как весьма существенный момент, ибо \emph{борьба противоположностей есть двигательная сила, источник развития}.

Сказанное позволяет сделать ещё один вывод --- о том, что \emph{единство} противоположностей \emph{условно}, всегда преходяще, а \emph{борьба} их --- \emph{абсолютна}, безусловна.

Борьба противоположностей имеет своим закономерным следствием \emph{исчезновение} существующего объекта как определённого единства противоположностей \emph{и возникновение} рано или поздно другого, нового объекта в новым, с характерным уже для него единством противоположностей.

Закон единства и борьбы противоположностей объясняет одну из самых важных особенностей развития: движение, развитие осуществляется как \emph{самодвижение, саморазвитие}.

Схватывание именно этого момента действительного развития вещей и делает столь \emph{значительной} гегелевскую идеалистическую систему диалектической философии.

Понятие самодвижения, саморазвития имеет прежде всего глубокий \emph{материалистический смысл}. Оно означает, что мир развивается не вследствие каких-то внешних по отношению к нему причин (например, «божественного» первотолчка), а \emph{в силу собственных законов} движения самой материи.

При таком понимании развития \emph{не остаётся места} для какой-либо мистической «творческой силы».

Устанавливая, что всем вещам и процессам свойственны внутренние противоречия, составляющие \emph{двигательную силу саморазвития} природы и общества, материалистическая диалектика делает попытку объяснить, как совершается этот процесс.

Сами противоречия \emph{не есть нечто неподвижное}, неизменное.

Раз возникнув, те или иные конкретные противоречия \emph{развиваются}, проходят определённые стадии, ступени.

То или иное явление не исчезает, не уступает место другим до тех пор, \emph{пока его противоречия не раскроются}, не развернутся в полной мере, так как только в процессе такого развития создаются предпосылки для скачка в новое качественное состояние.

В этом процессе можно выделить \emph{два} основных этапа:

--- I) \emph{этап развития}, развёртывания противоречий, свойственных предмету, и

--- 2) \emph{этап разрешения} этих противоречий.

В начале своего развития противоречие имеет \emph{характер различия}, т.е. ещё не развернувшегося противоречия.

Затем \emph{различие углубляется}, превращается в противоположность, которую следует понимать как уже раскрывшееся противоречие, противоположные стороны которого всё меньше и меньше могут оставаться в рамках прежнего единства. На этом этапе развития противоречие становится \emph{таким соотношением противоположностей}, которое само побуждает противоречие к разрешению.

Поэтому \emph{закономерным завершением} процесса развития и борьбы противоположностей является \emph{второй этап} --- этап разрешения противоречия.

Если весь предыдущий процесс происходит \emph{в рамках единства}, взаимосвязи противоположностей, то этап разрешения противоречия означает \emph{снятие данного единства}, его исчезновение, что совпадает с коренным качественным изменением предмета.

При этом диалектико-материалистическая философия не претендует на то, чтобы дать какой-то «\emph{реестр}» всевозможный противоречий. Её задача, скорее, указать «\emph{стратегию}» подхода к явлениям.

Каковы же \emph{конкретные противоречия конкретных предметов}, как они разрешаются, --- это вопросы, которые выясняют учёные в соответствующих областях познания.

Здесь важно остановиться на \emph{различении таких} противоречий общества, как антагонистические и неантагонистические.

\emph{Антагонистический вид противоречий} --- это противоречия \emph{враждебных сил}, близкие отношениям хищника и жертвы в животном мире, сил, имеющих в корне противоположные цели интересы.

Такие противоречия завершаются \emph{резким конфликтом} между противоположными сторонами, превращением последних в полярные крайности.

Отсюда и способы разрешения такого рода противоречий --- социальные \emph{революции}, подавление, репрессии, уничтожение противника.

\emph{Неантагонистический вид противоречий} --- это противоречия тех социальных сил, жизненные условия которых \emph{определяют общность} того или иного рода их коренных целей и интересов.

Важнейшая особенность таких противоречий заключается в том, что в них уже \emph{не заложена} объективная необходимость превращения противоположных сторон и тенденций в полярные, враждебные крайности.

Единство коренных интересов всех людей \emph{делает возможным} постепенное преодоление подобных противоречий, какой бы обостренной характер в тот или иной момент они ни имели.

Кроме того, между антагонистическими и неантагонистическими противоречиями нет непроходимой границы.

Те же неантагонистические противоречия, \emph{приобретая запущенный} характер, становятся практически неотличимыми от антагонистических.

Далее, каждая отдельная вещь, а тем более такое сложное образование, как общество, представляет собой \emph{целую систему} противоречий, находящихся в определённой структурной связи.

Таковы \emph{основные и неосновные}, \emph{главные и неглавные}, \emph{внутренние и внешние противоречия} и др.

Под \emph{основными} нужно понимать противоречия, которые характеризуют объект и определяют его развитие от момента его возникновения \emph{до самого конца} и которые обусловливают все остальные, т. е. \emph{неосновные}, противоречия.

Каждый этап развития предмета имеет в качестве \emph{главного} какое-то противоречие, определяющее \emph{сущность этого} этапа его развития.

\emph{Неглавные} противоречия --- это те, которые имеют место на этом же этапе, но не определяют его сущности.

А каково различие между \emph{внутренними и внешними} противоречиями?

В философии существовали и существуют подходы, которые \emph{сводят противоречи}я к одному лишь соотношению внешних друг другу вещей, сил, к их столкновению.

Это механистические трактовки, например, «\emph{теория равновесия}», которые рассматривают вещи как находящиеся в состоянии покоя, свободные от внутренних противоречий, и, следовательно, отрицают диалектический подход.

Всякий объект, будучи относительно самостоятельной системой, имеет свои внутренние противоречия, которые и являются \emph{основным источником} его развития.

Различия между несколькими такими объектами выступают как \emph{внешние} противоречия.

Внешние противоречия тесно \emph{связаны с внутренними} противоречиями, взаимодействуют с ними.

Если рассматривать объект как \emph{элемент более широкой системы}, то противоречия между ним и другими объектами -- элементами такой системы становятся уже внутренними противоречиями данной, более широкой системы.

Закон единства и борьбы противоположностей имеет \emph{большое значение} для познания.

И сегодня можно встретить подходы, которые трактуют человеческие понятия как неподвижные отображения, механические «\emph{снимки}» изменяющихся вещей.

Отсюда делается \emph{вывод}, что между объектами и познанием неизбежен \emph{вечный антагонизм} и что только некое непостижимое непосредственное чувство (\emph{мистическая интуиция}) может выразить движение объектов.

В диалектико-материалистической философии показано, что истинное, \emph{конкретное мышление} оперирует противоречиями, схватывающими противоположные стороны явлений в их единстве.

Это мышление способно \emph{видеть не одну сторону} противоречия и фиксировать её в жёстком, неподвижном понятии, \emph{а все стороны} противоречия, и не только их рядоположенность, но и их связь, переходы друг в друга.

Понятия должны \emph{быть столь же диалектичными}, гибкими, пластичными, связанными и преходящими друг в друга, \emph{как те объекты}, которые они отображают.

Подобным качеством должна обладать \emph{в своей основе} сама способность человека образовывать понятия и оперировать ими.

Человеческие понятия \emph{должны воплощать} в идеальной форме реальные противоречия, связь и взаимопроникновение противоположностей вещей, их переходы друг в друга.

\emph{Подводя итог} сказанному, мы можем определить теперь сущность закона единства и борьбы противоположностей.

\emph{Это закон, в силу которого всем вещам, явлениям, процессам свойственны внутренние противоречия, противоположные стороны и тенденции, находящиеся в состоянии взаимосвязи и взаимоотрицания; борьба противоположностей даёт внутренний импульс к развитию, ведёт к нарастанию противоречий, разрешающихся на известном этапе исчезновением прежнего и возникновением нового.}

Знание этого закона помогает \emph{критически осмыслить} совершающиеся процессы, видеть то, что устаревает, и то, что идёт ему на смену, преодолевать то, что мешает прогрессу.

\subsection{Закон отрицания отрицания}

Рассмотрим ещё один важный вопрос учения о развитии: существуют ли какие-либо \emph{тенденции}, определяющие направление бесконечного процесса развития, и если существуют, то в чём они заключаются?

В прежние времена, эпохи имели место \emph{теории круговорота}, признававшие восходящее развитие в обществе, но полагавшие, что достигнув какой-то высшей точки, общество отбрасывается назад, к исходному пункту («\emph{коллективный Сизиф}»), и развитие начинается сначала.

Такой была теория итальянского философа \emph{Дж. Вико}.

Идеологи начала Нового времени отстаивали точку зрения \emph{непрестанного развития} общества, \emph{хотя и считали вершиной} общественного прогресса общество по типу современного им, только усовершенствованного.

Позже появились всевозможные \emph{пессимистические теории}, принимавшие \emph{неизбежную гибель} современного их авторам общества, а вместе с ним и человеческого общества вообще (например, взгляды немецкого философа \emph{О. Шпенглера}).

Уже рассматривая переход количественных изменений в качественные и борьбу противоположностей, мы видели, что существенную роль в процессе развития играет \emph{отрицание}.

Качественное превращение возможно лишь как \emph{отрицание прежнего}, старого состояния.

Противоречивость вещи означает, что она содержит в себе основание своего преодоления, или своё \emph{собственного отрицание}.

\emph{Отрицание} --- \emph{неизбежный и закономерный момент во всяком развитии}. Без такого моменты ничего нового \emph{не могло бы} возникнуть.

Но \emph{что такое} отрицание? --- В обыденном сознании отрицание, его понятие ассоциируется со словом «\emph{нет}»; \emph{отрицать} --- значит \emph{сказать} «\emph{нет}», отвергнуть что-либо и т. д.

Несомненно, без отвержения чего-либо \emph{не может быть} никакого отрицания.

\emph{Диалектика} рассматривает отрицание \emph{как момент} развития. И поэтому данное понятие имеет в ней несравненно \emph{более глубокий} смысл, более всестороннее содержание, чем в обыденном словоупотреблении.

\emph{Смысл} диалектического развития --- в таком способе отрицания, который обусловливает \emph{дальнейшее развитие}.

Диалектическое отрицание характеризуется двумя существенными чертами:

\begin{enumerate}
\item оно есть \emph{условие и момент развития} и
\item оно есть \emph{момент связи нового со старым}.
\end{enumerate}

\emph{Первая черта} означает, что только то отрицание, которое служит предпосылкой для возникновения каких-то новых форм, есть «\emph{положительное отрицание}».

\emph{Вторая черта} означает, что новое в качестве отрицания старого, предшествующего, не оставляет за собой пустыню, не просто уничтожает, разрушает его, а как бы «\emph{снимает}» его, \emph{сохраняя} немало его элементов.

Термин «\emph{снятие}» хорошо выражает смысл и содержание диалектического отрицания: предшествующее \emph{одновременно} и отрицается, и сохраняется.

Оно сохраняется в \emph{двояком} смысле.

\emph{Во-первых}, без предшествующего развития \emph{не было бы основы} для новых форм.

\emph{Во-вторых}, всё, что сохраняется от предшествующей ступени развития, \emph{переходит} на следующую ступень в существенно изменённом, преобразованном виде.

Так, некоторые формы психической деятельности, развившиеся у животных, \emph{перешли и к человеку}, перешли в «\emph{снятом}» виде: в человеке, у человека они \emph{преобразованы} на основе тех особенностей, которые свойственны лишь ему (трудовая деятельность, способность к мышлению и др.).

Однако одним актом отрицания развитие какого-нибудь объекта далеко \emph{не исчерпывается}.

\emph{Первое отрицание} есть полная противоположность предшествующего, непосредственно отрицаемого.

Отношения исходной формы и первого отрицания --- это \emph{отношения противоположностей}, двух противоположных форм.

\emph{Что происходит дальше}, когда путём отрицания возникла новая, противоположная исходной, форма?

Это лучше рассмотреть на примере развития какого-либо объекта от начала до конца, например, \emph{на познании}.

Исходным пунктом всего развития человеческого познания была такая форма, при котором \emph{существовало единство} познающего и средств его познавательной деятельности.

Такую форму познания можно назвать в известном смысле «\emph{детской}» (в смысле детства человеческого рода), поскольку она была свойственна первобытному человеку.

Однако накопление познавательных элементов на этом уровне \emph{достигает со временем} такого состояния, когда первоначальная примитивная форма соединения познающего субъекта и средства познания \emph{стало тормозом} дальнейшего развития познания.

Возникает \emph{специализация} в использовании даже существовавших тогда примитивных средства познания, которая становится \emph{первым отрицанием исходной} формы познания.

Но, \emph{достигая} своего полного развития в древнем обществе --- Древней Индии, Китае, Египте, Греции, эта форма специализации познания, бывшая в своё время отрицанием единства познавательной деятельности и её средства, \emph{сама} закономерно подготавливает свое собственное отрицание.

Это уже \emph{второе} отрицание, \emph{отрицание первого} отрицания, почему оно и определяется как \emph{отрицание отрицания}.

Из рассмотренного выше примера \emph{видно}, что необходимость второго отрицания, или новой ступени отрицания, \emph{обусловлена следующим}: исходная форма и то, что её отрицает, представляют собой противоположности, они содержат в себе \emph{абстрактную односторонность}, которая должна быть преодолена, чтобы стало возможным дальнейшее развитии.

\emph{Гегель} был прав, когда он второе отрицание (отрицание отрицания) определял как тот \emph{синтез}, который преодолевает первые «абстрактные, неистинные моменты», в смысле их односторонности, \emph{незавершенности} (См. \emph{Гегель}. Соч. Т. VI. М, 1939, с. 312).

С этим связана \emph{ещё одна} важная черта отрицания отрицания.

В заключительном звене всего цикла развития, на ступени второго отрицания \emph{неизбежно восстанавливаются} некоторые черты исходной формы, с которой начинается развитие (своего рода механизм \emph{общедиалектической} «наследственности признаков»).

Этот диалектический характер развития \emph{также хорошо} видно на примере познания.

Так, при исследовании природы света \emph{вначале} была выдвинута идея о том, что он есть поток световых корпускул, особых частиц света. \emph{Затем} возникла противоположная ей волновая теория. Физика ХХ в. \emph{столкнулась} с тем фактом, что ни один из этих взглядов сам по себе не объяснял реальности. Это противоречие двух односторонне противоположных взглядов \emph{разрешилась} путём их синтеза на более высоком уровне, в новой теории, которая рассматривает свет как единство корпускулярных (дискретность, фотоны) и волновых свойств.

В силу действия закона отрицания отрицания развитие имеет \emph{форму не линии}, а своеобразного \emph{круга}, в котором конечная точка \emph{как бы} совпадает с начальной.

Но так как это совпадение происходит на некотором \emph{ином уровне}, то развитие имеет вид, который скорее можно охарактеризовать как \emph{спираль}, каждый круг, или, \emph{точнее, виток} которой обозначает иное качественное состояние.

В этом смысле в диалектической теории употребляется термин «\emph{спиралевидность}».

Часто процесс отрицания отрицания изображают, вслед за \emph{Гегелем}, в терминах: «\emph{тезис}» (=исходный пункт развития), «\emph{антитезис}» (=первое отрицание) и «\emph{синтез}» (=второе отрицание), усматривая сущность развития в этой \emph{троичности}.

В результате закон отрицания отрицания \emph{нередко сводится} к чисто формальному и внешнему приёму, которым всё разнообразие и сложность объективного развития \emph{произвольно втискивается} в эту жёсткую схему.

\emph{Уже сам} Гегель резко \emph{протестовал} против такого понимания диалектики, утверждая, что троичность есть \emph{лишь поверхностная}, внешняя сторона способа познания.

Закон отрицания отрицания, как всякий закон, принцип диалектики, \emph{не навязывает} никаких схем, он \emph{лишь ориентирует} в правильном исследовании действительности.

Анализ закона отрицания отрицания позволяет теперь ответить на поставленный выше вопрос о том, \emph{существует ли} какая-нибудь закономерная тенденция в бесконечной смене одних явлений другими, определяющая направление развития.

\emph{Развитие есть цепь диалектических отрицаний}, каждое их которых не только \emph{отвергает} предшествующие звенья, но \emph{и сохраняет} положительное, содержащееся в них, всё более и более \emph{концентрируя} в высших звеньях богатство развития в целом.

Бесконечность развития поэтому \emph{нельзя понимать} как уходящий в непостижимую даль ряд, в котором к существующим объектам просто прибавляются другие объекты, и так без конца.

Развитие заключается \emph{не в арифметическом добавлении} к существующей единице другой единицы, а в \emph{возникновении новых}, высших форм, содержащих в себе предпосылки для дальнейшего развития.

Отсюда общая закономерная тенденция развития от простого к сложному, от низшего к высшему, \emph{тенденция поступательного}, \emph{восходящего движения}.

Характерная черта процесса отрицания отрицания --- это его \emph{необратимость}, т.е. такое развитие, которое в качестве общей тенденции \emph{не может быть} движением вспять, от высших форм к низшим, от сложных к простым.

Это объясняется тем, что каждая новая ступень, синтезируя в себе всё богатство предыдущих, составляет основу \emph{для ещё более высоких} форм развития.

По отношению к миру в целом, к бесконечной Вселенной \emph{неправильно говорить}, конечно, об одной линии развития --- о его поступательности.

Однако по отношению к отдельным системам или их элементам тенденция к восходящему развитию \emph{вполне} реализует себя.

В то же время поступательность развития \emph{нельзя} понимать упрощённо. Как и всякий диалектический процесс, она реализуется в противоречиях, \emph{через борьбу} противоположностей.

Восходящей ветви в развитии одних форм \emph{соответствует} нисходящая ветвь в развитии других форм.

Каждая конечная форма, развиваясь по восходящей линии, создаёт предпосылки \emph{для собственного преодоления}, отрицания.

Сама поступательность движения реализуется \emph{в борьбе} различных тенденций, пробивает себе дорогу лишь в итоге, в массе и через массу перекрещивающихся линий развития.

Отдельные линии общего развития \emph{могут быть направлены} не вперёд, а назад, выражать моменты регресса, движение вспять.

Короче говоря, поступательность \emph{тоже нельзя понимать} метафизически, как плавный процесс, без отклонений и зигзагов.

Особенно это \emph{важно учитывать} в общественном развитии, где действуют различные силы, преследующие интересы зачастую несовпадающие.

Не нужно забывать, что \emph{закон отрицания отрицания есть закон, действием которого обусловливается связь, преемственность между отрицаемым и отрицающим, вследствие чего диалектическое отрицание выступает как условие развития, удерживающего всё положительное содержание прошлого, повторяющего на высшей основе некоторые черты исходных ступеней и имеющего в целом поступательный, восходящий характер.}

\section{Категории диалектико-материалистической философии}

\subsection{Общая характеристика категорий диалектики}

Каждая наука вырабатывает свои понятия, \emph{чтобы точнее} и глубже отразить содержание изучаемых объектов.

Совокупными усилиями многих поколений учёных вырабатывались понятия, общие некоторым группам наук, а также \emph{категории --- наиболее общие, фундаментальные понятия, пронизывающие собой все виды теоретического мышления}.

Философия вырабатывает свои категории, с помощью которых она изучает и фиксирует \emph{наиболее общие} свойства, отношения и связи вещей, закономерности развития, действующие и в природе, и в обществе, и в человеческом мышлении.

Как универсальные, всеохватывающие формы научного мышления, философские категории \emph{возникли и развиваются} на основе общественной практики.

По своему содержанию философские категории отражают \emph{вне нас} существующую действительность, свойства, отношения и связи объективного мира.

В ходе истории познания \emph{изменялись} и место, и роль отдельных категорий. Особенно \emph{подвижным} является содержание категорий.

Достаточно сравнить, например, \emph{как понимали} \emph{материю} в древности, и как эта категория осмысливается в системе современной научной картины мира.

Так как категории отражают \emph{всеобщие} свойства, отношения и связи материального мира, то от сюда вытекает их огромная \emph{методологическая} ценность, необходимость применения к исследованию \emph{любых} конкретных явлений.

Методологическую роль выполняют и важнейшие понятия каждой науки. Категории диалектики отличаются от общих понятий частных наук тем, что если последние применимы лишь в определённой сфере мышления, то \emph{философские категории как методологические принципы} пронизывают собой всю ткань научного мышления, все области знания.

Никакие частные области познания \emph{не могут обойтись} без предельно общих понятий, т.е. философских категорий. Лишь с помощью этих понятий \emph{возможно} последовательное теоретическое воспроизведение действительности и её мысленное творческое преобразование. А без этого \emph{не может быть} и её чувственно-предметного изменения, созидания вещей и переустройства общественных отношений.

Категории диалектики \emph{выражают и} закономерности мышления. Благодаря категориям единичные вещи воспринимаются и осмысливаются как \emph{частные проявления} общего.

Усвоение категорий в ходе индивидуального развития человека --- \emph{необходимое условие} формирования у него способности теоретического мышления.

Отражающие мир категории определённым образом \emph{связаны} между собой.

Каждая из категорий отражает какую-либо сторону объективного мира, а вместе они охватывают, \emph{всегда приблизительно}, универсальную закономерность вечно движущейся и развивающейся материи.

Категории связаны между собой так, что каждая из них может быть осмыслена \emph{лишь как элемент} определённой \emph{системы категорий}.

Категории диалектики находятся \emph{в тесной связи} с её основными законами.

Сами эти законы формулируются \emph{лишь через} определённые категории, и иначе они никак не могут быть выражены.

Так, закон единства и борьбы противоположностей \emph{выражается через} категории противоположности, противоречия, конфликта и др.

\emph{В свою очередь} законы диалектики определяют отношения между категориями как выражающими всеобщие стороны и отношения вещей.

Например, отношения между содержанием и формой, сущностью и явлением, необходимостью и случайностью представляют собой \emph{специфическое проявление} закона единства и борьбы противоположностей.

В предыдущих главах уже рассмотрен \emph{ряд} категорий. В данной главе, а также практически во всех последующих, будут рассмотрены и \emph{многие другие} категории.

\subsection{Единичное, особенное и общее}

\emph{Первое}, что обращает на себя внимание, когда мы воспринимаем окружающий нас мир, --- это его \emph{изменчивое} количественное и качественное \emph{многообразие}.

Мир един, но он существует \emph{в виде совокупности} различных вещей, явлений, событий, обладающих своими индивидуальными, неповторимыми признаками.

Существование отдельных, отграниченных друг от друга в пространстве и во времени предметов и явлений, обладающих индивидуальной качественной и количественной определённостью, характеризуется категорией \emph{единичного}.

Эта категория выражает \emph{то, что отличает} один объект от другого, что свойственно лишь данному объекту.

Любой предмет и процесс являются \emph{лишь моментами} некоторой целостной системы. Ни одна вещь, ни одно явление \emph{не существует} само по себе. Они не могут ни возникнуть, ни сохраниться, ни измениться \emph{вне связи} со множеством других вещей, явлений.

Общность свойств и отношений вещей выражается в категории \emph{общего}.

Эта категория отражает \emph{сходство} свойств, сторон объектов, связь между элементами, частями данной системы, а также между системами.

Общее может выступать в виде сходства \emph{свойств}, \emph{отношений} вещей, составляющих определённый класс, множество. Оно может быть выражено, например, \emph{в таких понятиях}, как «кристалл», «животное», «человек», «юридическая норма» и т.п.

Общее не существует \emph{до и вне} единичного, точно так же единичное не существует вне общего.

\emph{Всякий объект есть единство общего и единичного}.

Как бы связующим звеном между единичным и общим выступает \emph{особенное}.

\emph{По отношению} к единичному (например, тому самому Иванову) особенное (скажем, русский) является общим, \emph{а по отношению} к ещё большей общности (национальный) оно может быть единичным и т.д.

Общее \emph{не привносится} в единичное из сферы чистой мысли. И различие, и единство (общее) присущи \emph{самим} предметам и событиям реального мира.

Общее и единичное \emph{объективны}, не зависят от сознания человека как две \emph{неразделимые} стороны всякого бытия.

Любая вещь \emph{и отлична} от всех других, и вместе с тем в каком-то отношении \emph{сходна} с ними.

Общность и различие --- это отношение объекта к самому себе и к другим, характеризующее устойчивость и изменчивость, равенство и неравенство, сходство и несходство, одинаковость и неодинаковость, повторяемость и неповторяемость, непрерывность и прерывность его свойств, отношений, связей и тенденций развития.

Общее и его отношение к единичному \emph{по-разному} истолковываются в различных философских системах.

\emph{Метафизически} мыслящие философы обычно \emph{отрывают} единичное от общего и \emph{противопоставляют} их друг к другу.

В эпоху средних веков так называемые \emph{номиналисты} утверждали, что общее не имеет никакого реального существования, что общее \emph{суть лишь имена}, слова, реально же существуют \emph{только отдельные} вещи с их свойствами, отношениями.

Другие, \emph{реалисты}, напротив, полагали, что общие понятия существуют реально, как некие духовные сущности вещей, что они \emph{предшествуют} отдельным предметам и могут существовать независимо от них.

\emph{Спор} номиналистов и реалистов \emph{продолжался} и в последующие времена.

Некоторые мыслители утверждали и утверждают, что область социального бытия \emph{исключительно} «\emph{уникальна}» и что все отношения в ней неповторимы в своей индивидуальности, конкретности, абсолютно единичны. На этом основании \emph{отрицается} закономерность исторического процесса.

\emph{Состоятельна ли} такая позиция? --- \emph{Нет}, отдельные события во всей их конкретности \emph{действительно} никогда не повторяются. Каждая война, например, во всей своей индивидуальности \emph{не похожа} на другие.

Но в этой неповторимой индивидуальности конкретных событий \emph{есть всегда что-то общее}: их существенные свойства, типы внутренних и внешних связей.

Тот факт, что вторая мировая война не была похожа на греко-персидские войны, \emph{не является} препятствием, например, для социологического изучения различных типов войн.

Общность предметов, явлений ни в коей мере \emph{не нивелирует} индивидуальности событий. Она \emph{лишь свидетельствует} о том, что эта неповторимая индивидуальность --- конкретная форма обнаружения существенно общего.

Конкретной формой своего существования единичная вещь \emph{обязана} той системе закономерно сложившихся связей, \emph{внутри которых} она возникла и существует в своей качественной определённости.

Над единичным «\emph{властвует}» общее, но эта «власть» общего \emph{не является} чем-то сверхъестественным. Она кроется не в каких-то силах, стоящих над единичными вещами, а \emph{в них самих}, в системе взаимодействующих единичных вещей, где каждая вливает в «чашу» общего и берёт из него живительную влагу.

Существуя и развиваясь по законам общего, \emph{единичное} вместе с тем \emph{служит предпосылкой общего}.

Так обстоит дело, например, \emph{в развитии} живой природы.

Организм путём индивидуальной изменчивости \emph{приобретает} какой-либо новый полезный признак. Этот единично существующий признак может быть \emph{передан} по наследству и со временем стать признаком уже ряда особей, т.е. свойством разновидности в рамках данного вида.

В дальнейшем данная разновидность может \emph{превратиться} в новый вид, т.е. \emph{признак из единичного может стать общим}, видовым.

В развитии организмов происходят \emph{и прямо противоположные} процессы, когда тот или иной видовой признак начинает отмирать, атрофироваться. Такой признак становится свойством лишь немногих организмов, а потом может существовать только \emph{как исключение}.

Так \emph{общее превращается в единичное}.

Действие общего, общих признаков как закономерности \emph{выражается в} единичном и через единичное.

Однако подобная закономерность \emph{неприменима} к миру в целом, нельзя сказать, что общее возникает из единичного или наоборот. И то и другое существует \emph{в единстве}.

Правильный учёт диалектики единичного, особенного и общего имеет \emph{большое значение} для познания и практики.

Наука \emph{стремится} к обобщениям, она \emph{оперирует} общими понятиями, что даёт ей возможность устанавливать законы и тем самым обеспечивать практику средствами предвидения и проектирования.

Исследование в науке может идти \emph{двумя} путями:

\begin{enumerate}
\item \emph{путём восхождения} от единичного, как отправного пункта, к особенному, и от последнего к общему и всеобщему;
\item \emph{путём движения} от всеобщего и общего к особенному, и от последнего к единичному.
\end{enumerate}

\subsection{Причина и следствие}

Все науки \emph{стремятся вскрывать} причины возникновения явлений, причины их развития, преобразования, или гибели, разрушения.

Знание явлений, процессов \emph{есть, прежде всего}, знания причин.

Причинность («\emph{каузальность}», от латинского «\emph{causa}» -- причина) --- одна из форм всеобщей закономерной связи явлений.

Образуя понятие «\emph{причина}» и «\emph{следствие}», человек разделяет те или иные стороны единого объективного процесса.

\emph{Причина} и \emph{следствие} --- соотносительные понятия, потому что в реальном мире соотносительны сами причины и следствия.

Явление, которое \emph{вызывает} к жизни другое явление, выступает по отношению к нему как \emph{причина}.

\emph{Результат} действия причины есть \emph{следствие}.

\emph{Причинность} --- это такая внутренняя связь между явлениями, при которой \emph{всякий раз}, когда существует одно, за ним следует другое.

Например, \emph{нагревание воды} является причиной её превращения в пар, ибо всякий раз, когда происходит нагревание, возникает процесс парообразования.

Понятия о причине и следствии выработались \emph{в процессе} общественной практики и познания мира. В них мышление \emph{отразило} важнейшую закономерность объективного мира, знание которой необходимо для практической деятельности людей.

Познавая причины возникновения явлений и процессов, \emph{человек получает возможность} воздействовать на них, искусственно воссоздавать их, вызывать к жизни или, наоборот, предотвращать их возникновение.

\emph{Незнание причин}, условий, вызывающих явления, \emph{оказывается в свою очередь причиной} бессилия человека, его беспомощности перед этими явлениями.

Наоборот, \emph{знание причин выступает причиной} формирования возможностей действовать со знанием дела.

Причина во времени \emph{предшествует} следствию и вызывает его. Но это \emph{не значит}, что всякое предшествующее явление находится в причинной связи с последующим.

Ночь предшествует утру, \emph{но она не является} причиной утра.

\emph{Нельзя смешивать} причинную связь с временной последовательностью явлений.

Суеверный человек \emph{склонен считать} причиной войны, например, появившуюся перед её началом комету, солнечное затмение или другое предшествующее ей природное или социальное явление.

Причину нужно отличать от \emph{повода} --- от события, которое \emph{непосредственно предшествует} другому событию, делает возможным его появление, но не порождает и не определяет его.

Связь между поводом и следствием имеется, \emph{но она внешняя}, несущественная.

Так, поводом для восстания матросов на небезызвестном броненосце «\emph{Потёмкин}» в июне 1905г. послужила \emph{выдача матросам} еды из тухлого мяса.

Причиной восстания стало \emph{обострение противоречий} между существовавшим тогда царским строем и жизнью народа, большинства его, по крайней мере.

Если \emph{не то, так другое} событие непременно бы вызвало восстание.

\emph{Причинная связь явлений носит объективный и универсальный, всеобщий характер}.

Все явления в мире, все изменения, процессы \emph{непременно} возникают в результате действий определённых причин.

В мире \emph{нет, и не может быть} беспричинных явлений.

Всякое явление \emph{с необходимостью} имеет свою причину.

\emph{Человек} с различной степенью точности \emph{познаёт} причинную связь явлений.

Причины многих явлений нам \emph{до сих пор ещё неизвестны} или малоизвестны, но они объективно существуют.

Так, медицина ещё не открыла полностью причину (причины) раковых заболеваний, но эта причина (причины) \emph{существует и будет}, в конце концов, обнаружена.

Вокруг понимания причинности \emph{идёт} противоборство между материалистической и идеалистической философией.

\emph{Материалисты признают} объективную, не зависящую от воли и сознания причинную связь явлений и более или менее верное отражение её в сознании человека.

\emph{Идеалисты} или \emph{отрицают} причинную обусловленность всех явлений действительности, или \emph{выводят} причинность не из самого объективного мира, а \emph{из сознания}, из разума, из действия вымышленных сверхъестественных сил.

То, что, например, знание или другие элементы сознания человека включаются в причинно-следственные связи в обществе, \emph{не меняют} общего смысла причинной обусловленности в мире, хотя и существенно \emph{дополняют} его.

Положение, что все явления в мире причинно обусловлены, выражает \emph{закон причинности}.

Философы, признающие этот закон, распространяющие его действие на все явления, называются \emph{детерминистами} (от латинского determinate --- определять).

Философы, отрицающие закон причинности, называются \emph{индетерминистами}.

Закон причинности \emph{требует} естественного объяснения всех явлений природы и общества и исключает возможность их объяснения с помощью сверхъестественных, потусторонних сил.

Последовательно проведённый \emph{материалистический детерминизм} не оставляет место для бога, различного рода чудес, мистики и т. п., в науке, по крайней мере.

В истории философии с отрицанием объективной причинности выступил английский философ \emph{Д. Юм}, который утверждал, что знание о причинной связи явлений мы получаем из опыта, но, с другой стороны, \emph{допускал} серьёзную неточность.

Дело в том, что Юм \emph{сводил опыт} к субъективным ощущениям и отрицал в нём объективное содержание.

В опыте мы наблюдаем, что одно явление \emph{следует} за другим.

Но, как полагал Юм, \emph{во-первых}, у нас нет оснований считать, что предшествующее может быть причиной последующего; \emph{во-вторых}, нет оснований, исходя из прошлого и настоящего опыта, делать заключение о будущем.

\emph{Вывод Юма} сводится к следующему: причинность --- \emph{это только} определённая последовательная причинная связь ощущений и идей, а предвидение на её основе есть ожидание этой связи.

Наш \emph{прошлый опыт} даёт нам основание ожидать, что и в будущем трение будет порождать тепло. Но у нас нет, и не может быть никакой уверенности в объективности и необходимости этого процесса.

Диалектико-материалистическая философия, опираясь на данные науки и весь совокупный опыт человечества, утверждает, что доказательством объективности причинности служит \emph{вся человеческая практика}.

Другой, немецкий, философ \emph{И. Кант} не был согласен с Юмом в том, что причинность является только привычкой, привычной связью ощущений.

Кант признавал существование причинной связи как необходимой по своему характеру, \emph{но не в объективном мире, а лишь} в нашем рассудке.

По мнению Канта, причинная связь не устанавливается в опыте, она существует как \emph{априорная}, \emph{доопытная} (внеопытная), \emph{врождённая категория рассудка}, на основе которой различные восприятия связываются в суждение.

В целом идеалистические взгляды Юма и Канта на причинность воспроизводятся в различных вариациях в других, более поздних направлениях --- \emph{неокантианстве}, \emph{позитивизме}.

Например, \emph{Б. Рассел}, крупнейший западный философ ХХ века считает понятие причины \emph{донаучным обобщением}, служащим лишь некоторым руководством к действию.

Различие между Юмом и Расселом в понимании причинности состоит лишь в том, что, согласно Расселу, закон причинности основывается не на \emph{привычке}, как у Юма, а на \emph{животной вере}, которая глубоко укоренилась в языке. (\emph{Б. Рассел}. \emph{Человеческое познание}. М., 1957, с. 489).

Вслед за Расселом, многие современные философы на Западе настойчиво подчеркивали мысль, что слово «причина» надо \emph{исключить} из философской терминологии.

Причинность, с этой точки зрения, \emph{будто бы изжила себя}, подобно монархии.

Закон причинности заменяется законом \emph{функциональной связи}: нельзя говорить, что явление \emph{А} порождает явление \emph{Б}, а надо указывать, что \emph{А} и \emph{Б находятся в зависимости} друг от друга (\emph{А} всегда сопровождается \emph{Б}, предшествует ему, а \emph{Б} следует за \emph{А}).

В форме функциональной связи можно представить самые различные зависимости, \emph{в том числе}, внешние, малосущественные и даже произвольные.

Отношение между причиной и следствием \emph{тоже можно представить} в форме функциональной зависимости: следствие есть функция причины.

При этом однако \emph{теряется главное} в причинности: причина, как реальное явление порождает и обусловливает следствие --- другое реальное явление. То есть \emph{причинность растворяется в функциональной зависимости}.

Некоторые западные философы подменяют причинную связь \emph{логической связью} основания и следствия.

Но причинную связь \emph{следует отличать} от связи основания и следствия. Основанием \emph{в формальной логике} называют мысль (мысли), из которой следует какая-то другая мысль.

Причинность --- это \emph{связь не мыслей} в умозаключении, как например: «Все люди смертны. Сократ --- человек. Сократ --- смертен».

Причинность есть \emph{связь реальных} явлений, при которой одно явление вызывает, порождает другое.

Логическая связь мыслей \emph{есть отражение} отношений вещей в действительности, в том числе и их причинной обусловленности.

Конечно, из различия между причиной и основанием отнюдь \emph{не следует}, что в сфере мышления действуют только чисто логические связи, что \emph{принцип причинности} здесь заменяется \emph{принципом достаточного основания}.

Любая мысль, \emph{и сам принцип} достаточного основания, причинно обусловлены.

Принцип причинности подвергается критике и со стороны \emph{некоторых физиков}, утверждающих, что современная физика якобы отвергает представление, согласно которому все явления имеют причину своего существования.

По мнению этих физиков, \emph{в микропроцессах нет} причинной обусловленности: ни одна микрочастица, например, электрон, не подчиняется закону причинности, а свободно выбирает среди разных возможностей путь своего движения. При этом обычно \emph{ссылаются} на соотношение неопределённостей.

Действительно, если в макропроцессах \emph{можно определить одновременно} и положение и скорость тела, то положение (координаты) и скорость (импульс) микрочастицы \emph{нельзя одновременно определить} с неограниченной точностью.

Эта открытая физиками закономерность в движении микрообъектов \emph{не укладывается} в то определение причинности, которое было характерно для науки XVII и XVIII столетий и вошло в историю под названием \emph{лапласовского} (по имени французского учёного \emph{Лапласа}) детерминизма.

\emph{Лапласовская}, или механистическая, \emph{форма детерминизма} возникла на базе изучения внешнего, механического движения макрообъектов, она \emph{предполагает возможность} одновременного точного знания координат и импульса.

Но при описании внутриатомных процессов \emph{мы сталкиваемся} с особыми свойствами частиц (они обладают одновременно и корпускулярными и волновыми свойствами), поэтому \emph{прежние понятия} координат и импульса, выработанные для макрообъектов, здесь неприменимы.

Из принципа соотношения неопределённостей в явлениях микромира \emph{не вытекает} отрицания причинности.

\emph{Закон причинности утверждает только одно: все явления причинно обусловлены}.

Как выступает причинность в отдельных конкретных случаях, можно ли одновременно с неограниченной точностью определить и координаты и скорость частицы --- \emph{это уже другой вопрос}, решая который нужно учитывать конкретные свойства объектов.

Современная физика \emph{даёт богатый} фактический материал, подтверждающий универсальность закона причинности и многообразие форм её проявления.

Так, \emph{зная} угол, под которым сталкиваются электрон и позитрон (а они при соответствующих условиях превращаются в два фотона) \emph{можно определить} (предсказать) направление движения двух образовавшихся фотонов. Это доказывает существования причинности в микромире.

Если бы не действовал закон причинности, если бы движение микрочастиц \emph{происходило как угодно}, произвольно, то в одном случае электрон и позитрон порождали бы два фотона, а в другом при этих же условиях --- один фотон или два протона, к тому же нельзя было бы определить направление движения образовавшихся частиц.

Все микропроцессы \emph{подчинены} объективным законам, в них есть определённая последовательность.

Объективность причинной связи всех явлений действительности обосновывали и \emph{материалисты прежних эпох}, до появления диалектико- материалистической философии. Но они \emph{ограничивались} рассмотрением механической формы причинности (макромеханической), когда причина выступает как внешняя по отношению к следствию.

Диалектический материализм \emph{стремится преодолеть} ограниченность механистического и метафизического понимания причинности.

Связь причины и следствия \emph{носит характер} взаимодействия: не только причина порождает следствие, но и следствие может действовать на причину и изменять её.

В процессе взаимодействия причина и следствие \emph{могут меняться} местами.

Например, развитие капитализма в России послужило причиной отмены крепостного права, но отмена крепостного права \emph{в свою очередь} явилась причиной дальнейшего ускоренного развития капитализма.

Взаимодействие причины и следствия означает \emph{постоянное влияние} их друг на друга, \emph{в результате} чего происходит изменение, как причины, так и следствия.

Взаимодействие выступает внутренней причиной изменений явлений действительности (\emph{causa sui} --- причиной самого себя).

Мир как взаимодействие различных явлений для своего движения, развития \emph{не нуждается ни в каком внешнем} толчке, ни в какой потусторонней силе, вроде бога и т.п.

Именно взаимодействие является истинной причиной, конечной причиной (\emph{causa finalis}) всех вещей (\emph{Гегель}).

Кончено, взаимодействующие силы, факторы \emph{не равнозначны}. В системе взаимодействующих сил наука \emph{должна раскрывать} определяющие причины.

На взаимодействие причины и следствия \emph{оказывают влияние} окружающие их явления, совокупность которых носит название \emph{условий}.

\emph{Условия --- это такие явления, совокупность которых необходима для наступления данного события, но сами по себе они его не вызывают}.

Так, соответствующий возбудитель болезни, например, вирус спида, как причина \emph{может вызвать} соответствующее заболевание в зависимости от условий, в которые он попадает, т.е. от состояния организма.

Среди условий могут быть такие, которые \emph{способствуют} возникновению следствия, а могут быть и такие, которые \emph{блокируют} действие причины.

В зависимости от условий одно и то же явление может порождаться \emph{различными} причинами, и наоборот, \emph{одна и та же} причина может приводить к различным следствиям.

Так, огромная энергия может быть получена \emph{и в результате} расщепления ядер урана, \emph{и в результате} синтеза ядер водорода в ядра гелия (термоядерный синтез).

Причинные взаимосвязи явлений, несмотря на их многообразие, \emph{не исчерпывают} все богатство связей в мире.

Явления вступают друг с другом в \emph{различные отношения}: временные, пространственные и т.д., которые связаны с каузальностью, \emph{но не сводятся} к ней.

Наука не может ограничиться \emph{изучением только} причинных взаимосвязей явлений, она \emph{призвана изучать} явления во всем многообразии их закономерных связей.

\subsection{Необходимость и случайность}

Ранее \emph{было показано}, что закономерные связи и отношения вещей существенны и необходимы. \emph{Необходимость --- это устойчивая, существенная связь явлений}, процессов, объектов действительности, обусловленная всем предшествующим ходом их развития. \emph{Необходимое} вытекает из сущности вещей и при определённых условиях \emph{должно} обязательно произойти, \emph{не может не произойти}.

При этом следует различать необходимость и \emph{неизбежность}. Не всё необходимое неизбежно. Необходимость может быть неизбежной, когда исключаются все другие возможности и остается лишь одна из них.

Но всё ли, что появляется в мире, возникает \emph{как нечто} необходимое? Нет, в мире имеют место и \emph{случайные}, не необходимые явления, события. \emph{Случайное --- это то, что в данных условиях может иметь место, а может и не иметь, может произойти так, а может совершиться и иначе}.

В рамках религиозного мировоззрения существуют взгляды, согласно которым в мире, в жизни общества и отдельного человека всё заранее \emph{предопределено} богом, или судьбой, или мировым духом, слепая сила которых неотвратима. \emph{Вера в судьбу}, в предопределение --- это \emph{фатализм}.

Незнание диалектики ведёт обычно к \emph{противопоставлению} необходимости и случайности; одно будто бы исключает другое.

Так, древнегреческий философ-атомист \emph{Демокрит} утверждал, что всё совершается только по необходимости. «Люди измыслили идол (образ) случая, чтобы пользоваться им как предлогом, прикрывающим их собственную нерассудительность». (Цит. по кн.: «\emph{Материалисты древней Греции}». М., 1955, с. 69).

Почти все мыслители, отрицавшие случайность, \emph{отождествляли} её с отсутствием причины. Отсюда и вытекает \emph{ложный} вывод: поскольку всё происходящее имеет свою причину, то случайность (равно отсутствие причины) \emph{невозможна}. Мы будто бы называем случайными явления, причины которых мы \emph{ещё не в состоянии} с точностью установить и предвидеть, тогда как сами по себе эти явления не случайны, а необходимы.

Так, \emph{Б. Спиноза}, голландский философ, считал, что в природе вещей нет ничего случайного, но всё определено к существованию и действию по известному образу из необходимости природы.

\emph{Французские материалисты XVIII в.} тоже утверждали, что всё совершается с абсолютной необходимостью, и в мире вообще нет случайности. Вся наша жизнь, по словам \emph{П. Гольбаха}, --- это линия, которую мы должны по велению природы описать на поверхности земного шара, не имея возможности удалиться от неё ни на один момент.

Абсолютизация необходимости и отрицание случайности логически вытекает из \emph{механистической картины мира}.

Это получило своё наиболее характерное выражение в позиции \emph{Лапласа}. «Все явления, --- писал он, --- даже те, которые по своей незначительности как будто не зависят от великих законов природы, суть следствия столь же неизбежные следствия этих законов, как обращение солнца. Не зная уз, соединяющих их с системой мира в её целом, их приписывают конечным причинам или случаю, в зависимости от того, происходили ли и следовали ли они одно за другим с известной правильностью или же без видимого порядка; но эти мнимые причины отбрасывались по мере того, как расширялись границы нашего знания и совершенно исчезали перед здравой философией, которая видит в них лишь проявление неведения, истинная причина которого --- мы сами» (\emph{П. Лаплас}. «\emph{Опыт философии теории вероятностей}», с. 8.).

При абсолютизации необходимости она \emph{превращается} в свою противоположность.

Отрицая случайность, французские материалисты XVIII в. необходимость \emph{низводили} до степени случайности.

Тот же \emph{Гольбах} утверждал, что излишек едкости в жёлчи фанатика, разгорячённость крови в сердце завоевателя, дурное пищеварение у какого-нибудь монарха, прихоть какой-нибудь женщины \emph{являются достаточными} причинами, чтобы заставить предпринимать войны, чтобы посылать миллионы людей на бойню, погружать народы в нищету и траур, чтобы вызывать голод и заразные болезни и распространять отчаяние и бедствия на длинный ряд веков.

Существование необходимости в природе и обществе отрицали некоторые позитивисты XX в. Так, по мнению \emph{Л. Витгенштейна}, существует только \emph{логическая необходимость} --- необходимость следования одного суждения из другого, при этом логическая необходимость не отражает никакой объективной закономерности, а возникает \emph{из природы языка}.

В метафизическом мышлении получается \emph{ложная альтернатива}: или в мире господствует \emph{только} случайность --- и тогда нет необходимости, \emph{или} никакой случайности в мире нет --- и тогда всё осуществляется с неотвратимой неизбежностью.

В действительности же необходимость \emph{не существует} в «чистом виде». Любой необходимый процесс осуществляется во множестве случайных форм, \emph{при участии} множества случайных факторов.

Если бы в мире господствовала только необходимость, то в нём всё было бы фатально детерминировано и \emph{не было бы места} для свободной деятельности человека.

Точно так же \emph{не бывает} абсолютно случайных явлений, в противном случае не могло быть и речи о закономерной связи явлений, всё зависело бы от удачного или неудачного стечения обстоятельств.

Необходимое и случайное различаются между собой, прежде всего тем, что появление и бытие необходимого обусловлено \emph{существенными факторами}, а случайного --- чаше всего \emph{несущественными факторами}.

Диалектика необходимости и случайности состоит в том, что \emph{случайность выступает как форма проявления необходимости и как её} до\emph{полнение}.

Случайность в ходе развития может \emph{превращаться в необходимость}.

Так, как уже отмечалось выше, закономерные признаки того или иного биологического вида \emph{вначале появляются} как случайные отклонения от признаков другого вида. Эти случайные отклонения, сохраняясь и накапливаясь, \emph{становятся со временем} необходимыми элементами качества живого организма.

Случайности \emph{никогда не оставались} вне поля зрения научного познания, даже тогда, когда от них пытались абстрагироваться как от чего-то второстепенного.

\emph{Основная цель познания --- вскрывать закономерное, необходимое}.

Из этого, однако, \emph{не следует, будто} случайное принадлежит лишь области нашего субъективного представления и поэтому должно быть игнорировано в научном исследовании.

Через анализ различных случайных, единичных фактов наука \emph{движется к обнаружению} того, что лежит в их основе, --- к определённой необходимости.

\emph{Учёт диалектики, взаимодействия необходимости и случайности --- важное условие правильной практической, творческой деятельности}.

Немало открытий в науке и изобретений в технике осуществлено \emph{в силу благоприятного} стечения случайных обстоятельств.

\emph{Как бы ни были} рассчитаны наши поступки, они связаны с тем, что мы что-то оставляем на долю случая.

Развитие производства и науки \emph{все больше выводит} человека из-под власти неблагоприятных случайностей, порождая, правда, некоторые новые --- техногенные катастрофы.

Современное общество \emph{получает всю большую} возможность управлять собственными процессами, планировать развитие экономики, социальной и культурной сфер \emph{и тем самым} ограждать общество от пагубного действия случайностей.

На учёте влияния случайностей построено \emph{различение} статистической и динамической закономерностей, играющих значительную роль в науке.

\emph{Динамическая закономерность (закон)} --- это такая форма необходимой причинной связи, при которой взаимоотношение между причиной и следствием однозначно; другими словами, зная начальные условия той или иной системы, мы можем точно предсказать ее дальнейшее развитие, поведение.

Так, предсказание явлений солнечного и лунного затмений \emph{строится на} учёте динамических по своему качеству закономерностей движения небесных тел.

\emph{Статистическая закономерность (закон)} --- это, в отличие от динамической, единство необходимых и случайных признаков явлений. В этом случае из начального состояния системы её последующее состояние следует \emph{не однозначно}, а с определённой вероятностью, неопределенностью.

Приведём примеры.

Если вы купили \emph{лотерейный билет}, из этого не следует, что вы обязательно выиграете. Вы можете и выиграть, и не выиграть.

Вы бросаете вверх \emph{монету}. Вы заранее не знаете, что выпадет, «орёл» или «решка».

Выигрыш по лотерее или выпадение «орла» при подбрасывании монеты --- это \emph{типичные примеры} случайных явлений.

Мера осуществимости того или иного случайного события характеризуется понятием \emph{вероятности.}

Если событие \emph{никогда} не произойдёт, то его вероятность равна нулю.

Если оно произойдёт \emph{обязательно}, то его вероятность равна единице.

Все случайные события характеризуются \emph{вероятностью}, заключённой между нулем и единицей.

\emph{Чем чаще} происходит случайное событие, \emph{тем больше} его вероятность.

Понятие вероятности оказывается \emph{тесно связанным} с понятием неопределённости.

\emph{Неопределенность} возникает тогда, когда из нескольких предметов происходит выбор. Если вы имеете дело с одним предметом, то выбирать не из чего. Здесь нет никакой неопределённости.

Вероятность \emph{выбрать один} предмет равна единице. Но когда имеется \emph{два} предмета, то возникает уже неопределённость: вы можете выбрать или тот, или другой предмет.

Вероятность и мера неопределенности оказываются \emph{в весьма простой} зависимости: чем меньше вероятность выбора, тем больше неопределённость.

Когда степень неопределённости \emph{равна} нулю, вероятность равна единице.

Когда степень неопределённости \emph{равна} бесконечности, вероятность равна нулю.

Характерной особенностью \emph{статистических законов} является и то, что они основываются на \emph{случайности, обладающей устойчивостью}. Это значит, что они применяются только к \emph{большим совокупностям} явлений, каждое из которых носит случайный характер.

Статистическим закономерностям \emph{подчиняются}, например, такая совокупность массовых явлений, как скопление молекул газа. Движение отдельной молекулы по отношению к закономерностям, господствующим в совокупности, \emph{в целом}, является случайным. Но \emph{из перекрещивания} случайных движений отдельных молекул складывается необходимость, которая проявляется не полностью или даже вовсе не проявляется в каждом отдельном случае.

Существует \emph{закон больших чисел}, выражающий диалектику необходимого и случайного. Этот закон гласит: \emph{совокупное действие} большого числа случайных фактов приводит при некоторых общих условиях к результату, почти \emph{не зависящему} от случая.

Другими словами, суммирование большого числа случаев, отдельных явлений приводит к тому, что их случайные отклонения в ту или иную сторону нивелируются --- образуется \emph{определённая тенденция}, нечто закономерное. Эта закономерность и называется статистической.

Статистическая закономерность, проявляющаяся в массе единичных явлений, вместе с её специфическими взаимоотношениями между причиной и следствием, необходимым и случайным, единичным и общим, целым и его частями, возможным и вероятным составляет ту \emph{объективную основу}, на которой стоится применение \emph{статистических методов} научного исследования.

\subsection{Возможность и действительность}

Одно из важных мест в богатом арсенале средств современного теоретического мышления занимают категории \emph{возможности} и \emph{действительности}. Подобно всем другим категориям, они отражают универсальные связи и отношения вещей, процесс их изменения, развития

Как известно, из ничего \emph{не может возникнуть} нечто, и новое может возникнуть лишь из определённых предпосылок, заложенных в лоне того, что уже имеет место.

\emph{Бытие нового в его потенциальном состоянии и есть возможность.}

Ребёнок появляется на свет. Он заключает в себе \emph{множество потенций} --- возможность ощущать, чувствовать, мыслить, говорить.

В соответствующих условиях \emph{возможность превращается в действительность.}

\emph{Под действительностью в широком смысле слова имеют в виду всё актуально существующее.} Причём, как в зародышевом, так и в зрелом и в увядающем состоянии. Это единство единичного и общего, сущности и многообразных форм её проявления, необходимого и случайного.

В узком смысле \emph{под действительностью имеют в виду реализованную возможность} --- нечто уже ставшее, развившееся.

В мире нет ничего, что не было бы или в возможности, или действительности, или в «пути» от одного к другому.

\emph{Процесс развития --- это диалектическое единство возможности и действительности.}

Возможность органически связана с действительностью. Они взаимопроникают. Ведь возможность --- это одна из форм действительности в широком смысле слова, внутренняя, \emph{потенциальная действительность}.

Во взаимосвязи возможного и действительного \emph{«первенство» принадлежит действительности}. Правда, со временем возможность предшествует действительности. Но сама возможность является лишь одним из моментов того, что уже существует как реальная действительность.

Существуют различные \emph{виды возможности}.

Возможности могут быть \emph{общими и единичными.}

\emph{Общая возможность} выражает предпосылки общих элементов единичных предметов как систем, а \emph{единичная возможность} --- предпосылка единичных моментов, индивидуальных особенностей явлений.

Общая возможность обусловлена закономерностями развития действительности, а единичная --- специфическими условиями существования и действия этих общих закономерностей.

Каждая единичная возможность \emph{неповторима}.

Возможности могут \emph{быть реальными (конкретными) и формальными (абстрактными).}

Мы называем возможность \emph{реальной}, если она выражает закономерную, существенную тенденцию развития объекта и в действительности существуют необходимые условия её реализации.

\emph{Формальная} возможность выражает несуществующую тенденцию развития объекта и в действительности отсутствуют условия, необходимые для её реализации. В её пользу \emph{можно привести} только надуманные, основания.

Возможно, например, «что турецкий султан сделается папой, ибо он --- человек, может как таковой обратиться в христианскую веру, сделаться католическим священником и т.д.». (\emph{Гегель}. Сочинения, т. 1, с. 241).

Формальная возможность \emph{сама по себе} не противоречит объективным законам. И в этом смысле она коренным образом отличается от \emph{невозможности, т.е. того, что принципиально, ни при каких условиях не может быть реализован}о.

Например, \emph{невозможно} создание вечного двигателя. Это \emph{противоречит} законам сохранения энергии.

И в теоретической и в практической деятельности чрезвычайно важно уметь отличать возможное от невозможного, \emph{реально возможное} от \emph{возможного лишь формально}.

\emph{Формальная возможность} может рассматриваться как возможность только при отвлечении от всех других возможностей.

Огромная масса формальных возможностей \emph{не превращается} в действительность.

Но различие между реальной и формальной возможностями \emph{в известной мере} относительна.

Вполне реальная возможность может оказаться \emph{упущенной} или объективно не реализованной в силу каких-то обстоятельств. Она \emph{превращается} в формальную возможность.

Вместе с тем, возможности полётов человека в космос \emph{были совсем недавно} лишь формальными, теперь же они стали реальными.

Во времени, как отмечалось, возможность \emph{предшествует} действительности. Но действительное, будучи результатом предшествующего развития, является в то же время \emph{исходным пунктом} дальнейшего развития.

Возможное \emph{возникает} в данном действительном \emph{и реализуется}, воплощается в новой действительности.

Как скрытые тенденции, выражающие различные направления в развитии объекта, возможности характеризуют действительность \emph{с точки зрения} её \emph{будущего}.

Все возможности «\emph{нацелены}» к реализации и обладают определённой направленностью. Но эта обращенность к будущему не означает, что, как утверждают \emph{фаталисты}, конечный результат любого процесса в мире предначертан уже в самом начале и наступает с неотвратимой силой.

Диалектико-материалистическая философия \emph{исходит их того}, что развитие --- это не развёртывание готового набора возможностей, а постоянный процесс зарождения возможностей в рамках действительности и их превращения в \emph{новую действительность}.

Как и всё в мире, \emph{возможности развиваются}: одни из них растут, другие угасают.

Чтобы возможность перешла в действительность, необходимо наличие соответствующих \emph{условий.}

Имеется \emph{существенная разница} в процессе превращения возможности в действительность в природе и в человеческом обществе.

В природе превращение возможности в действительность происходит в целом \emph{стихийно}.

Историю же делают \emph{сами люди}. От их воли, сознания, активности \emph{зависит очень много} в процессе реализации заложенных в общественном развитии возможностей.

\subsection{Содержание и форма}

Любой объект действительности представляет собой \emph{единство} содержания и формы.

В мире \emph{нет, и не может быть} содержания вообще, а есть только определённым образом оформленное содержание.

\emph{Под содержанием имеется в виду состав всех элементов объекта, единство его свойств, внутренних процессов, связей, противоречий и тенденций развития}.

Например, содержанием живого организма является \emph{не просто} совокупность его органов, \emph{а весь} реальный процесс его жизнедеятельности, протекающий в определённой форме.

\emph{Под формой понимается} способ внешнего выражения содержания, \emph{относительно устойчивая определённость связи элементов содержания и их взаимодействия, тип и структура содержания}.

Форма и содержание представляют собой определённое отношение \emph{не только} различных, \emph{но и} противоположных моментов объекта.

Разделение объекта на форму и содержание \emph{существует только в рамках} их неразрывного единства, а их единство существует лишь как внутренне расчленённое.

Между содержанием и формой, содержательной и формальной сторонами объекта \emph{нет непроходимой} пропасти. Они \emph{могут} переходить друг в друга.

Так, \emph{мысль} есть идеальная форма отражения объективной реальности и вместе с тем составляет содержание нервно-физиологических процессов.

Форма и содержание в каждом конкретном объекте \emph{неотделимы} друг от друга.

Форма \emph{не есть} что-то внешнее, наложенное на содержание.

\emph{Например, жидкость} в состоянии невесомости, предоставленная самой себе, приобретает форму шара.

Самая превосходная идея \emph{ещё не даёт} произведения искусства, если она не облекается в соответствующую художественную форму, в художественные образы.

То, «что делает Илиаду Илиадой, есть та поэтическая форма, в которой выражено содержание». (\emph{Гегель}. Сочинения, т.1, с. 225).

Форма представляет собой \emph{единство} внутреннего и внешнего.

\emph{Как способ связи элементов содержания форма есть нечто внутреннее}.

Форма \emph{составляет} структуру объекта \emph{и становится} как бы моментом содержания.

\emph{Как способ связи данного содержания с содержанием других вещей форма есть нечто внешнее}.

Так, \emph{внутренней формой} художественного произведения являются, прежде всего, сюжет, способ связи художественных образов, идей, составляющих содержание произведения.

\emph{Внешнюю форму} произведения составляет его чувственно воспринимаемый облик, его внешнее оформление.

Форма «\emph{одновременно} и содержится в самом содержании и представляет собою нечто внешнее ему» (Там же, стр. 224).

Формы различаются по \emph{степени общности}.

Форма может быть способом организации \emph{единичного} предмета, некоторого \emph{класса} предметов и \emph{бесконечного множества} предметов.

Проблема соотношения содержания и формы \emph{по-разному решалась} представителями различных философских направлений.

Так, согласно \emph{Аристотелю}, содержание и форма изначально существуют как нечто \emph{самостоятельное}, независимое друг от друга и \emph{только впоследствии}, при образовании какой-либо вещи, они вступают между собой в тесную связь. При этом в роли первичной формы, или \emph{формы форм} выступает бог.

Нередко и сегодня форма \emph{абсолютизируется}, отрывается от содержания. Форма \emph{в таких случаях} становится самодовлеющей ценностью.

Форма и содержание --- это противоположности, находящиеся в единстве, это \emph{разные полюсы} одного и того же.

Неразрывное единство содержания и формы \emph{выявляется в том, что} определённое содержание «\emph{облачается}» в определённую же форму.

\emph{Ведущей стороной} в этом отношении является содержание: форма организации зависит от того, что организуется.

Не какая-то внешняя сила, а \emph{само} содержание формирует себя.

Между формой и содержанием \emph{имеется} внутреннее противоречие.

Возникновение, развитие и преодоление \emph{противоречий между содержанием и формой} вещей, процессов является одним из наиболее существенных и всеобщих выражений развития путём борьбы противоположностей.

Категории содержания и формы имеют \emph{большое значение} для осмысления диалектики процессов развития.

Форма, соответствующая содержанию, \emph{способствует} его успешному развитию.

Форма, не соответствующая содержанию, \emph{начинает тормозить} его дальнейшее нормальное развитие. В таком случае возникает \emph{конфликт между формой и содержанием}, который разрешается путём изменения прежней формы и появления формы, соответствующей изменившемуся, новому содержанию.

Единство формы и содержания \emph{предполагает} их относительную самостоятельность и активную роль формы по отношению к содержанию.

\emph{Относительная самостоятельность} формы выражается, например, в том, что она может несколько \emph{отставать} от развития содержания, или, наоборот, \emph{опережать} его развитие.

\emph{Изменение формы} представляет собой перестройку связей внутри предмета. Этот процесс развёртывается во времени, осуществляется \emph{через} противоречия, коллизии.

Отставание формы от содержания \emph{ведёт к} несоответствию одного другому.

Относительная самостоятельность формы и содержания выявляется и в том, что \emph{одно и то же содержание может облекаться в различные формы}.

Вместе с тем \emph{одна и та же форма может встречаться с разным содержанием}.

Так, \emph{одной и той же}, например, математической формулой можно выразить законы разных по своей природе явлений.

Учёт взаимосвязи содержания и формы и их относительной самостоятельности имеет \emph{особенно большое} значение для практической деятельности человека, \emph{когда умелое использование} формы организации труда, производственного процесса, расстановки людских сил может решить ход и исход дела.

\subsection{Сущность и явление}

\emph{Сущность} и \emph{явление} --- это категории, выражающие различные стороны явлений, вещей, ступени познания, разный уровень глубины постижения объекта.

Движение человеческого познания идёт \emph{от внешней формы предмета к его внутренней организации}.

Познание объекта \emph{начинается с установления} внешних свойств, пространственных отношений вещей.

Установление причинных и иных глубинных, закономерных отношений и свойств вещей является \emph{переходом} к раскрытию сущности.

Логика развития познания и потребности общественной практики привели человека к необходимости \emph{строго отличать} то, что составляет существо объекта, от того, каким он нам является.

Философия диалектического материализма исходит из того, что \emph{сущность} и \emph{явление} --- \emph{это универсальные объективные характеристики вещей}.

Что значит \emph{постигнуть сущность} какого-либо объекта? --- Это значит \emph{понять} причину его возникновения, законы его жизни, свойственные ему внутренние противоречия, тенденции развития, его определяющие свойства.

Сущность того или иного процесса можно раскрыть с различной степенью \emph{полноты}.

Наше мышление движется \emph{не только от} явления к сущности, \emph{но и от} менее глубокой ко всё более глубокой сущности.

Та особая реальность, которая составляет \emph{как бы «основание» объекта} и выступает как нечто устойчивое, главное в его содержании, и выражается в категории сущности.

С категорией сущности тесно связана \emph{категория общего}. То, что является сущностью определённого класса предметов, \emph{есть в то же время} их общность.

\emph{Существенное} --- \emph{значит важное}, \emph{определяющее (необходимое) в объекте}.

Когда мы говорим о сущности, то мы \emph{имеем в виду} именно закономерное.

Например, периодический \emph{закон Д.И. Менделеева} вскрывает существенную внутреннюю связь между атомным весом (теперь зарядом) элемента и его химическими свойствами.

Однако сущность и закон \emph{не есть} нечто тождественное.

\emph{Сущность шире} и \emph{богаче закона}.

Например, сущность жизни заключается \emph{не просто в каком-то одном} законе, а в целом комплексе законов.

Характеризуя сущность объекта, мы выражаем её с помощью близких к категории сущности, но не тождественных с ней категорий: \emph{единое} во \emph{многом}, \emph{общее} в \emph{единичном}, \emph{устойчивое} в \emph{изменчивом}, \emph{внутреннее}, \emph{закономерное}.

А что такое \emph{явление}? --- Это \emph{внешнее обнаружение сущности, форма её внешнего выражения}.

В отличие от сущности, которая скрыта от человека, непосредственно не наблюдаема, явление \emph{лежит как бы} на поверхности вещей, доступна непосредственному восприятию.

Сущность, как нечто \emph{внутреннее}, противопоставляется внешней, изменчивой стороне вещей.

Когда говорится, что явление --- это нечто внешнее, а сущность --- внутреннее, \emph{то имеется в виду} не пространственное отношение, а объективная значимость внутреннего и внешнего для характеристики самого предмета.

Явление не может существовать \emph{без того, что} в нём является, т.е. без сущности.

«Сущность является. Явление существенно». (\emph{В.И. Ленин}. ПСС, т.29, с. 227).

В сущности \emph{нет ничего, что} не проявляется так или иначе.

Но явление красочнее сущности \emph{хотя бы потому}, что оно индивидуализировано, связано с неповторимой совокупностью внешних условий.

В явлении существенное связано с несущественным, случайным.

Сущность обнаруживается и в массе явлений, и в единичном явлении.

В одних явлениях сущность проступает полно и «прозрачно», а в других завуалировано.

\emph{Сущность} и \emph{явление --- соотносительные категории}. Они характеризуются \emph{друг через} друга.

Если сущность есть \emph{нечто общее}, то явление --- \emph{нечто единичное}, выражающее лишь какой-то момент сущности; если сущность есть нечто \emph{глубинное и внутреннее}, то явление --- нечто \emph{внешнее}, более богатое и красочное; если сущность есть нечто \emph{устойчивое, необходимое}, то явление --- нечто \emph{преходящее, изменчивое, случайное}.

Впрочем, сущность \emph{есть тоже нечто} преходящее, но \emph{иным способом}.

Отличие существенного от несущественного \emph{не абсолютно}, а относительно.

В своё время, например, существенным свойством химического элемента \emph{считался} атомный вес. Свойство атомного веса \emph{не перестало} быть существенным. Оно существенно \emph{в первом приближении}, являясь сущностью менее глубокого порядка, и своё объяснение оно получает через заряд атомного ядра.

Сущность \emph{выражается во} множестве её внешних проявлений.

Однако в явлениях сущность может \emph{не только} выражаться, \emph{но и} \emph{маскироваться}.

В процессе \emph{чувственного познания} мы нередко сталкиваемся с тем, что явления кажутся нам не такими, каковы они есть на самом деле.

Имеет место то, что мы характеризуем как \emph{видимость}, или \emph{кажимость}.

Видимость \emph{не есть} порождение нашего сознания.

Видимость также \emph{объективна, как и само явление}, возникая в результате воздействия на субъект реальных отношений в объективных условиях восприятия.

Так, многочисленные поколения людей, признававшие вращение Солнца вокруг Земли, \emph{принимали видимое} явление (видимость) \emph{за нечто} действительное.

Таким образом, чтобы правильно понять то или иное событие, разобраться в нём, необходима \emph{критическая проверка} данных непосредственного восприятия, \emph{чёткое различение} кажущегося и реального, поверхностного и существенного.

\emph{Постижение сущности вещей} --- \emph{основная задача науки.}

\chapter{Учение диалектико-материалистической философии. О природе, материи, сознании и познании}

\section{Материя и основные формы её существования}

\emph{Исходным положением диалектического материализма} является признание материальности мира, его несотворимости и неуничтожимости, вечности существования во времени и бесконечности в пространстве, его неугасающего саморазвития, которое необходимо приводит на определённых этапах к возникновению жизни и мыслящих существ.

\emph{Только через человека} материя становится способной к познанию законов своего собственного существования и развития.

Каковы же \emph{основные свойства}, формы бытия и общие законы развития материи?

К систематическому изложению современного диалектикоматериалистического решения этих вопросов \emph{мы и переходим}.

\subsection{Философское понимание материи}

В окружающем нас мире \emph{мы наблюдаем} бесчисленное множество различных предметов и явлений. Есть ли между ними \emph{что-либо общее}, какова их природа, что лежит в их основе?

Различные попытки решить эти вопросы исторически привели к возникновению понятия \emph{субстанции} всех вещей (лат. \emph{substancia} -- \emph{сущность}).

Под субстанцией понималась некая \emph{всеобщая первичная основа} всех вещей, которая является их последней сущностью.

Если различные предметы и явления могут возникать и исчезать, то субстанция остается \emph{несотворимой} и \emph{неуничтожимой}, она лишь меняет формы своего бытия, переходит из одних состояний в другие.

Субстанция --- причина самой себя и \emph{основание всех} изменений, самый фундаментальный и устойчивый слой реальности.

Понятие субстанции какой-либо формы \emph{означает} возникновение вещи с тем качеством, которое соответствует этой форме.

Само формирование философии происходит именно в эпоху возникновения понятия \emph{субстанции} и \emph{единства} окружающего нас мира, закономерной связи явлений действительности.

В учениях \emph{милетской школы} в Древней Греции в ранг субстанции возводились конкретные формы вещества: \emph{вода} (\emph{Фалес}), \emph{воздух} (\emph{Анаксимен}), \emph{земля}, которые, как полагали античные мыслители, могут превращаться друг в друга.

В философии \emph{Гераклита} субстанцией считается \emph{огонь}, который образует Солнце, звезды, все другие тела и определяет вечное изменение мира.

В философии \emph{Анаксимандра} субстанцией считается не конкретное вещество, а некоторая \emph{бесконечная и неопределенная} материя --- \emph{апейрон}, вечная во времени, неисчерпаемая в структуре и непрерывно меняющая формы своего существования.

Все эти представления, однако, \emph{не позволяли} выразить идею всеобщности и сохранения субстанции в последовательной и непротиворечивой форме.

Каждое \emph{из четырёх} вещественных «\emph{первоначал}» не обладало необходимой всеобщностью с устойчивостью, а идея апейрона была слишком неопределённой и допускающей многозначные толкования.

Более развитой была \emph{атомистическая теория} субстанции, выдвинутая \emph{Левкиппом} и \emph{Демокритом} (VII в. до н.э.). Она была развита в дальнейшем \emph{Эпикуром} (III в. до н.э.) и \emph{Лукрцием Каром} (I в. до н.э.).

Эти мыслители допускали существование \emph{первичных простейших частиц} --- атомов, которые несотворимы и неразрушимы, находятся в непрерывном движении, различаются по весу, форме и взаимному расположению в телах.

Считалось, что различие качеств тел определяется \emph{различиями} в числе составляющих тела атомов, их формы, взаимного расположения и скорости движения.

\emph{Число атомов} во Вселенной \emph{бесконечно}, их \emph{вихри} образуют звезды, подобные Солнцу, планеты, а \emph{благоприятные сочетания атомов} приводят к возникновению живых существ и самого человека.

Атомистами впервые был выдвинут в конкретной и определённой форме \emph{принцип сохранения материи} как принцип нерушимости атомов. Именно эта конкретность и определённость выражения идеи материальной субстанции \emph{предопределила} в дальнейшем жизненность и широкое распространение атомизма во всех материалистических учениях.

Из идеи сохраняемости и абсолютности материи \emph{необходимо вытекало} положение о вечности и бесконечности мира, первичности материи по отношению к сознанию человека, о закономерной обусловленности всех явлений в мире.

Убеждение в материальности мира, в подчинении всех явлений необходимым законам природы \emph{создавало} у представителей атомистического материализма \emph{уверенность} в безграничных возможностях человеческого разума, в его способности последовательно объяснить все явления в мире.

Атомистический принцип получил дальнейшую разработку в философии и естествознании \emph{Нового времени} в трудах \emph{Ньютона}, \emph{Гассенди}, \emph{Бойля}, \emph{Ломоносова}, \emph{Гоббса}, \emph{Гольбаха}, \emph{Дидро} и других мыслителей.

На основе атомистической теории \emph{удалось объяснить} природу тепла, диффузии, теплопроводности, многие химические явления.

Атомистическая парадигма \emph{способствовала} возникновению корпускулярной теории света.

Но и атомизм, \emph{в силу неразвитости} научного типа познания на ранних ступенях, \emph{не мог объяснить} очень большого количества явлений, вывести из предполагаемых свойств и законов движения атомов особенности живых организмов, функции человеческого организма и множество других явлений природы и общества.

Нужно сказать, что и в современной науке большинство известных явлений \emph{ещё не имеет} своего причинного и структурного объяснения.

В качестве \emph{противовеса атомистике} возникли различные \emph{идеалистические концепции субстанции}, в которых в ранг всеобщей основы мира возводились божественная воля, мировой разум, абсолютный дух и т.п.

\emph{Психические качества}, свойственные человеческому \emph{мозгу}, здесь отрывались от него, возводились в абсолют, в мировой разум, творящий материю, пространство и время.

Ещё сравнительно недавно по меркам истории, в 1870 г., \emph{первый Ватиканский собор} во 2-м каноне «Догматической конституции католической веры» вновь подтвердил принцип: «\emph{Если кто-либо не постыдится утверждать, что вне материи ничего нет, --- да будет он проклят!}».

В \emph{философии Гегеля} мир --- это форма осуществления, инобытия абсолютного духа, некоего идеального, обожествленного разумного первоначала, которое в процессе саморазвития познаёт через природу и человеческую историю свою собственную сущность.

Развитие науки \emph{отказывается} от религиозно-идеалистических представлений о мире.

Крупными этапами на этом пути были познание строения \emph{Солнечной системы} и \emph{Галактики}, открытие методами спектрального анализа единого химического состава Солнца и других звезд, установление общих законов движения различных космических тел, познание геологической истории Земли, законов развития растительных и животных видов.

Открытие закона сохранения энергии, единого клеточного состава всех живых организмов, создание \emph{Ч. Дарвином} теории эволюции биологических видов \emph{послужили основой} для создания положений диалектико-материалистической философии, как отмечалось выше.

С прогрессом науки \emph{всё более обнаруживалась} ограниченност\emph{ь} метафизического метода мышления.

В рамках механической картины мира, господствовавшей в естествознании в XVII-XIX вв., \emph{абсолютизировались} известные тогда механические законы движения, физические свойства и состояния материи.

Законы механики \emph{распространялись как на} микромир, \emph{так и} на все мыслимые пространственно-временные масштабы Вселенной.

Единство мира \emph{понималось как} однородность и единообразие его строения, как бесконечное повторение одних и тех же звёзд, планет и других известных форм материи, подчиняющихся вечным и незыблемым законам движения.

Казалось, что \emph{абсолютная истина} \emph{уже близка}, основные законы мироздания открыты и лишь чисто технические трудности \emph{не позволяют} вывести свойства различных химических соединений и даже живых организмов из динамических законов движения атомов.

Выдающийся учёный начала XIX в. \emph{П. Лаплас} писал: «Ум, которому были бы известны для какого-либо данного момента все силы, одушевляющие природу, и относительное положение всех её составных частей, если бы вдобавок он оказался достаточно обширным, чтобы подчинить эти данные анализу, обнял бы в одной формуле движения величайших тел вселенной наравне с движениями легчайших атомов: не осталось бы ничего, что было бы для него недостоверно, и будущее, так же как и прошедшее, предстало бы перед его взором». (\emph{П. Лаплас}. \emph{Опыт философии теории вероятностей}. М., 1908, с.9).

Однако природа оказалась \emph{гораздо сложнее}, чем думали многие физики и философы.

Во второй половине XIX в. исследованиями \emph{Фарадея} и \emph{Максвелла} были установлены законы изменения качественно новой, по сравнению с веществом формы материи --- \emph{электрического поля}.

Законы электрического поля \emph{оказались несводимыми} к законам классической механики.

В конце XIX --- начале XX в. последовала \emph{новая серия открытий}: радиоактивности, сложности химических атомов, электронов, изменяемости массы тел в зависимости от скорости, кванта действия.

Была \emph{установлена} неприменимость ряда законов механики к объяснению структуры атомов и движения электронов, \emph{открыта} зависимость пространственно-временных свойств тел от скорости их движения.

В физике возник \emph{кризис механической картины мира} и метафизического понимания материи.

Однако идеалисты, и прежде всего представители так называемого \emph{эмпириокритицизма}, истолковали этот кризис как кризис всей физики и даже как \emph{крушение материализма вообще}, который при этом отождествлялся с механическим пониманием природы.

Так, радиоактивный распад атомов при этом истолковывался как «\emph{исчезновение}» материи, превращение материи в энергию.

Именно в этой ситуации \emph{В.И. Ленин}, в силу ряда обстоятельств занявшийся вопросами теории познания, формулирует обобщенное понятие диалектико-материалистической философии, \emph{понятие материи}: «Материя есть философская категория для обозначения объективной реальности, которая дана человеку в ощущениях его, которая копируется, фотографируется, отображается нашими ощущениями, существуя независимо от них». (\emph{В.И. Ленин}. \emph{Полн.собр.соч}.,т. 18, с. 298).

В этом определении, в отличие от прежних, \emph{преодолевается сведение} материи лишь к каким-либо определённым её видам --- частицам вещества, чувственно воспринимаемым телам и т. д.

Материя \emph{охватывает всё} бесконечное многообразие самых различных объектов и систем природы, которые существуют и движутся в пространстве и времени, обладают неисчерпаемым многообразием свойств.

Наши органы чувств могут воспринимать лишь \emph{ничтожную часть} этих реально существующих форм материи, но благодаря конструированию всё более совершенных приборов, измерительных устройств, а также развитию способности теоретического мышления, человек неуклонно расширяет границы познанного мира.

Данное В.И. Лениным определение материи охватывает \emph{не только} те объекты, которые познаны современной наукой, но и те, которые могут быть открыты в будущем, и в этом его особое методологическое значение.

Для каждого материального образования \emph{существовать --- значит обладать объективной реальностью} по отношению к другим телам, находиться с ними в объективных связях и взаимодействиях, быть элементом общего процесса изменения и развития материи.

\emph{Понятие материи как объективной реальности} характеризует материю вместе со всеми её свойствами, формами движения, законами существования и т. д.

Но \emph{это не значит}, что каждый отдельный, произвольно взятый фрагмент объективной реальности обязательно должен быть материей. \emph{Это может быть} и конкретное свойство материи, некоторый закон её существования, вид движения и т. п., которые неотделимы от материи, но всё-таки не тождественны ей.

В структуре объективной реальности \emph{следует различать} конкретные материальные объекты и системы (виды материи), свойства этих материальных систем (общие и частные), формы их взаимодействия и движения, законы существования, имеющие различную степень общности.

Так, движение, пространство, время, законы природы обладают объективной реальностью, \emph{но их всё же нельзя} считать самих по себе материей.

Материя существует в виде \emph{бесконечного разнообразия} конкретных объектов и систем, каждая из которых обладает движением, структурностью, связями и взаимодействиями, пространственно-временными и другими общими и частными свойствами.

Вне конкретных объектов и систем материи \emph{не существует}, и в этом смысле, нет объективно «\emph{материи как таковой}», материи «\emph{в чистом виде}» как первичной и бесструктурной субстанции.

Понятие субстанции претерпело \emph{радикальные изменения} в диалектико-материалистической философии по сравнению с предшествующей философией.

Диалектический материализм признаёт \emph{субстанциальность материи}, но лишь в том смысле, что именно она (а не сознание, не абсолютный дух, не божественный разум и т.п.) является единственной всеобщей основой, субстратом для различных свойств, связей, форм движения и законов.

Но внутри самой материи \emph{нет оснований допускать} существование некоторой \emph{первичной бесструктурной субстанции} как самого нижнего и фундаментального слоя реальности.

Любая форма материи (в том числе и микрообъекты) \emph{обладает} сложной структурой, многообразием внутренних и внешних связей, способностью к превращениям в другие формы.

Всякая \emph{научная теория материи} может быть только открытой, неограниченно развивающейся системой знаний.

Иногда, характеризуя тот или иной предмет, вещь, их рассматривают лишь как \emph{совокупность различных свойств}. В этом случае и материя, по сути дела, сводится к сумме свойств.

Однако материю \emph{нельзя растворять} в свойствах, пусть даже её собственных. Последние никогда \emph{не существуют сами по себе}, без материального субстрата, они всегда присущи определённым объектам.

Материальная действительность всегда обладает определённой \emph{организацией}, она существует в виде конкретных материальных систем.

\emph{Система --- это внутренне} (или внешне) \emph{упорядоченное множество взаимосвязанных} (или взаимодействующих) \emph{элементов}.

В системе связь между составляющими её элементами является \emph{более сильной}, устойчивой и внутренне необходимой, чем связь каждого из элементов с окружающей средой, с элементами других систем.

\emph{Внутренняя упорядоченность системы} выражается в комплексе законов и связей и взаимодействий между её элементами.

Каждый закон выражает \emph{определённый} порядок или тип связей между некоторыми явлениями.

\emph{Структура системы} выступает как совокупность внутренних связей между её \emph{элементами}, а также законов данных связей.

\emph{Понятие системы и элемента соотносительны}.

Всякая система может быть элементом \emph{ещё большего} образования, в которое она входит.

Точно так же и элемент выступает в качестве системы, если учитывать его \emph{собственную структурность}, наличие более глубоких структурных уровней материи.

Но эта соотносительность данных понятий \emph{не означает}, что системы придуманы человеком для удобства классификации явлений.

Системы \emph{существуют объективно} как упорядоченные целостные образования.

\emph{Границы} современного знания материи простираются от масштабов порядка \emph{10\textsuperscript{-15} см} («керн» нуклона) до \emph{10\textsuperscript{28} см} (примерно 13 миллиардов световых лет). В этом диапазоне материя \emph{всюду} обладает системной организацией.

Можно выделить следующие \emph{основные типы материальных систем} и соответствующие им структурные уровни материи.

В \emph{неживой природе} --- элементарные частицы (включая античастицы) и поля, атомные ядра, атомы, молекулы, агрегаты молекул, макроскопические тела, геологические образования, Землю и другие планеты, Солнце и другие звёзды, местные скопления звёзд, Галактику, системы галактик, Метагалактику. Границы и структура последней \emph{ещё не} установлены.

В \emph{живой природе} существуют \emph{внутреорганизменные} и \emph{надорганизмнные биосистемы}.

\emph{К первым} относятся молекулы ДНК и РНК как носители наследственности, комплексы белковых молекул, клетки (состоящие из подсистем), ткани.

\emph{Далее} --- органы, функциональные системы (нервная, кровеносная, пищеварения, газообмена и др.).

\emph{Наконец}, организм в целом.

\emph{К надорганизменным системам} относятся семейства организмов, колонии, различные популяции --- виды, биоценозы, биогеоценозы, географические ландшафты и вся биосфера.

\emph{В обществе} также существует большое количество типов взаимопересекающихся систем: человек, семья, различные коллективные образования (производственные, учебные, научные, спортивные и др.), сообщества людей, объединения и организации, социальные группы и слои, государства, системы государств и земное сообщество людей в целом.

\emph{Эта классификация} является весьма общей и \emph{далеко не полной}, так как на каждом структурном уровне можно выделить дополнительно большое количество взаимопроникающих материальных систем, возникающих на основе различных форм связей и взаимодействий элементов.

Факторы, определяющие целостность систем, \emph{непрерывно усложняются} по мере восходящего развития материи.

В неживой природе целостность систем \emph{определяется} ядерными (в атомных ядрах), электромагнитными и гравитационными силами связи.

Система будет \emph{целостной} в том случае, если энергия взаимодействия между её элементами \emph{больше} суммарной кинетической энергии этих элементов совместно с энергией внешних воздействий, направленных на разрушение системы. \emph{В противном случае} (если меньше) система не возникает либо распадается.

В живой природе дополнительно к этим факторам целостность определяется \emph{информационными процессами} связи и управления, саморегуляции и воспроизводства биосистем на разных структурных уровнях.

\emph{Целостность социальных систем} определяется многочисленными социальными связями и отношениями (экономическими, политическими, культурными, семейными и т.д.).

\emph{Классификация основных форм материи по типам материальных систем и соответствующих им структурных уровней материи является наиболее точной и детализированной}.

Наряду с этим \emph{распространена} классификация форм материи по ряду фундаментальных физических свойств.

Так, прежде всего, выделяется \emph{вещество} --- совокупность частиц, макроскопических тел и других систем, обладающих определённой массой покоя.

Реально существует \emph{также антивещество}, состоящее из античастиц (антипротонов, позитронов, антинейтронов и др.), которое иногда \emph{неправильно называются антиматерией}.

Атомы и молекулы из античастиц, при отсутствии обычных форм вещества, \emph{могут быть устойчивыми} и образовывать макроскопические \emph{тела,} и даже космические системы («антимир»).

В этом «\emph{антимире}» законы движения и развития материи («\emph{антиматерии}») будут аналогичными тем, которые проявляются в окружающем нас мире.

Кроме того, существуют \emph{невещественные виды материи} --- электромагнитные и гравитационные поля, а также нейтрино и антинейтрино различных типов.

Невещественные виды материи \emph{не обладают} конечной массой покоя.

Следует отметить, что \emph{поле и вещество нельзя противопоставлять}, так как поля существуют в структуре всех вещественных систем и объединяют их элементы в целостность.

\emph{Учение диалектико-материалистической философии о материи и формах её существования представляет собой определённое методологическое обоснование для конкретно-научных исследований}, для разработки целостного научного мировоззрения и соответствующей действительности трактовки открытий науки.

При этом это учение само \emph{постоянно совершенствуется и углубляется} с прогрессом научного знания, формулируются новые категории и принципы, всё более полно отражающие действительность, которая всегда будет сложнее всех наших даже самых, на первый взгляд, совершенных представлений о ней.

\subsection{Движение и его основные формы}

Познавая окружающий мир, мы видим, что в нём \emph{нет ничего абсолютно застывшего} и неизменного, всё находится в движении, переходит из одних форм в другие.

Во всех материальных объектах \emph{происходит движение} элементарных частиц, атомов, молекул, каждый объект взаимодействует с окружающей средой, а это взаимодействие заключает в себе движение того или иного рода.

\emph{Любое тело}, покоящееся по отношению к Земле, \emph{движется} вместе с ней вокруг Солнца, вместе с Солнцем --- по отношению к другим звёздам Галактики, а она, в свою очередь, перемещается относительно других звёздных систем и т.д.

\emph{Абсолютного покоя}, равновесия и неподвижности \emph{нигде нет}, всякий покой, равновесие относительны, являются определёнными состояниями движения.

Стабильность структуры и внешней формы тел \emph{обусловлена} определённым взаимодействием между составляющими их микроскопическими частицами.

Всякое взаимодействие, развёртывающееся в пространстве и времени, выступает \emph{как движение}.

Равным образом и любое движение \emph{включает в себя} взаимодействие различных элементов материи.

Взятое в самом общем виде, движение оказывается тождественным всякому \emph{изменению}, любому переходу из одного состояния в другое.

\emph{Движение -- это всеобщий атрибут, способ существования материи.}

\emph{В мире нет и не может быть материи без движения, как нет и движения без материи}.

Это важное положение \emph{можно, в частности, обосновать} методом рассуждения от противного.

\emph{Предположим}, что существует некая форма материи, \emph{лишённая} всякого движения, как внутреннего, так и внешнего.

Поскольку \emph{движение равнозначно взаимодействию}, то эта гипотетическая материя \emph{должна быть лишена} всех внутренних и внешних связей и взаимодействий. Но \emph{в таком случае} она должна быть бесструктурной, не заключать в себе никаких элементов, ибо последние из-за отсутствия способности к взаимодействиям не могли бы объединиться друг с другом и образовать данную форму материи.

Из этой \emph{гипотетической материи} в свою очередь не может ничего возникнуть, поскольку она лишена связей и взаимодействий. Она ни в чём \emph{не могла бы обнаруживать} своего существования по отношению ко всем другим телам, ибо она оказала бы на них влияние, а этого не может быть.

Она не обладала бы \emph{никакими свойствами}, поскольку всякое свойство представляет собой результат внутренних и внешних связей и взаимодействий и также раскрывается во взаимодействиях.

Наконец, она была бы \emph{принципиально непознаваема} для нас, поскольку всякое познание внешних предметов осуществимо лишь при их воздействии на наши органы чувств и приборы. У нас не было бы \emph{никаких оснований} допускать существование такой материи, поскольку от неё не поступало бы никакой информации.

Суммируя все эти негативные «\emph{признаки отсутствия»}, мы получаем \emph{чистое нечто}, некоторую фикцию, которой абсолютно ничто не соответствует в действительности.

\emph{Следовательно}, если любые возможные объекты внешнего мира обладают некоторыми свойствами, структурой, обнаруживают своё существование по отношению к другим телам и могут быть в принципе доступны познанию, то \emph{всё это --- результат} внутренне присущего им движения и взаимодействия с окружением.

Будучи неразрывно связанной с движением, обладая внутренней активностью, \emph{материя не нуждается} ни в каком внешнем божественном толчке для того, чтобы быть приведённой в движение.

Но именно такой метафизической концепции «\emph{первотолчка}» придерживались в своё время некоторые философы-метафизики, рассматривавшие материю как косную, инертную массу.

\emph{Материя является субстратом всех изменений в мире.}

\emph{Движения, оторванного от материи, не существует}, как нет и энергии без материи.

Возможность существования \emph{движения без субстрата}, перехода материи в энергию допускали представители \emph{энергетизма} (прежде всего, немецкий естествоиспытатель \emph{В. Оствальд}, взгляды которого подверг критике \emph{В.И. Ленин} в книге «\emph{Материализм и эмпириокритицизм»}).

Они отождествляли массу и материю, \emph{после чего делался вывод} о тождественности материи энергии.

В духе энергетизма рассуждают некоторые современные учёные, которые на основании формулы \emph{Е=mc\textsuperscript{2}} (\emph{Е} --- энергия, \emph{М} --- масса, \emph{с} --- скорость света) делают вывод об эквивалентности материи и энергии.

Превращение частиц и античастиц (при их взаимодействии) в фотоны они рассматривают как уничтожение («\emph{аннилиляцию»}) материи, преобразование её в «\emph{чистую энергию»}.

На самом деле кванты электромагнитного поля (\emph{фотоны}) представляют собой особую форму движущейся материи.

Происходит \emph{не уничтожение материи, а переход} её из одной формы в другую при строгом выполнении законов сохранения массы, энергии, электрического заряда, импульса, момента импульса и некоторых других свойств микрочастиц.

\emph{Энергия вообще не может существовать отдельно от материи}, она всегда выступает в качестве одного из важнейших \emph{свойств} материи.

\emph{Энергия} --- \emph{это количественная мера движения}, выражающая внутреннюю активность материи, способность материальных систем к совершению определённой работы или преобразованиям во внешней среде на основе внутренних структурных изменений.

Энергия из связанного состояния (соответствующего массе покоя) \emph{переходит в активные формы}, например, в энергию излучения.

В природе имеется \emph{бесчисленное множество} качественно различных материальных систем, и каждая из них обладает специфическим для неё движением.

Современной науке \emph{известна лишь небольшая} часть этих движений, которые можно подразделить на \emph{ряд основных форм движения}. К ним относятся \emph{способы} существования и функционирования материальных систем на соответствующих структурных уровнях.

Основные формы движения включают в себя такие группы процессов, которые подчиняются общим законам (различных для разных форм движения).

\emph{Ф. Энгельс} в своём произведении «\emph{Диалектика природы}» отмечал существование нескольких форм движения. Это:

\begin{enumerate}
\item \emph{механическая форма движения} (пространственного перемещения);
\item \emph{физическая} (электромагнетизм, гравитация, теплота, звук, изменения агрегатных состояний вещества и др.);
\item \emph{химическая} (превращение атомов и молекул веществ);
\item \emph{биологическая} (обмен веществ в живых организмах) и
\item \emph{социальная} (общественные изменения, а также процессы мышления).
\end{enumerate}

Эта классификация сохраняет своё значение и сейчас. Она исходит из принципа исторического развития материи и качественной \emph{несводимости высших форм движения к низшим формам}.

За истёкшие более чем сто лет наукой было открыто много новых форм движения в \emph{микро-} и \emph{мегамире}: движения и превращения элементарных частиц, процессы в атомных ядрах, в звёздах, в сверхплотных состояниях вещества, расширение Метагалактики и т.д.

В настоящее время из основных форм движения можно выделить прежде всего такие, которые проявляются \emph{во всех} известных пространственных масштабах и структурных уровнях материи.

К ним относятся: \emph{1)} \emph{пространственные перемещения} --- механическое движение атомов, молекул, макроскопических и космических тел; распространение электромагнитных и гравитационных волн (бестраекторное); движение элементарных частиц; \emph{2) электромагнитное взаимодействие}; \emph{3) гравитационное взаимодействие} (тяготение).

Далее необходимо выделить формы движения, проявляющиеся лишь

\emph{на определённых} \emph{структурных уровнях} в неживой природе, в живой природе и в обществе.

\emph{В неживой природе} --- это, прежде всего, взаимодействия и превращения элементарных частиц и атомных ядер.

\emph{Частным проявлением} данной формы движения выступают все виды ядерной энергии.

В \emph{результате перераспределения} связей между атомами в молекулах, изменения структуры молекул одни вещества превращаются в другие. Этот процесс составляет \emph{химическую форму} движения.

Следует указать на \emph{формы движения макроскопических тел}: теплота, процессы кристаллизации, изменения агрегатных состояний, структурные изменения в твёрдых телах, жидкостях, газах и плазме.

\emph{Геологическая форма движения} включает в себя комплекс физико-химических процессов, связанных с образованием всевозможных минералов, руд и других веществ в условиях больших температур и давлений.

\emph{В звёздах} проявляются такие формы движения, как самоподдерживающиеся термоядерные реакции, образование химических элементов (особенно при вспышках новых и сверхновых звезд).

При особенно больших массах и плотностях космических объектов возможны процессы типа \emph{гравитационного коллапса} и перехода системы в \emph{сверхплотное состояние}, когда её поле тяготения уже не выпускает наружу частицы вещества и электромагнитное излучение (так называемые «чёрные дыры»).

\emph{В масштабах мегамира} мы являемся свидетелями грандиозного \emph{расширения Метагалактики}, которое, по-видимому, является отдельным этапом формы движения этой гигантской материальной системы.

На каждом структурном уровне материи проявляются \emph{свои формы} движения и функционирования соответствующих материальных систем.

\emph{Формы движения в живой природе} включают в себя процессы, происходящие как внутри живых организмов, так и в надорганизменных системах.

\emph{Жизнь представляет собой способ существования белковых тел и нуклеиновых кислот}, содержанием которого является непрерывный обмен веществ между организмом и окружающей средой, процессы отражения и саморегуляции, направленные на самосохранение и воспроизводство организмов.

Все живые организмы представляют собой \emph{открытые системы}.

Постоянно обмениваясь веществом и энергией с окружающей средой, живой организм \emph{непрерывно воссоздаёт} свою структуру и функции, поддерживает их относительно стабильными. \emph{Обмен веществ} приводит к постоянному самообновлению клеточного состава тканей.

\emph{Жизнь представляет собой систему форм движения} и включает в себя процессы взаимодействия, изменения и развития в надорганизменных биологических системах --- колониях организмов, видов, биоценозах, биогеоценозах и всей биосфере.

\emph{Высшим этапом развития материи на Земле является человеческое общество} с присущими ему \emph{социальными формами движения}.

Эти формы движения непрерывно усложняются с \emph{прогрессом общества}. Они включают в себя всевозможные проявления \emph{целенаправленной деятельности} людей, все формы социальных изменений и виды взаимодействия между различными общественными системами --- от человека до государства и общества в целом.

Проявлению социальных форм движения служат и процессы \emph{отражения действительности в мышлении}, которые основываются на \emph{синтезе всех} физико-химических и биологических форм движения в мозгу человека.

Между всеми формами движения материи существует \emph{тесная взаимосвязь,} которая \emph{обнаруживается}, прежде всего, в историческом развитии материи и в возникновении высших форм движения на основе относительно низших.

Высшие формы движения \emph{синтезируют} в себе относительно низшие.

Так, \emph{человеческий организм} функционирует на основе взаимодействия физико-химических и биологических форм движения, находящихся в нём в неразрывном единстве, и одновременно человек проявляет себя как \emph{личность} --- носитель социальных форм движения.

При изучении взаимоотношения форм движения материи \emph{важно избегать} как отрыва высших форм от низших, так и механического сведения первых к последним.

\emph{Отрывая} высшие формы от низших, нельзя объяснить их происхождение и структурные особенности.

\emph{Игнорирование специфики} высших форм движения и грубое сведение их к низшим формам ведёт к механицизму и недопустимым упрощениям.

Познание взаимоотношения форм движения имеет \emph{большое методологическое значение} для раскрытия материального единства мира, особенностей исторического развития материи.

Процесс познания материи в значительной мере \emph{совпадает} с исследованием форм её движения, и если бы мы познали полностью движение, мы познали бы и материю во всех её проявлениях.

Но этот процесс \emph{бесконечен}.

Выяснение законов взаимоотношения форм движения материи \emph{важно} для познания сущности жизни и других высших форм, \emph{для моделирования} функций сложных систем, включая и мозг человека, на всё более сложных технических системах.

Прогресс науки и техники в этом направлении \emph{открывает} необъятные перспективы.

\subsection{Пространство и время}

Все окружающие нас предметы обладают определёнными \emph{размерами}, протяженностью в различных направлениях, перемещаются относительно друг друга или вместе с Землей --- по отношению к космическим телам.

Точно так же \emph{все объекты возникают и изменяются во времени}.

\emph{Пространство и время являются всеобщими формами бытия всех материальных систем и процессов.}

Не существует объекта, который находился бы вне пространства и времени, как и нет пространства и времени самих по себе вне движущейся материи.

Мы часто понимаем пространство и время как \emph{всеобщие условия} сущевтвования тел. Такой подход не приводит к ошибкам, пока рассматриваются тела и системы конечного порядка.

Каждая конкретная вещь существует и движется в пространственной структуре некоторой ещё большей по своим размерам системы --- Галактики, скоплений галактик и т.д.

Возникновение и весь цикл развития малой системы проявляется как определённый отрезок во времени развития большой системы, в которую она входит. Пространство и время последней выступают в качестве условий развития входящих в неё \emph{подсистем.}

Но понимание пространства и времени как условий бытия становится неправомерным, когда мы переходим к рассмотрению материи в целом. Ведь в таком случае необходимо было бы признать, что помимо материи реально существуют ещё пространство и время, в которые как-то «\emph{погружена»} материя.

В прошлом подобный подход приводил к \emph{концепции абсолютного пространства и времени} как внешних условий бытия материи (\emph{И. Ньютон}). Пространство рассматривалось как бесконечная пустая протяжённость, вмещающая в себя все тела и не зависящая от материи.

Абсолютное время в этой концепции рассматривалось как \emph{равномерный поток длительности}, в которой все возникает и исчезает, но которая сама не зависит ни от каких процессов в мире.

Развитие науки \emph{опровергло} эти представления. Никакого абсолютного пространства как бесконечной пустой протяженности \emph{не существует}. Всюду имеется материя в тех или иных \emph{формах} (вещество, поле и т.д.), а пространство выступает как \emph{всеобщее свойство} (\emph{атрибут}) материи.

Пространство и время существуют объективно и независимо от сознания, но вовсе не от материи.

\emph{Пространство --- это такая форма бытия материи, которая выражает её протяженность и структурность, сосуществование (рядоположенность) и взаимодействие элементов в различных материальных системах.}

\emph{Время --- это форма бытия материи («или атрибут»), характеризующая длительность существования всех объектов и последовательность смены состояний.}

Все свойства пространства и времени \emph{зависят от} движения и структурных отношений в материальных системах и должны выводиться из них.

Из свойств пространства и времени можно выделить \emph{всеобщие}, проявляющиеся на всех известных структурных уровнях материи, и \emph{частные}, а также \emph{особенные}, присущие лишь некоторым состояниям материи, и даже отдельным объектам.

\emph{Всеобщие свойства} неразрывно связаны с другими атрибутами материи и диалектическими законами её бытия. Они представляют для философии, её онтологической составляющей, первостепенный интерес.

\emph{К всеобщим свойствам пространства материи} относится прежде всего \emph{протяженность}, означающая рядоположенность различных элементов (отрезков, объёмов), возможность прибавления к каждому данному элементу некоторого следующего либо уменьшения числа элементов. Пространство без протяженности исключало бы возможность количественного изменения её элементов, а также структурность материальных образований.

Именно благодаря тому, что в материальных системах имеют место сосуществующие и взаимодействующие элементы, внутреннее пространство таких систем протяженно.

Таким образом, протяженность \emph{органически связана} со структурностью систем.

К всеобщим свойствам пространства относится его \emph{неразрывная связь со временем и с движением материи}, зависимость от структурных отношений: в материальных системах.

Пространству (точнее, пространственным свойствам материи) присуще \emph{единство прерывности и непрерывности}.

Прерывность относительна и проявляется в \emph{раздельном существовании} материальных объектов и систем, каждая из которых имеет определённые размеры и границы.

Но материальные поля (электромагнитные, гравитационные и др.) непрерывно распределены в пространстве всех систем.

\emph{Непрерывность пространства} проявляется также в пространственном перемещении тел. Тело, движущееся к определённому месту, проходит всю бесконечную последовательность элементов длины между ними. Тем самым пространство обладает \emph{связностью}, в нем отсутствуют «\emph{разрывы}»

Пространству присуща \emph{трёхмерность}, которая органически связана со структурностью систем и их движением.

\emph{Представим себе} материальный объект, размерами которого можно пренебречь (материальная точка). Движение такого объекта даст \emph{линию} --- \emph{одномерную протяжённость}. Перемещение линии в перпендикулярном направлении дает \emph{плоскость} --- \emph{двумерную протяженность}. Движение плоскости даёт \emph{объём} --- \emph{трехмерную протяженность.}

Если же мы будем далее перемещать любым возможным способом объём, то он не перейдет в пространство большего числа измерений. Три измерения оказываются тем \emph{необходимым и достаточным минимумом}, в котором реализуются все движения и взаимодействия материальных объектов.

\emph{В теории относительности} формулируется понятие \emph{четырехмерного континуума}. Но здесь в качестве \emph{четвертого измерения} добавляется время, само же пространство считается трёхмерным.

С протяжённостью пространства тесно связаны \emph{метрические отношения}, которые выражают структуру связи пространственных элементов, порядок и количественные законы этих связей.

Метрические отношения на плоскости, сфере, псевдосфере (фигуре, напоминающей грамофонную трубу) и других поверхностях отражаются в различных \emph{типах геометрии --- евклидовой и неевклидовой} (\emph{Лобачевского}, \emph{Римана}).

Наличие определённых метрических свойств у пространства относится к числу его всеобщих характеристик.

Из \emph{всеобщих свойств} времени (точнее, \emph{временных отношений} в материальных системах) следует отметить его неразрывную связь с пространством и движением материи, длительность, асимметрию, необратимость, нецикличность, единство прерывности и непрерывности, связность, зависимость от структурных отношений в материальных системах.

\emph{Длительность} выступает как последовательность существования материальных объектов, их сохранение, в относительно устойчивой форме. Длительность образуется путём возникновения одного момента времени за другим благодаря конечности скорости изменения любых процессов. Она аналогична протяженности пространства и является следствием сохранения материи и движения.

Сохранение материи и движения обусловливает также связность времени, отсутствие разрывов в нём, общую и абсолютную \emph{непрерывность}.

\emph{Прерывность} характеризует лишь время существования конкретных качественных состояний материи, каждое из которых возникает и исчезает, переходя в другие формы. Но составляющие их элементы материи (например, элементарные частицы) могут при этом не возникать и не исчезать, а только менять формы связей, образуя различные тела.

В этом смысле \emph{прерывность времени} существования материи \emph{относительна}, а \emph{непрерывность абсолютна}. Выражением этого факта являются законы сохранения материи и её важнейших свойств.

\emph{Асимметрия} или \emph{однонаправленность} времени означает его изменение \emph{только от прошлого к будущему}, необратимость этого изменения.

В пространстве можно двигаться в любом направлении. Во времени же \emph{движение в прошлое невозможно}, всякое изменение происходит таким образом, что наступают следующие, будущие моменты времени.

Невозможно и абсолютно полное повторение пройденных состояний или циклов изменения. \emph{Всякая цикличность относительна}, выражает большую или меньшую повторяемость процессов. Но в каждом из циклов всегда есть нечто новое, в силу чего и время всегда необратимо.

\emph{Необратимость времени} определяется асимметрией причинно-следственных отношений, общей необратимостью процесса развития материи, в котором всегда появляются новые возможности, качественные состояния и тенденции изменения.

Развитие науки за последние десятилетия \emph{пролило новый свет} на связь сеойств пространства и времени с материальными процессами.

\emph{Теория относительности обосновала}, что с возрастанием скорости движения тел относительно уменьшаются их размеры в направлении движения, в них происходит замедление всех процессов (по сравнению с таковыми в состоянии относительного покоя).

Замедление процессов происходит также под действием очень мощных гравитационных полей, создавемых большими скоплениями вещества. В результате этого спектральные линии излучения, испускаемого белыми карликами и квазарами, оказываются \emph{смешенными в красную сторону спектра}.

Под воздействием полей тяготения происходит \emph{своеобразное «искривление пространства»,} что проявляется в эффектах искривления световых лучей в гравитационных полях. Если масса и плотность системы достигают достаточно больших значений, то метрика её пространства изменяется столь сильно, что световые лучи начинают двигаться в ближайшей окрестности \emph{по замкнутым линиям}. Такой эффект был бы, например, если бы вся масса нашего Солнца была сконцентрирована в шаре диаметром в 2.5 км.

За последние годы в Галактике были обнаружены подобные объекты, возникшие в результате \emph{гравитационного коллапса} (катастрофически быстрого сжатия вещества).

Вначале полагали, что данные объекты (предсказанные в теории и названные «\emph{чёрными дырами}») являются абсолютно замкнутыми, поскольку не выпускают излучения. Но затем стало ясно, что они создают статическое гравитационное поле и поглощают из окружающего пространства межзвездную пыль и газ. При падении на данный сверхплотный объект частицы вещества постоянно сталкиваются между собой, в результате чего возникает \emph{мощное рентгеновское излучение}, которое и было зарегистрировано земными приборами.

\emph{Это ещё раз доказало}, что нет никаких оснований говорить о существовании абсолютно замкнутых в пространстве систем. Во всяком случае, такие системы никак не обнаруживали бы своего существования по отношению к другим телам, и о них мы не могли бы иметь никакой информации. Но тогда исчезают все основания утверждать, что они вообще существуют.

К числv всеобщих свойств пространства и времени относится их \emph{бесконечность}.

Поскольку материя абсолютна, несотворима и неуничтожима, она существует вечно, а эта вечность --- не что иное, как бесконечность любых интервалов времени (лет, тысячелетий и т.д.).

Всякое \emph{допущение конечности} времени неизбежно ведут к религиозным выводам о сотворении мира и времени богом, что полностью неприемлемо для современной науки и практики.

Бесконечность времени \emph{нельзя понимать} как неограниченное монотонное существование одних и тех же форм и состояний.

Материя всегда находилась и будет находиться в неугасающем саморазвитии, которое включает в себя бесконечное возникновение качественно новых форм, состояний, тенденций и законов изменения.

Бесконечность времени имеет не только количественный (\emph{неограниченная длительность}), но и качественный аспект, связанный с историческим развитием материи и её \emph{структурной неисчерпаемостью}.

Материя бесконечна в своих пространственных формах бытия.

Из теоретических принципов космологии и современных наблюдательных данных следует, что пространство непосредственно окружающей нас области Вселенной имеет \emph{отрицательную кривизну} и незамкнуто. Спектральные линии всех галактик смещены в красную сторону спектра, что свидетельствует об их взаимном удалении друг от друга. Скорость удаления возрастает с растоянием и достигает у наиболее отдаленных наблюдаемых объектов половины скорости света.

Есть основания считать, что данное расширение является локальным процессом и \emph{во Вселенной кроме нашей Метагалактики существует бесчисленное множество других космических систем} с самыми различными формами структурной организации и пространственно-временными свойствами.

Бесконечность пространства также имеет свои качественные аспекты, связанные со \emph{структурной неоднородностью материи}.

Процесс познания материального мира включает в себя в качестве важной составляющей части исследование пространственно-временных свойств и отношений тел.

Наряду с рассмотренными всеобщими свойствами пространства и времени первостепенное значение имеет познание \emph{свойств частных}.

\emph{К ним относятся} конкретные пространственные формы и размеры материальных систем, продолжительность их существования в единицах земного времени, ритм процессов в системах, метрические свойства, наличие симметрии или асимметрии в структуре системы, отношения пространственного подобия и т.д.

Все эти свойства \emph{производны} от движения и взаимодействия материи.

Исследование пространственно-временных отношений осуществляется в той или иной форме почти \emph{всеми науками}.

Так, \emph{в биологии} на первый план выдвигаются проблемы ритмов в различных подсистемах живых организмов («\emph{биологические часы}»), асимметрии в пространственной структуре молекул живого вещества.

\emph{В общественной жизни} мы наблюдаем ускорение темпов развития, в одну и ту же единицу физического времени сейчас укладывается всё большее количество научно-технических открытий, социальных изменений.

\subsection{Единство мира}

\emph{В мире не существует ничего, что} не было бы определённым состоянием материи, её свойством, формой движения, продуктом её исторического развития, что не было бы обусловлено в конечном счете материальными причинами и взаимодействиями.

\emph{Сам человек} --- это наиболее сложная из всех известных материальных систем, и все проявления его деятельности, включая высшие формы идеального отражения и творчества, имеют материальное происхождение и обусловлены социальными отношениями.

\emph{Осознание материального един}ства мира явилось результатом тысячелетнего развития науки и практики.

Когда-то было весьма распространено противопоставление \emph{земного и небесного миров}. В последний помещали всех небожителей. Он считался вечным и нетленным, в отличие от бренной материи.

Развитие астрономии, физики и других наук \emph{опровергло эти верования}.

Были познаны \emph{законы движения} планет и других космических тел, исследован их химический состав.

Оказалось, что наиболее распространенным веществом в космосе является \emph{водород}, на долю которого приходится более \emph{98\%} массы всех звёзд и галактик. Из оставшихся двух процентов примерно \emph{1 \%} приходится на гелий и \emph{1 \%} --- на все остальные элементы. (Иное соотношение элементов на Земле и других планетах Солнечной системы объясняется особыми условиями их образования и развития).

Законы движения материи, обнаруженные в земных условиях, проявляются и в космосе.

На основе развития физики и химии \emph{удалось достоверно предсказать} такие состояния материи, которые отсутствуют на Земле и в Солнечной системе, --- сверхплотные состояния вещества, нейтронные звезды, объяснить в общих чертах природу энергии звёзд, этапы их эволюции.

Мощный \emph{процесс интеграции наук} способствует формированию \emph{единой естественнонаучной картины мира} как движущейся и развивающейся материи.

\emph{Представители религиозно-идеалистической философии} всегда выводили единство мира из направляющей \emph{божественной воли}. По их мнению, бог создал этот мир и является его последней сущностью (субстанцией). Он предопределил всемирную связь и развитие всех явлений.

Это понимание единства мира является исходным в \emph{современном неотомизме}. Объективная реальность и существование материи здесь не отрицаются, но рассматриваются как \emph{вторичная реальность} по отношению к высшей реальности --- богу.

В системе объективного идеализма \emph{Гегеля} единство мира понималось в том смысле, что все явления в мире представляют \emph{форму инобытия} саморазвивающегося абсолютного духа, под которым понимался божественный мировой разум.

Но религиозно-идеалистическое понимание мира не продвигало познаний ни на шаг вперед, так как оно \emph{одно неизвестное сводило к другому}, ещё более неизвестному --- божественной воле, абсолютному духу и т.п.

\emph{Реалистически мыслящие ученые} никогда не удовлетворялись таким «объяснением» и стремились раскрыть естественные материальные причины всех явлений, вывести их из объективных законов природы. В результате этого \emph{получили мощное развитие естественные науки}, в которых последовательно раскрывалось материальное единство мира и есетственная детерминированность всех явлений.

В трудах выдающихся материалистов прошлого --- \emph{Демокрита, Эпикура, Лукреция Кара, Френсиса Бэкона, Д. Дидро, Л. Фейербаха и др.} было глубоко разработано учение о материальном единстве мира, его вечном изменении и развитии, о естетсвенном происхождении всего живого на Земле и человеческого общества.

Правда, эти мыслители \emph{не могли последовательно-материалистически объяснить} движущие силы и законы развития общества, сводя их к идеальным побуждениям людей.

Этот недостаток предшествующего материализма попыталась преодолеть диалектико-материалистическая философия, в которой проводится линия на \emph{последовательно материалистически монистическое объяснение} сущности природных и социальных явлений.

\emph{Общество как высший продукт развития природы} представляет собой социально организованную форму материи. Его развитие определяется материальными связями и отношениями: взаимодействием с природой, прогрессом способа производства материальных ценностей, совершенствованием материальной и духовной культуры, развитием материальных средств связи и коммуникаций (торговли, транспорта, печати, средств массовой информации и др.).

Но и \emph{высшие духовные ценности} оказывают также воздействие на прогресс общества.

Достижения науки, политические взгляды, моральные и эстетические принципы, \emph{овладевая сознанием многих людей}, воплощаются в материальные ценности --- новые средства производства, предметы быта, экспериментально-измерительное оборудование, материальные средства управления производством, произведения искусства.

\emph{Диалектико-материалистический монизм} даёт естественное и целостное объяснение природы и общества, служит методологической основой для поиска и раскрытия сущности всех новых, неизвестных ранее явлений.

Единство мира \emph{нельзя сводить} к однородности физико-химического состава либо к подчинению всех явлений одним и тем же известным физическим законам.

В силу действия всеобщего закона перехода количественных изменений в качественные каждое конкретное качество существует \emph{в определённых границах меры}, в конечных пространственно-временных масштабах. Его нельзя экстраполировать, переносить, распространять на бесконенность.

Поэтому и всякая конкретная научная теория имеет \emph{ограниченную сферу применимости}.

\emph{Истина всегда конкретна.}

Любая научная теория неизбежно является \emph{незамкнутой системой} знаний.

\emph{Материя бесконенно разнообразна} в своих проявлениях.

С изменением (увеличением или уменьшением) пространственно-временных масштабов на определенных этапах неизбежно происходят качественные изменения в частных свойствах, формах структурной организации, законов движения материи.

Многие законы микромира качественно отличны от законов макроскопических явлений, а в гигантских масштабах Вселенной существуют особые, необычные процессы и состояния материи, теория которых ещё не создана.

\emph{И всё же}, несмотря на все качественные различия и структурную неисчерпаемость материи, \emph{мир един}.

Это единство проявляется в глобальных масштабах в абсолютности, субстанциональности и вечности материи и её атрибутов; во взаимной связи и обусловленности всех материальных систем и структурных уровней, в естественной детерминированности их свойств, в многообразных взаимных превращениях форм движущая материи, \emph{в соответствии со всеобщим законами сохранения материи} и её основных свойств.

Единство мира проявляется также и в \emph{историческом развитии материи}, в возникновении более сложных форм материи и движения на базе относительно менее сложных форм.

\emph{Наконец}, оно находит выражение в действии универсальных диалектических закономерностей бытия, проявляющихся в структуре и развитии всех материальных систем.

\emph{Локальными проявлениями} единства мира выступает \emph{однородность} физико-химического состава тел, общность их количественных законов движения, сходство в структуре и функциях систем, подобие свойств, делающее возможным \emph{моделирование} сложных систем и процессов на основе более простых явлений с целью получения новой информации о мире.

Учение диалектико-материалистической философии о материи и формах её существования представляет \emph{фундамент} довольно целостного монистического мировоззрения.

Оно имеет вполне \emph{заслуженное методологическое значение} для современной науки, способствует интеграции отдельных наук и выработке \emph{целостного понимания мира} как движущейся и развивающейся материи.

\section{Сознание как свойство высокоорганизованной материи}

Человек владеет \emph{самым прекрасным даром} --- сознанием, мыслящим разумом с его способностью устремляться и в отдаленное прошлое, и в грядущее, проникать в область неведомого, с его миром мечты и творческой фантазии.

\emph{Что же такое сознание}, как оно возникло, как оно соотносится с действительностью и мозгом человека?

\subsection{Сознание --- функция человеческого мозга}

Над \emph{тайной своего сознания} человек, надо полагать, начал задумываться ещё в глубокой древности.

Лучшие умы человечества в течение многих веков пытались раскрыть природу сознания. Они искали ответ на вопросы о том, \emph{как} неживая материя на определённом уровне своего развития порождает живую, а последняя --- сознание, \emph{какова} его структура и функции, \emph{каков} механизм переходе от ощущений, восприятий к мысли, от чувственно-конкретного --- к отвлеченно-теоретическому, \emph{каким образом} сознание соотносится с материальными физиологическими процессами, протекающими в коре головного мозга.

Эти и многие другие проблемы в течение длительного времени \emph{оставались недоступными} строго объективному научному исследованию.

В трактовке явлений сознания большое распространение имели различного рода \emph{идеалистические и религиозные концепции}.

Согласно этим представлениям, сознание есть проявление некоей нематериальной субстанции «\emph{души}», будто бы не зависящей от материи вообще, от человеческого мозга в частности, способной вести самостоятельное существование, бессмертной и вечной.

\emph{Не умея объяснить} естественными причинами сновидения, обморок, смерть, разного рода познавательные и эмоционально волевые процессы, древние приходили к ложным взглядам на эти явления.

Так, \emph{сновидения толковались} как впечатления «души», \emph{покидающей во сне тело} и странствующей по различным местам.

\emph{Смерть} представлялась в виде \emph{разновидности сна}, когда «душа» по неведомым причинам не возвращается в покинутое тело.

Эти \emph{наивные, фантастические взгляды} получили в дальнейшем своё «\emph{теоретическое обоснование}» и закрепление в различных идеалистических философских и богословских системах.

Любая идеалистическая система так или иначе провозглашала сознание (разум, идею, дух) самостоятельной \emph{сверхъестественной сущностью}, не только не зависящей от материи, но и, более того, создающей весь мир и управляющей его движением, развитием.

В противоположность этому \emph{материалистическая философия} исходит из того, что \emph{сознание есть функция человеческого мозга, сущность которой заключается в активном, целенапрвленном отражении действительности}.

Вместе с тем \emph{проблема сознания} оказалась чрезвычайно трудной и для материалистически мыслящих философов и психологов.

Некоторые материалисты, оказавшись в затруднении перед проблемой возникновения сознания, стали рассматривать \emph{сознание в качестве атрибута материи}, как её вечное свойство, присущее всем --- как высшим, так и низшим --- её формам. Они объявили всю материю одушевленной.

Это воззрение именуется \emph{гилозоизмом} (от греческого \emph{hyle} --- материя, \emph{zoe} --- жизнь).

Диалектико-материалистическая философия исходит из того, что сознание является свойством не всякой, а \emph{лишь высокоорганизованной материи,} оно связано с деятельностью человеческого мозга, со специфически человеческим, социальным образом жизни.

Сознание никогда не может быть чем-либо иным, как \emph{осознанным бытием}, а бытие людей еcть реальный процесс их жизни.

Диалектико-материалистическая концепция сознания основывается на \emph{принципе отражения}, а именно, психического воспроизведения объекта, его содержания в мозгу человека в виде ощущений, восприятий, представлений, понятий, теорий.

Содержание сознания определяется в конечном счете окружающей действительностью, а его \emph{материальным субстратом}, носителем служит \emph{мозг человека}.

У \emph{животных} в ходе эволюции появилась \emph{способность психического отражения} внешних воздействий лишь тогда, когда у них возникла \emph{нервная система}.

Совершенствование психики животных под влиянием \emph{изменения образа жизни} тесно связано с развитием мозга.

Сознание человека возникло и развивается в тесной связи с возникновением и развитием специфически человеческого мозга \emph{под влиянием трудовой деятельности}, общественных отношений, общения.

\emph{Мозг есть орган сознания} как высшей формы психического отражения действительности.

Человеческий мозг --- это \emph{тончайший нервный аппарат}, состоящий из громадного числа нервных клеток. Их насчитывается \emph{до 15 миллиардов}. Каждая из них находится в контакте с другими, и все они вместе с нервными окончаниями органов чувств образуют сложнейшую сеть с неисчислимым \emph{множеством связей}.

Мозг человека имеет чрезвычайно сложное \emph{«иерархическое» строение}.

\emph{Наиболее простые} формы отражения, анализа и синтеза внешних воздействий и регуляции поведения осуществляются \emph{низшими отделами} центральной нервной системы --- спинным, продолговатым, средним и межуточным мозгом (появилась также \emph{гипотеза о «брюшном» мозге}).

\emph{Наиболее сложные} формы осуществляются \emph{высшими «этажами»}, и прежде всего большими полушариями головного мозга. В различные участки коры больших полушарий по нервным волокнам поступают нервные импульсы, вызванные воздействием внешних агентов, образований на органы чувств.

\emph{«Подкорковый» аппарат} мозга является органом сложнейших форм передающейся по наследству, врожденной (инстинктивной) деятельности. Он выполняет самостоятельную роль у низших позвоночных животных и теряет свою самостоятельность у высших позвоночных --- млекопитающих, и особенно у человека.

\emph{Взаимодействие} между организмом и окружающим мирок, а также между отдельными частями организма и его органами обеспечивается с помощью \emph{рефлексов}, т.е. реакций организма, которые вызываются раздражением органов чувств и осуществляются при участии центральной нервной системы.

Рефлексы разделяются на \emph{две основные группы} --- \emph{безусловные} и \emph{условные.}

\emph{Безусловные рефлексы} --- это врожденные, передающиеся по наследству реакции организма на воздействия внешней среды.

\emph{Условные рефлексы} являются приобретенными в процессе жизнедеятельности реакциями организма; их характер зависит от \emph{индивидуального опыта} животного \textsc{или} человека.

\emph{Учение о рефлекторной деятельности} мозга развивалось многими учеными различных стран мира. Большой вклад в него внесли такие русские ученые, как \emph{И.М. Сеченов, И.П. Павлов, Н.Е. Введенский, А.А. Ухтомский, Л.А. Орбели} и др.

Они исходили из представления о \emph{неразрывном единстве} физиологического и психического.

Важное значение для разработки физиологических механизмов психики, сознания имеют разработки, идеи \emph{учёных советского периода}: \emph{П.К. Анохина} --- об интегративной деятельности мозга как целостной функциональной системы, о физиологическом механизме опережающего отражения действительности, \emph{Н.А. Бернштейна} --- о конструировании в процессе деятельности мозга задачи, цели действия.

\emph{Мозг представляет собой исключительно сложную функциональную систему}. И верное функционирование этой системы предполагает объединение данных, полученных при изучении отдельных нервных клеток и при исследовании внешнего поведения человека.

Вне физиологических процессов в мозгу \emph{невозможно} возникновение никакого ощущения, мысли, никакого чувства и побуждения.

Идея о том, что \emph{мозг есть орган мысли}, возникла в глубокой древности и ныне является общепринятой в науке.

\emph{Сознание есть продукт деятельности мозга}, и оно возникает лишь благодаря идущему извне воздействию на мозг через органы чувств.

\emph{Органы чувств} \emph{--- это «аппараты»,} служащие для отражения, для информирования организма об изменениях в окружающей среде или внутри самого организма. Они поэтому являются \emph{специализированными} на внешние и внутренние.

\emph{Внешние органы чувств} --- это зрение, слух, обоняние, вкус и кожная чувствительность. Сигналы, поступающие от органов чувств, несут информацию о свойствах вещей, их отношениях и связях.

\emph{Совокупность органо}в чувств и соответствующих нервных образований \emph{И.П. Павлов} называл \emph{анализаторами}.

Анализ воздействий среды \emph{начинается} в периферической части анализаторов --- \emph{рецепторах} (концевых образованиях нервных волокон), где из всего многообразия воздействующих на организм видов энергии выделяется какой-либо определённый.

Высший и тончайший анализ достигается только при помощи \emph{коры головного мозга}.

Вызванное тем или иным воздействием на органы чувств возбуждение только тогда \emph{порождает ощущение, становится фактом сознания}, когда оно достигает мозга.

Корковые физиологические процессы --- \emph{необходимые} материальные механизмы отражательной психической деятельности, явлений сознания.

\subsection{Сознание --- высшая форма психического отражения объективного мира}

Физиологические механизмы психических явлений не тождественны содержанию психики, которая представляет собой отражение действительности в форме \emph{субъективных, идеальных образов}.

Диалектико-материалистическая философия выступает против примитивного истолкования сути сознания сторонниками так называемого \emph{вульгарного материализма}, сводящего сознание к его материальному субстрату --- физиологическим нервным процессам, протекающим в мозгу.

Отождествление сознания и материи --- \emph{грубая ошибка}.

Подвергая критике вульгарно-материалистические ошибки, следует отметить, что мысль и материя действительно связаны, но \emph{назвать мысль материальной --- значит совершать ошибку} в виде смешения материалистических и идеалистических элементов.

Не менее ошибочной является дуалистическая концепция \emph{психофизического параллелизма}, согласно которой психические и материальные (физиологические) процессы представляют собой абсолютно разнородные сущности, между которыми лежит пропасть.

Сознание --- это \emph{не особая}, отделённая от материи сущность. Однако создаваемый в голове человека образ предмета не сводится к самому материальному объекту, находящемуся вне субъекта, ни к тем физиологическим процессам, которые происходят в мозгу и порождают этот образ.

Мысль, сознание \emph{реальны}. Но это \emph{не объективная реальность}, а нечто субъективное, идеальное.

\emph{Сознание есть субъективный образ объектишого мира.}

Когда мы говорим о субъективности образа, то имеем в виду, что он представляет собой \emph{не искажённое} отражение действительности, а нечто идеальное, т.е. преобразованное, переработанное в голове человека содержание материального объекта.

\emph{Вещь в сознании} человека --- это её \emph{образ}, а сама \emph{реальная вещь} --- его \emph{прообраз}.

Возникновение, функционирование и развитие сознания теснейшим образом связано с \emph{приобретением} человеком \emph{знаний} о тех или иных предметах, явлениях.

«Способ, каким существует сознание и каким нечто существует для него, это --- знание...» (\emph{К. Маркс} и \emph{Ф. Энгельс}, Соч., т. 42, с. 165). Следовательно, сознание невозможно без \emph{познавательного отношения} человека к объективному миру.

Вместе с тем когда мы говорим о сознании, то имеем в виду прежде всего характеристику его как \emph{духовной деятельности}, как идеального явления, качественно отличного от материального.

\emph{Познание --- это деятельность субъекта, направленная на отражение окружающего мира.}

Не всякая психика человека сознательна. Понятие психического значительно \emph{шире} понятия сознания.

Психика есть и \emph{у животных}, но у них отсутствует сознание.

Психическая жизнь свойственна \emph{только что родившемуся ребёнку}, но у него ещё нет сознания.

Когда человек погружается в \emph{сон} и видит причудливые картины --- это психические явления, но это не сознание.

Да и в \emph{бодрствующем состоянии} далеко не все психические процессы у него озарены светом сознания. Человек идёт по улице и о чём-то думает, а в это время он видит, почти или совсем не осознавая, целый калейдоскоп явлений, как-то ориентируется в потоке людей и машин, действуя почти автоматически.

Жизнь требует от человека не только осознанных форм поведения, но и \emph{бессознательных}, освобождающих его от постоянного напряжения сознания там, где в этом напряжении нет необходимости.

Бессознательные формы поведения основаны на \emph{скрытом учёте} информации о свойствах и отношениях вещей. \emph{Диапазон бессознательного} довольно широк. Он охватывает ощущения, восприятия, представления, когда они протекают вне фокуса сознания, а также инстинкты, навыки, интуицию, установку и др.

\emph{Проблема бессознательного} всегда была предметом острых споров.

В настоящее время одним из самых распространенных учений о бессознательном является учение австрийского психиатра \emph{3. Фрейда}. Он исследовал сферу бессознательного, её место и роль в душевных расстройствах. Фрейд утверждал, что сознание определяется бессознательным, которое рассматривалось им как заряжённая высокой энергией совокупность инстинктивных стремлений.

По Фрейду, \emph{структура личности}, её поведение, характер, а также вся человеческая культура определяются в конечном счёте \emph{врожденными} эмоциями людей, их инстинктами, влечениями, ядром которых является \emph{половой инстинкт}.

Диалектико-материалистическая философия \emph{не соглашается} с иррационалистическими, преувеличивающими роль биологических факторов взглядами на душевную жизнь человека и утверждает, что ведущим началом в человеческой личности является разум, сознание.

В отличие от животных, \emph{у нормального человека} господствует сознание, осознанное состояние психики.

Сознание представляет собой \emph{целостную систему} различных, но тесно связанных друг с другом познавательных и эмоционально-волевых элементов.

Исходным чувственным образом, элементарным фактом сознания является \emph{ощущение}, через которое осуществляется непосредственная связь субъекта с объективной реальностью.

\emph{Ощущение --- это отражение отдельных свойств предметов объективного мира во время их непосредственного воздействия на органы чувств}. Оно есть превращение энергии внешнего раздражения в факт сознания. Потеря способности ощущать неизбежно влечёт за собой потерю сознания.

Если ощущения отражают лишь отдельные свойства вещей, то \emph{вещь в целом, в единстве её различных чувственно воспринимаемых свойств отражается в восприятии}.

Восприятие у человека обычно включает в себя \emph{осмысление} предметов, их свойств и отношений.

Характер восприятия зависит от \emph{уровня знаний}, которыми обладает человек, от его интересов.

Процесс чувственного отражения \emph{не ограничивается} ощущениями и восприятиями.

Высшая форма чувственного отражения --- \emph{представление.} Это \emph{образное знание} об объектах, которые воспринимались нами в прошлом, но которые не воздействуют в данный момент на наши органы чувств. Представления возникают в результате восприятия воздействий и их \emph{сохранения затем в памяти}.

Образы, которыми оперирует сознание человека, \emph{не ограничиваются воспроизведением} чувственно воспринятого. Человек может творчески \emph{комбинировать} и относительно свободно создавать новые образы в своём сознании.

\emph{Высшей формой представления является продуктивное творческое воображение}.

Относительная свобода от непосредственного воздействия объекта и обобщение совокупности показаний органов чувств в единый наглядный образ делает представление \emph{важной ступенью} процесса отражения, идущего от ощущений к мышлению.

Диалектико-материалистическая философия \emph{признаёт} качественное различие между этими двумя ступенями, \emph{но не разрывает} их.

Мышление выступает в форме \emph{понятий, суждений} и \emph{умозаключений}, представляет собой \emph{отражение существенных}, закономерных отношений вещей.

Мышлению доступны такие стороны мира, которые недоступны чувственному восприятию. На основании видимого, осязаемого, слышимого и т.п. мы благодаря мыслительной деятельности \emph{проникаем в невидимое}, неосязаемое, неслышимое.

Мышление даёт нам \emph{знание о существенных} свойствах, связях и отношениях.

С помощью мышления мы осуществляем диалектический \emph{переход от внешнего к внутреннему}, \emph{от явлений к сущности} вещей, процессов и т.д.

Будучи высшей формой отражения, отражательной деятельности, мышление вместе с тем \emph{присутствует и на чувственной ступени}: ощущая и воспринимая что-либо, человек \emph{осознаёт} результаты чувственных восприятий.

Сознание есть не только процесс познания и его результат --- знание, но вместе с тем и \emph{переживание} познаваемого, определённая \emph{оценка} вещей, свойств и отношений. Без эмоциональных переживаний, помогающих мобилизовать или сдерживать наши силы, невозможно то или иное отношение к миру.

Движущей «\emph{пружиной»} поведения и сознания людей является \emph{потребность} --- определённая \emph{зависимость} человека от внешнего мира, субъективные \emph{запросы} личности к объективному миру, её \emph{нужда} в таких предметах и условиях, которые необходимы для её нормальной жизнедеятельности, для самоутверждения и развития.

Важной стороной сознания является \emph{самосознание}.

Жизнь требует от человека, чтобы он познавал не только внешний мир, но \emph{и самого себя}.

Отражая объективную реальность, человек осознаёт не только этот процесс, но и самого себя как чувствующее и мыслящее существо, свои идеалы, интересы, нравственный облик. Он выделяет себя из окружающего мира, отдавая себе отчёт в своем отношении к миру, в том, что он чувствует, думает, делает.

\emph{Осознание человеком себя как личности и есть самосознание.}

Самосознание формируется под влиянием \emph{социального образа жизни}, требующего от человека контроля над своими действиями, ответственности за свои поступки.

Сознание имеет не только внутриличностное бытие. Оно \emph{объективируется} и \emph{существует надличностно} --- в открытиях науки, в творениях искусства, в правовых, нравственных нормах и т.д. Все эти проявления общественного сознания, сознания общественного человека --- необходимое условие формирования \emph{личного, индивидуального сознания}.

Личное и общественное сознание находятся \emph{в неразрывном единстве}.

Сознание каждого отдельного человека \emph{вбирает в себя} знания, убеждения, верования, оценки той общественной среды, в которой он живёт.

\emph{Человек --- существо специфически общественное}.

Исторически сложившиеся правила мышления, нормы права, морали, эстетические вкусы и т.д. с самого начала формируют поведение и разум человека, делают из него \emph{представителя определённого образа жизни}, уровня культуры и психологии.

Психические способности и свойства человека складываются в процессе его жизни в обществе и \emph{определяются} конкретными социальными условиями.

Человек становится социальным существом, \emph{поднимается} до уровня личности, до вершин современного мышления лишь в ходе общественного развития.

Исходным принципом диалектико-материалистической философии в трактовке сознания является признание неразрывной \emph{связи сознания и деятельности, практики}.

Сознание и объективный мир --- \emph{противоположности}, образующие \emph{единство}. Основой этого единства является \emph{практика --- активная чувственно-предметная деятельность людей}, выражающаяся в труде, социальных взаимодействиях различных видов, научном эксперименте и т.п.

Именно практика и порождает необходимость отражения действительности в сознании людей.

\emph{Необходимость сознания}, дающего верное отражение мира, лежит, следовательно, в условиях и требованиях самой общественной жизни.

Хотя сознание является функцией мозга, но осознает действительность не мозг сам по себе, а человек, выступаюший как \emph{субъект} преобразовательной деятельности, как субъект истории.

\emph{Сущность сознания} нельзя вскрыть, исходя лишь из анатомо-физиологических свойств мозга. Возникновение, функционирование и развитие сознания возможно \emph{только в обществе}, на основе практической деятельности людей.

Объективный мир, воздействуя на нас, отражается в сознании, \emph{превращается в идеальное}, специфическую информационную модель.

В свою очередь сознание, идеальное посредством практической деятельности \emph{претворяется} в действительность, в реальное.

Сознание характеризуется \emph{активо-творческим отношением} к внешнему миру, к самому себе, к человеческой деятельности.

\emph{Активность сознания} проявляется в том, что человек отражет внешний мир целенаправленно, избирательно. Он \emph{воспроизводит} в своей голове предметы и явления сквозь призму уже сложившихся у него знаний --- представлений, понятий.

Действительность воссоздается в сознании человека \emph{не в зеркально} мёртвом, а в творчески преобразованном виде.

Сознание способно создавать образы, опережающие действительность. Оно обладает \emph{способностью предвидения}.

Мозг человека устроен так, чтобы не только получать, хранить и перерабатывать информацию, но и \emph{формулировать план действий}, осуществлять активное управление действиями.

Действие человека всегда направлено на достижение конечного результата, т.е. определённой \emph{цели}.

Любой значимый поступок человека представляет собой решение той или иной жизненно важной задачи, осуществление того или иного намерения.

Между каждым предыдущим и последующим этапами процесса действия и деятельности в целом имеется более или менее чёткая \emph{согласованность}, поскольку весь этот процесс \emph{предопределён} целью, планом.

\emph{Цель}, которую человек стремится достичь, это \emph{то, что должно быть создано, но чего пока ещё реально не существует}.

Цель представляет собой \emph{идеальную модель желаемого будущего}, основанную на идеальных моделях уже отраженного действительного.

Действие, поступок человека имеет своей предпосылкой \emph{два} тесно связанных между собой \emph{процесса}: один из них --- \emph{постановка цели}, т.е. предусмотрение, предвидение, предвосхищение будущего, вытекающего из познания соответствующих связей и отношений вещей, другой --- \emph{программирование, планирование действия}, которое должно привести к реализации цели,

\emph{Целеполагание}, т.е. предвидение того, «\emph{для чего»} и «\emph{ради чего»} человек осуществляет свои действия, --- непременное условие любого сознательного поступка.

Однако, как отмечал \emph{Гегель}, --- «суть дела исчерпывается не своей \emph{целью}, а своим \emph{осуществлением}». (\emph{Гегель}. \emph{Сочинения}, т. 17, с.2).

\emph{Реализация цели} предполагает применение определенных \emph{средств} , т.е. того, что создаётся и существует ради достижения цели.

Человек создает то, чего природа, \emph{до него} не производила.

Конструкция, масштабы, формы и свойства преобразованных и созданных людьми вещей продиктованы потребностями людей, их целями; в них \emph{воплощены} человеческие замыслы, идеи.

Именно в \emph{творческой} и \emph{регулирующей деятельности}, направленной на преобразование мира и \emph{приспособление, но не простое подчинение} его интересам человека, общества, состоит основной \emph{жизненный смысл} и историческая необходимость возникновения и развития сознания.

«Сознание человека \emph{не только отражает} мир, но и \emph{творит} его...». (\emph{В.И. Ленин}. Полн.собр.соч., т. 29, с. 194).

\subsection{Эволюция форм отражения}

Способность человеческого мозга отражать действительность есть \emph{результат} длительного развития высокоорганизованной материи.

В философии и психологии существуют \emph{концепции}, согласно которым сама проблема возникновения сознания из его биологических предпосылок снимается тем, что наличие психики признается лишь у человека. Эта концепция восходит к \emph{Декарту}, полагавшему, что \emph{животные} --- это лишь сложные \emph{машины}.

Прямо противоположную \emph{позицию} занимают те, кто считает, что не только животные, но и \emph{вся природа является одухотворенной} (\emph{Ж. Робине} и др.).

Между этими крайними концепциями, допускающими существование психики или только у человека или \emph{у всей материи}, имеется как бы промежуточная позиция «\emph{биопсихизма»}, согласно которой психика есть свойство лишь живой материи (\emph{Э. Геккель} и др.).

Диалектико-материалистическая философия \emph{не разделяет} как признание всеобщей одушевлённости материи, так и представления о психике как присущей только человеку. Она не разделяет и позиции «биопсихизма». Она исходит из того, что \emph{психическое отражение} внешнего мира --- это такое свойство материи, которое появляется лишь на высоком уровне развития живого, когда образуется \emph{нервная система}.

В ясно выраженной форме ощущение связано только с высшими формами материи, а в основании материи существует \emph{способность,} \emph{сходная с ощущением}, ---\emph{свойство отражения}.

\emph{Отражение как общее свойство материи} обусловлено тем, что предметы и явления находятся в универсальной взаимосвязи и взаимодействии. Воздействуя друг на друга, они производят при этом те или иные изменения. Эти изменения выступают в виде определенного «\emph{следа}», который фиксирует особенности воздействующего предмета, явления.

\emph{Формы отражения} зависят от специфики и уровня структурной организации взаимодействующих тел. А \emph{содержание отражения} выражается

в том, какие изменения произошли в отражающем предмете и какие стороны в воздействующем предмете и явлении они \emph{воспроизводят}.

\emph{Соотношение} между результатами отражения («следами») и отражаемым (воздействующим) предметом может выражаться в виде \emph{изоморфизма} и \emph{гомоморфизма}.

Под \emph{изоморфизмом} понимается сходство между какими-либо объектами, подобие их формы, структуры, как это имеет место, например, в фотографии. \emph{Изоморфное подобие}, отражение --- это \emph{адекватное} воспроизведение оригинала.

Под \emph{гомоморфизмом} имеется в виду лишь \emph{приблизительное} отражение, например, изображение местности на карте.

Отражение присуще материи \emph{на всех уровнях} её организации, но высшие формы отражения связаны с живой материей, с жизнью.

\emph{Что такое жизнь?}

Это \emph{особая, сложная форма движения материи}.

Её важными признаками являются раздражимость, рост, размножение, в основе которых лежит \emph{обмен веществ}. Он-то и составляет сущность жизни.

Обмен веществ связан с определённым \emph{материальным субстратом} (в условиях Земли --- с белками и нуклеиновыми кислотами).

Жизнь --- это прежде всего \emph{процесс взаимодействия} организма и окружающей его среды.

\emph{На нашей планете} жизнь представлена в виде бесчисленного множества различных организмов, начиная от простейших до самых сложных, таких, как человек.

В процессе \emph{биологической эволюции} вместе с усложнением строения организмов, их поведения \emph{совершенствуются} и свойственные живой материи формы отражения.

Отражение и его формы у организмов \emph{находятся в прямой зависимости} прежде всего от характера и уровня их поведения, деятельности. В ходе её совершенствования у живых существ \emph{возникают и развиваются} органы чувств, нервная система. Вместе с тем сама деятельность \emph{зависит} от регулирующего влияния отражения.

Элементарной и \emph{исходной формой отражения}, свойственной всем живым организмам, является \emph{раздражимость.} Она выражается в \emph{избирательном} реагировании живых тел на внешние воздействия (на свет, изменения температуры и т.д.).

\emph{На более высоком уровне} эволюции живых организмов раздражимость переходит в качественно новое свойство --- \emph{чувствительность}, т.е. \emph{способность отражать отдельные свойства вещей в виде ощущений}.

Более высокого уровня отражение достигает у \emph{позвоночных животных}. У них возникает способность анализировать сложные комплексы одновременно действующих раздражителей и отражать их в виде \emph{восприятия --- целостного образа} ситуации.

Ощущения и восприятия, как уже говорилось, являются образами вещей. Это означает появление \emph{элементарных форм психики} как функции нервной системы и формы отражения действительности. (О роли отражения в процессе \emph{познания} см. подраздел 4, гл. VII).

Обычно \emph{различают два} тесно между собой связанных типа поведения животных: \emph{инстинктивное} --- врождённое, которое передается по наследству, и \emph{индивидуально приобретенное}.

Животным присуща способность отражать \emph{биологически значимые} (т.е. помогающие удовлетворять потребности в пище, избегать опасностей, размножении и т.д.) свойства предметов окружающего мира, других животных.

С совершенствованием этой способности связано формирование различных \emph{сложных форм поведения}. У высших животных --- \emph{обезьян} --- они выражаются, например, в отыскании необходимых путей при достижении цели, в употреблении различных предметов в качестве \emph{орудий}, словом, в том, что в обиходе называется»\emph{сообразительностью»} животных.

Высокий уровень развития психики животных показывает, что сознание человека имеет свои \emph{биологические предпосылки} и что между человеком и его животными предками не существует непроходимой пропасти, а имеет место известная преемственность. Однако это ни в коей мере \emph{не означает} тождества их психики.

\subsection{Сознание и речь, их происхождение и взаимосвязь}

Происхождение сознания и речи связано с переходом наших обезьяноподобных предков \emph{от присвоения} с помощью естественных органов готовых предметов, в том числе простейших орудий, к труду, \emph{к изготовлению} \emph{искусственных орудий}, к человеческим формам жизнедеятельности и вырастающим на её основе общественным отношениям.

Переход к сознанию и речи представляет собой \emph{величайший качественный скачок} в развитии психики.

Психика животных помогает им ориентироваться в меняющейся среде, приспосабливаться к ней, \emph{однако они не могут} целенапрвленно и систематически преобразовывать окружающий мир.

\emph{Труд как целесообразная деятельность} является основным условием всей человеческой жизни и формирования сознания.; труд --- как \emph{целесообразная материально-предметная деятельность}.

\emph{Исходная форма} этой деятельности --- процесс \emph{изготовления} особых \emph{орудий} из дерева, из камня, кости и т.д. и производство с их помощью средств существования.

Использовать различные предметы в качестве орудий \emph{могут и животные}. Например, обезьяны иногда берут камень и \emph{разбивают} им орехи, они \emph{достают палкой} приманку и т.п. Но ни одна обезьяна \emph{не изготовила} даже самого примитивного орудия.

Около миллиона лет назад наши обезьяноподобные предки жили \emph{на деревьях} (по одной из версий). Под влиянием изменившихся условий они вынуждены были вести другой образ жизни, \emph{начали спускаться с деревьев} на землю.

В новой обстановке им приходилось \emph{систематически применять} камни, палки, кости крупных животных в качестве средства обороны от хищников, а затем и нападения на других животных.

\emph{Потребность в систематическом использовании орудий} вынуждала их постепенно переходить к обработке материалов, находимых в природе, к производству самих орудий.

Всё это приводило к существенному изменению функции \emph{передних конечностей} (верхних). Они \emph{приспосабливались ко всё новым} операциям, становились естественным орудием трудовой деятельности.

Развившаяся в процессе трудовой деятельности \emph{рука} (\emph{теперь уже действительно рука}) оказывала влияние на мозг, \emph{на совершенствование} всего организма, в том числе, и прежде всего, мозга.

Сознание могло возникнуть лишь как функция сложно организованного \emph{мозга, который сформировался под влиянием труда и речи}.

Под влиянием трудовой деятельности в связи с развитием мозга \emph{совершенствовались и органы чувств} человека: всё более точным и тонким становилось осязание, слух, который стал способным воспринимать тончайшие различия и сходства звуков человеческой речи, более зорким оказывалось зрение. Многие птицы, например, \emph{орёл}, видят значительно дальше, чем человек, но человек замечает в вещах значительно больше, чем глаз птицы.

Способы организации практических действий, их логика фиксировались в голове и превращались в \emph{способы организации мышления}, его логики.

Формировалась \emph{способность к целеполаганию}.

\emph{На начальных этапах} осознание первобытным человеком своих действий и окружающего мира носило ограниченный характер, не выходило за пределы чувственных представлений, их комбинаций и простейших обобщений. Сознание \emph{вначале} представляло собой всего лишь осознание ближайшей чувственно воспринимаемой среды, непосредственных связей с другими людьми. В дальнейшем в ходе \emph{усложнения форм труда} и общественных отношений формировалась способность к мышлению в виде понятий, суждений и умозаключений, отражающих всё более глубокие и многообразные связи между предметами и явлениями действительности, в том числе социальной.

Возникновение сознания непосредственно связано с \emph{зарождением языка}, \emph{членораздельной речи,} выражающей в материальной форме представления, мысли людей.

Как и сознание, речь могла сформироваться лишь в процессе совместной материально-предметной, трудовой деятельности, которая требовала \emph{согласования действий людей}, не могла совершаться без тесного контакта, без постоянного общения людей друг с другом.

Речи предшествовал длительный период развития \emph{звуковых и двигательных реакций} у животных. Но \emph{у животных нет потребности в речевом общении}. А то немногое, что животные, сообщают друг другу, происходит и без помощи членораздельной речи.

Речь представляет собой деятельность, которая осуществляется с помощью \emph{языка, т.е. определённой системы средств общения}.

Существуют различные \emph{виды речи}: устная, письменная и внутренняя (беззвучная, невидимая речь, являющаяся материальной формой сознания, когда человек думает о чём-либо «\emph{про себя»}).

Основными \emph{единицами речи} являются \emph{слово} и \emph{предложение}.

Слово представляет собой \emph{единство значения и звучания} (позднее написания).

\emph{Материальная сторона слова} (звучание, письменное начертание) обозначает предмет и является \emph{знаком}.

\emph{Значение} же слова \emph{отражает предмет} и является чувственным или умственным \emph{образом}.

\emph{Предложение} --- это материальная форма, знаковый \emph{носитель} более менее законченной \emph{мысли, суждения}, как утверждения или отрицания (усмотрения) наличия и ли отсутствия тех или иных связей предметов и их признаков.

С помощью \emph{языка, как системы знаков}, осуществляется переход от живого созерцания, от чувственного познания к обобщённому, отвлечённому мышлению.

\emph{Объективируя} свои мысли и чувства в речи, \emph{как бы ставя их перед собой}, человек подвергает их анализу как вне его лежащий объект.

\emph{Проблема сознания и речи} с давних пор привлекала к себе пристальное внимание философов, вызывала горячие споры между ними.

Одни мыслители (например, немецкий философ \emph{Ф. Шлейермахер}) полностью отождествляли речь и мышление, утверждая, что \emph{разум есть язык}.

Другие (например, также немецкий философ \emph{Ф. Бенеке}) отрывали сознание от речи, считали, что мышление осуществляется без языка, что \emph{язык --- продукт мышления}.

Диалектико-материалистическая философия рассматривает \emph{сознание в тесной связи с языком}, речью.

Не только язык не существует вне мышления, но и мысли, идеи \emph{не существуют оторванно} от языка. \textsc{Otрыb} мышления от речи неизбежно приводит, с одной стороны, к \emph{мистификации сознания}, когда оно лишается внешних материальных средств своего формирования и реализации, а с другой --- к трактовке языка, речи как какой-то \emph{замкнутой в себе самодовлеющей сущности}, отторгнутой от жизни общества, от развития культуры.

\emph{Сознание и речь едины}, но это внутренне противоречивое единство различных явлений.

\emph{Сознание отражает} действительность, а \emph{язык обозначает} её и \emph{выражает} мысли.

Облекаясь в речевую форму, мысли, идеи \emph{не теряют} своего своеобразия.

\emph{В речи} наши представления, мысли и чувства облекаются в материальную, чувственно воспринимаемую форму и тем самым из личного достояния \emph{становятся достоянием других} людей, общества. Это превращает речь в могучее \emph{орудие воздействия} одних людей на других, общества на индивида, а индивида на обществе, особенно индивида значимого для общества.

Если \emph{видовой опыт животных} передаётся с помощью \emph{механизмов наследственности}, что обусловливает исключительно медленный темп прогресса, то \emph{у людей передача опыта}, различных приёмов воздействия на мир происходит \emph{через орудия труда} и \emph{через речь}.

Теперь уже наряду с биологическим фактором --- наследственностью --- \emph{человек выработал} более мощный и притом непосредственный способ передачи опыта --- социальный, во много раз ускоривший прогресс как материальной, так и духовной культуры.

\emph{Благодаря речи} сознание формируется и развивается как \emph{общественное явление}; как духовный продукт жизни общества.

Являясь \emph{средством взаимного общения} людей, обмена опытом, знаниями, чувствами, идеями, \emph{речь связывает людей} не только данной социальной группы и не только данного поколения, но и \emph{разных поколений}. Так создается \emph{преемственность исторических эпох}.

\emph{Философы-идеалисты} утверждают, будто \emph{сознание} развивается из своих внутренних источников, и поэтому оно может быть понято исключительно \emph{из самого себя}.

Диалектико-материалистическая философия исходит из того, что сознание \emph{нельзя рассматривать изолированно} от других явлений общественной жизни. Сознание не замкнуто в самом себе, оно развивается, изменяется в процессе исторического развития общества. Хотя свою родословную \emph{сознание} ведет от биологических форм психики, оно есть не продукт природы, а \emph{общественно-историческое явление}.

\emph{Не в мозге} как таковом таятся причины того, какие ощущения, мысли, чувства возникают у человека. Мозг становится органом сознания только тогда, когда человек вовлекается в \emph{водоворот общественной жизни}, когда он действует в условиях, которые питают мозг соками исторически выработанной и развивающейся культуры, заставляют его функционировать в направлении, заданном требованиями общественной жизни, ориентируют на постановку и решение нужных человеку, всему обществу проблем.

\subsection{Сознание и кибернетика}

Существенный вклад в познание природы отражения, сознания внесла и продолжает вносить \emph{кибернетика --- наука о сложных саморегулирующихся динамических системах}. К ним относятся живые организмы, органы, клетки, объединения биологических особей, общество, определённые технические устройства. Для всех них характерна способность получать информацию, перерабатывать, запоминать её, действовать по \emph{принципу обратной связи} и осуществлять на этой основе управление.

Что же такое \emph{информация}?

Каково её \emph{соотношение} с отражением?

По этим вопросам не существует единства взглядов.

Одни учёные склонны к полному \emph{отождествлению информации и отражения}, другие полагают, что это понятия, близкие между с обой, \emph{но не тождественные}.

В процессе отражения неизбежно происходит \emph{передача информации}, т.е. такое перенесение от одного предмета к другому определённой \emph{упорядоченности} (структуры, формы, наконец, \emph{многообразия}), на основе которой можно судить о тех или иных признаках, свойствах воздействующего предмета.

\emph{На каждом уровне организации материи} происходят свои специфические информационные процессы.

\emph{В неживой природе} осуществляется обмен информацией, но там она не подвергается \emph{расшифровке}.

\emph{Способность} не только получать, но и активно \emph{использовать информацию} --- это фундаментальное \emph{свойство живой материи}.

У животных возникает особая приспособительная деятельность --- \emph{поведение}, а вместе с ним и \emph{управление}, немыслимое без использования информации.

Под \emph{управлением} в кибернетике имеется в виду основанное на определённой программе \emph{регулирование} действий одной системы (\emph{управляемой}) со стороны другой системы (\emph{управляющей}).

Так, мозг --- управляющая система, а, например, органы движения и т.п. --- управляемые системы.

Информация передается с помощью определенных \emph{сигналов}, т.е. каких-либо \emph{материальных процессов} (импульсов электрического тока, электромагнитных колебаний, запахов, звуков, цвета и т.п.).

Обладая определённой структурой\emph{, сигнал} в силу этого может нести ту или иную информацию.

\emph{Информация представляет собой содержание сигнала.}

На принципе передачи информации с помощью сигналов основано конструирование \emph{электронно-вычислительных машин}, выполняющих разнообразные формально-логические операции.

В связи с появлением таких машин, помогающих человеку перерабатывать огромные потоки информации, большую остроту приобрёл \emph{вопрос о возможностях моделирования мышления с помощью машин}, о сходстве и различии процессов, происходящих в моделирующих устройствах и в голове человека.

Так, есть машины, «\emph{опознающие}» зрительные образы. Правда, они способны «опознавать» лишь ограниченный класс объектов, который был введён в них в процессе их «\emph{обучения}» или «\emph{самообучения}».

\emph{Принципиальная разница} между человеческим восприятием и «опознающей» функцией машины состоит в том, что в первом случае результатом является \emph{субъективный образ}, а во втором --- \emph{код} различных признаков объекта, которые необходимы для решения машиной определённых задач.

Наибольшие практические результаты сейчас даёт \emph{моделирование памяти}.

Создаются машины, обладающие чрезвычайно высокой \emph{скоростью запоминания}. Они способны как угодно долго сохранять в своей «\emph{памяти}» информацию и с безупречной точностью воспроизводить её.

Вместе с тем машины обладают и достаточно \emph{большим объемом}, можно сказать, \emph{огромным объёмом} «памяти».

Но машинная «память» \emph{существенно отличается} от человеческой.

В мозгу человека существует \emph{смысловая система} обращения к памяти, позволяющая извлекать нужную информацию, \emph{не перебирая} её \emph{всю подряд}.

Благодаря смысловой организации знаний (а не за счёт быстроты протекания физиологических процессов) достигается скорость их воспроизведения в человеческой памяти.

Накопление информации человеком --- это не механическое «складывание» её, а целенаправленный процесс, \emph{осмысленный процесс}.

Не менее разительные результаты, чем моделирование восприятия и памяти, дает \emph{моделирование} некоторых аспектов \emph{мышления}.

В настоящее время \emph{успешно выполняются машинами} такие умственные операции, как доказательство геометрических теорем, перевод с одного языка на другой, игра в шахматы и т.п.

Киберенетические устройства весьма успешно моделируют присущий человеку \emph{механизм формально-логического мышления}. Однако он далеко не исчерпывает сознания человека. Последнее характеризуется невероятной \emph{гибкостью} и \emph{точностью} в решении задач, не обусловленной какой-либо жесткой системой формальных правил.

При этом важно учитывать, что \emph{способность человека мыслить} заключена не только в структуре его мозга. Она формируется через \emph{приобщение} его к исторически накопленной \emph{культуре}, \emph{через воспитание} и обучение, через oneделённую деятельность с помощью созданных обществом приёмов и средств.

Богатство \emph{внутреннего мира человека} является следствием богатства и разносторонности его общественных связей.

Чтобы можно было \emph{полностью смоделировать сознание}, его структуру и все его функции, недостаточно воспроизведения лишь \emph{структуры мозга}. Для этого потребовалось бы \emph{воспроизвести логику всей истории человеческой мысли}, а следовательно, повторить весь исторический путь развития человека, снабдить его всеми потребностями, в том числе и потребностями политическими, нравственными, эстетическими и т.д.

Человек как существо, наделённое сознанием, возникает в ходе социального развития, поэтому \emph{проблема человека и его сознания} --- это не столько естественнонаучная, кибернетическая, сколько социально-философская проблема.

Рассмотрение вопроса о сознании, его особенностях, происхождений, связи с мозгом, речью \emph{подтверждает в целом правильность положения диалектико-материалистической философии} об отражательной сущности и общественно-историческом характере сознания.

\section{Природа человеческого познания}

Что такое познание, каковы его основные формы, каковы \emph{закономерности перехода от незнания к знанию}, от одного знания к другому, более глубокому, что такое \emph{истина}, что является её критерием, какими путями, методами достигается истина и преодолеваются \emph{заблуждения} --- эти и другие общие вопросы познания рассматриваются теорией познания, или \emph{гносеологией}. (Термин «\emph{гносеология}» происходит от греческого \emph{gnosis} --- знание и \emph{logos} --- слово, учение).

В зарубежной литературе наряду с этим термином применяется термин «\emph{эпистемология}» (от греческого episteme --- знание, в отличие от doxa --- мнение)

\emph{Эпистемологией обычно называют теорию научного познания (знания)}.

\subsection{Материалистическая диалектика --- теория познания диалектико-материалистической философии}

Проблемы теории познания возникли \emph{вместе} с возникновением философии.

В древнегреческой философии начало анализу природы познания положили \emph{Демокрит, Платон, Аристотель, эпикурейцы, скептики} и \emph{стоики}.

В дальнейшем \emph{Ф. Бэкон, Р. Декарт, Дж. Локк, Б. Спиноза, Г. Лейбниц, И. Кант, Д. Дидро, К. Гельвеций, Г. Гегель, Л. Фейербах и другие} философы внесли существенный вклад в анализ процесса познания.

Проблема познания занимает \emph{одно из центральных мест} в диалектико-материалистической философии.

Последняя стремится \emph{вскрыть недостатки} других традиций в подходе к познанию, несостоятельность философских учений, подвергающих сомнению или даже отрицанию принципиальную познаваемость природной и социальной реальности. Эти учения , несмотря на различия между ними, могут в целом быть охарактеризованы как \emph{философский (гносеологический) скептицизм}, как они назывались преимущественно в древности, или \emph{агностицизм}. Последнее название возникло в середине XIX в.

Идеи философского скептицизма высказывались уже древнегреческими философами \emph{Пирроном, Карнеадом, Энесидемом}.

Античные скептики, ссылаясь на тот факт, что по каждому обсуждаемому теоретическому вопросу высказываются противоположные, взаимоисключающие мнения, приходили к выводу, что \emph{истина принципиально недостижима}.

Античные скептики доказывали, что ни чувственные восприятия, ни правила логики не обеспечивают возможности познания вещей, что всякое \emph{знание есть не более чем верование}.

В новое время аргументы античного скептицизма были возрождены и развиты рядом философов, среди которых в первую очередь следует отметить английского философа XVII в. \emph{Д. Юма}. Он утверждал, что \emph{всякое знание является в сущности незнанием}.

Юм писал: «Самая совершенная естественная философия только отодвигает немного дальше границы нашего незнания, а самая совершенная моральная или метафизическая философия, быть может, лишь помогает нам открыть новые области такового. Таким образом, убеждение в человеческой слепоте и слабости является результатом всей философии...» (\emph{Д. Юм}. Соч. в двух томах, т. \emph{2,} М., 1965, с. 33).

В качестве основы для практического действия \emph{Юм} рекомендовал взять \emph{не знание, а веру и привычку}.

Кантианство --- следующая разновидность агностицизма. \emph{И. Кант} подверг детальному анализу познавательный процесс, его отдельные моменты --- чувства, рассудок, разум. Этот анализ был важным вкладом в гносеологию. Но направленность и общий вывод всех его теоретико-познавательных рассуждений \emph{в целом неверны}.

Кант обнаружил сложный и противоречивый мир познания, но оторвал его от вещей реального мира.

«...О том, --- писал \emph{Кант}, --- каковы они (вещи -- \emph{Ред}.) сами по себе, мы ничего не знаем, а знаем только их явления, т.е. представления, которые они в нас производят, воздействуя на наши чувства». (\emph{И. Кант}. Соч. в шести томах, т. 4, ч. I, М., 1965, с. 105).

\emph{Кант} прав, что познание начинается с опыта, с ощущений. Но опыт, в его понимании, вместо того, чтобы соединять человека и \emph{мир «вещей в себе»}, разделяет их, отгораживая друг от друга, поскольку предполагается наличие в сознании существующих до и независимо от опыта, форм чувственности и рас судка (\emph{априорное, доопытное знание}).

Знание складывается, по \emph{Канту}, из того, что даёт опыт, и этих априорных форм.

Априоризм приводит \emph{Канта} к \emph{безысходному агностицизму}.

В философии XIX -- XX вв. агностицизм не исчезает. Он принимается различными направлениями преимущественно западной философии, в первую очередь \emph{позитивзмом} и различными его разновидностями (\emph{махизм}, \emph{прагматизм} и др.). Ничего особенно оригинального в обосновании агностицизма они не внесли, воспроизводя идеи либо \emph{Канта}, либо \emph{Юма}, а нередко преподнося как новейшее достижение смесь воззрений того и другого.

\emph{Как относится} агностицизм к главным философским направлениям --- материализму и идеализму?

Было бы неправильным полагать, что все философы-идеалисты являются агностиками. \emph{Декарт, Лейбниц, Гегель} не были агностиками.

\emph{Гегель} опровергал агностицизм со своей идеалистической точки зрения.

Но идеалисты \emph{непоследовательно критикуют} агностицизм, допуская уступки агностицизму в ряде коренных вопросов.

С другой стороны, не всякий агностик является последовательным сторонником идеализма. Часто он стремится занять \emph{компромиссную позицию} в противостоянии материализма и идеализма. Агностик часто «не идёт дальше ни к материалистическому признанию реальности внешнего мира, ни к идеалистическому признанию мира за наше ощущение». (\emph{В.И. Ленин}. Полн. собр. соч., т. 18, с. 112).

\emph{Агностицизм} как теоретико-познавательная концепция, отрицая содержание ощущений, представлений, понятий, становится в силу этого на путь субъективистского, т.е. идеалистического по сути решения второй стороны основного вопроса философии.

Правда, не всегда мыслители, называвшие себя агностиками, являются действительными сторонниками идеализма. Некоторые ученые-натуралисты, вроде англичанина \emph{Т. Гексли} (который в XIX в. и \emph{ввёл термин «агностицизм»}), считали себя агностиками, прикрывая этим словом свой естественнонаучный материализм, свои враждебные религиозному мировоззрению убеждения о теоретической необоснованности теологических допущений.

\emph{Так же противоречиво} отношение агностицизма к диалектике и метафизике, как противоположным методам философствования.

Агностицизм \emph{субъективистски истолковывал} диалектическую противоречивость познания.

Действительно, \emph{момент скептицизма} является необходимым моментом процесса познания. И философский скептицизм начиная с античности содержал в себе определённую диалектическую тенденцию.

Скептики нередко \emph{обнаруживали богатство}, сложность и противоречивость процесса движения знания к истине. Но агностицизм абсолютизирует подвижность, относительность знания, в нём скепсис, элемент сомнения, приобретает отрицательный характер.

Агностики успокаиваются на установлении относительности знания, его противоречивости и не идут от них к законам объективного мира.

\emph{Отрыв субъективной диалектики} (движения знания) от \emph{объективной диалектики} (движения самой материальной действительности) --- основной гносеологический \emph{источник агностицизма}.

Агностицизм подвергался \emph{справедливой крити}ке с момента его возникновения. Противники агностицизма вскрывали противоречивость учения самих агностиков, неоправданность, деже абсурдность их конечных выводов, но в критике агностицизма раньше нередко \emph{было больше остроумия}, чем аргументов, вскрывающих несостоятельность подобных философских представлений.

Агностическое представление о знании возникает как отражение противоречивости процесса познания, трудностей в определении критерия истинного знания. Но агностицизм не может быть последовательно преодолён на основе старых форм материализма, впрочем, как и идеалистическая диалектика.

\emph{Действительное преодоление агностицизма} наметилось только в теории познания диалектико-материалистической философии, исходящей из следующих положений (\emph{В.И. Ленин}. Полн. бобр, соч., т. 18, с. 102):

\begin{enumerate}
\item «Существуют вещи \emph{независимо} от нашего сознания, независимо от нашего ощущения, вне нас...
\item Решительно \emph{никакой принципиальной разницы} между явлением и вещью в себе нет и быть не может. Различие есть просто между тем, что познано, и тем, что ещё не познано...
\item В теории познания, как и вовсех других областях науки, \emph{следует рассуждать диалектически}, т.е. не предполагать готовым и неизменным наше познание, а разбирать, каким образом из \emph{незнания} является \emph{знание}, каким образом неполное, неточное знание становится более полным и более точным».
\end{enumerate}

\emph{Гносеология обязана} философии диалектического материализма двумя вещами, которые существенным образом меняют её подходы: 1) \emph{распространением} материалистической диалектики на область познания; 2) введением в теорию познания \emph{практики} как основы и критерия истинности знания. Этим самым преодолевается обособление законов мышления от законов объективного мира, с отрывом первых от вторых.

Субъективная по форме \emph{диалектика познания} является отражением в процессе познания содержания объективной реальности, внутренне присущих ей закономерностей. Основой этого познавательного процесса является вся сфера общественной практики.

\subsection{Субъект и объект}

Знание не существует в голове человека изначально, а приобретается в ходе его жизни, является результатом познания.

\emph{Процесс обогащения человека новым знанием и носит название познания.}

Чтобы понять сущность, закономерности познания, необходимо определить, кто является его \emph{субъектом}.

Казалось бы, здесь все ясно: \emph{субъект познания --- человек}.

Но, \emph{во-первых}, история философии знает мыслителей, которые отрицали принципиальною возможность познания мира, а вместе с ней, по существу, и субъект познания.

\emph{Во-вторых}, некоторые мыслители и естествоиспытатели утверждают, что познание, в частности теоретическое мышление, свойственны не только людям, но и созданным ими устройствам, вроде \emph{ЭВМ}.

Наконец, недостаточно просто утверждения: человек --- субъект познания, необходимо выявить, \emph{что его делает таковым}.

Как известно, уже \emph{Л. Фейербах} подверг критике идеалистическое понимание, согласно которому субъектом познания является само сознание ---самосознание, правильно отмечая, что сознание присуще лишь человеку.

\emph{Человек, по Фейербаху}, телесное существо, обитающее в пространстве и во времени, обладающее благодаря своей органической связи с природой способностью её познавать.

Казалось бы, \emph{Фейербах} в своём понимании познания имеет в виду \emph{конкретного человека}, обладающего природной сущностью. Однако человек в учении Фейербаха оказывается \emph{лишь природным}, а не исторически развивающимся социальным существом.

Каким образом человек обретает свою конкретную \emph{реальную сущность}?

Человеку присущи свойства природного существа, в том числе и чувственность, но он создаёт свою \emph{вторую, социальную природу --- культуру}, цивилизацию, посредством труда творит себя, не просто присваивая предметы природы, а изменяя их соответственно своим потребностям.

\emph{Вне общества} \emph{нет} человека, а следовательно, нет и субъекта познания.

Но разве, могут спросить, познаёт сразу всё человечество, общество, а не отдельные люди: \emph{Пифагор, Аристотель, Ньютон, Эйнштейн и другие} выдающиеся личности?

Конечно, общество не может существовать \emph{без и вне} отдельных людей, мыслящих, производящих, обладающих индивидуальными особенностями и способностями. Но эти отдельные люди могут быть субъектами познания лишь благодаря тому, что они связаны с определённым социальным качеством.

Процесс познания \emph{обусловлен} исторически сложившейся структурой познавательных способностей человека, уровнем развития познания, который в свою очередь обусловлен существующими общественными условиями.

Как бы ни был гениален \emph{Ньютон}, но теории относительности он никак не мог создать.

\emph{Объективный идеализм} своим утверждением о независимости сознания, разума от реальных, существующих в обществе человеческих индивидов \emph{мистифицировал} ту особенность познания, что оно представляет социальный процесс. Взяв совокупный результат деятельности людей, зафиксированный в формах сознания, идеализм представил его в виде \emph{самостоятельной сущности}, движущейся по своей собственной логике. Поэтому \emph{мышление оказалось оторванным} не только от его конкретного носителя --- человека, но и от объекта --- находящихся вне субъекта познания предметов, явлений.

Однако для познания необходимы не тольк субъект, \emph{но и объект}, с которым субъект (человек) взаимодействует.

Явления, процессы объективной реальности существуют \emph{независимо от сознания}.

О самом субъекте познания --- человеке можно судить по тому, \emph{что выступает объектом} его познания и практики.

Например, электрон во времена не только \emph{Демокрита} и \emph{Аристотеля}, но и \emph{Галилея} и \emph{Ньютона} хотя и существовал как реальность, но не входил в сферу познавательной деятельности человека, который не был способен выявить его в качестве объекта своей мысли и действия.

Лишь зная \emph{степень развития общества}, можно сделать вывод о том, какой предмет природы станет объектом познавательной деятельности людей.

Например, уровень общественной практики \emph{сейчас} таков, что в сферу деятельности человека постепенно входит практическое освоение окружающего нашу Землю космического пространства, других планет Солнечной системы.

Человек живёт \emph{в очеловеченной} в той или иной мере природе. Он включает всё новые и новые явления природы в орбиту своего бытия, превращая их в объекты деятельности. \emph{Так расширяется} и углубляется человеческий мир.

Значительная часть объектов познания представляет собой явления природы, \emph{преобразованные человечеством}. Эти объекты познания находятся в известной зависимости от практической деятельности человека. Посредством этой деятельности и \emph{создается культура}, элементом которой и является знание.

\subsection{Практика. Общественно-исторический характер познания}

Воздействие предметов природы и социальных процессов на человека является необходимым \emph{условием познания}, однако основу этого процесса образует воздействие человека на объективную реальность.

Познание развивается благодаря тому, что человек своим действием \emph{вмешивается} в объективные явления, преобразует их, испытывая их воздействие.

Понять сущность человеческого познания можно только путём выведения его из особенностей \emph{практического взаимодействия субъекта и объекта}.

Человечество и природа --- \emph{две} качественно различные материальные системы.

Человек --- социальное существо и действует предметным образом, способом.

Наличие у человека \emph{сознания} и \emph{воли} оказывает существенное влияние на это взаимодействие, но при этом взаимодействие субъекта и объекта не теряет своей материальной природы.

Человек действует всеми своими средствами, естественными и искусственными орудиями на явления и вещи природы, преобразуя их, а вместе с тем и самого себя.

\emph{Предметная материальная деятельность человека носит название практики.}

Понятие практики является \emph{фундаментальным} для диалектико-материалистической философии.

\emph{Общественное производство} --- важнейшая форма практической деятельности людей. Однако \emph{не следует ограничивать} практическую деятельность лишь сферой производства. В этом случае человек превращается лишь в экономическое существо, удовлетворяющее посредством труда свои потребности в пище, одежде, жилище и т.п., а его сознание приобретает чисто технический характер.

В практику, взятую \emph{в самом широком смысле}, входит вся совокупность предметных форм деятельности человека, она охватывает все стороны его общественного бытия, в процессе которого создается материальная и духовная культура, \emph{включая такие социальные явления}, как взаимодействия социальных групп, классов, развитие искусства и науки.

В производственной, трудовой деятельности человек относится к природе \emph{не так, как животное}, которое добывает лишь то, в чём оно или его детеныш непосредственно нуждается.

\emph{Человек --- универсальное существо}, он создает то, чего нет в природе, он творит по своим меркам и масштабам, сообразно возникающим и развивающимся целям. А такого рода деятельность \emph{невозможна без сознания}.

\emph{На фундаменте труда}, производства строятся все формы предметной деятельности человека, которые и порождают такое явление, как познание вещей, процессов, закономерностей объективной реальности.

Первоначально познание \emph{не отделялось} от материального производства, а было непосредственно вплетено в него. Однако потом, в процессе развития цивилизации, производство идей \emph{отделяется} от производства вещей, процесс познания превращается в относительно самостоятельную духовную деятельность человека. На это почве возникло затем \emph{разделение, даже противопоставление теории и практики}, противоречие между ними, пути разрешения которого исследуются диалектической теорией познания, в том числе диалектико-материалистической.

Выясняя взаимоотношение \emph{теоретической (умственной) деятельности} и практики, можно установить зависимость теории от практики и в то же время её относительную самостоятельность. Для гносеологии важно как одно, так и другое.

\emph{Зависимость познания от практики объясняет нам общественно-историческую природу познания.}

В познании все стороны связаны и определены обществом.

\emph{Субъект познания} --- \emph{это человек} в его общественной сущности.

\emph{Объект познания} --- предмет природы или социальное явление, которые \emph{вычленяются} идеально познанием или практически-материальной деятельностью людей.

От природы человек унаследовал \emph{биологические предпосылки}, являющиеся условиями функционирования познания в виде достаточно развитой нервной системы, мозга.

Но естественные органы человека в процессе общественного развития \emph{изменили} свое назначение и функцию. Именно \emph{благодаря общественной деятельности} органы чувств, мозг, руки человека стали способны создавать такие чудеса, как картины и статуи великих художников, творения гениальных музыкантов, шедевры литературы» науки, философии.

Из общественной природы познания следует, что \emph{источником} его развития являются изменения в предметной деятельности человека, в социальных потребностях, которые определяют цель познания, его объект, стимулируют людей на всё более, глубокое теоретическое овладение им.

Относительная самостоятельность познания позволяет ему \emph{опережать} непосредственные запросы практики, предвидеть новые явления, активно воздействовать на производственную и иные сферы жизни людей.

Например, научное представление о сложной структуре атома возникло \emph{до того}, как общество сознательно поставило перед собой цель практического использования внутриатомной энергии.

\emph{Опережение познанием практики} обусловлено развитием самой общественной практики, с одной стороны, и специфическими законами, закономерностями познания --- с другой.

При этом связь познания с практическими задачами, которые ставит перед собой человек и человечество, часто носит сложный, \emph{опосредованный характер}. Например, результаты современных математических исследований \emph{прежде всего} находят применение в других областях науки --- физике, химии, биологии» социологии и т.п., \emph{а потом уже} в технике и технологии производства, управлении.

Конечно, существует \emph{возможность отрыва} теоретической деятельности от практической. В познании это может привести к превращению его в замкнутую внутри себя систему, не имеющую выхода в практику людей. Поэтому \emph{систематическое обращение познания к практике}, например, в форме наблюдения или эксперимента, --- залог его объективности, всё более глубокого проникновения его в сущность вещей и процессов объективной реальности.

\subsection{Знание как духовное освоение действительности. Принцип отражения}

Процесс познания имеет своим результатом \emph{знание}.

\emph{Понятие знания} весьма сложно и содержательно.

Многие гносеологи, занимающиеся анализом знания, \emph{пытались выделить} то одну, то другую его сторону и представить её в качестве выражения всей природы знания. Эта односторонность приводила к тому, что терялись важнейшие моменты,составляющие сущность знания, а в результате представления о знании оказывались ошибочными.

Первое определение знания фиксирует его место в общественной жизни людей.

Посредством знания человек теоретически (\emph{идеально}) овладевает объектом, и так же его преобразует.

Знание \emph{идеально по отношению} к находящемуся вне его объекту. Оно не есть сама познаваемая вещь, явление, свойство, а специфическая форма освоения действительности, способность человека в своих мыслях, целях, желаниях воспроизводить вещи, процессы, \emph{оперировать их образами}, понятиями.

Значит, \emph{знание}, будучи идеальным, существует не в виде чувственно-материальных вещей или их вещественных копий отпечатков, а как нечто противположное материальному, как момент, сторона предметного взаимодействия субъекта и объекта, как \emph{форма деятельности человека}.

Знание как идеальное \emph{вплетено} в материальное, в функционирование нервной системы, в созданные человеком знаки (слова, математические и другие символы и т.п.). В результате этого и создаются идеи, посредством которых осуществляется духовное освоение человеком объектов,создаются образы существующих и возможных вещей и процессов.

\emph{Отношение знания к объективной реальности выражено в понятии отражения.}

\emph{Принцип отражения} был сформулирован философией \emph{ещё в античности}.

\emph{Материалисты нового времени} разрабатывали этот принцип, обогащая его новым содержанием, но вместе с тем истолковывали его в механистическом духе: отражение мыслилось как воздействие предметов на человека, органы чувств которого, \emph{подобно воску}, запечатлевают форму предметов.

Хотя отражение не является понятием, специфическим только для диалектико-материалистической теории познания, но в ней оно заняло своё место, было переосмыслено, \emph{приобрело новое содержание}.

\emph{Почему} это понятие \emph{необходимо} для выявления особенностей знания, познания?

Когда речь идёт о \emph{содержании знания}, его источнике, о том, как и в какой форме оно связано с объективной реальностью, то нельзя оставаться на позициях материализма без понимания \emph{знания как отражения вещей}, свойств, закономерностей объективной реальности, своеобразного их аналога.

\emph{Материализм в теории познания} исходит из признания существования независимо от сознания человека объективной реальности и способности познать её. Понятие отражения как раз и связано с признанием объективной реальности, которая входит в содержание знания.

\emph{Знание отражает объект} --- это значит, что субъект создаёт такие формы мыслительной деятельности, которые в конечном счёте воспроизводят свойства, закономерности данного объекта, т.е. \emph{содержание знания объективно}.

\emph{Идеалистическая гносеология} избегает понятия отражения, стремится заменить его термином «\emph{соответствие}» и т.п., представляет знание не как образ объективной реальности, а как заменяющий её \emph{знак, символ}.

Так, неокантианец \emph{Э.Кассирер}, защищая концепцию \emph{знания как символа} по отношению к объекту, пишет: «Наши ощущения и представления суть знаки, а не \emph{отображения} предметов. Ведь от образа мы требуем некоторого \emph{подобия} с отображаемым объектом, а в этом подобии мы здесь никогда не можем быть уверены» (\emph{Э.} \emph{Кассирер. Познание и действительность}. Спб., 1912, с. 394).

Конечно, отражение, представленное в виде \emph{мёртвого копирования} существующих вещей и процессов, взятое вне субъективной, активно творческой деятельности человека, не может служить характеристикой знания.

\emph{Смысл бытия человека} заключается в свободной творческой деятельности, в практическом переустройстве мира, а знание служит целям и задачам этой деятельности.

Но знание \emph{только тогда может быть} орудием переустройства мира, когда оно обладает объективным содержанием и является активным, практически направленным отражением действительности.

\emph{Знание овладевает объективно существующей реальностью}, имеет её в качестве своего содержания, т.е. отражает свойства и закономерности явлений, процессов, существующих вне его.

Субъективная деятельность без такого отражения приведёт не к творчеству, не к созданию необходимых вещей, а к практически \emph{безрезультатному произволу}.

Другими словами, отрицание того, что знание есть отражение, \emph{равносильно выхолащиванию} его объективного, предметного содержания.

Диалектико-материалистическая теория познания обосновывает природу познания посредством \emph{принципа отражения}, наполняет само понятие отражения новым содержанием, включает в него чувственно-практическую, активную, творческую деятельность человека.

\emph{Знание представляет собой адекватное отражение действительности}, проверенное общественной практикой.

\emph{Знание} \emph{форма} деятельности человека, определяемая свойствами, закономерностями явлений, объективной реальности, т.е. \emph{способ} целесообразного, творчески активного отражения объекта.

\subsection{Язык --- форма существования знания. Знак и значение}

\emph{Знание} --- идеально как отражение материальной действительности, которое необходимо отличать от последней. Но знание не существует вне отражаемого им мира, оно \emph{необходимо} принимает специфическую материальную форму выражения.

\emph{Человек как предметное существо} действует предметным образом: его знания существуют также в предметной форме.

Оперировать знанием можно лишь постольку, поскольку оно принимает \emph{форму языка}, выражается системой чувственно воспринимаемых предметов --- \emph{знаков}. Иначе, как через язык, человек не может \emph{передать другому} идею вещи, её образ.

\emph{На поверхности} знание выступает в виде системы знаков, \emph{указывающих} на предметы, события, действия и т.п.

То, на что указывает знак, составляет его \emph{значение}.

Знак и значение неразделимы: \emph{нет знака без значения и наоборот}.

Различают \emph{знаки языковые и неязыковые.}

К неязыковым знакам относятся \emph{знаки-сигналы}, \emph{знаки-признаки} и т.п.

\emph{Знание существует в языковых знаках}, которые в составе своего значения имеют познавательный образ тех или иных явлений, процессов объективной реальности (\emph{смысловое значение знака}).

Между чувственно воспринимаемым предметом, выполняющим роль знака, и его значением \emph{нет} внутренне необходимой связи. Одно и то же значение можно связывать \emph{с разными} предметами, выполняющими роль знаков.

В качестве знаков могут выступать искусственные образования --- \emph{символы} (\emph{условные обозначения}).

Развитие познания привело к возникновению разветвленной системы \emph{искусственных, символических языков} (таков, например, язык символов математики, логики, химиики т.д.). Эти языки тесно связаны с естественными, но представляют собой \emph{относительно самостоятельные системы знаков}.

Наука всё чаще и эффективнее использует символику как средство выражения результатов познания и самого процесса получения этих результатов --- \emph{научного исследования.}

Та же \emph{современная формальная логика}, например, вводит различение \emph{значения} и \emph{смысла} (\emph{смыслового значения}).

\emph{Значение в логике} --- класс предметов, обозначаемых языком, а \emph{смысл} --- мысленное содержание знаков или их совокупности.

Например, \emph{значением} слова «\emph{кит}», являются \emph{все киты}, которые были, есть и будут, а \emph{смыслом} --- млекопитающее животное, обитающее в океанах, и т.п.

В данном случае термин «\emph{значение}» применяется \emph{в широком смысле} слова, объединяя оба указанные аспекта, и как значение, и как смысл.

Используются также понятия «\emph{предметное значение знака}» и «\emph{смысловое значение знака}», «\emph{денотат}» и «\emph{десигнат}».

Широкое применение \emph{символики} в современном познании \emph{используется} некоторыми философскими течениями для зашиты идеалистических представлений.

То обстоятельство, что знание существует в виде системы знаков, а этими знаками в современной науке все в большей мере выступают искусственные образования (символы), истолковывается ими как подтверждение концепции, согласно которой \emph{знание является символом, а не отражением} действительности; что якобы переход науки от естественного языка к искусственному означает \emph{потерю знанием} своей \emph{объективности}.

Конечно, наука, \emph{например, физика} имеет свой язык, не похожий на любой национальный естественный язык, но она создаёт его не для того, чтобы удалиться от изучаемых ею процессов, а с целью \emph{более глубокого и полного} их постижения.

Хотя знание но форме своего знакового выражения всё более становится \emph{символическим} и научная теория нередко выступает сейчас в виде системы знаков, но в своём значении эти символы, уравнения точнее и глубже отражают объективную реальность.

\emph{Не сами символы} являются результатами познания, а то их идеальное значение, которое своим содержанием имеет вещи, процессы, свойства, закономерности, изучаемые данной наукой.

\emph{Не символы} знаменитой формулы А. Эйнштейна \emph{Е = Мс²} являются знанием, а значение знаков, составляющих эту формулу, и отношение между ними выражают одну из физических закономерностей --- связь энергии и массы, т.е. дают реальное знание.

Правда, выявить значение, т.е. класс предметов, к которым относятся отдельные символы и теория в целом, \emph{не всегда легко}.

\emph{Прошло то время}, когда всякое знание, по существу, \emph{было наглядным}, за любым понятием усматривался определённый чувственный образ, предмет. Поэтому не случайно сейчас во весь рост встала \emph{проблема интерпретации}, \emph{истолкования} теорий, выраженных символическим языком, в большей или меньшей степени формализованным.

Сам \emph{термин «истолкование», или «интерпретация»}, приобрёл иной, по сравнению с традиционным, смысл. Под ним разумеют сейчас \emph{не только научное объяснение}, включающее поиски причин, законов явлений (от этой задачи никогда не освобождалась наука, она и сейчас является важнейшим элементом научного исследования), но и \emph{некоторую логическую операцию}, посредством которой определяется познавательное \emph{значение абстрактных, символических систем} в различных областях знаний, устанавливается возможное эмпирическое содержание и сфера применения как отдельных \emph{терминов} (символов) и \emph{высказываний} (выражений) теории, так и самой теории в целом.

Логическая мысль XX в. \emph{много} занимается вопросами, связанными с интерпретацией абстрактных теоретических систем.

\emph{На первый взгляд} задача кажется простой: имеется некоторая научная теория со своим языком; чтобы понять её, необходимо \emph{свести} её \emph{язык к другому языку} --- более универсальному и формализованному, например к тому, который дает современная формальная логика.

Вообще такое сопоставление двух языков \emph{весьма плодотворно}, оно позволяет проверить научную теорию критериями строгого языка, установить её \emph{непротиворечивость}, точность употребляемых терминов и т.п.

Однако этим путём нельзя выявить \emph{предметную область} теории, т.е. её познавательное значение и объективное содержание.

Существует \emph{другой способ} интерпретации научной теории: сравнение её языка с \emph{языком наблюдения}, эксперимента, отыскание не только абстрактных объектов, стоящих за терминами и выражениями теории, но и эмпирических, чувственно-наглядных объектов.

Эта операция, называемая \emph{эмпирической интерпретацией,} даёт возможность связать абстрактную теоретическую систему с явлениями объективной реальности, однако и она \emph{не решает главной задачи} --- выявления того \emph{познавательного значения} теоретической системы во всём его объеме. Ведь одна и та же теория может быть интерпретирована на материале различных экспериментов, которые даже в своей совокупности не могут заменить содержащегося в ней знания о законах явлений.

Некоторые направления современной философии, в частности \emph{логический позитивизм}, исходят из того, что знание складывается как бы из двух моментов: \emph{правил оперирования знаками} языка и \emph{совокупности чувственных восприятий}.

Согласно концепции неопозитивистов, научную теорию можно интерпретировать \emph{только языковыми средствами формальной логики} или \emph{путём сведения к языку наблюдения}, эксперимента, который ближе к естественному языку, а, следовательно, к чувственным образам.

\emph{Уязвимость} неопозитивистских концепций состоит в том, что они, анализируя язык науки, \emph{не учитывают} в достаточной степени \emph{содержание знания}, между тем как уже \emph{Кант} убедительно показал, что содержание знания независимо от его Формы, той формы, которую придаёт ему процесс познания.

\emph{Чтобы понять теорию}, выявить её познавательное значение, осмыслить содержащееся в ней знание об объективной действительности, необходимо, не ограничиваясь интерпретацией в языке формальной логики и эмпирического наблюдения, включить её в общий процесс развития познания, а вместе с тем и человеческой цивилизации вообще.

Таким путём \emph{можно понять}, что даёт теория для интеллектуального развития, для духовного овладения явлениями и процессами объективной реальности, куда ведёт человеческую мысль и действие.

В этом обнаружении \emph{познавательного значения теории} огромная роль принадлежит \emph{категориям} философии.

Из всего вышеизложенного можно сделать вывод, что \emph{знание является необходимым для практической деятельности духовным освоением действительности, в процессе которого создаются понятия, теории. Это освоение творчески целесообразно, активно отражает явления, свойства, закономерности объективного мира и реально существует в форме языковой системы.}

\subsection{Объективная истина}

Знание есть результат человеческой деятельности.

Люди, руководствуясь своими исторически сформировавшимися целями, создают \emph{специальные орудия}, приборы, другие средства, помогающие познанию действительности.

\emph{Вмешательство} человека в изучаемые им процессы всё больше усиливается.

Для практической деятельности нам необходимо знание, с наибольшей степенью полноты и точности отражающие объективный мир, каким он существует сам по себе, независимо от сознания человека и его деятельности.

\emph{Встаёт вопрос об истинности знания}: что представляет собой истина, как она возможна, \emph{где критерий}, по которому можно отделить истинное знание от неистинного, ложного?

По традиции, уходящей в философию античности, \emph{истиной называется знание, соответствующее действительности}.

Но это определение настолько широко, расплывчато, что его нередко принимали и взаимоисключающие философские направления --- как материалистические, так и идеалистические.

\emph{Даже агностические} мыслители обычно соглашаются с этим определением, истолковывая на свой лад термины ``\emph{соответствие»} и «\emph{действительность»}. Они заявляют, что не отрицают существования знания; они-де отрицают лишь знание как отражение вещей, процессов такими, какими они существуют сами по себе.

\emph{Подавляющая часть философов} считают целью познания достижение истины и тем самым признают, что истина существует.

Понятно, почему диалектико-материалистическая философия, её гносеологический раздел \emph{не могут остановиться} на приведенном выше абстрактном, расплывчатом определении истины. Они идут дальше, конкретизируя понятие истины и определяя её как \emph{объективную истину, т.е. знание, содержание которого не зависит от субъекта, не зависит ни от отдельного человека, ни от человечества в целом.}

Как было отмечено раньше, независимо от практической деятельности человека \emph{не существует} никакого знания, а значит и истины.

Принципиально \emph{несостоятельны} те концепции истины, которые выводят её вообще за пределы человека и человечества, в некий, например, \emph{трансцендентный, потусторонний мир}.

Но с другой стороны, истина лишь постольку является таковой, поскольку ей присуща \emph{объективность}, т.е. её содержание составляет объективную реальность, которую она адекватно отражает.

Так, \emph{утверждение}: «В структуру атома любого элемента входят электроны» \emph{является объективной истиной}, ибо его содержание взято из объективной реальности, из того положения вещей, которое существует независимо от познания.

В объективной истине выражена \emph{диалектика субъекта и объекта}.

С одной стороны, истина \emph{субъективна}, поскольку является формой человеческой деятельности, а с другой стороны, она \emph{объективна}, ибо её содержание не зависит ни от человека, ни от всех людей вместе взятых.

\emph{Формы отрицания} объективной истины различны.

\emph{Субъективный идеализм}, отвергая существование независимой от сознания реальности, отрицает тем самым и объективное содержание человеческих знаний, объективную истину.

\emph{Прагматизм} выводит истинность из практики, но которая понимается как субъективная деятельность, направленная на достижение пользы и удобства.

\emph{Б. Рассел}, виднейший английский философ, близкий неопозитивизму, считатет истину \emph{формой убеждения, веры}. «...Истинной или ложной, --- писал он, --- является, прежде всего, вера; предложения становятся истинными или ложными только благодаря тому, что они выражают веру» (\emph{Б. Рассел. Человеческое познание}. М., 1957, с. 147)

С его точки зрения Рассела, \emph{истина --- вера}, которой соответствует какой-то факт, а \emph{ложь --- тоже вера}, но не подтверждающаяся фактами.

При этом то, что представляет собой \emph{факт}, подтверждающий веру, --- это остается открытым; им может служить какая-либо внешняя ассоциация, например, и т.п. Иными словами, признание явной объективности содержания знания как решающего момента истины \emph{здесь не просматривается}.

Объективная истина, знание, соответствующее действительности, \emph{не есть нечто застывшее}, статичное. Они суть \emph{процесс}, включающий в себя различные качественные состояния.

Диалектико-материалистическая философия исходит из \emph{разграничения понятий абсолютной и относительной истины.}

Термин «\emph{абсолютная истина»} у потребляется в литературе неоднозначно. Часто за ним стоит представление о \emph{полном} и \emph{законченном} знании о мире в целом. Это истина «\emph{в последней инстанции»}, как осуществление предела стремлений и потенций человеческого разума.

Спрашивается, \emph{возможно ли} достижение подобного знания?

\emph{Человек в принципе способен познать любое явление}, но реально эта его способность осуществляется в процессе практически бесконечного исторического развития общества. Поэтому стремление во что бы то ни стало достигнуть истины «в последней инстанции» подобно \emph{погоне за химерами}.

Иногда в качестве такой истины «в последней инстанции» рассматривается \emph{фактическое знание} отдельных явлений, процессов, достоверность которых уже доказана наукой. В таком случае истина получает наименование \emph{вечной}: «Лев Толстой родился в 1828 г.», «Птицы имеют клюв», «Химические элементы обладают атомным весом».

Существуют ли такие истины? --- Конечно, да. Но кто желает ограничить познание достижением подобного рода знания, тот, \emph{мало чем} может поживиться.

Развитие науки шло путём преодоления различных утверждений, претендовавших на абсолютность, но оказавшихся \emph{истинами лишь в определённых границах} (например, «Атом неделим», «Все лебеди белые» и т.п.).

Реальная научная теория нередко содержит \emph{элемент неистинного}, иллюзорного, что обнаруживается последующим ходом познания и развитием практики.

Но \emph{не встаём ли мы} при таком понимании познания на путь отрицания объективной истины? Ведь если в процессе познания в истинном обнаруживается \emph{момент иллюзорного}, \emph{заблуждение}, противоположность между истинным и неистинным становится относительной.

Сторонники \emph{гносеологического релятивизма} (от латинского \emph{relativus} --- относительный), не видя противоположность между истиной и заблуждением, приходят к выводу, что всякая истина в конечном итоге оказывается заблуждением и история познания, науки представляет собой \emph{смену одного заблуждения другим}.

Релятивизм содержит в себе \emph{верный момент} --- \emph{признание текучести}, подвижности всего существующего, в том числе и познания, но он метафизически \emph{отрывает} движение познания от объективной реальности.

Теория познания диалектико-материалистической философии, преодолевая и догматизм и релятивизм, \emph{признаёт существование абсолютности и относительности} знаний, но при этом устанавливает их связь между собою в процессе достижения объективно истинного знания.

\emph{Абсолютная истина, абсолютность знаний существует}, ибо в нашем знании имеется нечто такое, что не отбрасывается последующим ходом познания, науки, а обогащается новым объективным содержанием.

Вместе с тем в каждый данный момент наше знание \emph{относительно}, поскольку оно \emph{неполно отражает} действительность и, следовательно, является \emph{истиной лишь в определённых пределах}, которые расширяются или суживаются в ходе развития познания.

Объективная \emph{истина --- это ещё и процесс} движения познания от одной ступени к другой, в результате которого знание наполняется содержанием, почерпнутым из объективной реальности. Она всегда является единством абсолютного и относительного.

\emph{Ещё в античности} была создана \emph{геометрия}, вошедшая в науку под названием \emph{евклидовой}. Истинна она или нет? Конечно, она является объективной, \emph{абсолютно-относительной истиной}, ибо её содержание взято из пространственных отношений, существующих в объективной реальности.

Однако она \emph{истинна до определенных пределов}, т.е. до тех пор, пока мы \emph{абстрагируемся от кривизны пространства} (приравниваем её к нулю). Как только рассматривают пространство с положительной или отрицательной кривизной (впадина, сфера), то переходят к \emph{неевклидовым геометриям} (\emph{Лобачевского и Римана}), которые расширили пределы наших знаний и внесли свой вклад в развитие геометрических знаний по пути дальнейшего углубления объективной истины.

\subsection{Критерий истинности знания}

Стремясь к достижению объективно истинных знаний, человек испытывает \emph{потребность в критерии}, с помощью которого он мог бы отличать их от заблуждений.

Казалось бы все просто: наука даёт объективную истину, и человек выработал множество способов доказательства и проверки её. Но это не так, не так просто.

\emph{Доказательство в строгом смысле} --- это выведение одного знания из другого, когда одно знание с необходимостью следует из другого --- как тезис из аргументов в \emph{дедуктивном выводе}.

Таким образом знание, в процессе своего доказательства, не выходит из своей собственной сферы, как бы замыкается в себе. На этом основании возникло представление о существовании \emph{формального (логического)} \emph{критерия истины}, когда последняя устанавливается посредством соответствия одного знания другому.

Так называемая \emph{теория когеренции}, которая в XX в. особенно распространялась \emph{неопозитивистами}, вообще исходит из того, что никакого иного критерия и не существует, а сама \emph{истина есть согласие знания со знанием}, устанавливаемое на основе формально-логического закона недопустимости противоречия (\emph{закона непротиворечия}).

Но формальная логика обеспечивает нам \emph{гарантию истинности выводного суждения}, если объективно истинны посылки, из которых оно следует: «\emph{а}» следует из «\emph{Ь}»,«\emph{b}» следует из «\emph{с}» и т.д. до бесконечности.

Возникает вопрос: \emph{откуда, из какого знания следуют всеобщие принципы}, аксиомы, да и сами правила логического вывода, которые лежат в основании всякого доказательства? Этот вопрос ставил ещё \emph{Аристотель}.

Следуя теории ковергенции, остается только одно: \emph{признать всеобщие принципы условным соглашением (конвенциями)} и таким образом \emph{поставить крест} на всех попытках установления объективной истинности знания, склониться к субъективизму и агностицизму в теории познания.

В истории философии \emph{были различные подходы} к решению проблемы критерия истинности знания.

Одни философы видели его \emph{в эмпирическом наблюдении}, в ощущениях и восприятиях человека.

Конечно, эмпирическое наблюдение является одним из способов проверки знания.

Но, \emph{во-первых}, не все теоретические понятия можно проверить непосредственно путём наблюдения.

\emph{Во-вторых}, «эмпирическое наблюдение само по себе никогда не может доказать достаточным образом необходимость\ldots». (\emph{Ф. Энгельс.} К. Маркс и Ф. Энгельс. Соч., т. 20, с. 544). А ведь знание, фиксирующее законы, обязательно включает в себя необходимость и всеобщность.

Конечно, в научной практике имеет место проверка суждений и теорий посредством \emph{обращения к чувственному опыту}. Но она не может служить окончательным критерием истинности, ибо из одной и той же теории могут следовать самые \emph{различные следствия}, допускающие опытную проверку. Соответствие опыту одного такого следствия или некоторой суммы их ещё не гарантирует объективной истинности всей теории.

Кроме того, \emph{не все положения} науки можно проверить путём непосредственного обращения к чувственному опыту. Поэтому \emph{даже позитивисты}, поднявшие на щит \emph{принцип верификации} (проверяемости знания путём сопоставления его с данными опыта, наблюдения, эксперимента), почувствовали недостаточность этого способа как всеобщего критерия истинности знания, особенно когда дело касается \emph{научных теорий, обладающих большой степенью общности}.

Чтобы спасти принцип верификации, они стали предлагать всё более широкие способы толкования понятия «\emph{опытная проверяемость}», а с другой стороны ограничивать сферу её приложения (не все истинные идеи поддаются опытной проверке и т.д.).

Некоторые из них, например, английский философ \emph{К. Поппер}, полагают, что \emph{верифицируемость, проверяемость, следует заменить фальсифицируемостью}, т.е. поиском опытных данных, не подтверждающих, а опровергающих её.

Конечно, поиски фактов, опровергающих теорию, необходимы в науке, в частности, так устанавливаются \emph{пределы применимости} той или иной теоретической системы. Но этим путём никак нельзя доказать её объективную истинность.

Если эмпирическое наблюдение не является критерием, то, может быть, всеобщие принципы, аксиомы, правила логического вывода и т.п. \emph{истинны сами по себе в силу своей ясности, отчётливости, т.е. самоочевидны} и не требуют никакого доказательства, поскольку противоположное им просто немыслимо?

Но современная наука, критическая в своей сущности, \emph{не склонна уповать} ни на веру, ни на самоочевидность, а парадоксальность её утверждений стала обычным явлением.

Диалектико-материалистическая философия решает проблему критерия истинности, выявляя, что он находится в конечном счёте в деятельности, являющейся основой знания, т.е. в \emph{общественно-исторической практике.}

\emph{В чём сила практики как критерия истины?}

Критерий истинности познания должен обладать, как минимум, \emph{двумя качествами}.

\emph{Во-первых}, он несомненно должен носить \emph{чувственно-материальный характер}, выводить человека из сферы сознания, в предметный мир, ибо надо установить объективность содержания знания.

\emph{Во-вторых}, знание, особенно когда речь идёт о законах науки, носит всеобщий характер, а одна всеобщность доказывается другой всеобщностью. \emph{Одним единичным} и даже сколь угодно большой суммой их \emph{нельзя доказать} всеобщее и бесконечное.

Такой особенностью обладает \emph{практика человека}, в природе которой заключена всеобщность.

К тому же в практике всеобщее приобретает чувственно-конкретную форму вещи, процесса.

Иными словами, в практике объективность знания, носящего всеобщий характер, приобретает форму \emph{чувственно достоверного}.

Это, конечно, \emph{не означает}, будто с точки зрения диалектико-материалистической гносеологии \emph{надо каждое} понятие, всякий акт познания непосредственно проверять на практике, в производственной или иной материальной деятельности людей.

Реально процесс доказательства происходит в форме выведения одного знания из другого, т.е. \emph{в форме логической цепи рассуждений}, некоторые звенья которой проверяются путём выхода в практику.

Но не возникает ли тогда представление о существовании наряду с практикой \emph{критерия, основанного на логическом аппарате мышления}, на сопоставлении одного знания с другим?

Конечно, формы и законы логического вывода не зависят от отдельных актов практического действия, \emph{но это не означает}, что они вообще не связаны с практикой и не порождены ею.

\emph{Практическая деятельность «миллиарды раз должна была приводить сознание человека к повторению разных фигур, дабы эти фигуры могли получить значение аксиом»}. (\emph{В.И. Ленин}. Полн. собр. соч. т. 29, с. 172).

Сама \emph{практика не является застывшим состоянием, а процессом}, складывающимся из отдельных моментов, этапов, звеньев. Ведь познание может опережать практику того или иного исторического периода.

Существующей \emph{практики бывает недостаточно} для установления истинности тех теорий, которые уже построены наукой.

Всё это говорит об \emph{относительности критерия и самой практики}.

Но, \emph{во-первых}, другого, \emph{более объективного и точного нет}, а \emph{во-вторых}, этот критерий \emph{одновременно и абсолютен}, поскольку только на основе практики сегодняшнего или завтрашнего дня можно установить объективную истину.

\emph{Практика преодолевает свою ограниченность} как критерия истинности в процессе развития.

Развивающаяся \emph{практика очищает знание} от всего неистинного и способствует его прогрессивному развитию к новым открытиям и всё более полным, объективным истинам.

\section{Диалектика процесса познания}

\emph{Познание} осуществляется как \emph{переход из состояния незнания к состоянию знания}, от одного знания к другому, более глубокому, как движение к объективной, всё более полной, многогранной истине.

Процесс этот \emph{складывается из множества моментов}, сторон, необходимо связанных друг с другом.

Теория познания диалектического материализма стремится \emph{вскрыть} \emph{взаимодействие} основных компонентов познания, их роль в ходе достижения истины.

\subsection{Познание как единство чувственного и рационального}

Философия давно уже выделила \emph{два элемента}, составляющие познание: \emph{чувственное} (ощущения, восприятия и представления) и \emph{рациональное} (мышление в многообразных формах: понятиях, суждениях, умозаключениях, гипотезах, теориях).

Сразу же возникли вопросы: \emph{каково значение} этих элементов в возникновении и развитии знания, как они относятся друг к другу и т.п. Само собой разумеется, \emph{ответы} на эти вопросы не были одинаковыми.

Сторонники \emph{сенсуализма} (от латинского \emph{sensus} --- чувство, ощущение) полагали, что решающая роль в познании принадлежит чувственному моменту: ощущениям и восприятиям. Здесь \emph{есть верная мысль}, ибо, действительно, только посредством ощущений сознание человека связывается с внешним миром. Но само \emph{понимание природы ощущений} и восприятий человека, их роли в познании может быть различным.

Ощущения --- источник знания, \emph{но что же} является источником самих ощущений?

\emph{Идеалистический сенсуализм} (\emph{Беркли, Юм, Мах и др.}) счиатет ощущения и восприятия \emph{последней реальностью}, с которой мы имеем дело. Он либо вообще \emph{отрицает} существование независимой от познания реальности, либо \emph{провозглашает бессмысленной} саму постановку вопроса об источнике ощущений и восприятий.

При этом \emph{нередко спекулируют} на действительных противоречиях чувственного отражения действительности.

Так называемый \emph{«физиологический» идеализм}, возникший ещё в XIX в., \emph{односторонне толкуя} данные физиологии органов чувств, полагает, что внешний раздражитель выполняет лишь \emph{функции толчка, повода} для ощущения, но совершенно не определяет его содержания. Последнее зависит от некое «\emph{внутренней энергии}», которая специфична для каждого органа чувств.

При такой постановке проблемы, ощущения, по существу, \emph{изолируются} от внешнего мира, их содержание истолковывается как субъективное, которое в лучшем случае способно выполнять лишь \emph{роль символа, иероглифа} по отношению к предметам внешнего мира.

Аностический характер такого вывода \emph{очевиден}.

Другой крайностью в понимании ощущений является «\emph{наивный реализм»}. Его сторонник считают, что вещи и процессы, существующие вне сознания человека, являются \emph{буквально такими}, какими их непосредственно воспринимает человек. Человек и его нервная система \emph{якобы не оказывают} никакого воздействия на форму ощущений.

В действительности органы чувств \emph{оказывают влияние} на формирование ощущений.

\emph{Ощущение --- это субъективный образ объективного мира.}

Ощущения и восприятия, будучи источником знаний человека, обладают \emph{достоверностью}. В определённых границах они дают такие представления о внешнем мире, которые верно отражают действительность.

\emph{Согласованность} между чувственными данными и внешним миром является \emph{результатом эволюции} живых существ, их приспособления к окружающей среде.

Но данные чувственного отражения действительности, будучи источником знания, \emph{не составляют всего содержания} знания.

\emph{Тезис сенсуализма}, провозглашённый английским философом \emph{Локком} («\emph{Нет ничего в разуме, что первоначально не было бы в чувствах}»), выражает метафизическую ограниченность, которая носит название \emph{эмпиризма} (от греческого \emph{empeiria} --- опыт).

С точки зрения эмпиризма знание не только по своему источнику происходит из ощущений, но и всё его содержание \emph{исчерпывается} им.

\emph{За мышлением} эмпиризм оставляет только роль \emph{суммирования, упорядочения} данных \emph{опыта}, под которым подразумевается совокупность ощущений и восприятий человека.

\emph{Эмпиризм матераилистической философии XVII -- XVIII вв}. имел \emph{прогрессивное значение}, поскольку способствовал опытному исследованию природы, очищению знания от схоластического умозрения.

Впоследствии эмпиризм стал \emph{одним из источников агностицизма} и других ошибочных подходов, ибо, \emph{пренебрегая теоретическим мышлением}, он, по существу, толкал науку на оперирование устаревшими понятиями.

Современный эмпиризм существовал в XX в. в форме \emph{неопозитивизма}, или \emph{логического позитивизма}.

Он \emph{не выступает против мышления} вообще, но допускает его лишь в форме \emph{логических исчислений} (логического доказательства, операций со знаками).

Неопозитивисты стремились найти и выделить в современном научном знании некоторые \emph{исходные моменты} (\emph{высказывания и термины}), которые можно отнести к \emph{непосредственным чувственным данным}. Эти данные принимаются за \emph{базис знания}, всё остальное знание сводится либо к нему, либо к логическим правилам вывода, носящим характер \emph{конвенции} (соглашения между учеными).

Однако весь ход развития научного познания убедительно демонстрирует, что знание \emph{нельзя свести} к двум элементам: данным опыта и логическим операциям со знаками. Оно \emph{включает в себя всю} сложную, синтетическую деятельность человеческого разума.

Если эмпирики преувеличивают роль чувственного отражения, то представители другого направления, называемого \emph{рационализмом} (от латинского \emph{ratio} --- разум, рассудок) \emph{абсолютизируют роль мышления} в познании.

Чувственному созерцанию эмпириков рационалисты (\emph{Декарт, Спиноза и др}.) противопоставляют «\emph{сверхчувственное}», т.е. якобы независимое от чувственных данных «\emph{чистое мышление»}, способное без опоры на опыт логически дедуцировать новое знание.

Они выдвинули понятие \emph{интеллектуальной интуиции}, посредством которой разум, минуя чувственные данные, «\emph{непосредственно»} постигает сущность вещей и процессов. Роль чувственного опыта при этом \emph{принижалась}. Получалось, что опыт дает \emph{лишь толчок, повод} для деятельности мышления или \emph{служит простым подтверждением} умозрительных выводов.

Логически продолжая свои концепции, некоторые рационалисты, например, \emph{Декарт}, приходили к идее существования «\emph{врожденного знания}», в частности в виде основных \emph{понятий математики и логики.} Объявляя эти «врожденные идеи» \emph{абсолютными истинами}, рационалисты пытались \emph{дедуцировать} из них основное содержание научного знания.

Несколько смягченную, ослабленную форму рационализма представляет собой \emph{априоризм Канта}.

По мнению Канта, знание имеет \emph{два независимых друг от друга} источника: 1) \emph{данные чувственных восприятий}, составляющие содержание знания, и 2) \emph{формы чувственности и рассзадка}, носящие априорный (\emph{независимый от опыта}) характер.

Верной является мысль Канта, что знание возникает в результате \emph{синтеза чувственного и рационального}, однако эти два момента у него отгорожены друг от друга: чувственные восприятия связаны с воздействием на органы чувств не зависящих от сознания «\emph{вещей в себе}», тогда как рациональные формы познания (\emph{категории}) коренятся в априорных, доопытных способностях рассудка.

В результате, правильно поняв категории (наиболее общие, всеобщие понятия) как формы познания, \emph{Кант не увидел} того, что они являются таковыми лишь постольку, поскольку отражают действительные отношения, объективно существующие формы всеобщности.

Конечно, формы мышления существуют независимо от конкретного опыта, \emph{от опыта отдельного человека}, но они возникли и развились на основе чувственно-предметной деятельности человечества в целом.

\emph{Кант ошибался}, принимая их за формы, которые, по существу, врождены человеку.

Соотношение чувственного и рационального, данные опыта и мышления в познании было понято последовательно, в общем, \emph{только с позиций диалектико-материалистической теории познания}.

\emph{Познание начинается с живого, чувственного созерцания действительности.}

Чувственный опыт человека (ощущения, восприятия, представления) --- \emph{источник знания}, связывающий человека с внешним миром. Это \emph{не означает}, конечно, что каждый отдельный познавательный акт непосредственно начинается с опыта.

Знания \emph{не наследуются в биологическом смысле}, но они \emph{передаются} от одного поколения людей другому.

Существуют формы знания, обобщающие в теоретическом виде \emph{опыт предшествующих поколений}, эти формы независимы от особого опыта каждой отдельной личности.

\emph{Знание --- это не только то, что дают органы чувств.}

\emph{С помощью различных форм мышления знание выходит за пределы чувственного представления.}

Даже такое простое суждение, как «\emph{Роза красна}», представляет собой форму связи ощущений и восприятия человека на основе понятий о цветах, их окраске и т.д.

\emph{Без понятий человек} не может выразить в языке своего чувственного опыта.

\emph{Нет чистого» чувственного созерцания}.

У человека чувственное созерцание всегда \emph{пронизано мышлением}.

\emph{Но нет и «чистого» мышления}, последнее всегда связано с материалом чувственности, хотя бы в форме наглядных образов и знаков.

Живое чувственное содержание действительности можно считать непосредственным только в том смысле, что оно связывает сознание с миром вещей, их свойств и отношений, но оно \emph{обусловлено предшествующей практикой}, выработанным языком и т.п.

\emph{Если нет осмысления результатов ощущений, то нет и знания.}

Таким образом, \emph{знание является единством чувственного и рационального отражения действительности.}

\emph{Вне чувственного} представления у человека нет никакого реального знания.

Например, \emph{многие понятия современной науки} весьма абстрактны, и всё же они не свободны от чувственного содержания, связи с ним. Не только потому, что они своим происхождением \emph{обязаны в конечном счёте опыту людей}, но и потому, что по своей форме они существуют \emph{в виде системы чувственно воспринимаемых знаков}.

С другой стороны, знание не может обойтись без \emph{рациональной обработки данных опыта} и включения их в результаты и ход интеллектуального развития человечества.

\subsection{Уровни знания: эмпирическое и теоретическое, абстрактное и конкретное. Единство анализа и синтеза}

Чувственное и рациональное --- \emph{основные моменты} всякого познания.

Но в процессе познания можно выделить и \emph{различные уровни}, качественно своеобразные ступени знания, различающиеся между собой по полноте, глубине, всесторонности охвата объекта, по способу достижения основного содержания знания, по форме своего выражения.

К ним в первую очередь следует отнести такие уровни, как \emph{эмпирическое} и \emph{теоретическое.}

\emph{Эмпирический} --- это такой уровень знания, содержание которого в основном получено из опыта (\emph{из наблюдений и экспериментов}), подвергшегося некоторой рациональной обработке, т.е. выраженного определенным языком.

На эмпирическом уровне знания предмет познания отражается со стороны свойств и отношений, доступных чувственному созерцанию.

Например, даже такой объект современного физического знания, как \emph{элементарные частицы}, доступен для эмпирического познания. В \emph{камере Вильсона}, в мощных \emph{ускорителях} частицы чувственно воспринимаются исследователем в виде фотографии следов их движения и т.п. Результаты их наблюдений и измерений фиксируются \emph{определённым языком}.

\emph{Данные} наблюдения и зкспериментов являются тем \emph{эмпирическим базисом}, из которго исходит теоретическое исследование.

Получение этих данных приобретает такое большое значение в познании, что в некоторых науках складывается \emph{разделение труда}, при котором одни учёные специально занимаются \emph{экспериментальным исследованием}, а другие --- главным образом \emph{теоретическим}.

Не случайно говорят об \emph{экспериментальной} физике, биологии, физиологии, психологии и т.п.

\emph{Эксперимент} всё шире применяется и в различных отраслях \emph{наук об обществе}.

\emph{Теоретическое познание} --- отличный от эмпирического уровень исследования. На этом уровне объект отражён \emph{со стороны} его \emph{связей и закономерностей}, полученных не только в опыте, но и \emph{путём абстрактного мышления}.

\emph{Задача теоретического анализа} «заключается в том, чтобы видимое, лишь выступающее в явлении движение свести к действительному внутреннему движению...» (\emph{К. Маркс}). (К. Маркс и Ф.Энгельс. Соч., т. \emph{25,} ч. 1, с. 343).

\emph{Чувственное в теоретическом} знании служит некоторым \emph{исходным пунктом} и формой выражения достигнутых мышлением результатов в виде системы знаков.

В любой области научного познания мы встречаемся с разработкой теоретических построений, в которых знание не только \emph{выходит далеко} за пределы чувственного опыта, но и \emph{вступает нередко в противоречия} с непосредственными чувственными данными.

Эти противоречия носят \emph{диалектический характер}; они не опровергают ни теоретических положений, ни эмпирических данных.

Взять, к примеру, \emph{теорию относительности Эйнштейна}, \emph{квантовую механику}, \emph{геометрию Лобачевского} и т.п.

Опыт не давал данных для утверждения о постоянстве скорости света.

Когда \emph{Планк} выдвинул предположение о том, что \emph{свет испускается квантами} (порциями), эта гипотеза также не основывалась на эмпирически установленных фактах.

\emph{Лобачевский}, выдвигая постулат «\emph{Через точку, не лежащую на данной прямой, проходят но крайней мере две прямые, лежащие с ней в одной плоскости и не пересекающие её}», не только не опирался на какие-либо наглядные представления о пространстве, но и вступал даже в противоречие с ними.

Эмпирический и теоретический уровни познания \emph{тесно связаны} между собой.

\emph{Во-первых}, теоретические построения возникают \emph{на основе обобщения} предшествующих знаний, в том числе и полученных из наблюдений, экспериментов. Это, конечно, \emph{не означает}, что все теории непосредственно исходят из опыта, некоторые из них в качестве исходного используют уже имеющиеся понятия и теории. Но если взять не отдельные теории, а теоретическое знание в целом, то оно, конечно, \emph{прямо или косвенно связано} с эмпирическим знанием.

Теоретическое знание \emph{может и должно опережать} данные опыта.

\emph{Теоретическая физика} пришла к идее существования \emph{античастиц} задолго до их экспериментального обнаружения.

Но \emph{было бы неверным} полагать, что в данном случае за наблюдением и экспериментом остаётся только \emph{роль регистратора} результатов теории.

Когда учёные обнаружили в космических лучах \emph{позитрон}, то это было блестящим опытным \emph{подтверждением} квантового уравнения английского физика \emph{Дирака}, из которого следовало, что существуют электроны с двумя противоположными зарядами --- отрицательным и положительным.

Однако эмпирические наблюдения \emph{внесли коррективы} в рассуждения \emph{Дирака}, который частицу, симметричную электрону, считал не позитроном, а протоном.

Таким образом, развитие познания предполагает \emph{непрерывное взаимодействие опыта и теории}.

\emph{Абсолютизация} одного из них пагубно сказывается на развитии науки.

Тем не менее \emph{целью научного знания} являются не эксперименты, а теории.

\emph{Развитие науки определяется} не столько количеством добытых эмпирических данных, сколько количеством и качеством выдвинутых и достаточно обоснованных теорий.

Современная наука во многих областях естествознания и обществознания, накопив значительный эмпирический материал, испытывает \emph{потребность в новых фундаментальных теориях}, на основе которых можно было бы обобщить, систематизировать этот материал и двигаться дальше.

\emph{Уровень знания} определяется не только тем, каким способом получено знание --- опытным путем или теоретическим мышлением, но и тем, как в нем отражен объект --- \emph{во всех} своих связях и опосредованиях \emph{или с одной}, хотя и очень важной, стороны.

С этой точки зрения знание принято разделять на \emph{конкретное} и \emph{асбтрактное}.

В принципе \emph{знание стремится стать конкретным}, \emph{т.е. многосторонним, охватывающим объект как некоторое целое}.

Но сама конкретность может быть \emph{разной}.

В чувственном опыте человека объект может быть дан \emph{во многих связях}, отношениях. Но, как известно, эмпирическому знанию доступны преимущественно \emph{внешние} связи и отношения, поэтому чувственная конкретность ограничена по своему содержанию. Она \emph{не даёт знания} закономерностей, целостного отражения явлений.

Чтобы подняться на более высокую ступень конкретности, \emph{надо сначала} взять предмет или группу предметов с какой-то одной определённой стороны, абстрагировавшись от других сторон. В этом смысле само мышление можно рассматривать как способ постижения действительности \emph{посредством абстракции}.

\emph{Абстракция} (от латинского \emph{abstractio} -- отвлечение) --- важнейший способ отражения мышлением объективной действительности.

Посредством абстракции выделяется \emph{существенное в данном отношении} свойство, сторона.

Выделяя какое-либо свойство или отношение, мысль может \emph{абстрагироваться и от самих вещей}, явлений, которым принадлежит данное свойство, отношение. Так возникают \emph{качества} «белизна\emph{»}, «красота», «наследственность», «электропроводность» и т.п.

Подобного рода абстракции в логике носят название \emph{абстрактных предметов}, т.е. предметов только мыслимых.

В процессе абстрагирования мышление \emph{не ограничивается} выделением и изоляцией некоторого чувственно доступного свойства или отношения предмета (в таком случае абстракция не преодолеет недостатков чувственной конкретности), а \emph{пытается обнаружить} связь, скрытую и недоступную для эмпирического познания.

«\emph{Погружение в абстракцию}» является способом более глубокого постижения объекта. Современная наука, сделавшая \emph{абстракцию главным орудием} проникновения в сущность вещей и процессов, подтверждает это.

Так, \emph{А. Эйнштейн}, \emph{В. Гейзенберг} отмечали, что посредством \emph{математической абстракции} современная физика схватывает объективную природу явлений.

Но никакая \emph{абстракция не всесильна}.

Посредством абстакции человеческое мышление выделяет в объекте \emph{отдельные свойства}, закономерности.

Посредством абстракции предмет в мысли \emph{анализируется, разлагается} на абстрактные определения, образование которых выступает средством достижения нового, \emph{конкретного знания}.

Это движение мысли носит название \emph{восхождения о т абстрактного к конкретному}.

В процессе такого восхождения и происходит воспроизведение мыслью \emph{объекта в его целостности}.

Впервые этот процесс был описан \emph{Гегелем}.

Но если \emph{Гегель считал}, что в процессе восхождения от абстрактного к конкретному \emph{создаётся сам объект}, то диалектико-материалистическая философия видит здесь только \emph{воссоздание объекта} в мысли во всей доступной полноте его связей путём синтеза различных абстрактных (в данном случае односторонних) определений.

\emph{Движение от чувственно-конкретного через абстрактное к конкретному в мышлении является законом развития теоретического познания}.

Конкретное в мышлении является \emph{самым глубоким} и содержательным знанием.

Истина не может быть объективной, если она не конкретна, если она не представляет собой \emph{развивающуюся систему знания}, если она непрерывно не обогащается новыми элементами, выражающими новые стороны, связи объекта, углубляющими прежние научные представления.

Истина всегда является \emph{теоретической системой знаний}, направленной на отражение объекта в его целостности.

Движение от чувственно-конкретного через абстрактное к конкретному в мышлении, происходящее \emph{на основе практики}, включает в себя такие приемы, как \emph{анализ} и \emph{синтез}.

Абстрагирование предполагает \emph{мысленное расчленение} явления, предмета на его свойства, отношения, части, ступени развития и т.д.

Например, человек, наблюдая \emph{солнечное затмение}, разлагает это явление на отдельные составляющие его моменты. Он видит, как на Солнце с западной стороны \emph{надвигается} круглый черный диск, который или полностью на короткое время \emph{покрывает} Солнце и \emph{постепенно сходит} с него, или закрывает только часть Солнца, продвигаясь на восток. В то же время он \emph{наблюдает} изменения в атмосфере Солнца и его короне, \emph{выделяет} так называемую хромосферу, протуберанцы и т.п.

С другой стороны, создание конкретного в мышлении происходит на основе \emph{синтеза, сведения к единству многообразных} свойств и отношений, обнаруженных как в данном предмете, так и в других.

Например, современная наука \emph{сводит к одному принципу} выделение солнечной энергии и термоядерную реакцию на Земле.

Это \emph{соединение в мышлении} различных сторон, явлений, свойств само возможно по объективным законам.

Синтетическая способность мышления лежит в \emph{основе научного творчества}, познание не может сделать действительного шага вперёд, только анализируя или только синтезируя.

Анализ \emph{предшествует} синтезу, но и сам возможен \emph{только на основе} результатов проделанной синтетической деятельности.

\emph{Связь} анализа и синтеза --- органическая, внутренне необходимая.

\subsection{Историческое и логическое. Формы воспроизведения мышлением объекта}

Воспроизвести в мышлении объект во всей его объективности, конкретности --- значит \emph{постичь его в развитии}, в истории.

В многообразии способов познания выделяют \emph{два метода}: \emph{исторический} и \emph{логический.}

\emph{Исторический метод} связан с освещением различных этапов развития объектов в их \emph{хронологической последовательности}, в конкретных формах исторического проявления.

У этого метода имеются свои \emph{достоинства}, поскольку он даёт возможность описать исторический процесс во всем его \emph{многообразии}, с учётом его неповторимых, индивидуальных \emph{особенностей}.

Но \emph{чтобы вскрыть историю} объекта, предмета, выделить главные этапы его развития и основные исторические связи, \emph{необходимо теоретическое понятие} об этом предмете, его сущности.

\emph{Другой метод --- логический} --- как раз и ставит своей задачей \emph{воспроизвести} в теоретической форме, \emph{в системе абстракций} сущность, основное содержание исторического процесса.

При этом исходным пунктом исследования становится рассмотрение предмета в его \emph{наиболее развитом виде}.

Логический метод, имеет свои \emph{достоинства} и некоторые преимущества перед историческим.

\emph{Во-первых}, он отражает объект \emph{в самых} его существенных связях.

\emph{Во-вторых}, он даёт о\emph{дновременно} возможность постичь его историю.

Логический метод в теоретической форме отражает одновременно \emph{и сущность} предмета, необходимость, закономерность \emph{и историю} его развития, ибо, воспроизводя предмет в высшей, зрелой форме его, включающей \emph{как бы в снятом виде} предыдущие ступени, мы тем самым познаём и основные, главные вехи его истории.

Логический метод \emph{не умозрительное} выведение одного понятия из другого, он также основывается на отражении реального объекта, однако только в необходимых моментах его развития, \emph{не обязательно следуя} временной и эмпирически конструируемой связи этих моментов, как она выступала на поверхности.

Логический метод имеет и то \emph{преимущество перед историческим}, что даёт возможность соединить в себе два необходимых элемента исследования: \emph{изучение структуры} данного предмета с \emph{пониманием его истории}, в их неразрывном единстве.

Исторические и логические исследования \emph{тесно связаны} между собой.

Исторический метод без логического \emph{слеп}, а логический без изучения реальной истории \emph{беспредметен}.

На основе единства исторического и логического можно в зависимости от конкретных задач исследования \emph{сделать специальным предметом} теоретического анализа как развитие объекта, так и его современную структуру.

Исторический метод исследования вполне правомерен, когда задачей исследования является изучение \emph{истории самого предмета}. Однако и в этом случае в качестве исходного принципа должно выступать единство логического и исторического, т.е. изучение истории предмета \emph{во всем многообразии}, со всеми зигзагами и случайностями, должно подводить нас к пониманию его логики, закономерностей, основных вех его развития.

Не только логика приводит к истории, но и историческое исследование исходит из некоторых понятий, \emph{в качестве своего результата} подходит к формированию новых понятий, обобщающих историю, охватывающих сущность предмета.

\emph{Логическое воспроизведение} объекта в мысли происходит в определённых формах.

\emph{Форма мышления --- это определённая структура мысли}, посредством которой в системе последовательных, взаимосвязанных абстракций отражается объективная реальность --- предмет в его историческом развитии.

\emph{Абстракции отличаются друг от друга} не тем, что в одной постигается один объект природы или общества, а в другой --- иной, а их \emph{функцией} в мышлении.

Различные мыслительные структуры сложились \emph{в связи с целями} познавательной деятельности человека, благодаря им тот или иной предмет постигается всесторонне, в его действительной расчленённости и целостности.

В качестве \emph{основных логических форм мышления} принято выделять \emph{суждение, понятие, умозаключение.}

Под \emph{суждением} в логике по традиции понимается \emph{мысль, утверждающая или отрицающая что-нибудь о чём-нибудь}: «Водород --- химический элемент», «Товар имеет стоимость» и т.п.

В суждении \emph{можно видеть} все характерные особенности познающей мысли.

Процесс мышления \emph{начинается там}, где происходит \emph{выделение} отдельных признаков, свойств предметов и образование хотя бы элементарных форм абстракции.

Всякое знание находит свое выражение \emph{в форме суждения} или системы суждений.

Даже выражение результатов \emph{живого, чувственного созерцания} в рациональной форме приобретает форму суждения. Например: «Этот дом больше другого».

В любом суждении можно \emph{выделить связь} единичного и общего (всеобщего), тождества и различия, случайного и необходимого и т.п.

Познание закономерно доходит до \emph{выделения всеобщего} и \emph{существенного} в предмете, т.е. до \emph{понятия}.

Образование понятия является результатом длительного процесса \emph{погружения мысли} человека в объект, в нём подводится \emph{итог} того или иного этапа познания объекта и в концентрированном виде выражено достигнутое знание.

Вскрыть \emph{диалектику движения понятий} --- это значит обнаружить закономерности их развития.

Развитие понятий происходит \emph{в нескольких направлениях}:

1) возникают \emph{новые понятия}, отражающие предметы, явления, которые стали объектом теоретического исследования;

2) \emph{старые понятия} конкретизируются и поднимаются на более высокий уровень абстракции.

Особое значение имеет \emph{переосмысление}, уточнение и обогащение основных понятий, являющихся \emph{категориями} в данной науке.

\emph{Революции}, качественные изменения в науке сопровождаются \emph{ломкой}, изменением содержания старых и возникновением новых понятий, которые меняют строй и метод мышления учёных.

Понятие не существует вне, \emph{без своего определения} того или иного вида, в процессе которого происходит \emph{подведение} данного понятия под другое, более широкое (т.н. род-видовое определение как ведущий вид).

Чтобы вскрыть сущность предмета, надо \emph{выявить существенно общее}.

Однако одного общего \emph{ещё недостаточно} для определения понятия. Поэтому определение всегда включает в себя наряду с \emph{указанием рода}, т.е. более общего понятия, и признаки, составляющие \emph{особенность вида}.

Например, понятие «\emph{звезды}» можно определить следующим образом: «Звезды --- это \emph{небесные тела} (род), \emph{излучающие свет}» (вид).

Образование понятий в процессе мышления невозможно без \emph{умозаключения}, посредством которого на основе ранее установленного знания, не обращаясь к чувственному опыту, можно приобрести новое (\emph{выводное}) знание.

\emph{Умозаключение --- это процесс выведения одних суждений (заключений) из других (посылок), т.е. определённая система суждений.}

В умозаключении выражена способность теоретического мышления \emph{выйти за пределы} того, что дано ему непосредственным чувственным опытом, наблюдениями и экспериментеми.

Если бы умозаключение не давало бы возможности получить новое знание, то человек, например, \emph{никогда бы не определил} расстояние от Земли до других небесных тел, \emph{не знал бы} химического состава звезд, \emph{не смог бы} проникнуть в мир атома и составляющих его элементраных частиц.

Заключение в умозаключении вытекает из посылок, но \emph{не просто повторяет} их, а даёт нечто новое, обогащенное знание.

Суждения, понятия и умозаключения \emph{связаны между собой}. Изменения одного влекут за собой изменения другого.

\emph{Взаимозависимость} понятий, суждений и умозаключений проявляется в \emph{процессе мышления}, которое заключает в себе:

\begin{itemize}
\item \emph{выделение} свойств, признаков предмета (\emph{суждение});
\item \emph{подытоживание} предшествующего знания, образование научных \emph{понятий};
\item \emph{переход} от одного, ранее достигнутого знания к другому (\emph{умозаключение}).
\end{itemize}

Все перечисленные моменты присутствуют в \emph{научной теории, которая представляет собой относительно замкнутую и достаточно обширную} \emph{систему знаний}.

Суждения образуют \emph{принципы} и \emph{высказывания} теории, понятия --- её \emph{термины}, различные умозаключения --- способы получения знаний в ней с помощью \emph{вывода}.

\emph{Функция теории} --- не только \emph{привести} в систему достигнутые результаты познания, но и \emph{открывать} путь к новым знаниям.

\emph{Теории в науке бывают разные} в зависимости от предмета, который в них отражен, от того, насколько широкий круг явлений они описывают, от способов доказательства, применяемых в них.

Своеобразной формой теории является так называемая \emph{метатеория,} т.е. \emph{теория о природе, строении и функциях самих теорий}.

Возникновение \emph{метатеорий} и \emph{метанаук} --- это нечто новое, характерное для развития познания на рубеже XX и XXI веков. Это \emph{свидетельство} о необходимости исследования структуры, способов построения и путей развития теории.

Для нашего времени также характерным является \emph{процесс интеграции теори}й, создания так называемых \emph{объединенных теорий}.

Создание теорий, соединяющих в себе несколько теорий, построенных в разное время, для объяснения разных явлений, является \emph{подтверждением} движения знания по пути объективной истины.

В современном познании соединяются теории, созданные в т.ч. разными науками.

Решение проблем, связанных с построением метатеорий, с объединением, интеграцией теорий, требует \emph{усиленной разработки логики}.

\subsection{Диалектическая и формальная логика}

Изучением форм мышления занимается \emph{логика}. Её основоположником принято считать \emph{Аристотеля}, в трудах которого впервые были сведены воедино и систематизированы проблемы, которые впоследствии получили наименование логических.

\emph{В новое время} большой вклад в развитие логики был сделан \emph{Ф. Бэконом} и другими мыслителями.

К XVII -- XVIII вв. в лоне философии сформировалась \emph{традиционная, или классическая, формальная логика}.

Законы традиционной логики --- это \emph{законы тождества}, недопустимости противоречия (\emph{непротиворечия}), \emph{исключённого третьего} и \emph{достаточного основания}.

Эта логика рассматривала формы мышления в качестве \emph{принципов и самого бытия}.

Дальнейшее развитие формальной логики связано с применением, с одной стороны, \emph{новых средств логического анализа}, а с другой, с изучением \emph{новых форм доказательства}, выдвигаемых развитием научного познания.

Была разработана \emph{математическая символика} для решения логических задач.

Использование формальной \emph{логики в математике}, в частности с целью её обоснования, вызвало развитие и самой формальной логики. Так возникла новая разновидность формальной логики, которая носит название \emph{символической, или математической}.

В настоящее время предметом логического анализа служат по преимуществу \emph{искусственные, формализованные языки.} Изучается их синтаксис и семантика.

\emph{Логический синтаксис} формулирует \emph{правила} построения и преобразования языковых выражений лишь \emph{с формальной стороны}, без учёта выраженного ими содержания.

\emph{Логическая семантика} анализирует языковые системы с целью выявить \emph{значение} их элементов.

Формально-логический анализ теоретического знания дал \emph{большие результаты}.

Так, \emph{кибернетика была бы невозможна} без создания метода \emph{анализа знания на основе искусственных, формализованных языков}.

На базе этого метода можно \emph{проанализировать} имеющееся знание, соответствующим образом \emph{перестроить} его, \emph{выразить} по возможности в строго формализованной системе и \emph{передать} некоторые функции человеческого мышления машине.

Анализ знания средствами формальной логики \emph{способствует} и достижению нового знания, поскольку он \emph{помогает обнаружить} некоторые недостающие элементы, звенья, необходимые для построения строго формализованной теории, и направить человеческую мысль на их поиски.

Развитие логики шло \emph{не только по пути} выделения формальной логики в самостоятельную науку и превращения её в символическую со своим специфическим предметов и методом его изучения.

Наряду с этим в рамках философии развивалось \emph{учение о формах и методах теоретического мышления, ведущих к объективной истине}.

Материалистическая диалектика, продолжая эту линию развития, выступает одновременно не только как теория познания, но и как \emph{диалектическая логика}, включая в себя исследования форм мышления с помощью диалектических законов и категорий.

Диалектическая логика (её также нередко именуют \emph{содержательной логикой}) возникла как \emph{продолжение и развитие предшествующих логических учений}, она не отрицает значения формальной логики, а определяет её место в изучении научного знания.

Диалектическая логика не существует и не может существовать вне материалистической диалектики, так как она раскрывает значение наиболее общих законов развития объективного мира \emph{для движения мышления к истине}, т.е. становится наукой о совпадении содержания знания с объектом, \emph{наукой об истине в её содержательном аспекте}.

Диалектическая логика выступает логикой \emph{в качественно ином} по сравнению с формальной логикой смысле.

\emph{Диалектическая логика} не рассматривает форм мышления лишь с точки зрения их строения, \emph{не отвлекается от} выраженного в них \emph{конкретного содержания}.

Она берёт формы мышления \emph{не в застывшем}, не в изолированном виде, а во взаимосвязи, \emph{в движении, развитии}.

Если формальная логика сосредоточивается главным образом на анализе \emph{уже сложившихся} понятий, теорий, то диалектическая логика стремится вскрыть логические принципы перехода к новому знанию, \emph{исследует образование и развитие понятий и теорий}.

Основные \emph{требования} диалектической логики таковы:

\begin{itemize}
\item \emph{всесторонний охват} предмета в мышлении;
\item рассмотрение предмета \emph{в его развитии, самодвижении};
\item \emph{учёт} всей общественно-исторической \emph{практики};
\item нацеленность на \emph{поиски конкретной истины} (абстрактной истины нет).
\end{itemize}

\subsection{Образование и развитие научной теории. Интуиция}

Диалектико-материалистическая философия, её гносеологический раздел изучают \emph{движение научного знания}, вычленяя в нём формы и законы, фундаментальные понятия и принципы, следуя которым мышление приходит к объективной истине.

\emph{Фундаментальные понятия и принципы} в науке возникают как результат творческой деятельности людей.

\emph{Что же такое} \emph{научное творчество}? Следует ли творческая деятельность учёного каким-нибудь законам, или она абсолютно свободна, не связана никакой логикой?

Конечно, на \emph{творчество}, как мы увидим, влияет множество факторов и внелогического порядка, однако оно \emph{в основе своей} представляет собой деятельность человеческого разума, т.е. рационально, и, следовательно, \emph{является объектом логического анализа}.

\emph{Научное исследование} начинается с постановки \emph{проблемы}.

Понятие проблемы, как правило, связывается с \emph{непознанным}, и поэтому можно дать первоначальное \emph{определение проблемы:} то, что \emph{непознано} человеком и \emph{что нужно познанть.}

В этом довольно несовершенном определении содержится нечто весьма важное --- \emph{момент долженствования}, как то, что направляет процесс исследования.

Однако нетрудно заметить, что между \emph{областью непознанного} и \emph{должного} дистанция довольно значительная.

Человек многого не знает, а в принципе нет ничего такого, чего он \emph{не хотел бы знать}.

Необходимо, однако, выделить то, что он \emph{не знает, но что может знать на данной ступени} развития. А для этого уже требуется некоторое знание.

\emph{Проблема} --- хотя это звучит как парадокс --- \emph{не просто незнание, а знание о незнании.}

\emph{Проблемы вырастают} из потребностей практической деятельности человека, в виде некоторого \emph{стремления к новому знанию}.

\emph{Наука должна дорасти} до того, чтобы иметь необходимые и достаточные основания для постановки определённой проблемы.

\emph{Постановка проблемы} обязательно включает в себя какое-то предварительное, пусть несовершенное, знание путей её разрешения.

\emph{Умение правильно поставить проблему}, определить реальную \emph{потребность в новом знании}, которая в сложившихся условиях может быть удовлетворена, --- это уже почти пройти добрую половину пути к достижению нового знания.

Но как для постановки проблемы, так и тем более для её разрешения требуются \emph{факты}.

\emph{Термин «факт»} в литературе употребляется \emph{не однозначно}.

Фактом называют 1) \emph{само явление} (вещь, процесс объективной, или субъективной реальности), \emph{а также} 2) \emph{знание}, обладающее своими особенностями.

В данном случае нас интересует факт \emph{во втором} значении этого термина.

\emph{Какое знание называется фактическим?}

Фактами следует считать в первую очередь \emph{положения, полученные эмпирическим путём}, т.е. посредством наблюдения и фиксации его результатов.

\emph{Теория основывается на фактических данных}. Однако, как уже отмечаюсь, теории исходят из достоверного знания, независимо от того, как оно получено, эмпирически или умозрительно (теоретически).

Для постановки проблемы, её решения, проверки выдвинутых положений необходимо \emph{знание, объективная истинность которого установлена}. Это достоверное знание является \emph{фактом}, на который опираются в ходе исследования.

Фактом современной науки являются как результаты эмпирического научного наблюдения, \emph{так и законы}, достоверность которых установлена на практике.

Достоверность знания --- необходимое условие его \emph{превращения в факт}, поэтому о фактах и говорят как об \emph{упрямой вещи}, которую надо принимать вне зависимости от того, нравится ли она, удобна для исследователя или нет. Все остальные признаки факта, например его \emph{инвариантность, т.е. некоторая независимость от системы, в которую он включен}, являются производными от его достоверности.

Факт --- это \emph{то, что доказано} в качестве объективно-истинного и в этом своём содержании остаётся таковым независимо от того, в какую систему оно включается.

\emph{Гипотезы} и \emph{догадки} могут рассыпаться, не выдерживая критики, критерия практики, но факты, на основе которых они строились, остаются и переходят из одной системы знания в другую.

\emph{Накопление фактов} --- важнейшая часть научного исследования, но само по себе оно не решает проблемы. Необходима \emph{система знания}, представляющая и объясняющая интересующее нас явление или процесс.

Такая система может находиться на \emph{разных уровнях}: догадка, гипотеза, достоверная научная теория.

\emph{Догадка} --- это первоначальное \emph{предположение}, которое ещё в достаточной мере не исследовано, не выяснены его логические и эмпирические основания. Например, первоначальная мысль \emph{Резерфорда} и \emph{Содди} о радиоактивном распаде было только догадкой, развитой последующими исследованиями до уровня научной гипотезы.

Закономерно поставить вопрос: \emph{как возникают догадки}, почему исследователю приходит в голову эта, а не другая мысль, положенная в основу объяснения фактов?

При ответе на эти вопросы нельзя миновать понятия \emph{интуиции}.

Известно, что \emph{новые идеи}, изменяющие прежние представления, возникают, как правило, не в результате строгого логического выведения из предшествующего знания и не как простое обобщение опытных данных, а \emph{как некоторого рода скачок} в движении познающего мышления.

Мышление совершает подобные скачки \emph{в силу своей природы}, в силу непосредственной связи с практическим действием, которое толкает мысль в поисках новых результатов \emph{за пределы чувственно данного} и строго логически обоснованного.

«Всякий математик, и всякий естествоиспытатель согласится, что без воображения, без изобретательности, без способности придумывать гипотезы и планы нельзя выполнять ничего, кроме «механических» операций, то есть манипулирования аппаратами и применения вычислительных алгоритмов. Создание гипотез, изобретение технических приспособлений и придумывание экспериментов --- явные случаи творческих процессов, или, если предпочитаете, \emph{интуитивных действий}, противополагаемых «механическим» операциям» (\emph{М. Бунге}. Интуиция и наука. М., 1967, с. 109).

Но \emph{это не означает}, что интуиция ничем не обусловлена, возникает из ничего. Она \emph{отталкивается} от предшествующего уровня эмпирического и теоретического знания объекта.

Немалое значение приобретают здесь \emph{способности и опыт мыслителя}, строй его мышления.

На интуицию оказывают воздействие \emph{различные случаи} из его жизни. Влияние этих случайных факторов, быстрота и внезапность и выглядят как «\emph{озарение}».

История научных открытий \emph{полна легенд} по поводу тех случайностей, которые послужили толчком для гениальной интуиции.

Здесь и «\emph{яблоко Ньютона}», и «\emph{сон Менделеева»}, и т.п.

Но, не отрицая возможности подобных случаев, за каждым актом интуиции необходимо видеть \emph{напряжение человеческой мысли}, её настойчивый поиск решения поставленной проблемы.

В интуиции \emph{в свёрнутом} (\emph{или неразвёрнутом}) виде заложен опыт предшествующего общественного и индивидуального интеллектуального развития человека.

В интуиции \emph{нет ничего мистического}, её непосредственность относительна, и в дальнейшем интуитивно выдвинутые теоретические положения подвергаются сознательной \emph{логической обработке}, в результате которой первоначальная догадка либо опровергается, отвергается, как лишенная достаточных оснований, либо приобретает форму научно \emph{обоснованной гипотезы}.

\emph{Переход догадки в гипотезу} включает в себя нахождение аргументов, превращающих, по выражению \emph{Эйнштейна}, «чудо в нечто постижимое».

Здесь вступает в свои права логика, без которой \emph{интуиция повисает в воздухе}. Происходит процесс \emph{мобилизации имеющегося знания}, поиски новых фактов, которые превратили бы догадку в гипотезу.

\emph{Гипотеза --- это такой компонент предметного содержания сознания, в основе которого лежит предположение, опирающееся на ряд уже имеющихся знаний, но ещё явно недостаточных для принятия этого компонента предметного содержания сознания как достоверного знания}.

\emph{Само знание представляет собой такое предметное содержание сознания, которое достоверно, соответствует своему объекту, т.е. истинно.}

\emph{Истина же ---} \emph{это свойство знания, и только знания, представляющее собой соответствие заключённого в знании предметного содержания действительности.}

Обоснование и доказательство гипотезы предполагает \emph{поиск новых} фактов, постановку экспериментов, анализ прежних результатов познания.

Иногда \emph{для объяснения} одного и того же процесса выдвигается \emph{несколько гипотез}, которые «испытываются» разными способами (метод множественных гипотез).

При выборе гипотез известное значение приобретают даже такие моменты, как \emph{простота}, \emph{экономность}, \emph{эстетичность} (\emph{красота}), которые служат вспомогательными средствами определения наиболее правдоподобной теоретической системы.

\emph{Из множества} равноценных или почти равноценных в других отношениях гипотез \emph{предпочтение} отдаётся той, которая \emph{проще, яснее, экономнее и красивее} ведёт к своей цели.

Однако экономность, простота, эстетичность (красота) выступают только как некоторые \emph{средства выбора} среди равноценных гипотез, \emph{но не как критерии истинности} гипотезы.

Таким критерием является \emph{только практика} во всём её многообразии.

\emph{Обоснование и доказательство гипотезы превращает её в теорию.}

Теория не есть нечто абсолютное, она \emph{относительно завершённая} система знания, меняющаяся в ходе своего развития.

Изменение в теории происходит \emph{путём включения} в неё новых фактов и выражающих их понятий, \emph{уточнения} принципов.

Однако потом наступает момент, когда \emph{обнаруживается противоречие}, неразрешимое в рамках данной теории. Определить этот критический для данной теории момент можно путём конкретного анализа.

Когда это время наступает, \emph{совершается переход к новой теории} с другими или уточнёнными принципами.

Между новой и старой теориями существуют \emph{сложные отношения}, одно из которых выражено в \emph{принципе соответствия.}

Согласно этому принципу, новая теория приобретает право на существование, когда прежние теории оказываются некоторыми её \emph{предельными моментами}.

\emph{Например, классическая физика} является определённым частным случаем, пределом современных теорий.

Принцип соответствия выражает одновременно и \emph{преемственность} и \emph{развитие} знания.

Если объективная истинность какой-либо теории была установлена, то эта теория \emph{не может исчезнуть бесследно}, последующая теория лишь ограничивает сферу её применения.

Можно определить \emph{правила перехода} от новой теории к старой.

\emph{Включение} какой-либо теории в более широкую, общую теорию помогает установить её достоверность в более точно фиксируемых границах.

\subsection{Практическая реализация знаний}

\emph{Знание возникает} и развивается \emph{на основе} практической деятельности и \emph{служит ей}, поскольку \emph{создаёт прообразы} возможных и необходимых человеку вещей, процессов.

Знание \emph{должно быть}, в конце концов, так или иначе практически реализовано, \emph{применено}.

Для своей реализации \emph{знание должно} приобрести соответствующую форму, \emph{стать идеей.}

В философской литературе термин «идея» часто употребляется в широком значении как \emph{всякая мысль}, \emph{всякое знание} независимо от формы: понятие, суждение, теория и т.п.

Однако существует и другое, \emph{более точное значение} этого термина.

\emph{Идея --- это мысль, достигшая высокой степени объективности, полноты и конкретности и в то же время нацеленная на практическую реализацию}.

Следовательно, знание, чтобы быть реализованным, должно стать идеей, в которой \emph{слиты воедино} три момента:

1) конкретное, \emph{целостное знание} об объекте;

2) \emph{стремление к практической реализации}, к материальному, объективированному воплощению;

3) цель, \emph{проект действия субъекта}, план изменения им объекта.

Такой характер имеют \emph{идеи науки}, на основе которых происходит перестройка производства, глубокие социальные изменения в обществе.

\emph{Мы говорим об идеях} построения совершенного, справделивого общества, об идеях завоевания космического простаранства, об идеях мирного использования атомной энергии и т.п.

Идеи практически \emph{реализуются людьми} с помощью не только материальных средств (орудий труда, техники), но и с помощью духовных сил человека (\emph{воли, эмоций} и т.п.).

\emph{Решимость человека} в деле преобразования природы и общества основывается на знании, которое даёт ему \emph{интеллект, мышление}, которые, в свою очередь, всегда с необходимостью \emph{связаны с волей}, направляющей человека к преобразованиям мира.

У человека должно \emph{созреть решение действевать} в соответствии с идеей.

В формировании идеи большое место занимают \emph{убеждения} людей в истинности идеи, в необходимости действия в соответствии с ней, в реальной возможности воплощения идеи в действительность.

Убежденность субъекта или его \emph{сознательная, обоснованная} \emph{вера в правомерность своих действий, основанных на знании,} \emph{признаётся в диалектико-материалистической философии, в её теории познания.}

\emph{Диалектический материализм не признает замены знания верой} или привычкой, фанатичной веры. Он строго различает \emph{слепую веру} в догматы и \emph{убежденность} человека, основанную на знании объективной реальности и себя самого.

Научное знание должно перейти в \emph{личное убеждение человека}, которое создает решимость к действию, направленному на изменение тех или иных процессов.

Процесс практической реализации идей, превращения их в предметный мир, противостоящий человеку, носит название \emph{опредмечивания идеи}.

Идеи должны \emph{материализоваться}, принять форму предмета.

Опредмечивание имеет \emph{две} стороны: 1) социальную и 2) \emph{гносеологическую}.

\emph{Социальный аспект опредмечивания} связан с выявлением отношения, существующего между предметом, созданным деятельностью человека, и самим человеком.

Так, в определённых социальных условиях продукты деятельности людей превращаются в самостоятельную, \emph{чуждую им силу}, а отношения людей друг к другу принимают форму вещественного характера (\emph{стихийный обмен товаров}).

Этот \emph{вид опредмечивания} получил в философии название \emph{отчуждения,} о чём речь будет идти ниже.

Рассматривая опредмечивание с теоретико-познавательной, гносеологической стороны, необходимо поставить \emph{вопрос о соответствии} полученного в практике предмета той идее, которая в нём реализована.

Когда идея воплощается в действительность, то тем самым решается вопрос о её объективной истинности, \emph{отметается кажущееся, иллюзорное} в ней.

Этот процесс обнаруживает \emph{определённое несоответствие} между идеей и полученным в практике результатом, которое возникает либо вследствие несовершенства идеи, недостаточности содержащегося в ней знания и способов реализации, либо вследствие отсутствия необходимых материальных и духовных средств, условий для наиболее полного воплощения идеи в объективную реальность.

Поэтому опредмечивание \emph{подытоживает} один цикл исследований \emph{и открывает} новый.

Наконец, полученный в практике предмет подвергается анализу с точки зрения его \emph{соответствия разумным (рациональным) целям человека}.

\emph{Разумное, рациональное} не дано изначально, его нет в природе без человека, если не считать таковым законы природы. Оно является \emph{продуктом исторического развития человека}, его труда и познания.

С точки зрения диалектико-материалистической философии существует \emph{только один разум --- человеческий} (по крайней мере, на Земле, инопланетный разум нами пока не открыт).

Посредством труда и других форм практики \emph{человек вносит разум в мир}, создаёт на основе идей разума \emph{очеловеченную природу}.

Поскольку \emph{практика} как объективно-исторический процесс, с одной стороны, \emph{подчинена} целям человека, выраженным в его идеях, а с другой стороны, \emph{выходит} за их пределы, создавая что-то новое, она всегда одновременно \emph{и рациональна}, \emph{и нерациональна} (не путать с \emph{иррациональным}).

\emph{Иррационализм} (от латинского \emph{irrationalis} --- неразумный) абсолютизирует иррациональное, нерациональное, существующее в нашей жизни, отрывает его от рационального, рассматривает как господствующую тенденцию всего развития.

В противоположность иррационализму диалектико-материалистическая философия \emph{признаёт} нерациональное в качестве \emph{противоположного}, а нередко и \emph{сопутствующего} момента рационального.

\emph{Нет извечно иррационального}, но есть нечто нерациональное в тех или иных исторических условиях.

Например, \emph{построив гидростанцию} на реке, человек создаёт нечто рациональное --- дешёвый способ получения электроэнергии. Но наряду с этим он нередко получает заболоченные водоёмы, заполненные сине-зелеными водорослями, которые, естественно, воспринимаются в качестве нерационального результата практики.

Но нерациональное как некоторый побочный, непредвиденный результат деятельности \emph{не остается навечно}, оно \emph{преодолевается} последующим познанием и практикой.

\emph{Само знание} человека как момент его деятельности тоже \emph{можно оценивать} в категориях рационального и нерационального.

\emph{По природе своей знание рационально}, поскольку создаёт идеи, отвечающие целям и потребностям человечества, поскольку соответствует логике, известным, сложившимся формам разума.

Вместе с тем знание нередко \emph{выходит за} ранее созданные формы мышления, не может быть объяснимо только ими, т.е. содержит момент, который преодолевается \emph{путём изменения самой логики}, обогащения её арсенала новыми формами и категориями мышления.

Иррационализм \emph{акцентирует внимание} на этом нерациональном остатке, аспекте в знании, которое ещё \emph{не имеет объяснения в существующих формах разума}, считает его подлинной сущностью и, таким образом, создаёт неверное представление о ходе познания.

Рациональное как \emph{главная тенденция} развития познания существует в двух формах --- \emph{рассудка} и собственно \emph{разума}.

\emph{Рассудочная деятельность} --- это оперирование формами мысли, абстракциями по строго заданной \emph{схеме, шаблону}, без осознания самого метода, его границ и возможностей.

\emph{Рассудок разделяет} целое, единое на взаимоисключающие противоположности, не умея охватить их в единстве взаимопроникновения.

\emph{Наиболее ярко} особенности рассудка видны \emph{на примере алгоритма}, правила, в соответствии с которым выполняется, например, тот или иной вычислительный процесс, для которого характерна строго-однозначная определенность.

Каждая стадия вычислительного процесса \emph{определяет следующую}, сам процесс расчленяется на отдельные шаги, а предписание задаётся в виде комбинации символов.

Действия по алгоритмам может выполнять \emph{машина}.

Деятельность рассудка \emph{необходима} для теоретического мышления, без него мысль расплывчата и неопределенна.

Рассудок придаёт мышлению \emph{системность} и \emph{строгость}, стремясь превратить теорию в формализованную систему.

\emph{Но не рассудок}, тем более сам по себе, составляет \emph{характерную особенность} человеческого мышления. Её выражает собственно \emph{разум}.

В отличие от рассудка \emph{разум оперирует понятиями с осознанием их содержания и природы}, с его помощью творчески активно, целенаправленно отражается природа, её процессы.

Разум есть \emph{орудие} преобразующей деятельности, создания мира, отвечающего потребностям и сущности человека.

Человеческий разум стремится \emph{выйти за пределы} сложившейся системы знаний, создать новую, в которой с большей полнотой и объективностью выражены цели человека.

Если \emph{для рассудка} характерен \emph{анали}з, то \emph{для разума} --- \emph{синтез}, как доведённая до самого высшего уровня человеческая способность.

Человеческое \emph{знание --- результат единства} рассудочной и разумной деятельности, с высот которой осмысливается объективная реальность, определяются пути её рационального изменения.

\subsection{Знание и ценность}

Идея практически реализуется в \emph{культуре} --- материально и духовно определенной: в вещах, произведениях искусства, нормах морали и т.п. Отсюда правомерна постановка вопроса об их отношении к \emph{человеческим потребностям}, носящим общественный характер.

\emph{Люди начинают} не с чисто теоретического отношения к миру, а с активного овладения им, налагая на него субъективные формы своей деятельности .

На понимании этого отношения к предметам внешнего мира как к \emph{средству удовлетворения человеческих потребностей} выросла философская проблема \emph{ценности}.

\emph{Суть} проблемы ценности заключается \emph{не в том, называть ли} созданные человеком предметы материальной и духовной культуры, а также явления природы, служащие удовлетворению его потребностей, «\emph{благими}», «\emph{ценностями}» или какими-либо иными словами, нужно ли и по каким признакам их классифицировать и т.п.

\emph{Главное} в проблеме ценности --- это \emph{вопрос о природе ценности}, её отношении к субъекту и объекту, познанию и т.п.

Диалектико-материалистическая философия решает вопрос о природе ценности на основании понимания \emph{человека как творца} истории и самого себя, своей деятельной сущности.

\emph{Предметы} природы, материальной и духовной культуры обладают \emph{способностью удовлетворять} потребности человека, служить его целям. В этом смысле к ним возможен и необходим \emph{ценностный подход}.

\emph{Откуда берётся} у предметов эта способность удовлетворять потребности человека --- из их собственной природы, или она идёт от человека, его особенностей и возможностей?

Если признать, что \emph{ценность} коренится \emph{лишь в самих предметах}, то это будет означать наделение их изначальными свойствами служить человеку и его целям. Но мы знаем, что природа и её предметы существовали \emph{задолго до появления человека}.

С другой стороны, \emph{нельзя считать}, что предмет может удовлетворять материальные или духовные потребности человека \emph{вне зависимости} от своей собственной природы. \emph{Если бы хлеб} не содержал в себе определённых веществ, он не был бы продуктом питания, благом для человека.

Диалектико-материалистическая философия рассматривает ценность как \emph{общественно-историческое явление} и момент практического взаимодействия субъекта и объекта.

Общественный мир \emph{не является чем-то} потусторонним материальному природному процессу.

Продукт человеческого труда является \emph{продолжением природы}, поэтому \emph{ценность --- это свойство} \emph{предметов}, возникших в процессе развития общества, а вместе с тем \emph{и свойство предметов природы}, включённых в процессы общественной деятельности, трудовой, бытовой и т.п.

Некоторые философские направления \emph{противопоставляют} ценностный подход к предметам и явлениям объективно-научному их рассмотрению.

Однако разделить теоретико-познавательный и ценностный подход к предметам действительности \emph{можно только в абстракции}, для строго определенных целей.

\emph{Теоретико-познавательный подход} стремится зафиксировать постижение предмета таким, как он существует вне человека и человеческой деятельности, освободить сознание от отношения субъекта, человека к его содержанию и \emph{выделить в чистом виде собственно знание}, т.е. объективную истину.

\emph{Ценностный подход,} наоборот, стремится как в самом объекте, так и в его отражении сосредоточивить \emph{внимание на человеческом отношении}, оценить всё с точки зрения заложенных в нём возможностей удовлетворять потребности людей. Он берёт \emph{не знание} в чистом виде, \emph{а его воплощение} в материальной и духовной культуре, способной служить человеку и его целям.

Большую \emph{роль} играет ценностный подход, например, \emph{в моральном} или \emph{художественном сознании}, во многом выражая специфику их отношения к объективному миру.

Вместе с тем в реальной человеческой деятельности, как предметной, так и духовной, оба момента (\emph{объективно-научный} и \emph{ценностный}) соединены, не могут существовать друг без друга, и вытекают из одного источника --- практического отношения человека к объективной действительности.

Итак, культура --- это специфически человеческий способ активно-деятельностного освоения действительности, основу которого составляет развившаяся в определённых объективных, материальных условиях разумная способность человека, в структуре которого нужно выделять три основных компонента:
\begin{itemize}
\item знания (номологический компонент),
\item технологии (технологический компонент),
\item ценности (аксиологический компонент).
\end{itemize}

При этом ценности составляют ценностный базис всякой культуры, как типологического качества (типа культуры), а знания и технологии --- цивилизационную надстройку типа культуры.

\chapter{Учение диалектико-материалистической философии об обществе, человеке, личности (исторический материализм)}

\section{Исторический материализм как наука}

Диалектико-материалистическое учение об обществе --- \emph{исторический материализм} --- имеет свой особый предмет исследования --- \emph{наиболее общие законы и движущие силы развития человеческого общества и составляющих его людей, личностей}.

Это учение приобрело \emph{относительную самостоятельность} как \emph{общесоциологическая теория}, как научно-историческая основа общества \emph{объяснения} его настоящих и прошлых ступеней и \emph{предвидения} ступеней его будущего развития.

Вместе с тем, исторический материализм представляет собой \emph{неотъемлемую часть} диалектико-материалистической философии как единого целого.

\subsection{Становление исторического материализма}

Материализм, который имел место до появления диалектико-материалистической философии, был \emph{непоследовательным, ограниченным} в своих принципах вообще, и в своих подходах к обществу в частности. Ему \emph{не удавалось} применять принципы материализма к познанию общественной жизни, истории.

Последовательно материалистический подход к обществу мог возникнуть лишь \emph{при определённых} социальных и теоретических \emph{предпосылках}. Его возникновение, как и всей диалектико-материалистической традиции в целом, \emph{было подготовлено} закономерным развитием социально-политической и философской мысли.

Вместе с тем, \emph{возможность} познания законов общественной жизни определялась и социальными условиями.

Бурные \emph{события конца XVIII -- первой половины XIX в.} показали, что общество отнюдь не прочный монолит, а скорее своеобразный \emph{живой организм}, подверженный изменениям и \emph{подчинённый} в своём существовании и развитии объективным, не зависящим от воли и сознания людей законам.

\emph{Исторический материализм}, как \emph{социальный раздел диалектикоматериалистической философии}, в своих основах был разработан \emph{К. Марксом} и \emph{Ф. Энгельсом} в результате распространения ими философского (диалектического) материализма на понимание общества.

Наиболее общие законы, раскрываемые диалектическим материализмом, \emph{действуют и в обществе}, но они выступают здесь в особой, специфической форме.

Чтобы выявить закономерности развития человеческого общества, \emph{недостаточно знать общие принципы} диалектико-материалистической философии --- необходимо ещё изучить \emph{особые формы} их действия в истории общества, в социальной жизни.

Диалектический метод, примененный к обществу, и метод исторического материализма --- это, в сущности, \emph{тождественные понятия}.

В применении к обществу диалектический метод \emph{конкретизируется}, ограничивается неким специфическим содержанием. Что означает \emph{разработку} в дополнение к общефилософским категориям таких чисто \emph{социологических категорий}, как \emph{общественно-экономическая формация}, \emph{производительные силы} и \emph{производственные отношения}, \emph{способ производства}, \emph{базис} и \emph{надстройка} и т.д.

В категориях исторического материализма обобщаются и выражаются важнейшие \emph{закономерности социального бытия} и \emph{общественно-исторического познания}.

\emph{Новый взгляд на историю}, который был предложен в историческом материализме, первоначально представлял собой лишь \emph{гипотезу и метод}, но такие гипотезу и метод, которые создавали возможность подхода, \emph{ориентированного на критерии научного исследования} применительно к истории. Они позволяли \emph{устанавливать повторяемость} и \emph{правильность} в развитии общества, \emph{обобщать} порядки разных стран в понятие общественно-экономической формации, \emph{выявлять то общее}, что их объединяет, и вместе с тем те различия, которые свойственны отдельным странам в силу специфических, условий их развития.

С написанием \emph{К. Марксом} первого тома «\emph{Капитала}» исторический материализм \emph{был подтверждён} значительной массой последовательно исследованного социологического материала.

\subsection{Предмет исторического материализма}

\emph{Человеческое общество --- это самая сложная по своей сущности, структуре форма существования материи.}

Общество есть специфическая, качественно своеобразная \emph{часть природы}, в известном смысле \emph{противостоящая} остальной природе.

Такое понимание взаимоотношения общества и природы коренным образом \emph{отличает} общесоциологический блок диалектико-материалистической философии как \emph{от идеализма}, который в большинстве случаев противопоставляет общество и природу, так и \emph{от метафизического материализма}, который не видит качественного различия между ними.

Итальянский мыслитель \emph{Дж.Б. Вико} (конец XVII -- начало XVIII в.) отмечал, что \emph{история общества} отличается от истории природы тем, что она \emph{делается людьми, и только людьми}, тогда как в природе явления, процессы происходят сами собой, в результате взаимодействия \emph{слепых безликих}, \emph{стихийных сил}.

Тот факт, что \emph{в обществе действуют люди}, обладающие разумом и волей, ставящие перед собой те или иные цели, задачи и борющиеся за их достижение, \emph{служил в прошлом} и часто \emph{служит в наше время} камнем преткновения для социологов и историков при изучении ими сущности, коренных, глубоких причин общественных процессов, социальных явлений.

Некоторые философы, абсолютизируя специфику общественно-исторических событий, \emph{метафизически противопоставляют} \emph{науки о природе}, изучающие общие, повторяющиеся явления и процессы, \emph{наукам историческим}, которые якобы имеют дело лишь с индивидуальным, неповторимым.

Так, в XIX в. немецкие философы --- представители одной из школ \emph{неокантианства} (\emph{Г. Риккерт, В. Виндельбанд}) --- считали, что должны существовать \emph{два} различных, и даже противоположных, \emph{метода познания}:

\begin{itemize}
\item так называемый \emph{номотетический,} или \emph{генерализирующий} (обобщающий) \emph{метод}, который применяют науки о природе, и
\item \emph{идеографический}, или \emph{индивидуализирующий метод}, (имеющий дело лишь с индивидуальными, неповторимыми событиями), которым пользуются исторические науки.
\end{itemize}

Но такое метафизическое \emph{противопоставление наук} о природе наукам об обществе надуманно, \emph{неоправданно}.

\emph{Не только в истории общества}, но \emph{и в природе} нет двух явлений (например, двух особей животных или двух листьев на одном и том же дереве), которые были бы абсолютно тождественны.

С другой стороны, \emph{в обществе}, в истории наряду со специфическими, индивидуальными признаками \emph{есть общие явления}, проявляющиеся в экономике, в социальных отношениях, в политической и духовной жизни разных стран и народов, находящихся на одной и той же ступени исторического развития.

Выделение общего в истории даёт возможность открыть \emph{законы социальной жизни}.

\emph{Может показаться}, что если общественные события, социальные процессы являются \emph{результатом деятельности самих людей}, то в силу этого их познание представляет собой \emph{менее сложную проблему}, чем познание явлений природы. И установление власти человека и общества над общественными отношениями, по-видимому, \emph{более легкая задача}, чем подчинение человеку противостоящих ему сил природы.

Такое представление, как свидетельствует история общества, \emph{существенно неверно}.

В первой половине XIX в. науки о природе \emph{уже достигли} значительного развития, а наука об обществе \emph{ещё только} \emph{зарождалась}.

Человечество, познавая законы и силы природы, \emph{шаг за шагом} подчиняло их своей власти.

Познание сущности общества и его законов оказалось \emph{делом более длительным и сложным}.

Человеческое общество, общественные явления и процессы рассматриваются сегодня \emph{различными науками}, каждая из которых изучает лишь \emph{ту или иную сторону} общественной жизни, тот или иной вид общественных отношений или явлений (экономических, политических, правовых, идеологических и т.п.).

\emph{Предметом исторического материализма являются} не отдельные стороны жизни общества во всём их многообразии, а \emph{всеобщие законы и движущие силы его формирования и развития, общественная жизнь в её целостности, внутренней связи и противоречивости всех сторон и отношений}.

В отличие от специальных общественных наук \emph{исторический материализм изучает} прежде всего и главным образом \emph{наиболее общие законы развития общества, законы возникновения, существования, движущие силы развития общественно-экономических формаций} (о них речь пойдёт позднее)\emph{.}

Общесоциологические законы, \emph{наиболее общие законы развития общества}, относящиеся ко всем историческим эпохам, внутри каждой исторически определённой общественно-экономической формации, системы общества, в каждую эпоху проявляются по-особому, специфическим образом.

\emph{Чтобы правильно понять} характер, существо общесоциологических законов, надо изучать их действие, функционирование и в той \emph{специфической форме}, в какой они проявляются в разных системах общества, в различные исторические эпохи (например, \emph{при феодализме}, \emph{раннем капитализме}, \emph{современно обществе}).

В понятие «\emph{общесоциологические законы}» входят те внутренние связи и отношения, которые характеризуют наиболее общие закономерности определённых \emph{формаций, систем общества}.

Исторический материализм \emph{отличается} от такой науки, как история.

\emph{В задачу истории} входит изучение истории стран и народов, событий в их \emph{хронологическом порядке}.

В отличие от истории как конкретной науки \emph{исторический мате}риализм --- это прежде всего \emph{общетеоретическая, методологическая наука}, которая изучает не тот или иной народ, не ту или иную страну в отдельности, а человеческое \emph{общество в целом}, рассматриваемое со стороны наиболее общих законов и движущих сил его развития.

Исторический материализм, как и диалектико-материалистическая философия в целом, представляет собой \emph{единство теории (онтологии) и метода (методологии и гносеологии)}.

Он даёт диалектико-материалистическое решение \emph{основного гносеологического вопроса социальной науки}, социального познания в целом --- \emph{вопроса об отношении общественного бытия и общественного сознания}; даёт \emph{знание наиболее общих законов развития общества} и потому претендует на роль научной общесоциологической теории.

Исторический материализм одновременно выступает \emph{и как общий метод} \emph{изучения} явлений, процессов общественной жизни, \emph{и как общий метод преобразования} этой действительности.

\emph{Значение исторического материализма} существенно и при проведении конкретных социальных, \emph{социологических исследований}.

\emph{Применяя} в конкретных социологических исследованиях математические методы, методы опроса, интервьюирования, анкетирования и т.п., \emph{необходимо опираться на общесоциологическую теорию и ее метод}, каковым и стремится выступать в отношении социальных наук исторический материализм.

Исторический материализм, как \emph{общесоциологический раздел} диалектико-материалистической философии, \emph{опирается} в своём развитии на конкретные социальные (в том числе социологические) исследования, на широкое использование статистических и других эмпирических данных, относящихся к различным сторонам общественной жизни. (\emph{См. определение философии в начале учебника}).

Конкретные \emph{социальные исследования} призваны раскрывать и показывать механизм действия, функционирования социологических законов в самых разнообразных условиях.

\subsection{Законы развития общества и их объективный характер}

Основные \emph{положения} и \emph{принципы} исторического материализма заключаются в следующем (\emph{К. Маркс}. К. Маркс и Ф. Энгельс\emph{.} Соч., т. 13, с. 6-7):

«В общественном производстве своей жизни люди вступают в определенные, необходимые, от их воли не зависящие отношения --- \emph{производственные отношения}, которые соответствуют определённой ступени развития их материальных производительных сил. Совокупность этих производственных отношений составляет \emph{экономическую структуру общества}, реальный базис, на котором возвышается юридическая и политическая надстройка и которому соответствуют определённые формы общественного сознания. \emph{Способ производства материальной жизни} обусловливает социальный, политический и духовный процессы жизни общества. Не сознание людей определяет их бытие, а, наоборот, их общественное бытие определяет их сознание. На известной ступени своего развития \emph{материальные производительные силы} общества приходят в противоречие с существующими производственными отношениями, или --- что является только юридическим выражением последних --- с \emph{отношениями собственности}, внутри которых они до сих пор развивались. Из форм развития производительных сил эти отношения превращаются в их оковы. Тогда наступает \emph{эпоха социальной революции}. С изменением экономической основы более или менее быстро происходит \emph{переворот} во всей громадной надстройке. При рассмотрении таких переворотов необходимо всегда отличать материальный, с естественнонаучной точностью констатируемый переворот в экономических условиях производства от юридических, политических, религиозных, художественных или философских, короче --- от \emph{идеологических форм}, в которых люди осознают этот конфликт и борются за его разрешение. Как об отдельном человеке нельзя судить на основании того, что он о себе думает, точно так же нельзя судить о подобной эпохе переворота по её сознанию. Наоборот, это сознание надо объяснить из противоречий материальной жизни, из существующего конфликта между общественными производительными силами и производственными отношениями. \emph{Ни одна общественная формация не погибает раньше, чем разовьются все производительные силы, для которых она даёт достаточно простора}, и \emph{новые более высокие производственные отношения никогда не появляются раньше, чем созреют материальные условия их существования в недрах самого старого общества}. Поэтому человечество ставит себе всегда только такие задачи, которые оно может разрешить, так как при ближайшем рассмотрении всегда оказывается, что сама задача возникает лишь тогда, когда материальные условия её решения уже имеются налицо, или, по крайней мере, находятся в процессе становления». (\emph{К. Маркс}. К.Маркс и Ф. Энгельс. Соч., т. 13, с. 6-7).

Приведенная формулировка \emph{основных положений и принципов исторического материализма}, данная более полутора веков тому назад, демонстрирует \emph{две} важные особенности этого учения:

\emph{во-первых}, последовательное проведение материалистического взгляда на историю как закономерный процесс, и,

\emph{во-вторых}, строгий историзм, рассмотрение общества как находящегося в постоянном развитии.

Ещё до исторического материализма \emph{социологическая мысль}, в частности под влиянием успехов естествознания, стремилась постичь общественную жизнь, историю общества как \emph{закономерный процесс}. Однако социальные закономерности при этом \emph{отождествлялись} с закономерностями механическими, физическими или биологическими процессами, имеющими место в природе. \emph{Игнорировалось} то специфическое, что характеризует общественную жизнь, которая создается людь\emph{ми,} обладавшими разумом, волей.

\emph{Общественное развитие представляет собой естественно-исторический процесс.}

Естественно-исторический процесс --- это процесс столь же \emph{закономерный}, необходимый и объективный, как и природные процессы, не только независящий от воли и сознания людей, но и \emph{определяющий} их волю и сознание.

В то же время, в отличие от процессов природы, естественноисторический процесс представляет собой \emph{результат деятельности самих людей}.

На первый взгляд это положение содержит в себе \emph{логическое противоречие.} Как совместить то обстоятельство, что исторический процесс творится людьми, обладающими сознанием, волей, ставящими перед собой определенные цели, с тем, что история подчинена объективным, не зависящим от воли и сознания людей законам?

Но это противоречие \emph{вполне разрешимо}, если учесть, что люди, преследующие свои цели, руководствуются теми или иными идеями, стремлениями, вместе с тем всегда живут при определённых объективных, не зависящих от их воли и желания условиях, которыми в конечном счёте определяются направление и характер их деятельности, их идеи и стремления.

В полном соответствии с общим диалектико-материалистическим мировоззрением, исторический материализм исходит из положения о \emph{первичности общественного бытия по отношению к общественному сознанию.}

\emph{Общественное сознание} представляет собой более или менее верное (неверное) \emph{отражение общественного бытия}.

\emph{Общественное бытие в конечном счёте определяет общественное сознание}, идеи, стремления и цели людей, социальных групп.

Что же представляет собой понятие «\emph{общественное бытие»,} занимающее центральное место в историческом материализме?

В философском материализме категория бытия рассматривается как \emph{тождественное} понятию материи, природы. Соответственно этому \emph{под общественным бытием в диалектико-материалистической философии понимают материальную жизнь общества, её производство и воспроизводство.}

В состав общественного бытия входят общественные \emph{производственные отношения} и \emph{производительные силы}, необходимые условия их функционирования и развития, \emph{материальные стороны} жизни семьи, социальных групп, этносов, наций и других форм общности людей.

Общественное бытие \emph{первично}, так как оно существует вне и независимо от общественного сознания людей.

Общественное сознание \emph{вторично}, так как оно представляет собой отражение общественного бытия.

\emph{Как понимать} независимость общественного бытия от общественного сознания? \emph{Разве} не сами люди создают свои средства производства и т.п.? \emph{Разве} отличительной чертой человеческого труда не является целенаправленная деятельность людей? \emph{Разве} люди не сами устанавливают отношения друг с другом в процессе производства?

Рассуждая подобным образом, можно прийти к выводу, что люди не могут объединяться иначе, как \emph{при помощи сознания}, а следовательно, социальная жизнь во всех своих проявлениях есть жизнь сознательно-психическая. (\emph{А.А. Богданов}).

Отсюда \emph{заключение}: общественное бытие и общественное сознание тождественны.

Действительно, люди сами строят свою общественную жизнь. Однако они далеко не всегда и не во всём строят её сознательно.

Конечно, \emph{каждый отдельный акт}, например, производства, осуществляется людьми сознательно. Но отсюда не следует, будто люди всегда сознают, каков характер их общественных отношений, в каком направлении изменяются эти отношения, каковы социальные последствия их изменений.

Побуждаемые \emph{жизненной необходимостью}, люди трудятся, производят продукты, обменивают их, а складывающиеся при этом экономические отношения зависят не от их сознательного выбора или желания, а от достигнутой ими ступени развития общественного производства.

Общественное бытие не определяется сознанием \emph{и даже не охватывается} им полностью никогда.

Воля, цели, желания, стремления людей, обусловленные их общественными или личными интересами, воплощаясь в их действия и выступая на арене социальной жизни, взаимно \emph{сталкиваются}, приходят в противоречие друг с другом, переплетаются сложнейшим образом, и в результате нередко получается так, что желаемое достигается \emph{лишь в редких} отдельных случаях.

Лишь после того, как люди \emph{научаются понимать и управлять} общественными законами, всегда лишь ограниченно, возникает возможность во всё возрастающей мере достигать поставленных перед собой целей. \emph{Но и при этом} общественное развитие остаётся естественноисторическим процессом, обусловленным объективными причинами, законами, находящимися вне сознания людей и определяющими их волю, сознание, их цели и задачи.

Развиваясь, общество \emph{лишь постепенно} научается понимать и учитывать объективные условия, реальные возможности, из которых можно и нужно исходить в своих действиях.

\emph{Субъективизм} и \emph{произвол}, как показывают многочисленные исторические примеры, приводят лишь к отрицательным результатам.

Деятельность людей успешна лишь тогда, когда она соответствует объективным социальным, общественным законам.

\emph{Что же понимается под социальным законом?}

\emph{Законы}, устанавливаемые науками об обществе, в том числе историческим материализмом, выражают необходимую, устойчивую и притом повторяющуюся связь между социальными явлениями и процессами.

Среди социальных законов существуют такие, которые действуют \emph{на всех} \emph{ступенях} общественного развития. К ним относятся законы:

\begin{itemize}
\item определяющей роли \emph{общественного бытия} по отношению к общественному сознанию;
\item определяющей роли \emph{способа производства} по отношению к той или иной структуре общества;
\item определяющей роли \emph{производительных сил} по отношению к экономическим отношениям;
\item определяющей роли \emph{экономического базиса} по отношению к надстройке;
\item \emph{зависимости} социальной природы личности от совокупности общественных отношений и т.д.
\end{itemize}

Это \emph{общесоциологические законы}, они действуют во всех формациях, в том числе будут действовать и в возможной будущей формации.

Кроме общесоциологических существуют законы, \emph{присущие только ряду общественных формаций}, они исследуются в науках, делающих своим предметом отдельные стадии развития общества.

Каждый социальный закон действует \emph{при определённых условиях}, и результаты этого действия зависят от этих конкретных условий, которые изменяются не только от одной формации к другой, но и внутри каждой формации, от одной страны к другой. Но эти особенности касаются не самого главного в законах общества. Они \emph{не отменяют} и не могут отменить закономерности, присущие обществу вообще (общие), и той или иной ступени его развития, в частности.

\emph{Законы отдельных формаций}, являясь особенными по отношению к общесоциологическим, сами представляют собой общие законы для всех стран, которые входят в данную формацию.

Здесь, как и в других областях, существует \emph{диалектическое единство} общего и особенного, национального и не национального (интернационального).

\subsection{Сознательная деятельность людей и её роль в истории. Свобода и необходимость}

Рассматривая общественное развитие как естественноисторический процесс, не закрываем ли мы себе \emph{путь к правильному пониманию} роли созидательной, активной преобразующей деятельности людей? \emph{Не ведёт ли это к принижению} исторической активности, исторической инициативы авангардных общественных сил, \emph{к умалению} роли субъективного фактора в истории?

Сторонники идеалистических подходов к обществу нередко \emph{обвиняли и продолжают критиковать} диалектико-материалистическую философию, исторический материализм в принижении роли субъективного, личностного фактора, \emph{и даже в фатализме}. Они утверждают, что исторический материализм недооценивает свободную активную деятельность людей, принижает человека, что он попросту \emph{антигуманен}. Экономический фактор, дескать, --- всё, а идеи, различные формы общественного сознания --- ничто, не имеют будто бы никакого значения.

Всё \emph{это --- результат смешения} действительных подходов исторического материализма с разного рода его \emph{вульгаризациями}, если не преднамеренно созданным объектом критики.

Действительный, последовательный, приверженный исходным принципам исторический материализм, как и диалектико-материалистическая философия в целом, \emph{отнюдь не игнорирует} значение политики, религии, общественного сознания, духовных ценностей, общественных идей, а напротив, признает их огромную роль в общественной жизни.

Тем более он не может обойти вниманием, например, \emph{крайне отрицательную роль}, например, реакицонных идей, реакционной политики (\emph{расизма, милитаризма, национализма, шовинизма и т.п.}, не говоря уже о \emph{фашизме}, который объединяет всё, что перечислено перед ним).

Передовые, прогрессивные идеи и основанная на них политика, наоборот, играют огромную позитивную роль, особенно тогда, \emph{когда становятся убеждениями} значительной части, тем более большинства членов общества.

Исторический материализм вообще сложился исторически \emph{в противоборстве}, спорах:

\begin{itemize}
\item во-первых, с \emph{субъективизмом}, а
\item во-вторых, с \emph{провиденциализмом} (всё на основе божественной воли) и \emph{фатализмом}, принижающими значение активной, сознательной, творческой деятельности людей.
\end{itemize}

\emph{Многие критики} исторического материализма, особенно в связи с его объявлением государственной идеологией \emph{в период СССР}, пытались и пытаются в нём самом, как теории найти противоречие между деятельностью общественных сил и утверждениями об исторической необходимости, общественных законах.

На самом деле, как уже было отчасти показано выше, никакого \emph{непреодолимого противоречия здесь нет}, хотя налицо факт реального противоречия между сознательной деятельностью людей и теми объективными условиями, в которых она всегда совершается. Но это \emph{вопрос о механизмах} разрешения подобных объективных противоречий, а не причина их полного отрицания.

Законы общественного развития выступают чаще всего, как \emph{тенденции}. Они прокладывают себе дорогу \emph{через многие случайности}, через столкновения с противоположными тенденциями, за которыми стоят определённые силы и которые надо как-то преодолеть, чтобы достичь перевеса данной тенденции.

Столкновение различных тенденций в обществе ведет к тому, что в каждый исторический момент \emph{существует не одна возможность}.

Так, прежние стадии развития общества \emph{были чреваты} постоянными войнами, эта тенденция и сегодня реальна. Но наряду с возможностями войны ныне существует и другая реальная возможность --- \emph{возможность поддержания мира}. Она вытекает из роста \emph{сил демократии и прогресса}, противостоящих милитаристским тенденциям.

\emph{Историческая необходимость}, таким образом, \emph{не тождественна предопределению}.

В реальной жизни в результате действия объективных законов, различных тенденций общественного развития возникают определённые возможности, реализация которых \emph{зависит от деятельности людей}, от хода взаимодействия различных социальных групп, классов, от политики различных партий, движений, использующих различные средства, в том числе данные общественных наук.

\emph{Познание исторической необходимости}, объективных законов общественного развития не только не освобождает людей от действительности, а, наоборот, \emph{требует активной сознательной деятельности} для их гуманистической реализации.

\emph{Незнание законов}, игнорирование реальных условий и средств деятельности \emph{обрекает людей} на бесперспективность и пассивность, или на авантюризм и поражение.

\emph{Свобода человека} заключается не в независимости от законов природы и общества, а \emph{в познании} этих законов и в основанной на знаниях возможности планомерно заставлять эти законы действовать для определённых человеческих целей.

\emph{Свобода воли означает способность принимать решения со знанием дела.}

Так в диалектико-материалистической философии решается \emph{старая философская и социологическая проблема} соотношения свободы и необходимости, проблема свободы и детерминированности воли.

История человечества шла далеко \emph{не всегда по прямой} восходящей линии. Она выглядела бы очень \emph{мистически}, если бы в ней было бы только поступательное движение.

История общества \emph{напоминает не железную дорогу}, проложенную человеком по определённому, заранее начерченному маршруту, а \emph{многоводную реку}, встречающую на своем пути многочисленные препятствия, но, несмотря на все зигзаги, текущую к морю или океану.

Так и история человечества, \emph{несмотря на попятные движения}, зигзаги, даже исторические катастрофы, вроде войн, нашествий варваров, падение и распад могучих государственных образований, \emph{закономерно шла} по восходящей линии, от одного системного качества к другому, от низшего к высшему.

И это историческое движение \emph{не однолинейно} (сравни --- \emph{синергетика}). Оно \emph{многообразно} и включает в себя много специфического, связанного с особенностями, условиями развития разных народов, этносов.

\emph{Есть ли какой-то смысл в истории} человечества, в развитии общества, или это движение столь же бессмысленно (бессмысленно) и \emph{стихийно}, как и течение рек, сметающих на своем пути всё, что попадается?

Нельзя, конечно, признавать, \emph{стоя на почве науки} какой-то \emph{извне заложенный} в историю смысл --- нечто вроде божественного предопределения, заранее запрограммированного плана или сверхъестественных предначертаний для народов.

Вместе с тем, история общества в каждую эпоху имеет \emph{своё определённое содержание}.

\emph{Народы} прокладывают пути для новых общественных отношений, борются за решение определенных исторических задач, которые люди могут сознавать либо \emph{более или менее полно}, либо \emph{превратно}, нередко в мистифицированной, религиозно-фантастической форме.

В переломные исторические эпохи происходит \emph{известный подъём} сознательной, творческой деятельности людей, социальных групп, классов.

\emph{История человечества не бессмысленна}, она совершается \emph{не только стихийно}, в ней участвует и общественное сознание.

\section{Материальное производство --- основа общественной жизни}

Как выяснено в предыдущей главе, \emph{предметом} изучения исторического материализма является человеческое общество и наиболее общие законы его развития. Чтобы открыть эти законы, нужно было прежде всего выявить \emph{роль материальной основы общества --- материального производств}а.

Легко понять, что общество \emph{не может} существовать без производства необходимых для жизни людей материальных благ. Это положение \emph{очевидно}.

Но \emph{гораздо менее очевидно} положение исторического материализма о том, что имеет место \emph{закономерная зависимость} системы всех общественных отношений, всех общественных элементов от \emph{способов производства} материальных благ.

В процессе производства люди не только создают материальные продукты, средства существования.

\emph{Производя материальные блага, люди тем самым производят и воспроизводят свои собственные общественные отношения.}

Поэтому исследование общественного производства, его структуры, составляющих его элементов и их взаимосвязей, общих законов развития производства и воспроизводства материальной жизни общества \emph{даёт возможность проникнуть в сущность} исторического процесса» раскрыть глубинные социальные механизмы, действующие в общественной жизни.

\subsection{Обществo и природа, их взаимодействие}

Производстве --- это прежде всего процесс взаимодействия общества с природой. Именно в процессе этого взаимодействия люди добывают из окружающей природы необходимые им \emph{средства существования}.

\emph{Труд}, производство вместе с тем есть \emph{основа формирования} самого человека как социального существа, выделения его из природы.

Конечно, положение о том, что движущей силой, основой становления человека выступает труд, \emph{не следует понимать упрощённо}, в том смысле, будто сам труд появился до человека.

Первоначально \emph{наши предки} использовали в качестве орудий \emph{случайно оказавшиеся под рукой} простейшие предметы природы для защиты от хищников или захвата добычи. Эта их деятельность относилась ещё к разряду первых животнообразных \emph{инстинктивных форм труда}.

Но именно эта примитивная деятельность предков человека \emph{явилась началом} \emph{становления} самого человека, человеческого труда в такой форме, в какой он составляет исключительное достояние человека.

От простого использования данных природой предметов, которое иногда встречается и у животных, наши предки постепенно переходили к \emph{изготовлению орудий труда}, и это было главным моментом в возникновении собственно человеческого труда.

Трудовая деятельность имела \emph{два} решающих следствия.

\emph{Во-первых}, организм предков человека стал приспосабливаться не просто к условиям среды, а к трудовой деятельности.

Специфические \emph{особенности физической организации человеческого существа} --- прямая походка, дифференциация функций передних (верхних) и задних (нижних) конечностей, развитие руки как таковой, головного мозга --- выработались в процессе длительного приспособления организма к выполнению трудовых операций.

\emph{Во-вторых}, труд, будучи \emph{совместной} деятельностью, стимулировал возникновение и развитие членораздельной \emph{речи, языка как средства общения}, накопления, передачи трудового, социального опыта в целом.

В процессе формирования человека можно выделить \emph{два важных рубежа}.

\emph{Первый} из них связан с началом \emph{изготовления орудий труда}.

Это \emph{стадия формирующихся людей} (\emph{питекантропы} и \emph{неандертальцы}).

В последнее время в \emph{Южной и Восточной Африке} были открыты останки самых древних предков человека в слоях с геологическим возрастом в \emph{2,5 миллиона лет}. Их кости найдены вместе с примитивными каменными орудиями труда. Это подтверждает внутреннюю связь между развитием труда и формированием человека.

\emph{Вторым} крупным качественным рубежом, как полагает современная наука, явилась происшедшая около \emph{100 тысяч лет назад}, на грани раннего и позднего палеолита, \emph{смена} неандертальца современным видом человека (\emph{homo sapiens} --- человек разумный).

Если \emph{неандерталец сохранил} в своем строении ещё много особенностей, связывающих его с обезьянами, то со времени появления человека разумного коренных изменений физического типа человека \emph{уже не происходило}.

Этому периоду соответствуют и крупные изменения в производстве, характеризующиеся \emph{созданием разнообразных орудий труда} (каменных, из кости, рога).

Этапы формирования и совершенствования человека и орудий его труда были в то же время и \emph{этапами формирования самого человеческого общества} в его первоначальной форме, а имение \emph{родового общества}.

\emph{Человек --- общественное существо}, он никогда не жил и не мог появиться вне общества и до него.

Однако и общество не могло появиться до человека, \emph{новые формы связи между особями} развивались лишь в меру того, как предки человека становились людьми.

Человека можно \emph{отличить от животного} по самым различным признакам.

Если, однако, выделить \emph{наиболее существенные признаки}, то ими будут \emph{производство орудий} (по определению \emph{Б. Франклина}, человек это --- «\emph{a toolmaking animal}» --- животное, делающее орудия), \emph{членораздельная речь}, \emph{способность к абстрактному мышлению}.

Первый признак --- \emph{исходный}.

Процесс производства, взятый в его самом общем виде, есть воздействие людей на предметы и силы природы \emph{с целью добыть и создать} необходимые для их жизни средства существования: пищу, одежду, жилище и т.д. Этот процесс предполагает \emph{деятельность человека, или самый труд}, направленный на предметы труда.

В отличие от инстинктивных форм деятельности \emph{человеческий труд} в подлинном смысле этого слова представляет собой \emph{целенаправленную деятельность}, в результате чего создаётся предмет, который уже был до этого в представлении человека, т.е. \emph{идеально}.

\emph{Самый плохой архитектор}, однако, превосходит наилучшую \emph{пчелу} тем, что, прежде чем построить свое сооружение, он уже создал его \emph{в своей голове.} (\emph{К. Маркс}).

Трудовая деятельность осуществляется \emph{при помощи} соответствующих средств воздействия на предмет труда --- средств, орудий труда.

\emph{Благодаря орудиям труда} совершается переход от непосредственных действий, присущих животным, которые пользуются своими естественными органами --- \emph{когтями}, \emph{клыками} и т.п., к свойственным человеку действиям, опосредованным орудиями труда. Последние выступают как бы в качестве \emph{продолжения естественных органов человека}; они выполняют вначале те же функции, что и естественные органы, усиливая их действие.

Общество, в отличие от биологического типа существования, можно назвать \emph{социальным организмом}.

Если для биологического организма характерна система \emph{естественных органов}, выполняющих определённые функции, которые необходимы для жизни, то развитие человека, человеческого общества связано с совершенствованием его \emph{искусственных органов} --- орудий, средств труда.

Человеческий труд \emph{отличается} от деятельности даже наиболее развитых животных \emph{тем}, что:

\begin{itemize}
\item во-первых, он представляет собой \emph{активное воздействие человека на природу, а не простое приспособление} к ней, которое характерно для животных;
\item во-вторых, он предполагает \emph{систематическое использование и,} самое главное, \emph{производство орудий производства};
\item в-третьих, труд означает \emph{целенаправленную сознательную деятельность} человека;
\item в-четвёртых, он с самого начала \emph{носит общественный характер} и немыслим вне общества.
\end{itemize}

Вследствие этого социальное развитие \emph{отлично} от биологического.

Человек развивается как социальное существо \emph{без коренного изменения} своей биологической природы. Отсюда вытекает и \emph{различие} характера и темпов обоих процессов.

Коренные изменения в общественной жизни совершаются \emph{в такие сроки}, которых совершенно недостаточно для сколько-нибудь значительных изменений в развитии \emph{биологических видов} (разумеется, если не принимать в расчёт тех изменений, которые происходят в природе \emph{под воздействием самого человека}).

По выражению \emph{Дж. Льюиса}, «летать мы научились \emph{за 50 лет}, тогда как биологическая эволюция посредством генетических изменений достигла этого \emph{за 50 миллионов лет}» (\emph{J. Lewis}. Man and Tvolution. L., 1962, p. 49 ).

К тому же \emph{биологическое развитие} во многих случаях все более \emph{замедляется}, по мере того, как определённый вид организмов специализируется и приспособляется к среде.

Напротив, \emph{развитие общества} на протяжении его истории в общем и целом \emph{ускоряется}, несмотря на разного рода зигзаги и временные отступления назад. Во многом это стало возможным благодаря возникновению \emph{новых механизмов преемственности} в развитии общества по сравнению с биологической эволюцией.

\emph{В органическом мире} накопление и передача информации от одного поколения к другому осуществляется главным образом \emph{через механизм наследственности}, лежащей в основе прирожденных инстинктов, а \emph{у высших животных} также путём передачи потомству приобретенных родителями индивидуальных навыков.

\emph{В общественной жизни} громадную роль играет \emph{наследование} каждым поколением \emph{средств производства}, созданных предшествующими поколениями, а также \emph{социального опыта}, который воплощается в языке, мышлении, культуре и традициях.

Если биологическая передача свойств \emph{ограничена запасом} информации, который заложен в аппарате наследственности (в генах), то наследование социального опыта происходит непрерывно и \emph{не имеет границ}.

\emph{Культура}, рассматриваемая в самом общем смысле, представляет собою воплощение социального опыта, совокупность созданных в ходе человеческой истории материальных и духовных ценностей (смотрите определение куль туры, данное выше).

\emph{Каждое новое поколение} обогащает культуру новыми достижениями.

В отличие от \emph{биологического мира}, где все изменения совершаются \emph{стихийно}, \emph{бессознательно}, перед человеческим обществом открывается возможность --- и притом \emph{в ходе истории} по всё возрастающей мере --- \emph{сознательно} и \emph{целенаправленно} изменять условия своей материальной жизни и регулировать взаимоотношения с природой.

\emph{Всякая материальная система} предполагает определённый тип связи между составляющими её элементами.

Специфику общественной жизни определяет производственная, \emph{экономическая связь}. Все формы общественных отношений складываются в конечном счёте на базе отношений между людьми, возникающих в процессе производства, --- производственных отношений, которые цементируют социальный организм, определяют его единство.

Качественно \emph{новым формам связей}, образующим социальный организм, соответствуют и отличные от биологических \emph{специфические закономерности} его развития, биологические законы, как и другие законы природы, не регулируют и не определяют развития социальных явлений.

Общество управляется своими специфическими законами, которые раскрываются \emph{общественными науками}.

Это, однако, \emph{не означает}, будто общество развивается изолированно от природы. Развитие общества немыслимо без известных \emph{естественных предпосылок}.

К числу естественных предпосылок относятся прежде всего окружающие общество \emph{природные условия}, обычно называемые \emph{географической средой}, и \emph{телесная организация} самих людей, составляющих \emph{население}.

Различного рода \emph{натуралистические теории} в социологии пытались приписать этим естественным предпосылкам определяющую роль в истории.

Так, сторонники \emph{географического детерминизма} (французский философ \emph{Ш. Монтескье}, английский историк \emph{Г. Бокль}, французский географ \emph{Э. Реклю} и \emph{др.}) стремились объяснить различия в общественном строе и истории отдельных народов \emph{влиянием природных условий}, в которых они живут.

Однако в сходных географических условиях у народов бывает весьма \emph{различный} общественный строй, а при различных географических условиях может существовать \emph{один и тот же} общественный строй (например, \emph{родовой строй} имел место в разное время в Европе, Азии, Америке , Австралии).

Историческая \emph{смена общественно-экономических формаций}, систем общества также не может быть объяснена влиянием географической среды хотя бы потому, что она происходит \emph{гораздо быстрее}, чем не зависящие от общества изменения в географической среде.

Коренной методологический \emph{недостаток натуралистических теорий} в социологии состоит в том, что они видят источник развития общества \emph{вне его самого}.

Разумеется, влияние внешних условий на любую развивающуюся систему, в том числе и на общество, \emph{нельзя отрицать} или недооценивать.

Но изменение такой системы \emph{не является просто отпечатком} меняющейся среды, пассивным результатом её влияния. Система имеет \emph{внутреннюю логику развития} и в свою очередь оказывает воздействие на окружающую среду.

Если воспользоваться \emph{современной классификацией} систем, то общество можно отнести к числу так называемых \emph{открытых систем}, которые обмениваются с окружающей их средой не только энергией, но и веществом.

Между обществом и природой (\emph{остальной природой}) происходит постоянный \emph{обмен} веществ, совершающийся, как это показывают исследования, в процессе труда, производства.

Из растительного и животного мира \emph{человек берёт} средства питания, сырьё для изготовления предметов потребления. Минеральные и рудные богатства --- это \emph{кладовая}, служащая человеку при изготовлении средств производства.

В процессе производства используются различные \emph{источники энергии}: сначала собственная мускульная сила человека, затем сила прирученных животных, ветра и воды, наконец, сила пара, электричества, энергия химических и внутриатомных процессов.

На разных ступенях развития общества \emph{географическая среда влияет} на него теми или другими своими сторонами.

Но во всяком случае \emph{преимущественное значение} имеет не непосредственное влияние географических условий на природу человека, на его психический склад, а \emph{опосредованное влияние} --- через условия производства и общения.

На самых \emph{низких ступенях культуры}, когда человек по преимуществу присваивал готовые продукты, большое значение имеют \emph{естественные средства существования}: плодородие почвы, обилие рыбы в водах и т.п.

На \emph{более высоких ступенях культуры}, когда развивается промышленность, несравненно важнее наличие \emph{естественных средств производства}: водопадов, судоходных рек, лесов, металлов, угля, нефти и тд.

Конечно, \emph{направленность хозяйственной деятельности} людей, неодинаковая у различных народов, во многом зависит от географических условий их обитания.

Производительные силы получили у племен, обитавших в районах северных субтропиков, в плодородных областях Месопотамии, долине Нила, и др., \emph{более быстрое развитие, чем} у племен, обитавших в условиях Крайнего Севера и Крайнего Юга.

Вместе с тем \emph{неравномерность темпов развития производства} у различных народов связана и с различиями в общественных условиях жизни, с тем, как складывались отношения между народами --- с их взаимной связью или изолированностью, взаимным \emph{общением или столкновениями} и т.д.

\emph{Влияние географических условий всегда опосредуется общественными условиями}. В первую очередь уровнем развития производства.

Люди различным образом \emph{используют} свойства окружающей их природы, в производство вовлекаются новые и новые материалы, человечество \emph{проникает} в новые области природы (недра земли, глубины моря, космическое пространство и т.д.) и осваивает их для удовлетворения своих нужд.

Связи общества с природой \emph{всё более расширяются}, становятся всё более многостороннее.

Богатство природных ресурсов, конечно, \emph{никогда не утратит} своего значения, оно входит как важный элемент в экономический потенциал страны.

Но с развитием производства \emph{зависимость} общества от естественно сложившихся природных условий относительно \emph{уменьшается}.

\emph{Расширение связей и уменьшение зависимости} от естественных природных условий --- оба эти процесса обусловлены усилением воздействия человека на природу. Если естественные условия сами по себе изменяются сравнительно \emph{медленно}, то под воздействием человека их изменение \emph{ускоряется}. Окружающая человека природа \emph{носит на себе печать} его производственной деятельности.

\emph{Географические условия} на Земле есть в значительной мере результат деятельности живых организмов, с которыми связано, например, образование известняков, доломитов, мраморов, каменного угля, торфа, плодородной почвы и т.д.

Активная \emph{роль жизни} на Земле выражена в предложенном академиком \emph{В.И. Вернадским} понятии \emph{биосферы - планетной оболочки, в которую входят организмы и неживое вещество, охваченное и преобразуемое жизнью}.

Если бы жизнь на Земле прекратилась, то её облик напоминал бы по своей безжизненности \emph{лик Луны}.

С появлением человека «\emph{давление жизни}» на оболочку планеты стало неизмерима более могущественным.

\emph{Человек влияет} на растительный и животный мир, \emph{истребляет} одни виды растений и животных, \emph{насаждает} и изменяет другие.

\emph{Растительный мир} на значительной части земли сформирован человеком.

\emph{В Древней Греции} было всего \emph{два сорта яблонь}, теперь же их \emph{более 10 тысяч}.

Под воздействием человека \emph{расширялась область распространения} многих культурных растений. Благодаря человеку широкое распространение в различных странах получили \emph{картофель}, родиной которого являются плоскогорья между горными цепями Анд (Южная Америка), \emph{кукуруза}, произроставшая первоначально в Америке, \emph{арбуз} --- «\emph{выходец}» \emph{из Африки} и другие полезные растения.

\emph{Масштабы воздействия человека} на земную кору можно сравнить с действием самых мощных геологических сил.

По данным академика \emph{А.Е. Ферсмана}, люди извлекли из земли за последние пять столетий \emph{не менее 50 миллиардов} тонн углерода, \emph{2 миллиардов} тонн железа, \emph{20 миллионов} тонн меди, \emph{20 тысяч} тонн золота и т.д.

Как отмечал академик \emph{С.В. Калесник}, в результате производственной деятельности человека ежегодно выносится на поверхность \emph{не менее 5 кубических километров} горных пород.

Человек прорезает материки \emph{каналами}, отвоевывает у моря сушу.

\emph{Орошая} пустыни, \emph{осушая} болота, \emph{меняя течение} рек,, он \emph{изменяет даже} климатические условия своей жизни.

\emph{На климат} оказывает косвенное влияние и производственная деятельность человека, так как \emph{при сжигании} нефти, угля и торфа в атмосферу ежегодно возвращается \emph{около 1,5 миллиарда} тонн углерода.

А от содержания углерода в воздухе \emph{зависит температура} на Земле.

Если воздействие природы на общество носит \emph{целиком стихийный характер}, то воздействие общества на природу есть всегда \emph{результат сознательной борьбы}, которую люди ведут за своё существование.

\emph{Однако наряду} с целенаправленным изменением природы деятельность человека имеет и \emph{не предвидимые} им последствия, которые во многих случаях в дальнейшем носят огромный \emph{урон}.

\emph{Культура}, если она развивается стихийно, а не направляется сознательно, \emph{оставляет после себя пустыню} (\emph{К. Маркс}. К .Маркс и Ф. Энгельс. Соч., т. 32, с. 45).

Хищническая \emph{вырубка леса} нарушает течение рек, позволяет разрастаться оврагам, открывает дорогу засухе, огромные площади земель подвергаются эрозии, становятся непригодными для земледелия.

\emph{Применение химических средств} борьбы с сорняками и насекомыми нередко вызывает не только их гибель, но и отравление многих других видов растений и животных.

Особенность современного этапа взаимодействия общества и природы состоит в том, что \emph{вся поверхность земного шара} становится поприщем деятельности человека, который выходит даже за пределы Земли, в космос. Он использует почти все вещества, входящие в земную кору, и почти все виды источников известной природной энергии.

Однако вместе с расширением масштабов деятельности человека \emph{растёт и опасность} его неуправляемого воздействия на природную среду.

Побочным результатом деятельности человека является, например, \emph{нарушение равновесия} между различными процессами в природе, \emph{загрязнение} вод и воздуха таким количеством промышленных отходов, радиоактивных веществ и т.д., что это может создать угрозу его собственному существованию.

Французский ученый \emph{Ж. Дорст} пишет: «Может быть, это звучит парадоксом, но самая насущная современная проблема в области охраны природы --- это \emph{защита нашего вида от нас самих}. \emph{Homo sapiens нужно защищать от Homo faber}» (\emph{Ж. Дорст}. \emph{До того как умрёт природа}. М., I960, с. 128). (\emph{Homo sapiens} --- человек разумный, \emph{Homo faber} --- человек производящий).

Однако виновником этой угрозы является \emph{не человек вообще}, а \emph{подчинение} его деятельности соображениям \emph{корысти}, наживы или узкий практицизм, недальновидность.

Ныне \emph{жизненной необходимостью} для человечества становится \emph{разумное использование} процессов природы в планетарном масштабе, которое только и может сделать человека подлинным хозяином земли.

Эта необходимость нашла свое выражение и в выработанном естествознанием понятии \emph{ноосферы} (от греческого \emph{noos} --- разум), \emph{как сферы организованного посредством сознательной деятельности человека взаимодействия природы и общества.}

\emph{Биосфера} XX столетия \emph{превращается}, по представлению \emph{В.И. Вернадского}, \emph{в} «\emph{ноосферу}, создаваемую прежде всего ростом науки, научного понимания и основанного на ней социального труда человечества». (Цит. по сб. «\emph{Природа и общество}». М., 1968, с. 335 - 336).

Создание ноосферы предполагает \emph{планомерное} и \emph{гуманистически ориентированное использование} сил природы в масштабе целых стран и континентов, всей планеты Земля.

\emph{Итак, воздействие человека на природу зависит от уровня производительных сил, от характера общественного строя, от уровня} \emph{развития общества и самих людей}.

Принципиально \emph{так же решается вопрос} и о другой естественной предпосылке человеческой истории --- телесной организации людей, их биологических свойствах.

\emph{Биологические свойства порождают} у людей потребности в пище, одежде и т.д. Однако \emph{способ удовлетворения} этих потребностей определяется уже не биологическими, а социальными условиями жизни людей.

Размножение людей также осуществляется сообразно их биологическим свойствам, и всё же \emph{рост народонаселения} --- это прежде всего социальное явление, регулируемое законами развития общества.

При \emph{натуралистическом взгляде} рост народонаселения рассматривается \emph{как фактор, независимый} от законов развития общества и даже определяющий его.

При этом одни социологи приписывают данному фактору \emph{положительную роль}, и рассматривают увеличение населения как причину, заставляющую людей искать новые источники пропитания и тем самым толкающую вперёд развитие производства. (Таков был, например, взгляд русского социолога \emph{М.М. Ковалевского}).

Другие (английский экономист конца XVIII -- начала XIX в. \emph{Т. Мальтус} и его современные последователи --- \emph{неомальтузианцы}) видят в быстром росте народонаселения \emph{источник} общественных \emph{бедствий}.

\emph{Мальтус} сформулировал «\emph{закон}», согласно которому средства существования растут \emph{в арифметической прогрессии}, а народонаселение --- \emph{в геометрической}. Население размножается \emph{быстрее, чем} возрастают средства потребления, и \emph{отсюда якобы проистекают} голод, безработица, нищета значительной части населения.

\emph{Делается вывод}, что для улучшения своего положения, люди должны \emph{ограничивать рождаемость}.

(Мальтузианские концепции были использованы определенными силами \emph{для оправдания} войн, истребления «\emph{лишнего населения}» и т.д.).

\emph{В действительности} соотношение между темпами роста населения и производства средств существования \emph{не является} раз и навсегда данным.

При относительно консервативном техническом базисе, медленном развитии производства \emph{в докапиталистических} общественно-экономических формациях \emph{наблюдалось давление избытка населения} на производительные силы, которое нередко приводило к крупным миграциям (перемещениям) населения.

\emph{В условиях же более быстрого} технического прогресса рост производства средств существования значительно обгоняет темпы роста населения, о чём свидетельствует, например, \emph{увеличение производства на душу населения}.

В странах развитого капитализма \emph{не избыток} населения давит на производство, производительные силы, \emph{а, наоборот}, производительные силы давят на население, создают относительно избыточное население.

Всякому исторически определённому способу производства \emph{свойственны свои особенные}, имеющие исторический характер законы народонаселения.

\emph{Абстрактный закон населения} возможен только для растений и животных, пока в условия их существования не вмешивается человек.

Численность населения, его прирост, плотность, размещение по территории, \emph{бесспорно, оказывают влияние} на развитие общества.

\emph{Вместе с тем} сама численность людей, составляющих общество, зависит от степени развития производства.

\emph{К началу неолита} (т.е. \emph{около} 10 \emph{тысяч лет назад}) первобытные племена, расселившиеся по всем материкам, насчитывали \emph{несколько миллионов} человек.

\emph{К началу нашей эры} население Земли составляло \emph{приблизительно 150 -- 200 миллионов} человек.

\emph{К 1000г. н.э}. --- \emph{около 300 миллионов} человек.

\emph{Первого миллиарда} население земного шара достигло в 1850 г., \emph{второго --- в 1930 г}., \emph{третьего --- в I960 г}., \emph{четвертого --- в 1976 г.}

\emph{В 2020 г.} нас на Земле \emph{более 7 миллиардов}.

Ускорение темпов роста \emph{не причина, а следствие} изменения способов производства, условий жизни людей.

Прирост населения зависит от соотношения \emph{смертности} и \emph{рождаемости}.

На оба эти процесса влияет \emph{множество факторов} социального порядка:

\begin{itemize}
\item \emph{экономические} отношения,
\item уровень \emph{благосостояния} населения,
\item \emph{жилищные} условия,
\item состояние \emph{здравоохранения} и т.д.
\end{itemize}

От социально-экономических условий зависят и \emph{типы воспроизводства} населения.

Рост населения представляет в своей основе \emph{стихийный процесс}.

Но на него оказывает \emph{большее или меньшее влияние} политика государства, правовые и другие меры, направленные на поощрение или, наоборот, ограничение рождаемости.

Неомальтузианцы утверждают, что ныне происходящий «\emph{демографический взрыв}» \emph{не менее опасен}, чем взрыв атомной бомбы.

Американский биолог \emph{П. Эрлих} \emph{сравнивает} рост численности населения на Земле в ближайшие десятилетия \emph{с} бурным \emph{разрастанием раковых клеток}, что ведёт к возникновению новых обширных \emph{очагов голода}. Но подобно другим мальтузианцам, он \emph{не желает видеть} социальных причин голода.

Подсчёты учёных показывают, что \emph{при более полном} и \emph{планомерном использовании} пригодных для сельского хозяйства земель и повышения их урожайности можно свободно прокормить \emph{вдесятеро больше} людей, чем нынешнее население земли.

Дело, однако, в том, что реализация этих возможностей \emph{зависит не только} от изыскания более рациональных способов использования биосферы, \emph{но и от} решения социальных вопросов, \emph{преодоления} экономической и культурной отсталости многих стран, несправедливости в отношениях различных стран, регионов, групп стран и т.п.

Критика мальтузианства и неомальтузианства, \emph{конечно, не означает}, будто для общества вообще не существует проблемы регулирования роста населения.

\emph{Будущее общество} вынуждено будет \emph{регулировать производство людей}, подобно тому, как оно регулирует отчасти уже сегодня производство вещей.

\subsection{Производительные силы общества. Человек в системе производительных сил}

\emph{Материальное производство} есть та сфера общественной жизни, \emph{где создается} материальный продукт, идущий затем в общественное, производственное или личное потребление.

Какой бы высокой ступени развития не достигло общество, оно \emph{не может существовать} и развиваться без производства.

Если представить себе, что производство прекратилось хотя бы на некоторый срок --- \emph{перестали действовать} хлебопекарни, обувные и текстильные фабрики, остановились поезда, перестал идти по проводам электрический ток, прекратилась подача воды и т.д., --- то нетрудно убедиться: полная остановка производства означала бы \emph{гибель общества}.

\emph{Без производства нет общества.}

В процессе производства \emph{люди взаимодействуют} с природой и друг с другом.

Эти \emph{два ряда} отношений и составляют неразрывно связанные стороны любого конкретного способа производства --- \emph{производительные силы} и \emph{производственные отношения}.

Анализ способа проивзодства в его общем виде \emph{сводится к выяснению} того, \emph{что такое} производительные силы, производственные отношения и какова их взаимосвязь.

\emph{Производительные силы --- это те силы, с помощью которых общество воздействует на природу и изменяет её.}

Если производительные силы выражают отношение общества к природе, то естественно, что сама \emph{природа не может входить} в состав общественных производительных сил. Природа является всеобщим \emph{предметом труда}.

\emph{Труд есть отец богатства, природа --- его мать} (\emph{Маркс}).

Непосредственно предметом труда служит не вся природа, а та её \emph{часть, которая вовлечена} в производство, поскольку она используется человеком.

Из природы \emph{человек извлекает} вещество, подвергаемое переработке в процессе труда.

Но за исключением добывающей промышленности, распашки целины и т.п. в производстве обычно оперируют предметами, в которые \emph{предварительно вложен} уже какой-то труд.

Так, \emph{сталь}, из которой делается станок, была до этого \emph{выплавлена}.

\emph{Сырьё} (например, хлопок, зерно, руды), полуфабрикаты и т.д. являются предметами труда, к которым \emph{уже приложен} человеческий труд.

Человек \emph{не только обретает} в природе готовый предмет труда, \emph{но и создаёт} себе этот предмет.

\emph{Прогресс производства} связан с включением в него всё новых и новых материалов.

В современном производстве \emph{используются} различные редкие металлы, новые сплавы, разнообразные синтетические материалы --- пластмассы, синтетические волокна и т.д. И это вполне закономерно, ибо \emph{новые материалы} расширяют производственные возможности человека.

\emph{Средства труда есть вещь или комплекс вещей, которые человек помешает между собой и предметом труда} и которые служат активным проводником его воздействий на этот предмет.

Предметы и средства труда, т.е. \emph{вещественные элементы} процесса труда, составляют в совокупности \emph{средства производства}.

\emph{Состав средств труда} весьма разнообразен и \emph{меняется} от эпохи к эпохе.

В \emph{промышленном} и \emph{сельскохозяйственном производстве} ныне используются машины и двигатели, различные вспомогательные средства труда, необходимые для транспортировки, хранения продуктов и других целей.

Среди всех средств труда, применяемых в ту или иную эпоху и типичных для неё, \emph{особо выделяют те}, которые непосредственно служат проводником воздействия человека на природу, --- \emph{орудия производства}. Они составляют как бы \emph{костную и мускульную систему} производства (\emph{Маркс}).

Но средства труда превращаются в активную силу, преобразующую предмет труда \emph{лишь в контакте с живым трудом}, с \emph{человеком.}

\emph{Человек, трудящиеся массы являются производительной силой}, благодаря наличию у них знаний, опыта, навыков, необходимых для осуществления производства.

Таким образом, \emph{общественные производительные силы --- это созданные обществом средства производства, и прежде всего орудия труда, а также люди, приводящие их в действие и осуществляющие производство материальные благ.}

Средства труда представляют собой \emph{определяющий элемент} производительных сил, поскольку они определяют характер отношения человека к природе.

«Экономические эпохи различаются \emph{не тем, что} производится, \emph{а тем, как} производится, какими средствами труда». (\emph{К. Маркс и Ф. Энгельс}. Соч., т. 23, с. 191).

Люди, трудящиеся с их знаниями и опытом являются \emph{важнейшей производительной силой} общества. Именно человек \emph{использует} имеющуюся и создает новую технику, \emph{приводит в движение} орудия труда, \emph{осуществляет} производство, опираясь на свои знания, опыт, навыки.

В то же время сами эти качества людей их опыт и навыки зависят от того, какими средствами труда они пользуются.

Без автомобилей \emph{не было бы} шоферов, без самолетов --- летчиков.

С переходом к \emph{машинному производству} наряду с эмпирическим опытом и совокупностью усвоенных приёмов для использования техники, а также её совершенствования \emph{всё большее значение приобретают} образование, культура, научные знания.

\emph{Землекоп не может} бросить лопату и пересесть на экскаватор без овладения последним. Он \emph{должен овладеть} новой техникой, хотя экскаватор производит ту же работу, что и землекоп.

Но развитие производства носит \emph{противоречивый характер}.

Возникновение машинного производства \emph{приводит к обострению} противоположности между умственным и физическим трудом. Возрастает потребность в \emph{квалифицированном труде}, необходим подъём культурно- технического уровня рабочих до уровня инженерно-технического труда.

\emph{Показателем уровня} развития труда, производительных сил служит \emph{производительность общественного труда.}

Важнейший фактор роста производительности труда --- создание производительных орудий и средств труда, т.е. \emph{технический прогресс}.

За время существования общества производительные силы \emph{достигли} колоссального развития.

\emph{Производство исторически начинается} с изготовления и использования примитивных каменных, костяных, деревянных орудий --- каменного рубила, каменного остроконечника, дубины и копья, изделий из кости.

Величайшим достижением ранней стадии развития человечества было открытие способов \emph{использования и добывания огня}.

Значительно расширило возможности человека \emph{изобретение лука и стрел}, большим шагом вперед было возникновение \emph{гончарного производства}.

У людей \emph{накапливался}, таким образом, \emph{набор простейших орудий}, позволявший им заниматься охотой, рыболовством, собирательством.

\emph{Совершенствование орудий} идёт в направлении всё большей их дифференциации, специализации применительно к различным операциям.

При этом \emph{на самой ранней стадии} первобытного общества человек производил \emph{лишь орудия труда}, а средства к существованию брал готовыми из природы (\emph{хозяйство присваивающего типа}), что ставило его в сильную \emph{зависимость} от природных условий.

Поэтому \emph{величайшей революцией} в развитии первобытного производства был \emph{переход от присвоения к производству средств существования}, что связано с появлением \emph{земледелия} и \emph{скотоводства}.

Этот переход произошел \emph{в период неолита}.

\emph{Собирательство} плодов, злаков \emph{подготовило переход} к земледелию, а \emph{охота} --- к скотоводству.

Крайне примитивное \emph{мотыжное земледелие} требовало колоссального труда. Но это был принципиально \emph{новый шаг} в развитии, ибо позволил человеку использовать новое могучее средство производства --- \emph{землю}.

Развитие земледельческих орудий привело к \emph{появлению плуга} и других средств обработки земли и уборки урожая.

Дальнейший прогресс связан с использованием \emph{металлических орудий} --- сперва из меди и бронзы, потом из железа.

Земледелие, скотоводство, металлические орудия \emph{создали новый уровень} развития производства.

Возникла \emph{основа для разделения общественного труда} между скотоводством и земледелием, ремесленным и сельскохозяйственным производством, а \emph{затем между умственным и физическим трудом}.

Люди начинают производить больше, появляется \emph{возможность накопления} \emph{богатства}.

Всё это имело большие социальные \emph{последствия}, подготовило переход \emph{от первобытно-общинного строя} \emph{к классовому обществу}.

Следует также отметить то огромное значение для развития производства, всей человеческой культуры, какое имело \emph{появление письменности}.

В классовом обществе развитие производства первоначально происходило \emph{на базе ремесленных орудий} труда.

Конечно, и эти орудия совершенствовались, на их основе происходило развитие производства, возникали различные его отрасли, но не выходило за рамки \emph{индивидуального пользования орудиями}.

Помимо \emph{мускульной силы животных} люди начинают использовать \emph{энергию воды} и \emph{ветра} (ветряные и водяные мельницы, водяное колесо), появляются \emph{более сложные} орудия.

Человечество обогащается важными \emph{изобретениями}, играющими большую роль в развитии техники: механические часы, порох, книгопечатание и производство бумаги, компас и т.д.

Всё это подготавливает условия для \emph{нового качественного скачка} в развитии производительных сил --- \emph{возникновения машинного производства}.

Непосредственные технические предпосылки для появления машин создаёт \emph{мануфактура}.

В тех или иных масштабах \emph{кооперация труда}, \emph{т.е. объединение людей для производства работ}, имела место всегда --- на рудниках, в шахтах, ремесленных мастерских, в строительстве и т.д.

Мануфактура отличается от \emph{простой кооперации} тем, что она основывается на \emph{детальном разделении труда} при производстве какого-либо изделия.

Разделение труда в мануфактуре \emph{приводит к специализации} ремесленных орудий и самого рабочего.

Если \emph{ремесленник целиком создавал} изделие, то в мануфактуре производство этого изделия \emph{распадается на ряд частных операций}, что и создаёт предпосылки для их замены действием машины.

Возникновение \emph{машинного промышленного производства} относится к XVIII в., когда в Англии происходит \emph{первая промышленная революция}.

Эта революция связана с \emph{появлением рабочих машин} --- ткацкого станка и прядильной машины.

Машины \emph{заменяют} большое количество рабочих, производя те же операции, которые раньше совершались вручную.

Но рабочая машина \emph{требует двигателя}. И такой двигатель был создан в виде \emph{паровой машины}.

Двигатель, передаточный механизм и рабочая машина составили \emph{первоначальный производственный механизм машинного производства}. Его становление завершается, когда создается адекватная ему техническая база --- \emph{производство машин машинами}.

Был создан \emph{принципиально новый подход} в развитии производства, его производительных сил, положивший \emph{начало новой эпохе} в развитии производства.

Промышленный переворот, \emph{начавшись} в Англии в XVIII в., в XIX в. \emph{распространяется} на другие европейские страны, Северную Америку, в конце века \emph{захватывает Россию}, Японию.

Машинное производство выступает здесь в качестве \emph{материально-технической базы капитализма}, особого рыночного хозяйства.

\emph{Современный технический прогресс} происходит на основе машинного производства.

\emph{Машинное производство} придаёт самому процессу труда общественный (коллективный, групповой) характер, объединяя на фабриках, заводах большие массы людей «\emph{под одной крышей}».

Вместе с тем широко развиваются \emph{различные виды разделения труда} между специальностями, различными отраслями производства.

В свою очередь, разделение труда создает такую \emph{взаимосвязь между различными видами производства}, что изменения в одной отрасли быстро сказываются на всех других.

В отличие от ремесленного \emph{технический базис машинного производства} является основой практически неограниченного развития, а сознательное \emph{применение естествознания} к производству делает неизбежными постоянные \emph{технические перевороты}.

\emph{Ещё в} XIX \emph{в. понимали}, что развитие машинного производства ведёт \textsc{свой} путь от применения отдельных машин к системе машин и затем --- к созданию \emph{автоматизированного производства}, при котором \emph{человек выключается} из непосредственного процесса материального производства и \emph{на его долю остаётся} общий контроль, наладка, ремонтные работы, конструирование новых машин и т.п.

\emph{Развитие естествознания и техники} в первой половине XX в. создало предпосылки для нового грандиозного \emph{скачка} в производительных силах --- для \emph{современной научно-технической революции}, объединяющей качественные изменения в естествознании и технике.

Эта революция открыла \emph{эру автоматизированного производства}, привела к коренному \emph{изменению места человека} в производстве.

Рабочая машина и двигатель позволили \emph{передать от человека} техническим устройствам \emph{функцию человеческой руки} --- непосредственного воздействия на предмет труда. Но \emph{управление машиной}, самим процессом производства оставалось за человеком.

Развитие и техническое \emph{приложение кибернетики} и \emph{радиоэлектроники} привело к созданию \emph{электронно-вычислительной техники}, являющейся уже \emph{продолжением человеческого мозга}.

Открывается возможность передавать машинам и \emph{функции управления производством}, полностью \emph{автоматизировать} непосредственный процесс материального производства.

\emph{Формируется качественно новый уровень развития производительных сил}.

Мы пока находимся в первой фазе этого процесса, но его \emph{перспективы уже сейчас довольно ясны} --- развитие идёт \emph{от частичной автоматизации к полной}, когда постепенно между человеком и природой становится не просто орудие труда и \emph{даже не система машин}, а \emph{автоматизированный производственный процесс}.

Научно-техническая революция происходит и \emph{в области энергетики}, где она связана с мирным использованием \emph{атомной энергии}, а в перспективе --- с открытием использования \emph{энергии управляемой термоядерной реакции}.

Развивается космическая техника, открывшая человеку выход в мировое космическое пространство.

Научно-техническая революция \emph{меняет положение науки в обществе}, её отношение к производству.

\emph{Промышленная революция XVIII -- XIX вв}. происходила при участии естествознания в том смысле, что производство выдвигало \emph{определённые задачи} перед наукой, научное решение этих задач позволяло совершенствовать производство.

В условиях современной научно-технической революции процесс \emph{привлечения науки} идёт дальше.

Здесь сами новые виды производства \emph{возникают в результате развития науки}.

Производство и в этих условиях остаётся конечной \emph{материальной основой развития науки}, но здесь техническая необходимость состоит уже в том, \emph{чтобы наука опережала развитие техники}.

Так, «в ранние периоды своего развития наука следовала за промышленностью; теперь она имеет тенденции догнать её и руководить ею». (\emph{Дж.Д. Бернал}. \emph{Наука в истории общества}. М., 1956, с.30).

С развитием машинного производства вообще и в особенности в условиях научно-технической революции \emph{производство} во всё возрастающей степени \emph{становится технологическим применением науки}, а \emph{наука превращается в непосредственную производительную силу}.

\emph{Это не значит}, что наука вообще сливается с производством, теряет свою относительную самостоятельность, перестает быть сферой духовного производства.

\emph{Пвращение науки} в непосредственную производительную силу означает, что,

\begin{itemize}
\item во-первых, средства труда, технологические процессы становятся результатом \emph{материализации научного знания}; не только создание новой, но и функционирование существующей техники \emph{невозможно без науки};
\item во-вторых, научные знания становятся необходимым компонентом опыта и знаний \emph{всех работающих}, участвующих в процессе производства;
\item в-третьих, управление производством, технологическими процессами, в особенности в автоматизированных системах, становится \emph{результатом приложения науки};
\item в-четвёртых, расширяется само понятие производства, куда включается не только непосредственный производственный процесс, но и \emph{проектно-конструкторские разработки}, происходит \emph{сближение} и \emph{взаимо-проникновение} сферы науки и производства --- своего рода «\emph{онаучивание}» \emph{производства}.
\end{itemize}

Следствием всего этого является \emph{расширение человеческого компонента производительных сил}, уже в настоящее время включающих не только людей физического труда, но и техников, инженеров \emph{и даже научных работников}, которые осуществляют непосредственно научно-техническое обслуживание производственного процесса.

\emph{Автоматизация} и «\emph{онаучивание}» производства создают \emph{основу для сближения} физического и умственного труда, ведут к \emph{интеллектуализации труда} рабочих, делают его всё более содержательным и творческим, вызывают существенные изменения в профессиональной структуре труда, \emph{приводят к быстрому росту} слоя квалифицированных рабочих, инженерно-технических работников.

Автоматизированная современная техника предъявляет \emph{особые требования к личным качествам человека} --- его способности принимать самостоятельные решения, брать на себя ответственность, сочетать личные интересы с интересами других людей, коллектива и т.д.

Таким образом, если в мануфактуре труженик был «\emph{частичным рабочим}», если развитие машинного производства на первом этапе превращало рабочего в «\emph{придаток машины}», то развёртывание научно-технической революции в условиях современного развитого общества включает в себя как необходимый момент \emph{совершенствование творческих сил}, способностей человека, \emph{освобождение от} малоквалифицированного , монотонного труда.

\emph{Всестороннее развитие личности} --- идеал будущего общества --- становится в перспективе и потребностью самих производительных сил. В этом нельзя не видеть свидетельства того, что тенденции научно-технической революции \emph{совпадают с потребностями} прогрессивного социального развития общества.

Сегодня возникают различные варианты своеобразной «\emph{технической мифологии}», когда \emph{абсолютизируется роль техники} и последняя рассматривается как сила, враждебная человеку.

Утверждают, например, что развитие науки и техники \emph{не сулит ничего хорошего} людям. Техника-де уже сейчас порождает различные \emph{опасности} для человека, приводит к стандартизации жизни, обезличиванию людей.

Авторы подобных концепций \emph{отрывают технику от человека}, \emph{недооценивают} роль трудящихся в целом, \emph{игнорируют} значение социальных условий, от которых в первую очередь зависят те или иные последствия технического развития.

\emph{Развитие техники, производительных сил нельзя}, следовательно, \emph{рассматривать вне связи с общественными производственными отно- шениями.}

\subsection{Производственные отношения}

Производя материальные блага, \emph{люди взаимодействуют} не только с природой, но и друг с другом.

В процессе производства между людьми с необходимостью возникают определённые отношения, которые и носят название \emph{производственных отношени}й.

Эти отношения являются \emph{неотъемлемой стороной} всякой производственной деятельности людей, всякого материального производства, производство включает в себя производительные силы и производственные отношения, представляет собой их единство.

Производственный отношения --- очень \emph{важный компонент} любого общества, и в дальнейшем изложении вопрос об их месте и роли в жизнедеятельности общественного организма будет рассмотрен подробнее.

Отметим, что понять функционирование и развитие производства не просто как технологический, но и \emph{как социальный процесс} позволило именно выделение производственных отношений, определяющих и социальную характеристику каждого элемента производительных сил, и социальную природу способа производства в целом.

От производственных отношений \emph{зависит}, является ли работник рабом, крепостным или наёмным работником, служит ли машина \emph{средством эксплуатации труда} или, напротив, его облегчения, работают ли заводы и фабрики для обогащения немногих или в интересах удовлетворения потребностей всех и т.д.

\emph{Производственные отношения --- это отношения экономические}. Специально, во всех деталях их изучают \emph{экономические науки}.

Исторический материализм \emph{интересуют вопросы} об их специфике, структуре, законах взаимосвязи с производительными силами и другими общественными явлениями.

\emph{Что же отличает} производственные отношения от других общественных отношений, \emph{в чём} \emph{их специфика}?

Прежде всего, как и производительные силы, производственные отношения принадлежат к \emph{материальной стороне} общественной жизни.

Материальность производственных отношений выражается в том, что они \emph{существуют объективно}, \emph{независимо от сознания и воли людей} и определяются не желаниями людей, а необходимостью их соответствия уровню развития производительных сил.

Производственные, экономические отношения \emph{не только не зависят} от общественного сознания, но даже никогда и не охватываются им полностью.

Производственные отношения как \emph{общественные} отношения производства \emph{следует отличать} \emph{от организационно-технических отношений}, обусловленных самой технологией производства, техническим разделением труда между различными профессиями или специальностями.

\emph{Характер} общественно-производственных (экономических) отношений зависит от того, \emph{как используются} в данном обществе основные средства производства, или, иначе говоря, как решается в нём \emph{проблема владения}, распоряжения и пользования средствами производства (\emph{отношений собственности}).

Отношения собственности понимаются \emph{не просто как юридическое закрепление прав на вещи}, а как реальная совокупность экономических отношений между людьми, опосредованная их отношением к вещам --- \emph{средствам производства}.

\emph{Собственность на средства производства} может быть \emph{либо общественной}, в том числе групповой, \emph{либо частной}.

При этом как степень развития, так и конкретные формы той и другой собственности в истории \emph{весьма разнообразны}, не говоря уже о наличии \emph{переходных форм}, которые последнее время стали играть чуть ли не решающую роль.

Истории известны такие \emph{формы частной собственности}, как \emph{рабовладельческая}, когда раб был собственностью рабовладельца, \emph{феодальная}, и наиболее развитая и продолжающая развиваться дальше --- \emph{капиталистическая} частная собственность.

Экономическая структура капитализма определяется наличием \emph{частной собственности на основные средства} --- фабрики, заводы, шахты и т.д. --- и свободной \emph{рабочей силы} --- \emph{свободной отличной зависимости}, от средств труда, от средств к существованию.

Экономическая необходимость вынуждает человека \emph{продавать свою рабочую силу} владельцу капитала как особый товар, и только в этой форме он может соединиться со средствами труда и начать процесс производства.

Существует ещё и \emph{мелкая частная собственность}, основанная на личном труде. Как правило, она играет \emph{подчиненную роль}.

\emph{Общественная собственность} --- собственность определённых групп (коллективов) или всего общества --- ставит людей \emph{в равное положение} по отношению к средствам производства, и потому «\emph{обмен деятельностью}» выступает здесь в форме \emph{взаимного сотрудничества}.

\emph{Формы этого сотрудничества}, как и формы общественной собственности, различаются весьма значительно, поскольку общественная собственность \emph{доминировала} на самых ранних ступенях человеческого общества (\emph{собственность рода, племени}), её некоторые разновидности существовали и в докапиталистических классовых обществах (\emph{общинная собственность}).

Новую эпоху в истории человечества создает утверждение форм \emph{современной государственной собственности на средства производства}.

Как экономическая категория современная государственная собственность проявляется в \emph{планомерном развитии хозяйства}, в новых групповых \emph{отношениях взаимопомощи} и т.д.

\emph{Будущее за гармоничным сочетанием всех видов и форм собственности.}

\emph{Границы производственных отношений} определяются движением материального продукта, которое начинается в с\emph{фере непосредственного производства,} проходит там определённый цикл, затем \emph{через обмен и распределение} попадает к потребителю и завершается в форме \emph{индивидуального потребления}.

\emph{Производительные силы} и \emph{производственные отношения} --- это две стороны общественного производства, которые \emph{не существуют порознь} друг от друга.

\emph{Лишь в абстракции} можно рассматривать производительные силы без производственных отношений или производственные отношения без производительных сил. В действительности они \emph{неотделимы друг от друга}, \emph{как} неотделимы \emph{содержание} и \emph{форма}, если понимать в данном случае под содержанием производительные силы, а под социальной формой --- производственные отношения.

\subsection{Диалектика развития производительных сил и производственныx отношений}

Взаимодействие производительных сил и производственных отношений подчиняется общему социологическому \emph{закону соответствия производственных отношений характеру и уровню развития производительных сил.}

Этот закон выражает объективно существующую зависимость производственных отношений от развития производительных сил, устанавливает, что производственные отношения складываются и изменяются \emph{под определяющим воздействием} производительных сил.

Когда человечество только выделилось из животного состояния, употребляемые людьми каменные и другие орудия хотя и являлись инструментами индивидуального пользования, но \emph{были настолько примитивны, малопроизводительны,} что индивид, вооружённый этими орудиями, оказывался \emph{неспособным в одиночку} производить необходимые для своей жизни материальные блага. Люди вынуждены были действовать \emph{сообща}, поддерживать друг друга из-за слабости обособленной личности перед лицом могущественных сил природы.

Главной производительной силой здесь, следовательно, был прежде всего \emph{сам коллектив}. На этой основе \emph{сложились коллективистские первобытнообщинные отношения}.

Дальнейшее развитие производительных сил, переход от каменных к бронзовым и затем к железным орудиям повысили производительность труда, в результате чего \emph{стала возможной} производственная деятельность людей \emph{в одиночку} или в масштабах отдельной семьи.

Появился \emph{прибавочный продукт} (т.е. продукт, остающийся после удовлетворения самых необходимых потребностей).

Возникло \emph{разделение труда}, тенденция к обособлению отдельных производителей и, как следствие этого, \emph{частная собственность}, различные \emph{формы эксплуатации} людей.

\emph{Первая форма эксплуатации --- рабство} --- основана на прямом и грубом насилии, с помощью которого человека превращали в средство труда --- бесправного раба.

В широких масштабах \emph{внешнеэкономическое насилие}, принуждение применяется и \emph{в условиях феодализма} по отношению к крестьянам, которые сами были мелкими частными собственниками и в то же время являлись главной производительной силой этого общества.

\emph{В процессе генезиса капитализма} происходит отделение непосредственного производителя от средств производства и его превращение в «\emph{частичного рабочего}» мануфактуры.

Постепенно капитализм развивает машинное производство, придавая самому процессу производства \emph{общественный характер}.

При капитализме развивается \emph{противоречие между общественным характером процесса производства и частнокапиталистической формой присвоения}, являющееся коренным противоречием этого общества.

Это противоречие проявлялось в катаклизмах \emph{стихийной} капиталистической экономики, анархии производства и кризисах перепроизводства, в классовой борьбе пролетариата.

Оно же \emph{вызвало к жизни} все современные формы управления сферой экономики в западных развитых странах, требует их дальнейшего совершенствования.

Общественному характеру процесса производства может соответствовать \emph{особый механизм реализации отношений собственности на средства производства}, учитывающий интересы всех категорий участников процесса производства, всех категорий населения вообще.

Попытка реализации этой задачи посредством \emph{социалистической госсобственности} \textsc{в} условиях СССР была лишь первой.

Второй были \emph{реформы Рузвельта} в начале 30-х годов в США, за которой последовали \emph{многочисленные варианты} западноевропейских, китайской, японской, южнокорейской и т.д. реформ, связанных с качественно новыми подходами к регулированию отношений собственности на средства производства и примыкающих к ним других групп отношений.

Итак, каждая форма производственных отношений существует \emph{до тех пор, пока} она представляет достаточно простора для развития производительных сил.

Но в ходе дальнейшего развития производственные отношения \emph{постепенно вступают в противоречие} с развивающимися производительными силами и превращается в их \emph{тормоз}.

Тогда их сменяют \emph{новые, усовершенствованные производственные отношения}, роль которых состоит в том, чтобы служить формой дальнейшего развития производительных сил.

«Для того, чтобы не лишиться достигнутого результата, для того чтобы не потерять плодов цивилизации, люди вынуждены изменять все унаследованные общественные формы в тот момент, когда способ их отношений более уже не соответствует приобретённым производительным силам» (\emph{К. Маркс и Ф.Энгельс}. Соч., т. 27, с. 403).

\emph{Что же вызывает развитие самих производительных сил?}

Этот процесс имеет \emph{внутреннюю логику}.

Более сложные орудия труда \emph{возникают на основе} более простых.

Накопленный опыт и знания \emph{овеществляются} в новых средствах труда, а человек, в свою очередь, вынужден приспосабливаться к ним, раз они появились и вошли в употребление.

\emph{С развитием техники}, появлением новых, более производительных орудий, машин существующая техника морально \emph{устаревает} и требует замены.

Но внутренние потребности производительных сил \emph{всё-таки не объясняют} нам того, почему в одних случаях развитие производства идёт быстрее, а в других --- медленнее, в одних --- более или менее равномерно, в других --- через подъемы и кризимы.

\emph{Не может этого объяснить} и ссылка на развитие науки.

Всякая техника есть \emph{материализация знаний}, и без развития человеческого познания был бы невозможен пост техники.

Ныне \emph{решающим источником} технического прогресса является развитие и применение науки. Но развитие самой науки, её темпы в большей мере зависят от производства.

Важным обстоятельством, влияющим на развитие производства, являются \emph{потребности} общества, человека.

\textsc{Прямо} или косвенно \emph{производство всегда служит} удовлетворению каких-либо человеческих потребностей.

Между потребностями и производством в обществе устанавливается сложная \emph{диалектическая взаимосвязь}.

Сами потребности порождаются развитием производства, удовлетворение одних потребностей вызывает \emph{появление новых}, что так или иначе влияет на производство.

Однако отношение потребностей человека к производству \emph{опосредуется} производственными отношениями.

Каждая форма производственных отношений подчиняет производство \emph{своеобразной цели}, и этой целью далеко не везде и не всегда бывают потребности человека.

В рабовладельческом, феодальном, капиталистическом (раннем и современном) обществе имеют место \emph{разнообразные стимулы} к деятельности.

\emph{Активность производственных отношений} и проявляется в том, что они порождают у людей определённые \emph{стимулы к деятельности}.

\emph{В капиталистическом обществе} господствуют интересы получения прибавочной стоимости, интересы получения прибыли собственниками средств производства.

Именно объективные \emph{законы расширенного воспроизводства}, законы максимальной прибыли, капиталистической конкуренции составляют \emph{движущие силы} развития капиталистического производства, его производительных сил, что в современном обществе \emph{дополняется} особыми формами регулирования этого процесса в интересах больших групп людей, всего общества, \emph{не допускающими} стихийного разрушительного действия этих движущих сил капиталистического производства, \emph{подобно тому} как различные системы защиты позволяют обеспечить нормальное функционирование ядерного реактора, других опасных производств.

\emph{Нельзя} рассматривать причины развития производительных сил \emph{в отрыве от} социальных условий, от системы производственных отношений.

\emph{Нельзя} одними и теми же причинами объяснить развитие примитивных орудий первобытного общества и прогресс современной техники. \emph{Каждому} исторически определённому способу производства \emph{присуши свои} специфические причины (источники) и экономические законы развития производительных сил, и характер этих законов зависит от производственных отношений.

Но активность производственных отношений \emph{не означает}, что просто сами по себе формы собственности двигают или тормозят развитие производства.

\emph{Только люди} развивают производство или, наоборот, не заинтересованы в его развитии.

\emph{Сами люди} развивают и меняют свой способ производства, составляющий основу их истории.

\emph{Закон соответствия} производственных отношений характеру и уровню развития производительных сил определяет \emph{не только} развитие данного способа производства, \emph{но и} необходимость замены одного способа производства другим.

С развитием производительных сил в недрах старого общества \emph{зарождаются и новые} производственные отношения, образующие определённый уклад хозяйства, зародыш нового способа производства.

Рабство зарождается \emph{уже в недрах} первобытно-общинного строя, капиталистический уклад появляется \emph{в недрах} феодального общества.

Выросшие производительные силы \emph{вступают в конфликт} с господствующими в обществе старыми производственными отношениями.

Разрешение этого конфликта \emph{невозможно} путём простого количественного изменения. Здесь становится необходимым \emph{качественный переход}, быстрое (\emph{революционное}) \emph{преодоление} отживших и окостеневших старых экономических, социальных, а вместе с ними и политических форм, \emph{открывающее путь} к утверждению нового способа производства.

\emph{Не так давно} политическая экономия, ориентированная на диалектико-материалистическую философию, утверждала, что новые производственные отношения, идущие на смену капиталистическим, \emph{не могут} появиться в недрах капиталистического общества, что \emph{нужна} общественная собственность на средства производства, которая приходит путём национализации, фактически \emph{насильственным обобществлением} средств производства через насильственное же завоевание политической власти одной из групп населения --- пролетариатом и т.п.

Похоже, что сам такой подход в условиях современного общества, с его существенно изменившимся способом производства, сегодня должен \emph{претерпеть существенные изменения}, и прежде всего с точки зрения принципов самого диалектического материализма.

\emph{Закон соответствия} производственных отношений характеру и уровню развития производительных сил \emph{продолжает действовать} и в современном развитом западном обществе.

Но действие этого закона, похоже, \emph{уже не приводит} здесь к потрясениям по примеру прошлых времен, к разрушительным конфликтам.

\emph{Общество имеет возможность} хотя бы отчасти своевременно принимать меры для приведения производственных отношений в соответствие с развившимися производительными силами.

В то же время \emph{практика т.н. социалистического строительства} в бывшем СССР, с её нередко волюнтаристскими и субъективистскими подходами, показала, что данный \emph{закон не может быть нарушен никем} и ни при каких обстоятельствах, ни под какими лозунгами.

Для \emph{мирного разрешения} данного противоречия необходимы не только объективные условия, но и соответствующее действие субъективного фактора, т.е. \emph{правильная политика} и надлежащее руководство развитием общества.

Всемерное использование достижений научно-технического прогресса, повышение экономической эффективности производства, дальнейшее совершенствование системы управления экономикой, более быстрый рост материального благосостояния и культуры населения, развитие демократии --- \emph{претворение в жизнь} этих первоочередных задач обеспечит целостное, интенсивное, более менее пропорциональное развитие всех отраслей экономики, различных сфер общественной жизни современного общества.

Всё это также служит \emph{примером для подражания} (\emph{именно примером, а не навязываемым извне подходом}) странам, которые ещё только вступают в фазу современного развития производства.

\section{Общественно-экономическая формация. Единство и многообразие всемирно-исторического процесса}

Теория общественно-экономических формаций является своеобразным \emph{фундаментом социальной философии диалектического материализма}, его понимания истории как целостного закономерного естественноисторического процесса развития общества.

Выделяя в истории различные типы общества, представляющие в то же время качественное своеобразные этапы его развития --- отдельные общественно-экономические формации, \emph{эта теория позволяет} поставить изучение истории на конкретную почву.

Если история общества складывается из истории отдельных формаций, то \emph{надо исследовать} закономерности их развития и перехода от одной формации к другой.

\subsection{Понятие общественно-экономической формации}

\emph{Основой общественной жизни} и исторического развития служит способ производства материальных благ.

Какое бы общественное явление мы ни взяли --- государство или нацию, науку или мораль, язык или искусство и т.д., --- оно \emph{не может быть понято} из самого себя, а лишь как явление, порожденное обществом и отвечающее определённым общественным потребностям.

Поскольку же образ жизни людей в том или ином обществе в основе своей характеризуется способом производства, постольку и все другие общественные явления \emph{зависят в конечном счёте} от способа производства, вытекают из него.

\emph{Способ производства есть материально-экономическая основа общества}, определяющая всю его внутреннюю структуру.

В понятии экономической общественной формации \emph{отражается} прежде всего эта подчиненность, \emph{зависимость} всех общественных явлений от материальных отношений производства.

При исследовании общества \emph{выявляется также} взаимная связь общественных явлений друг с другом, тот факт, что все стороны общественной жизни органически связаны между собой.

Общественно-экономическая формация --- \emph{это не агрегат индивидов}, не механическая совокупность разрозненных общественных явлений, а \emph{целостная социальная система,} каждый компонент которой должен рассматриваться не изолированно, а лишь в связи с другими социальными явлениями и с обществом в целом, ибо каждое из них играет определённую и своеобразную роль в функционировании и развитии общества.

Целостность социальных систем и \emph{выражается понятием} общественно-экономической формации.

\emph{История общества складывается} из истории отдельных стран и народов, живущих в разнообразных географических и исторических условиях, имеющих свои этнические, национальные, культурные и другие особенности.

\emph{История многообразна}. На этом основании некоторые мыслители утверждали, что в ней \emph{якобы отсутствует} повторяемость, что все события, явления здесь \emph{сугубо индивидуальны} и задачей исторической науки может быть \emph{лишь фиксация} этих индивидуальных событий, их оценки с точки зрения какого-либо идеала.

Подобный подход к истории неизбежно \emph{приводит к} \emph{субъективизму}, ибо сам выбор идеалов и ценностей, с точки зрения которых должна оцениваться история, становится произвольным, \emph{теряется} объективный критерий для разграничения того, что является существенным, главным, определяющим в истории , а что производно, вторично, зависимо.

Понятие общественно-экономической формации \emph{позволяет}, наряду с фиксацией общих моментов в историческом развитии народов, \emph{чётко отделять} один исторический период от другого.

Каждая общественноэкономическая формация представляет собой \emph{определённую ступень} в развитии человеческого общества, качественно \emph{своеобразную систему} социально-экономических отношений.

\emph{История общества представляет собой историю развития и смены общественно-экономических формаций.}

Обычно \emph{выделяют следующие формации}: первобытнообщинную, рабовладельческую, феодальную, капиталистическую \emph{и некоторую будущую}, которую нередко называли коммунистической.

\emph{В рамках первобытнообщинной формации} происходило становление человека и были созданы предпосылки для дальнейшего развития.

Последовательность формаций \emph{не является обязательной схемой}, которой должна подчиняться история каждого народа, так как одни народы \emph{задерживаются} в своём развитии, другие \emph{минуют} целые формации.

История знает и различные \emph{переходные формы}.

\emph{Итак, общественно-экономическая формация --- это определённый тип общества, цельная социальная система, функционирующая и развивающаяся по своим специфическим законам на основе определённого способа производства.}

\emph{Экономическим скелетом}, т.е. базовой, фундаментальной структурой каждой общественно-экономической формации являются исторически определённые производственные отношения и связи.

Но формация \emph{включает в себя и другие} общественные явления и отношения, облекающие этот скелет плотью и кровью.

Поэтому возникает \emph{необходимость разобраться} в сложной структуре формации, \emph{хотя бы в общих чертах}.

\subsection{Структура общественно-экономической формации. Базис и надстройка}

Каждая общественно-экономическая формация (далее -- \emph{ОЭФ}) качественно отличается от других формаций. Но вместе с тем по своей структуре они имеют некоторые \emph{общие черты}, присущие всем или, во всяком случае, большинству формаций.

Каждое общество характеризуется определённым типом общественных отношений (\emph{структурой}).

\emph{Общественные отношения --- это} особый вид связей и взаимодействий, существующих только в обществе, возникающих в процессе социальной деятельности людей, т.е. деятельности в сфере производства, политики, духовной жизни и т.п.

Эти отношения называются общественными \emph{ещё и потому}, что они складываются из взаимодействия больших количеств людей, \emph{социальных групп} (классов).

Общественные отношения весьма \emph{многообразны}.

Существуют различные \emph{виды общественных отношений}: экономические, политические, правовые, социально-психологические, организационные, нравственные и т.п.

\emph{Чтобы разобраться} в этом многообразии, установить закономерную связь общественных отношений, \emph{необходимо выделить} отношения \emph{ведущие}, основные и отношения \emph{производные}, вторичные (по генезису, но не по значимости).

В историческом материализме \emph{принято делить} все общественные отношения прежде всего на \emph{материальные} и т.н. \emph{идеологические}, возникающие после и надстраивающиеся над материальными.

\emph{Материальные} (объективные) \emph{отношения} --- это прежде всего производственно-экономические отношения.

\emph{Материальными являются также} отношения человека к природе, отношения между производством и потреблением, исходные первичные отношения в сфере быта, в семье.

Понятие материальных общественных отношений \emph{шире} понятия экономических отношений.

\emph{Общим для всех} материальных отношений является то, что они формируются, \emph{не проходя} предварительно через сознание людей, \emph{первичны по отношению} ко всем другим видам общественных отношений.

\emph{Материальность в социальном смысле} (в смысле социальной объективности), например, отношений собственности, \emph{не следует} полностью отождествлять с материальностью в смысле вещественности.

Конечно, общество \emph{не может существовать} без материально-вещественного воплощения достижений человеческой деятельности.

Орудия труда, здания, пашни, парки, каналы --- \emph{всё это} создание человеческих рук, \emph{материализация} деятельности и идей человека.

Но «\emph{социальную материю}», объективную основу всех общественных отношений составляют прежде всего материальные общественные отношения, возникающие между людьми в процессе производства и воспроизводства их непосредственной жизни.

\emph{Надстроечные отношения}, т.е. \emph{отношения вторичные}, производные от материальных, надстраивающиеся над ними как в прямом, так и в переносном смысле, объединяются под общим названием \emph{идеологических отношений}.

\emph{Идеологические отношения} включают в себя отношения политические, правовые, нравственные, эстетические и т.п. \emph{Особенность} этих отношений состоит в том, что они возникают, проходя предварительно через общественное сознание, обязательно опосредуясь им.

\emph{Например}, отношения политические формируются на основе экономических отношений и интересов различных социальных групп, классов, но \emph{в соответствии с политической идеологией} этих групп, т.е. в соответствии с состоянием сознания этих групп, уровнем отражения в нём общих групповых интересов и целей.

\emph{Разграничение} материальных и идеологических отношений \emph{позволяет подойти} к определению понятий, характеризующих структуру и качественное своеобразие каждой ОЭФ, --- \emph{базиса} и \emph{надстройки}.

\emph{Базис ОЭФ --- это совокупность производственных отношений, составляющих экономическую структуру данного общества}.

\emph{Понятие базиса} \emph{выражает} социальную функцию производственных отношений как экономической основы общественной жизни.

\emph{Надстройка включает} в себя три группы явлений.

\emph{Во-первых}, общественные идеи, настроения, социальные чувства, т.е. \emph{идеологию и общественную психологию}.

\emph{Во-вторых}, различные организации и учреждения (\emph{материализовавшиеся идеи}) --- государство, суд, церковь и т.д.

\emph{В-третьих}, надстроечные (\emph{идеологические}) отношения.

\emph{Надстройка есть совокупность общественных идей, учреждений и отношений, возникающих на основе данного экономического базиса.}

Исторический материализм \emph{исходит из} признания первичности и определяющей роли базиса по отношению к надстройке, из того, что \emph{каждая ОЭФ} имеет свой базис и соответствующую надстройку.

\emph{Надстройка, как и базис, носит исторически конкретный характер.}

\emph{Каков} экономический базис данного общества, \emph{таковы и} господствующие в нём системы политических, правовых, религиозных, философских взглядов, а также соответствующие отношения и учреждения этого общества.

Отражая положение в базисе, \emph{надстройка} соответствующим образом \emph{фиксирует} и пронизывающие его противоречия.

\emph{Зависимость надстройки от базиса} состоит и в том, что в надстройке отражаются изменения, происходящие в экономическом строе общества. Это хорошо \emph{просматривается на примере} капиталистических отношений, которые в виде особого уклада зародились в недрах европейского феодализма.

Возникновение капиталистических отношений \emph{сопровождалось глубокими изменениями} в духовной жизни общества, связанными с появлением буржуазной идеологии и новой культуры, которые противостояли феодальной надстройке и \emph{носили прогрессивный характер}.

\emph{Развернулась борьба} всех антифеодальных общественных сил, борьба, кульминационным пунктом которой явились \emph{буржуазные революции}.

При капитализме движение трудящихся групп населения \emph{толкают буржуазию} на определённые изменения в экономическом базисе и политико-правовой надстройке, в других её элементах.

Движение молодого пролетариата в рамках раннего капитализма \emph{приводит к острейшим ситуациям}, вплоть до выдвижения лозунга \emph{пролетарской диктатуры} и попыток его реализации на практике.

\emph{Развитой капитализм} существенным образом сглаживает противоречия в экономическом базисе и политико-правовой надстройке, переводит их развитие в существенно более мирное русло.

В связи с последним традиционное учение о социалистической революции, диктатуре пролетариата \emph{должно быть последовательно проанализировано}, \emph{вплоть до замены} его \emph{обновлёнными концепциями} общественного развития, учитывающими современный \emph{опыт развитых стран}, а также попытку построения определённого типа общества \emph{в СССР} и ряде других стран.

\emph{Зависимость надстройки от базиса} выражается в том, что экономический базис определяет содержание политической и идеологической надстройки и её структуру.

\emph{Изменения в надстройке} происходят под воздействием изменений в базисе.

\emph{Преодоление старого базиса} и возникновение нового базиса вызывает преобразование во всей громадной надстройке.

Вместе с тем надстройка обладает \emph{относительной самостоятельностью} по отношению к своему базису.

Социальная система никогда \emph{не может быть} такой жёсткой и однозначно определённой, как системы механических зависимостей.

\emph{Воздействие базиса на надстройку} осуществляется через связь экономических и политических интересов социальных групп, сложную систему посредствующих звеньев между экономикой и различными формами идеологии и т.д.

\emph{Историю делают люди}, социальные группы, классы. Они преобразуют базис, совершают качественные изменения, меняют элементы надстройки, осуществляют политику, создают новые идеи и ведут идеологическое противоборство.

Зависимость надстройки от базиса \emph{не следует понимать упрощённо}, как автоматически действующий механизм.

\emph{Нельзя} любые изменения в надстройке \emph{объяснять только} экономическими причинами.

Взаимодействие самих элементов надстройки приводит к последствиям, иногда экономически \emph{никак не обусловленным}.

Экономика \emph{лишь в конечном счёте} определяет надстройку.

\emph{Надстройка всегда является} \emph{активной силой}, воздействующей на все стороны общественной жизни, в том числе и на свой базис, с целью его защиты, укрепления и развития.

В современную эпоху \emph{резко возрастает роль надстройки} как активного фактора истории.

Развитое западное общество \emph{всё больше надежд возлагает} именно на средства идеологического и политического воздействия, подкрепляемые целым рядом существенных изменений в экономическом базисе, прежде всего в виде системы более гибкого перераспределения производимого в этих странах богатства.

А вот надстроечная часть того \emph{реального социалистического общества}, которое развивалось в течение ряда десятилетий \emph{в СССР} оказалась в существенной мере односторонней, искажённой и оторванной от экономических процессов, \emph{что привело} в конечном итоге к необходимости радикального пересмотра её практики.

Базис и надстройка являются \emph{основными структурными элементами} любой \emph{ОЭФ}, характеризующими её качественное своеобразие, её отличие от других формаций.

Кроме базиса и надстройки \emph{ОЭФ включает в себя и \textsc{ряд} других} элементов общественной жизни, в том числе таких важных, как \emph{быт}, \emph{семья} и т.д.

Но именно базис и надстройка \emph{определяют специфику формации} как целостного социального организма.

\subsection{Единство и многообразие исторического процесса}

\emph{Развитие и смена ОЭФ} выражает поступательный ход истории.

При этом \emph{одна сторона} способа производства --- производительные силы --- является элементом, который \emph{обеспечивает преемственность} в поступательном развитии общества, определяет направление развития --- от низшего к высшему.

\emph{Вторая же сторона} способа производства --- производственные отношения --- \emph{выражает прерывность} в историческом развитии.

Становление, развитие \emph{ОЭФ}, переход к более высокой формации \emph{объясняются} действием \emph{закона соответствия} производственных отношений характеру и уровню развития производительных сил. Этот закон \emph{как тенденция пробивает} себе путь в развитии и смене формаций, достигая своей наиболее полной формы проявления в капиталистической формации.

С появлением капитализма \emph{история становится} в полном смысле слова \emph{всемирной}, поскольку \emph{устраняется} былая изолированность отдельных районов и сравнительная обособленность различных народов, \emph{впервые создаётся} единая мировая система хозяйства, единый мировой рынок.

\emph{Источником и основой} развития капиталистической ОЭФ являются \emph{производительные силы}, связанные с машинным производством, претерпевающим постоянные качественные изменения.

\emph{Темпы} экономического и социального развития с переходом к капиталистической ОЭФ, особенно на современном этапе, \emph{резко возрастают}.

\emph{Антагонистические формы}, свойственные раннему капитализму, сегодня \emph{в значительной мере преодолены}.

\emph{Распределение прибавочной стоимости} между владельцами средств производства и наемными работниками, а также другими категориями населения развитых стран, становится существенно \emph{более справедливым} (см. \emph{шведская}, вообще т.н. социал-демократическая \emph{модель} капитализма).

За сравнительно короткий исторический срок \emph{капитализм проходит ряд этапов}, начиная с периода первоначального капиталистического накопления через систему свободного предпринимательства \emph{к эпохе} государственно регулируемой экономики и развитой демократии.

Имевшие место на ряде этапов капитализма \emph{тенденции к застою} также существенно преодолены, не могут трактоваться как остановка в развитии производства и общества в целом.

\emph{Напротив,} современные наука и техника позволяют достигать высочайших темпов экономического развития.

Имевшая место в первой половине ХХ в. \emph{эпоха империалистического капитализма} также отчасти преодолена, приостановлен до известной степени рост милитаризма, процесс распространения политической реакции, особенно в связи с преодолением на пространстве бывшего СССР прямой опасности вступления в противоборство с развитыми странами Запада.

Всё это \emph{порождает надежду} на внедрение социалистических по сути преобразований разного уровня, конечно, при постоянном давлении на все структуры современного общества со стороны заинтересованных групп населения.

\emph{В то же время} \emph{сохраняется основное противоречие} капиталистической формации между общественным в значительной мере характером производства и в значительной же мере частной формой присвоения, между трудом и капиталом, между различными силами внутри самого этого общества на уровне различных государств и их групп.

\emph{Растёт потребность} в реализации последовательно социалистических отношений в рамках этого общества.

Современный развитой капитализм обнаруживает \emph{тенденцию к преодолению} огромного периода человеческой истории --- периода открыто антагонистического общества.

Вce крупные проблемы общества \emph{до сих пор разрешались в острой борьбе} социальных сил, классов, которая пронизывала все прежние формации, начиная с рабовладельческой. Менялся характер классов, менялся характер противоречий, \emph{но общим оставался сам тип исторического развития}, протекавшего в формах столкновения экономических и политических интересов социальных групп, борьбы классов.

\emph{Начав с самой жестокой формы} порабощения человека --- с рабства, история формаций шла по пути \emph{постепенного видоизменения} форм эксплуатации, \emph{замены} внеэкономических форм принуждения экономическими, \emph{развития} материальной заинтересованности в результатах производственной деятельности как непосредственных производителей, так и собственников средств производства.

\emph{Большим достижением} предшествующего развития в истории человечества \emph{стало могучее развитие} техники, науки, культуры, поднявшее человека на огромную высоту и создавшее предпосылки для преодоления социальных антагонизмов и перехода человечества на принципиально новый уровень социального бытия.

\emph{Задачей нового этапа истории}, начинающегося в настоящее время, является \emph{овладение} человеком своими собственными общественными отношениями и всестороннее \emph{развитие самого человека} на базе высшего развития материального и духовного производства, развития отношений сотрудничества и взаимопомощи на всех уровнях, от отдельных личностей, до мирового сообщества государств в целом.

\emph{Выше была рассмотрена} общая линия исторического развития в той мере, в какой оно определяется закономерностями движения материального производства. \emph{Но это не значит}, что тем самым уже объяснено общественное развитие в каждой точке исторического процесса.

\emph{Конкретная история гораздо богаче}, в ней действует масса факторов, которые разнообразят и видоизменяют исторический процесс, и потому его \emph{нельзя рассматривать} как нечто \emph{однолинейное}.

Историческое развитие \emph{есть результат} действия многих сил, и, чтобы понять конкретную историю, \emph{необходимо учитывать} все существенные факторы, участвующие в этом взаимодействии.

Диалектико-материалистическая философия \emph{должна преодолеть вульгаризацию} своих подходов к историческому процессу, \emph{не допускать превращения} своих положений \emph{в схему}, навязываемую конкретной \textsc{истории и} подменяющую изучение конкретных фактов.

\emph{Если кто-то заявляет}, «что экономический момент является будто \emph{единственно} определяющим моментом, то он \emph{превращает} это утверждение (утверждение об определяющей роли производства в \emph{конечном счете} -- \emph{Ред}.) в ничего не \textsc{говорящую,} абстрактную, бессмысленную фразу». (\emph{Ф. Энгельс}. К. Маркс и Ф. Энгельс. Соч., т. 37, \textsc{с.} 394).

На ход развития общества \emph{влияют различные моменты} надстройки, идеологии и т.д. \emph{Если не учитывать} этого влияния, не видеть случайностей, сквозь массу которых \emph{пробивает себе дорогу} экономическая необходимость, то «применять теорию к любому историческому периоду было бы легче, чем решать простое уравнение первой степени». (\emph{Там же}, с. 395).

Множество причин \emph{разнообразит} общий ход мировой истории.

Уже раньше говорилось о влиянии на общество \emph{условий географической среды}, которое было, в особенности на более ранних ступенях развития общества, одной из существенных причин, определивших неравномерность хода мировой истории, выдвижение одних и отставание других народов (всегда относительное).

\emph{Нельзя не учитывать} также воздействия на ход истории вторичных по отношению к экономике \emph{факторов, таких, как} государство, своеобразие культуры, традиции, идеология, общественная психология и т.д.

\emph{Существенным моментом} в истории \emph{является взаимодействие}, взаимовлияние различных \emph{народов}, которое может происходить в самых разнообразных формах, начиная от войн и завоеваний и кончая торговлей, культурным обменом. Оно может осуществляться \emph{во всех сферах} общественной жизни, от экономики до идеологии.

Своеобразия отдельных стран \emph{невозможно понять, не учитывая неравномерности} мирового исторического развития.

Одни народы \emph{уходят вперед}, другие \emph{отстают} в своём развитии, некоторые в силу ряда конкретных причин \emph{минуют} целые общественные формации.

\emph{В каждый период} на протяжении всей письменной истории на земле существовала не одна какая-либо формация, а народы, находящиеся на разных ступенях общественного развития, и между ними \emph{возникали сложные взаимодействия}.

\emph{Последовательность смены формаций} не у всех народов была одинакова.

Так, \emph{у славян и германских народов}, населявших Центральную и Восточную Европу, \emph{разложение доклассового строя} происходило в эпоху, когда рабовладельческая формация (\emph{Древний Рим}) исчерпала себя и шла к упадку, поэтому складывавшийся \emph{рабовладельческий уклад не развился у них} в формационное качество, и они перешли от родового строя прямо к феодализму.

\emph{Характер взаимовлияния народов}, находящихся на различных ступенях исторического развития, зависит от природы их общественного строя.

\emph{Признавая поступательный характер} общественного развития, смену низших по своему типу общественных формаций высшими, \emph{исторический материализм отнюдь не рассматривает} это развитие как фатальный процесс.

\emph{Многообразие истории}, особенности в развитии отдельных континентов, стран обусловлены, как уже отмечалось, целым \emph{комплексом} разнообразных \emph{причин}.

Но при всём разнообразии истории различных народов \emph{в каждый конкретный исторический период} имеются определённые \emph{ведущие тенденции} общественного развития.

Для характеристики \emph{того или иного отрезка} всемирной истории в соответствии с ведущими его тенденциями и используется \emph{понятие исторической эпохи.}

Так, в наше время в отдельных странах встречаются ещё \emph{остатки феодализма}, но об «\emph{эпохе феодализма}» применительно к современности \emph{говорить было бы нелепо}.

Понятие эпохи может связываться и с определёнными этапами \emph{ведущей формации}.

Для выделения \emph{основной тенденции эпохи} необходимо установить, какие силы стоят в центре эпохи, определяя главное её \emph{содержание}, главное \emph{направление} её развития, главные её \emph{особенности}.

\emph{В отличие от понятия ОЭФ}, характеризующего определённую ступень развития общества, \emph{понятие исторической эпохи более конкретно}, выражает многообразие процессов, совершающихся в данное время на данном этапе истории.

\emph{В одну и ту же эпоху} в различных частях земли сосуществовали \emph{разные формации}: например, рядом с народами \emph{Древней Греции} и \emph{Рима}, находившимися в рабовладельческой формации, жили народы, пребывавшие на ступени первобытнообщинного строя.

\emph{Рядом с капитализмом}, утвердившимся в Европе и Северной Америке, \emph{сохранялись} феодальные и дофеодальные отношения в некоторых странах и регионах.

Понятие исторической эпохи охватывает \emph{и типичное, и нетипичное} для данного отрезка истории.

В каждой эпохе \emph{были, есть и будут} отдельные, частичные движения \emph{то вперёд, то назад,} бывают различные уклонения от среднего типа и темпа движения.

\emph{Наконец}, понятие эпохи \emph{может связываться} с переходом от одной \emph{ОЭФ} к другой, когда человечество переживает переходное время, когда совершаются грандиозные перемены в его жизни.

Так, различные периоды перехода от феодализма к капитализму характеризуются как \emph{эпоха Возрождения}, \emph{эпоха буржуазных революций}.

\emph{Наше время} во всемирной истории является также переходной эпохой --- эпохой, как минимум, \emph{существенных трансформаций} традиционного капиталистического общества.

Что касается перехода капиталистического качества в некое принципиально иное --- социалистическое и коммунистическое, то \emph{вопрос этот} сегодня в теории исторического материализма \emph{должен быть} \emph{рассмотрен как бы заново}, с учётом всего того нового опыта, который обрело человечество за последние полтора столетия, и \emph{особенно последние пятьдесят лет}.

\emph{Современное западное рыночно-демократическое общество} \emph{усвоило} ряд важнейших признаков, которые традиционно связывали с социалистическим качеством.

В то же время \emph{в бывшем СССР} дали знать о себе ряд признаков, подобных феодальным, или даже дофеодальным.

Короче, \emph{вопрос} о перспективах трансформации современного капитализма в новую ОЭФ \emph{откладывается на некоторое время}, нужное как практике, так и теории, чтобы определиться точнее.

\section{Исторические формы общности людей: племя, народность, нация. Понятие социальной структуры общества}

Способ производства материальных благ \emph{лежит в основе} всех общественных отношений.

Он определяет структуру общества, типы социальных групп, а также более или менее устойчивые \emph{исторические формы общности} людей.

\subsection{Род и племя как исторические формы общности людей доклассового общества}

Первыми формами общности людей в истории человечества являются \emph{род} и \emph{племя}.

По данным антропологии, этнографии и археологии, родовая организация пришла \emph{на смену стадному образу жизни}, по-видимому, в эпоху верхнего палеолита, когда появился современный тип человека.

\emph{Род можно определить как первичный производственный социальный и этнический коллектив доклассового общества, обладающий} общностью происхождения, общностью языка, общими обычаями, верованиями, чертами быта и культуры.

\emph{Род} --- коллектив, в осуществлении всех функций которого первостепенную роль играли не только производственные, но и \emph{кровнородственные связи}.

Род имел \emph{общие места} поселения и охоты.

Экономической основой рода являлась \emph{первобытнообщинная собственность}.

Коллектив (сообщество) людей, составляющих род, вёл совместное хозяйство на основе \emph{групповой собственности} и уравнительного распределения продуктов.

Изменение и развитие хозяйственной деятельности \emph{приводили к} видоизменению форм родовой организации общества.

\emph{Племя --- более крупная, чем род, общность людей}, обычно насчитывающая в своём составе от нескольких сот до нескольких тысяч (а иногда и десятков тысяч) человек.

\emph{Каждое племя состояло} по крайней мере из двух, а развитое --- и из \emph{нескольких родов}.

Внутри племени \emph{каждый род продолжал оставаться} самостоятельной социально-производственной подсистемой.

Но в то же время \emph{племя вызвало к жизни} и новую форму общественной собственности, новый вид общественной организации.

Наряду с собственностью рода существует и \emph{племенная собственность} --- прежде всего территория (место поселения родов, место охоты, общие пастбища и другие угодья).

Возникает \emph{потребность управления всем племенем}, а в связи с этим \emph{появляются} вожди, жрецы, военачальники и такие органы управления, как \emph{совет племени}, наряду с общим собранием воинов или взрослых членов племени.

\emph{Родо-племенная форма} общности для своего времени была \emph{единственно возможной формой} функционирования и развития производства, а также первобытного общества в целом. \emph{Этим объясняется} как наличие подобной формы у всех народов, находящихся на стадии первобытнообщинного строя, так и её живучесть в течение многих тысячелетий.

Родо-племенная общность \emph{дала известный простор} для развития хозяйственной деятельности и развития первобытной культуры, содействовала сплочению людей. Она \emph{создала благоприятные условия для} хранения и накопления производственного опыта и зачатков кульутры, для совершенствования языка.

\emph{В то же время} кровнородственные связи \emph{ограничивали} численный рост первобытных коллективов, \emph{затрудняли} общение, в частности передвижение людей, развитие экономических отношений.

\emph{Сила традиций}, облегчающая функционирование общественного организма, была настолько велика, что \emph{препятствовала} закреплению каких-либо изменений в жизни первобытных общин.

\emph{Обострение противоречий} в родо-племенной организации \emph{в конце концов привело} к преодолению этой формы общности, \emph{замене} её другими формами.

Строго говоря, образование племён уже положило \emph{начало расчленению} единой многофункциональной общности.

Поскольку племя несло лишь часть общественных функций, то тем самым было положено \emph{начало обособлению этнической общности} от непосредственно хозяйственных функций.

Когда возникает \emph{парная семья}, появляется \emph{тенденция к обособлению} семейно-брачных отношений, кровнородственных связей от этнических общностей.

\subsection{Возникновение основных социальных групп и развитие форм общности людей. народность. Нация}

С возникновением общественного разделения труда (отделения скотоводства от земледелия, выделения ремесла), с появлением меновых отношений и отношений имущественного неравенства \emph{родо-племенная организация должна уступить место} новой форме общности людей.

В основу этой новой формы общности легли уже \emph{не кровнородственные связи}, а определённые \emph{территориальные связи} между людьми, принадлежащими к разным родам, но тесно связанными друг с другом характером хозяйственной деятельности, торговыми или другими экономическими отношениями.

Этой новой формой общности людей явилась \emph{народность}.

Формируясь на базе классовых отношений в производстве, сменивших первобытнообщинные, \emph{народность выступает как общность} людей, проживающих на одной территории, связанных общим языком, особенностями психического склада, особенностями культуры и образа жизни, закреплёнными в обычаях, нравах, традициях.

Хозяйство из примитивно-коллективного, каким оно было в родо-племенной организации, превращается в личную частную собственность, \emph{появляется и растёт частная собственность людей}, \emph{использующих труд других}, не относящихся к их семьям людей.

С формированием \emph{народности} \emph{постепенно утрачивается} непосредственная связь хозяйства с более широкой формой общности людей.

Представляя собой более развитую, чем племя, общность людей, \emph{народности содействовали} развитию производства, накоплению и обмену производственным опытом, достижениями культуры, совершенствованию языка, всех форм общения между людьми на сравнительно обширной территории \emph{с десятками и сотнями тысяч} населяющих её людей.

\emph{Но и эта форма общности} оказалась с течением времени \emph{слишком ограниченной} для развития производства материальных благ и обмена, когда последний стал охватывать самые различные виды деятельности людей.

\emph{Патриархально-натуральное хозяйство уступило место товарному производству}.

Позднее, товарно-капиталистические отношения \emph{ликвидировали экономическую разобщенность} отдельных хозяйственных районов, \emph{стали укреплять} связи между жителями данной народности и близких к ней народностей, \emph{содействовали образованию} общего для них языка, общих черт культуры, \emph{сплачивали} людей в ещё более устойчивые общности --- в \emph{нации}, в общности «с одним законодательством, с одним национальным классовым интересом, с одной таможенной границей». (\emph{К. Маркс и Ф. Энгельс.} Соч., т. 4, с.428).

Нередко в силу ряда причин \emph{создание централизованного государства} \emph{протекало более быстро} и завершалось ранее, чем все народности, проживающие на данной территории, успевали сложиться в нацию.

\emph{В таких случаях} формировалось многонациональное государство \emph{с привилегированным положением} одной или нескольких наций, сложившихся ранее других и ставших ведущей силой в создании централизованного государства.

\emph{Многонациональные государства} возникали также и тогда, когда основные социальные группы складывающейся нации, опираясь на централизованную государственную власть, \emph{подчиняли другие} народы, как правило стоящие на более низком у роже экономического развития.

Многие однонациональные государства XIX в. \emph{превратились в колониальные империи} с многообразным в национальном отношении населением.

Однако \emph{во всех случаях} нации \emph{формировались на базе} капиталистических производственных отношений.

\emph{Без общности экономической жизни нет нации}.

Однако этот признак \emph{лишь в сочетании} с другими признаками, которые возникают в более ранний, докапиталистический период, но развиваются на основе тесных экономических связей, \emph{даёт нацию}.

\emph{Помимо общности экономической жизни} общий язык, общая территория и некоторые особенности общественной психологии народа, проявляющиеся в специфических чертах культуры, составляют основные признаки нации.

При этом некоторые \emph{общие, не решающие черты} психологии и культуры, присущие данной нации, отнюдь \emph{не снимают коренного различия} психологии представителей различных социальных групп внутри данной нации.

\emph{Нация --- это такая устойчивая совокупность людей, которая связана общим языком, общей территорией; общностью экономической жизни и некоторых особенностей общественной психологии, закрепленной в специфических чертах культуры данного народа, которые отличают его культуру от культуры других народов.}

\emph{Формирование наций помогло} ликвидации феодализма и утверждению капитализма.

Но для капиталистического общества по мере его развития даже \emph{национальные рамки} со временем \emph{оказываются слишком узкими}.

\emph{Капитализм создаёт} как национальный, так и мировой \emph{рынок}, который не только консолидирует нацию как экономическую общность, но и устанавливает экономические связи между всеми нациями, \emph{превращая} в конце концов \emph{капитализм в мировую систему хозяйства}.

Всё это приводит к серьёзным противоречиям, к возникновению \emph{двух тенденций в развитии наций}.

\emph{Первая тенденция --- складывание наций}, пробуждение национальной жизни, борьба с феодальной раздробленностью и докапиталистическими формами принуждения человека.

\emph{Вторая --- усиление экономических связей} между нациями, ломка национальных перегородок «\emph{интернациональным}» капиталом.

Этот процесс принимал нередко \emph{характер острой борьбы} между тенденциями, происходил в виде захватов чужой территории, колонизации, экспансионизма, что в свою очередь \emph{порождало справедливую борьбу} порабощенных народов за свое освобождение.

\emph{Фаза острого противоборства} наций грубыми средствами в настоящее время в целом преодолена, хотя рецидивы продолжают случаться.

\emph{Нации вступили}, по крайней мере внешне, \emph{на путь} взаимоуважения, поддержки, сотрудничества практически во всех сферах жизни народов.

\emph{Таковы три типа общности} людей, исторически сменяющие друг друга в процессе поступательного развития человечества: родо-племенная общность, народность и нация.

\emph{Смена этих типов} показывает, что развитие общества, общественного производства и обусловленный последним социальный прогресс \emph{требовали} расширения этнических общностей, достижения их большей устойчивости, укрепления связей между ними.

Однако типы исторической общности лишь в основном соответствуют определённым формациям.

\emph{\textsc{В} истории редко бывают ситуации}, когда эти типы общности представлены в «\emph{чистом}» виде.

\emph{В результате неравномерности} экономического развития сейчас на нашей планете \emph{имеют место все} экономические отношения и соответственно все исторические формы общности --- от первобытнообщинных до развитых наций.

Если генетически племя \emph{предшествует} народности, а народность нации, то конкретно-исторически во всемирном масштабе, а очень часто в рамках одного народа они \emph{сосуществуют} и взаимодействуют друг с другом.

\subsection{Социальная структура и её изменения. Возникновение и сущность классов}

\emph{Классы} представляют собой большие группы людей, на которые делится общество.

Но в обществе есть \emph{множество других} больших групп людей, разграничительные линии между которыми лежат в иной плоскости, чем классовое деление.

\emph{Между людьми существуют различия} возрастные, половые расовые, национальные, профессиональные и т.д.

Естественные различия \emph{сами по себе не порождают} социальных различий и лишь при определённых условиях могут быть связаны с социальным неравенством, различиями.

Так, \emph{неравенство между расами} не естественного, а исторического происхождения.

Социальное \emph{неравенство полов} также объясняется не естественными, а историческими причинами.

\emph{Деление на классы} обычно вообще не связано с естественными различиями, оно существует внутри одних и тех же рас, этнических групп и т.д.

Разделение общества на классы --- \emph{это результат} экономических причин: оно имело место даже там, (например, \emph{в древних Афинах}), где не было никакого завоевания, что иногда рассматривают в качестве причины возникновения социальных классов.

\emph{Источник классов --- разделение труда} внутри общества, предполагающее \emph{обособление производителей}, занятых различными видами производства, и обмен между ними продуктами труда.

Вместе с общественным разделением труда и обменом \emph{развивается частная собственность на средства производства}. В результате этого в обществе появляются социальные группы, занимающие \emph{неодинаковое положение} в общественном производстве, --- \emph{классы}.

\emph{Классовое деление}, имея свою основу в экономическом строе общества, находит отражение и в его политическом строе, \emph{и в духовной жизни}.

\emph{Совокупность отношений между классами образует классовую структуру общества}, составляет основу взаимодействия классов, которое исторически нередко, даже скорее чаще, чем реже, принимало \emph{форму острого противоборства}, доходившего в решающие периоды жизни ОЭФ до вооруженных восстаний, революций, гражданских войн.

\emph{Наряду с классовыми различиями} в обществе существуют и другие социальные различия, например различия \emph{между городом и деревней}, между \emph{людьми физического и умственного труда} .

К социальным различиям относятся также \emph{внутриклассовые различия}, выражающие наличие внутри классов более дробных групп.

Существуют также \emph{особые группы людей}, которые не по всем признакам могут быть отнесены к социальным классам. Это \emph{так называемые социальные прослойки}, к числу которых можно отнести, \emph{например, интеллиненцию}, а с другой стороны --- \emph{деклассированные элементы}, люмпен-пролетариат.

\emph{Совокупность классов, общественных слоёв и групп, система их взаимоотношений образуют социальную структуру общества}.

\emph{При смене одного способа производства другим} изменяется социальная структура, одни классы (подклассы) \emph{сменяются другими}.

В условиях рабовладения, а также феодализма структура общества выступает \emph{в своеобразном облачении}.

В ряде стран Востока (\emph{например, в Индии}) социальное расчленение принимало вид \emph{деления на касты}, т.е. замкнутые группы людей, связанных единством наследственной профессии.

В других рабовладельческим странах (\emph{Древней Греции, Риме и др}.), а также в феодальном обществе классовые различия при помощи государственной власти юридически \emph{закреплялись в сословном делении} населения.

\emph{Сословное деление}, складываясь на базе классового, в то же время не соответствовало ему полностью, а привносило в него \emph{элемент иерархии власти} и юридических привилегий.

\emph{В дореволюционной России}, например, существовало сословное деление на дворян, духовенство, купцов, мешан и крестьян.

Капиталистическая формация \emph{упрощает классовое деление} общества, упраздняя, по крайней мере в принципе, сословные привилегии.

\emph{Капиталистическая система экономики} представляет собой систему \emph{отношений найма} и \emph{использования} (людей, предметов, земли и т.д.).

\emph{Различают основные} и \emph{неосновные}, или \emph{переходные классы}.

Основные классы возникают \emph{из господствующего способа производства}, его экономических отношений.

Неосновные классы \emph{связаны с уходящим} способом производства или \emph{зарождающимся} новым способом производства.

\emph{Классовый состав} общества отличается большой сложностью. \emph{К тому же} классы не являются столь замкнутыми группами людей, как сословия. Происходят \emph{перемещения} отдельных лиц и групп из одних социальных слоёв в другие.

\subsection{Национальные отношения в современном обществе и их перспективы}

Формирование наций вызвало к жизни и \emph{национальный вопрос}, в развитии которого можно выделить \emph{три этапа}.

\emph{Первый} --- \emph{эпоха становления} капиталистического рыночного общества и \emph{разложения} феодализма, эпоха \emph{преобразования} народностей в нации. В этот период человечество, в лице Европы, \emph{пережило первый тур} национальных войн и революций.

\emph{Второй этап} --- \emph{период распространения} капиталистических порядков, \emph{превращения} капитализма в мировую систему и \emph{перерастания} «\emph{свободного}» капитализма в капитализм, разделённый на зоны влияния наиболее развитыми его представителями. Это \emph{эпоха мощных национальных движений}, когда народы, ещё не успевшие консолидироваться в нации, начинают борьбу за своё конституирование и независимость.

\emph{Третий} --- \emph{современный} --- этап --- \emph{этап распада} колониальных систем, \emph{завершения} формирования наций у народов, поначалу попавших под власть передовых наций.

Процесс формирования наций \emph{охватил большую часть человечества}. Это эпоха \emph{качественных изменений} в самих развитых капиталистических странах.

\emph{При капитализме} \emph{любые общественные движения}, в том числе и национальные, непосредственно или косвенным образом \emph{приобретают политический характер,} выступают как общественно-политические движения и течения. Однако сама политическая деятельность при этом неминуемо развивается \emph{в национальных формах}.

\emph{Национальный вопрос} ещё и потому является политическим, что в условиях государственно-политических, государственно-правовых отношений он \emph{не может не быть вопросом} правовым и вопросом государственным, связанным с конституцией и государственным устройством страны, с политикой, направленной на регулирование национальных отношений.

К этому следует добавить, что \emph{межгосударственные отношения}, неизбежно носящие политический (внешнеполитический) характер, \emph{не могут не принимать}, как правило, характера национальных отношений.

Национальные отношения в современном мире \emph{играют громадную роль} и \emph{в области духовной жизни} народов даже самых развитых стран.

\emph{Все виды духовной деятельности}, особенно в сфере художественного творчества, \emph{протекают в национальной форме}.

С национальным вопросом тесно связаны такие явления современного мира, как \emph{национализм} и \emph{интернационализм}.

\emph{Националист} не просто исходит из наличия известной общности национальных интересов. В противоположность \emph{интернационалисту} он \emph{преувеличивает} их значение.

\emph{По отношению к другим народам} (в том числе и в рамках данного государства, если последнее многонационально) \emph{национализм} выражается в \emph{восхвалении} «\emph{своего}», национального, независимо от того, каково его социально-политическое содержание, \emph{в том числе и в восхвалении отживших} социальных и политических институтов, обычаев, традиций, \emph{в забвении} или \emph{пренебрежении} особенностями и интересами других наций и народностей, в явном или молчаливом \emph{признании неполноценности} других народов и \emph{исключительности} «\emph{своего}» народа.

\emph{Национализм} проявляется также \emph{в противодействии} установлению широких связей с другими народами или в признании таких связей лишь с народами, \emph{этнически близкими}, \emph{в противодействии} исторически прогрессивному процессу сближения наций и слиянию некоторых из них с другими нациями, даже если этот процесс осуществляется «\emph{естественным}» \emph{путём}, в ходе экономического развития страны, \emph{а не в результате} насильственных мер, насильственной ассимиляции.

\emph{В} своих \emph{крайних выражениях} национализм принимает форму \emph{шовинизма}, когда пренебрежение к особенностям и интересам других народов \emph{перерастает в неприязнь}, а подчас и \emph{в ненависть} к ним, доходящую до стремления не просто поработить и использовать их, но и \emph{уничтожить}, \emph{истребить}.

Особенно часто шовинизм доходит до подобных форм выражения, когда национализм \emph{сочетается с расизмом}.

Национализм неразрывно \emph{связан с космополитизмом}.

\emph{Внешне} они \emph{противостоят друг другу}: национализм \emph{преувеличивает} национальные особенности, космополитизм \emph{отвергает} их существенное значение.

\emph{Космополитизм} --- выражение \emph{тенденции к интернационализации} экономических и иных связей между странами. Но, \emph{обособляя} эту тенденцию и \emph{про\textsc{тиво}поставляя} её второй --- \emph{тенденции к национальному сплочению}, космополитизм \emph{оправдывает} экономические (а затем и политическое) \emph{подавление} других народов.

\emph{Космополитизм --- это оборотная сторона национализма более сильных}, более развитых стран, наций.

\emph{Интернационалист} в противоположность националисту и космополиту \emph{является патриотом}. Он \emph{понимает} подлинные интересы народа своей страны, его стремление к прогрессу и процветанию. Но \emph{интернационалист} \emph{объективно подходит} к подобным же интересам и других стран, стоит на позициях \emph{справедливого взаимного учёта} интересов различных стран и народов.

В современном обществе \emph{усиливается тенденция к интернационализации} жизни вообще, и прежде всего экономической, а также социальной и духовной.

Происходит \emph{сплочение народов}, их \emph{сближение}, \emph{возникновение} у сотрудничающих народов всё больше общих черт, \emph{развитие} потребности в новом общении, более широком, чем национальное.

Хотя \emph{продолжает сохраняться} и \emph{множество сложных проблем}, особенно в связи с неодинаковостью ступеней развития народов --- экономического и культурного.

Однако \emph{тенденция к сближению} и \emph{сотрудничеству} стран и народов \emph{на рубеже II} и \emph{III тысячелетий} является \emph{ведущей}.

\section{Политическая организация общества}

С появлением социально-классового членения общества структура общественной жизни \emph{усложняется}.

Возникают \emph{новые виды} общественных отношений --- политические и правовые.

Формируется \emph{сфера политической жизни}, включающая в себя ряд организаций и \emph{социальных институтов}, неизвестных доклассовому обществу.

Важнейшим из этих институтов является \emph{государство, представляющее собой организацию политической власти господствующих социальных групп}.

В классовом обществе возникают также \emph{политические движения}\underline{,} \emph{политические партии} и различные \emph{общественные организации}, создаваемые для выражения и осуществления интересов тех или иных общественных групп, слоёв, классов.

Все эти \emph{организации} и \emph{социальные институты}, взятые в их взаимосвязи, \emph{образуют политическую организацию общества.}

\emph{Политическую организацию общества} можно, следовательно, определить как \emph{систему институтов, организаций, движений и учреждений, регулирующих политические взаимоотношения между социальными группами, классами, нациями, государствами}.

\emph{Сфера политической жизни} общества охватывает политические институты и отношения, политическое сознание и политическую деятельность.

\subsection{Переход от неполитической (общинной) организации общества к политической}

\emph{В первобытном обществе} социальные \emph{отношения регулировались} силой привычки, обычаями и традициями, в которых закреплялся многовековой опыт совместной жизни и труда.

\emph{Главной силой} в общественной жизни являлся \emph{сам народ}, в необходимых случаях вооруженный.

\emph{С появлением более развитых форм} общественного разделения труда и частной собственности на средства производства родо-племенные органы управления \emph{оказались неспособными} функционировать в новых условиях.

Родо-племенной строй \emph{был заменён} государством.

\emph{Если родовые органы} власти покоились на примитивной общественной собственности и общности интересов первичных коллективов людей --- родов, племён, \emph{то государство призвано было обслуживать} потребности, выросшие на почве частнособственнических отношений.

Появляются новые группы (классы) с существенно различными интересами, между которыми \emph{появляются острые противоречия}.

С появлением таких противоречий возникает \emph{необходимость в особой организации власти}, уже не совпадающей со всем народом и призванной удерживать имеющийся \emph{статус-кво} между группами с несовместимыми интересами.

Государство является \emph{силой}, как бы «\emph{стоящей над обществом}», всё более «\emph{отчуждающейся от него}».

\emph{Три} \emph{основные черты} или признака характеризуют государство: публичная власть, налоги с населения и территориальное деление.

\emph{Публичная власть} противостоит непосредственной организации вооруженного народа, которая существовала в родовом обществе.

Власть включает постоянное \emph{чиновничество}, особые отряды вооруженных людей (\emph{армия}, \emph{полиция}, милиция, жандармерия), \emph{карательные}, разведывательные органы государства и соответствующие «вещественные» придатки --- \emph{тюрьмы} и т.п.

\emph{Налоги} с населения нужны для содержания аппарата власти, а также (в современном обществе) для перераспределения доходов в пользу других групп населения в обществе.

\emph{Территориальное деление} приходит на смену делению по кровнородственному признаку.

Территориальное деление людей \emph{содействует} развитию экономических связей и созданию политических условий для их регулирования.

\emph{Государство регулирует} отношения между людьми, их группами внутри своей территории и защищает эту территорию с её населением от посягательств других государств.

Итак, государство представляет собой \emph{политическую надстройку над экономическим базисом}.

Государство \emph{регулирует} (прежде всего \emph{через систему правовых норм}) всю совокупность социальных отношений: \emph{национальных} (если общество многонациональное), \emph{семейных} и др., содействуя упрочению определённого социально-экономического порядка, поддержание которого есть основная задача всякого государства.

Государство \emph{решает ряд} экономических и культурных задач.

\emph{Некоторые мыслители} объясняют возникновение государства главным образом \emph{духовными факторами} --- взаимной договоренностью, растущей духовной зрелостью людей, «\emph{осознавших}» невозможность без государства организовать общественную жизнь, а также свойствами человеческой природы, потребностями общественной психологии и морали и т.д. При таком подходе к вопросу политика, государство рассматриваются как явления, которые, раз возникнув, \emph{существуют вечно}.

Социальная философия диалектического материализма исходит из \emph{исторического характера государства}.

Государство выполняет свои функции с помощью \emph{права}.

\emph{В первобытнообщинном обществе} отношения между людьми регулировались обычаями, традициями, нормами морали (общепринятыми нормами поведения, нарушение которых вызывало общественное осуждение).

Разделение труда, появление частных интересов и частной собственности усложнили общественную жизнь и вызвали \emph{потребность в таких нормах}, которые \emph{в принудительном порядке навязывались} членам общества.

\emph{Право --- это совокупность норм поведения, закрепленных в законах, которые санкционированы государством.}

\emph{Право узаконивает} нормы, регулирующие прежде всего отношения между собственниками, \emph{даёт} юридическую санкцию, необходимую для функционирования экономических отношений.

\emph{Юридического регулирования требует} и передача собственности другому лицу или наследнику.

\emph{Право оформляет} экономические отношения и обусловленные ими социальные отношения социальных групп, классов, а также положение семьи и взаимоотношения её членов, положение национальных меньшинств и т.д.

\emph{Право} также \emph{определяет} правовой статус всех общественных институтов, организаций, религиозных общин, \emph{юридически оформляет} положение, права и обязанности отдельных граждан.

\emph{Право охватывает} в той или иной степени все стороны жизни общества, все виды деятельности людей, все формы общественных отношений.

\emph{Как государство} не может обойтись без права, \emph{так и право} --- ничто без государства, стоящего на страже правовых норм.

\emph{Право возникло} вместе с государством. Государственно-правовую надстройку над экономическим базисом \emph{можно с полным основанием считать} единым институтом.

С образованием государства и возникновением права сложились \emph{новые виды отношений} между людьми, неизвестные ранее, --- отношения \emph{политические} и \emph{правовые}.

\emph{Политические отношения} --- это отношения между большими социальными группами людей, классами.

Однако \emph{не всякие} отношения между социальными группами \emph{можно считать} политическими. \emph{Существуют ещё} отношения экономические, социальные и т.д.

\emph{Политические отношения} между социальными группами \emph{выражают} в концентрированном виде их \emph{коренные, прежде всего экономические интересы}. Эти отношения, как и всякие надстроечные отношения, складываясь, \emph{проходят через сознание} людей.

Политические отношения строятся в соответствии с \emph{политическими идеалами} и \emph{целями}, представлениями и взглядами политических партий и политических деятелей.

\emph{\underline{Политика}} социальной группы, класса \emph{есть более или менее сознательная} (во всяком случае, для его наиболее развитой части) \emph{линия поведения} этой \emph{группы, класса по отношению к другим группам, государству}.

\emph{Политика проводится} в экономической, социальной и культурной областях, хотя сами по себе эти области могут непосредственно лежать за границами политики.

Так можно говорить о \emph{хозяйственной} (например, торговой, финансовой) политике, проводимой посредством государства, о политике той или иной группы в области \emph{национальных отношений}, в области народного \emph{образования} и т.д.

Вся \emph{система политических отношений} выражает прежде всего экономические отношения групп того или иного общества и \emph{представляет собой} \emph{форму}, в которой только и могут функционировать эти отношения.

Экономическая жизнь на определённой ступени потому и \emph{порождает политику}, что вне этой формы она не может функционировать и развиваться.

Следует отметить несомненное \emph{возрастание роли политики}, и прежде всего политической организации, в жизни современного общества.

Возрастание роли политики \emph{объясняется многими причинами}:

\begin{itemize}
\item \emph{развитием} общественного производства и возрастанием масштабов управления хозяйством и обществом ;
\item \emph{ростом} сознательных начал в жизни современного общества, несмотря на наличие стихийности в развитии экономики;
\item \emph{возрастанием} удельного веса и значения политических форм отстаивания своих интересов;
\item \emph{повышением} экономической \textsc{роли} государства в жизни общества.
\end{itemize}

\subsection{Развитие политической организации и её роль в жизни ощества}

\emph{На протяжении истории} отношения собственности на средства производства \emph{меняли} свои типы и формы, а вместе с этим \emph{менялись} типы социальных структур.

В соответствии с этим \emph{становились иными} политические отношения, взгляды и учреждения --- словом, вся политическая организация общества.

\emph{В рабовладельческом обществе} положение рабов, разделение свободных членов общества на касты, замкнутые группы, часть которых занимала особо привилегированное положение в обществе, --- все это закреплялось в праве и правовых отношениях.

В этом обществе государство, по существу, являлось \emph{диктатурой рабовладельцев}: \emph{охраняло} их собственность и привилегии, \emph{держало в повиновении} остальную, большую часть населения.

Кроме того, государство и армия удерживали в повиновении \emph{население насильственно захваченных территорий}.

Формы правления в рабовладельческом обществе \emph{были многообразны}: \emph{восточные деспотии} (Китай, Индия, Ближний Восток), \emph{империи} (Александра Македонского, Римская империя), \emph{республики} (Афины, Рим в первый период).

Другие политические организации \emph{были развиты слабо}, хотя существовали определённые политические группировки и союзы, достаточно вспомнить \emph{борьбу партий в Греции} и в \emph{Риме}.

В политической жизни почти всех древних стран видную роль играли \emph{религиозные организации}.

\emph{Феодальный способ производства} вызвал к жизни новый тип политической организации общества, правовых и политических отношений.

\emph{Экономическое принуждение} и здесь \emph{дополнялось внеэкономическим}, хотя последнее видоизменилось (не было права на жизнь крепостных).

\emph{Феодальное право} давало \emph{привилегии одним} сословиям и делало почти \emph{бесправными другие}.

\emph{Государство} в феодальном обществе \emph{значительно разрослось} по сравнению с государством в рабовладельческом обществе.

Чиновничество, судейский аппарат, постоянная армия, офицерский корпус стали так многочисленны, что \emph{поглощали значительную часть}, а иногда и большую часть правящего класса, близких к нему групп.

Ведущей формой государства становится \emph{абсолютная монархия}, пришедшая на смену \emph{разрозненным образованиям} (княжествам, герцогствам и т.п.).

В то же время \emph{республиканские формы} правления \emph{встречались} в феодальную эпоху значительно реже и, как правило, в средневековых городах.

Отличительная черта феодального общества --- \emph{огромная роль церкви} в структуре государственной власти, и не только как силы идеологической, но и политической, а нередко и военной.

В Европе, \emph{например, католическая организация} являлась могучей ценртализованной силой, имеющей свои военные формирования (\emph{ордена}). С нею \emph{приходилось делить власть} светским феодалам, которые нередко подчинялись феодалам церковным.

\emph{Помимо государства}, правовых институтов, церкви в систему политической организации феодального общества входили \emph{сословные организации} и союзы.

Политических организаций и партий \emph{в современном смысле} тогда не было.

\emph{Своего наивысшего развития} политическая организация достигает \emph{в капиталистическом обществе}, и прежде всего в современном.

\emph{Формально-юридическое равенство} людей перед законом, в противовес сословному неравенству, --- характерная черта правовых норм этого общества.

Право \emph{утрачивает местный характер}, его нормы действую по всей стране.

Сложность и многогранность экономических и других общественных отношений определяет \emph{сложность и многогранность}, \emph{многообразие} правовых норм.

Правовые нормы в капиталистическом обществе \emph{регулируют не только} отношения взаимоотношения социальных групп, классов, \emph{но и все виды} экономических отношений в области производства и распределения продуктов, в частности товарно-денежных и финансовых отношений, \emph{положение} общественных организаций, политических партий, прессы.

\emph{Детально разрабатываются} нормы гражданского и уголовного права, личные права граждан и т.д.

При капиталистическом способе производства \emph{завершается процесс централизации государственной власти}, начатый ещё при феодализме, \emph{возникают} национальные (в ряде случаев многонациональные) \emph{централизованные государства}.

Феодальная монархия \emph{уступает место} республике или ограниченной монархии (с парламентом и ответственным перед ним правительством).

Происходит \emph{разделение властей}: законодательной, исполнительной и судебной.

Формирование наемной армии, как правило, \emph{уступает место} всеобщей воинской повинности и т.д.

\emph{Главные органы} государственной власти --- чиновничество, армия, полиция, разведка, тюрьмы \emph{совершенствуются} и \emph{растут}.

Возрастает удельный \emph{вес политических отношений}, роль политических форм и методов отстаивания своих интересов.

\emph{Политические партии} наряду с государством занимают важнейшее место в системе политической организации.

Орудием, \emph{играющим огромную роль} в политической жизни общества, становятся средства массовой информации (\emph{СМИ}) --- печать, радио, телевидение, наконец, в значительной своей части интернет --- «\emph{четвёртая власть}».

Правда, в ряде стран \emph{ещё и сегодня} имеют место факты наличия \emph{системы цензов} (имущественного, оседлости, образования и т.д.), неравенства женщин и мужчин, национального неравенства.

Имели место чудовищные \emph{факты фашизации} целых стран.

Политические институты были подвергнуты значительной \emph{деформации в бывшем СССР}.

В то же время \emph{в современном обществе} значительно \emph{выросла экономическая роль государства}.

\emph{Создание} значительной государственной собственности, \emph{использование} возросших налоговых поступлений, \emph{субсидирование} государством научных исследований в области новой техники \emph{расширили эту роль} в сфере программирования производства, регулирования цен, непосредственного управления социальными отношениями.

Политическая организация современного общества \emph{требует} \emph{дальнейшего совершенствования}.

\emph{Государство должно стать} представителем интересов действительно всех социальных групп. Его хозяйственная и культурная деятельность приобретает \emph{всё большее значение} и становится главной.

Современное цивилизованное государство \emph{должно отказаться от функций} агрессии, захвата чужих территорий, покорения других народов.

\emph{Внешние функции} современного государства должны \emph{служить целям}:

\begin{itemize}
\item \emph{обороны},
\item \emph{укрепления} взаимопонимания и сотрудничества государств,
\item \emph{поддержки} других государств, подвергшихся агрессии,
\item \emph{нормализации} хозяйственных и культурных связей со всеми другими странами,
\item \emph{утверждения} мирного сосуществования государств с существенными различиями в своём социальном строе,
\item \emph{осуществление} разрядки напряженности в международных отношениях.
\end{itemize}

С изменение государства происходит \emph{изменение и права} в современном обществе.

\emph{Право начинает служить} всем социальным группам и слоям, \emph{узаконивает} новые общественные отношения, \emph{охраняет} все виды собственности, \emph{определяет} правовое положение государственных органов и общественных организаций, права и обязанности отдельных граждан и т.д.

В современном обществе происходят масштабные \emph{изменения во всех звеньях политической организации}.

\emph{Возникают} многочисленные общественные организации и движения, \emph{включаясь} в политическую организацию.

Происходит \emph{дальнейшая демократизация} всей политической системы управления, \emph{возрастание роли} населения в управлении обществом.

Но и современное государство \emph{не может отказаться от мер принуждения} по отношению к тем членам общества, которые нарушают государственные законы, действую вопреки, тем более во вред, интересам других людей, групп и общества в целом.

\emph{Совершенствование} политической организации современного общества позволяет говорить о \emph{новых формах} государственного и общественного управления.

Есть \emph{основания говорить} о политической организации общества, \emph{охватывающей все или многие} страны мира.

Разумеется, государства продолжают функционировать в национальных, территориально ограниченных рамках, но в то же время налицо экономические и политические \emph{объединения во всемирном} масштабе (ООН, СБСЕ, Европейский Союз, ЕВРАЗЕС и т.д. и т.п.).

Заключаются \emph{многочисленные соглашения и договоры}, соответствующие интересам различных государств, в том числе имеющих существенные различия.

\section{Структура и формы общественного сознания}

От анализа закономерностей общественного Развития, экономических и политических отношений \emph{мы переходим теперь к рассмотрению} такой важной сферы жизни общества, как общественное сознание.

Когда имеют в виду общественное сознание, то \emph{отвлекаются от всего индивидуального}, личностного (о личности разговор ещё впереди) и берут взгляды, идеи, характерные для данного общества в целом или для определённой социальной группы.

Хотя общественное сознание формируется прямо или опосредованно отдельными людьми, но оно, объективируясь, \emph{выходит из-под их власти}, становясь достоянием всего общества, Так, научные открытия, художественные ценности \emph{принадлежат всему человечеству}.

\emph{Подобно тому}, как общество не есть простая сумма составляющих его людей, \emph{так и} общественное сознание \emph{не есть сумма} «\emph{сознаний}» отдельных личностей.

Общественное сознание \emph{есть нечто большее}, чем эта сумма, оно представляет собой качественно \emph{особую духовную систему}, которая будучи порождена и обусловлена в конечном счёте общественным бытием, \emph{живёт своей относительно самостоятельной жизнью} и \emph{оказывает} большое влияние на каждого человека, \emph{заставляет} его считаться с исторически сложившимися формами общественного сознания как с чем-то вполне реальным, объективным, хотя и нематериальным.

Общественное сознание в его исторически сложившихся формах является \emph{составной частью духовной культуры общества}.

\subsection{Понятие духовной культуры}

Термин «\emph{культура}» в буквальном смысле слова означает «\emph{обработка}» (от латинского \emph{cultura}) и употребляется обычно в сопоставлении с природой, рассматриваемой в её естественном состоянии, независимо от человека и его труда.

\emph{Под культурой} мы и понимаем прежде \emph{всего способы и результаты деятельности человека}, созданные им ценности (см. также определение культуры, приведенное в конце разделов о сознании и познании).

Культуру условно подразделяют на \emph{материальную} и \emph{духовную}, хотя элемент духовности в широком смысле, \emph{идеальности}, содержится во всяком культурном образовании.

Это деление \emph{относительно}. Ведь изготовление тех же орудий труда и вообще предметов, удовлетворяющих материальные потребности человека, \emph{невозможно без участия} его мысли.

С другой стороны, \emph{продукты духовного производства} --- идеи, художественные образы, общественные нормы и заповеди выступают в определенной \emph{вещественной форме}: в рукописях, книгах, картинах, нотах, чертежах и т.д.

К духовной культуре в собственном смысле слова \emph{относятся результаты} духовной деятельности человека --- наука, философия, искусство, мораль, политика, право, и \emph{соответствующие учреждения} (научные институты, школы, театры, библиотеки, музеи и др.), \emph{степень} его интеллектуального, эстетического и морального развития. А также сами \emph{способы, технологии} деятельности по производству духовно-культурных ценностей.

С понятием культуры связано \emph{приобретение} человеком \emph{знаний} и \emph{опыта} в той или иной области деятельности, \emph{усвоение} и \emph{принятие} какой-то \emph{системы ценностей}, \emph{выработка} определённого \emph{поведения}.

\emph{Каждый человек} с ранних лет \emph{находится под влиянием} определённой культуры --- предметов, идей, ценностей, образцов поведения.

\emph{Воспитание} и \emph{образование} человека и состоит в \emph{приобщении} его к существующей культуре, в \emph{усвоении} накопленных обществом знаний, умений, навыков, а также духовных ценностей и норм поведения, принятых в обществе.

Сама постановка образования и воспитания, развитие системы просвещения являются \emph{важными показателями уровня культуры} данного общества.

Но человек \emph{не только потребитель} созданной культуры. Он является и её \emph{продуктом}, и её \emph{творцом}.

Духовная культура \emph{несёт на себе} \emph{отпечаток} характерных черт общественно-экономической формации, социальных групп, которые её составляют, и в этом смысле она \emph{совпадает с надстройкой}.

\emph{Вне надстройки} находятся такие явления духовной культуры, как \emph{научные знания} о природе, \emph{язык}, являющийся формой национальной культуры, \emph{нормы логического мышления} и другие явления, которые могут обслуживать различные экономические системы, интересы различных социальных групп.

Весьма \emph{специфическим}, но при этом очень важным элементом всякой культуры является \emph{идеология}.

В культуре проявляются и особенности \emph{общественной психологии}, характерные для каждой эпохи, для той или иной социальной группы.

\subsection{Общественная психология и идеология}

Что же такое идеология и общественная психология?

Материальные экономические и иные \emph{отношения}, социальные \emph{условия} существования людей, их \emph{повседневная деятельность} и накапливаемый опыт \emph{отражаются в человеческой психике} в виде чувств, настроений, мыслей, побуждений, привычек.

Перечисленное в отражённом виде мы и называем обычно \emph{общественной психологией}.

Общественная психология \emph{вырастает непосредственно} под влиянием определённых условий социального бытия людей, их деятельности. Она \emph{не выступает в виде} обобщенной системы взглядов и воззрений, а \emph{проявляется в отдельных суждениях, эмоциях, чувствах, настроениях, волевых актах} и т.п.

Идеи и взгляды людей на уровне общественной психологии \emph{не имеют теоретического выражения}, они \emph{носят эмпирический характер}, интеллектуальные моменты здесь \emph{переплетаются} с эмоциональными до неразличимости.

Общественная психология представляет собой \emph{часть обыденного сознания} людей.

Термин «\emph{обыденное сознание}» употребляется в литературе в \emph{более широком} смысле, чем термин «\emph{общественная психология}».

Обыденное сознание \emph{содержит не только} отражение социальных условий, \emph{но и результаты} эмпирического наблюдения над природой, которое осуществляет человек в своей повседневной жизни, знания и навыки\textsc{,} почерпнутые в процессе труда, и др.

Психология \emph{в обществе, разделённом} на социальные группы, \emph{носит на себе печать} этого членения, выражет условия жизни различных групп.

\emph{Ещё до того}, как в сознании той или иной социальной группы формируется идеология, в нём \emph{на психологическом уровне} проступают определённые черты, резко или же заметно отличающие сознание той или иной группы от сознания других групп.

\emph{В современном обществе} многие различия в психологии различных социальных групп \emph{смягчаются} в связи с ослаблением социальных различий в жизни этих групп, в частности в плане стирания различий между городским и сельским населением, между видами преимущественно физического и умственного труда.

\emph{Постепенно преодолевается} социальногрупповая ограниченность в психологии людей. Формируется \emph{разносторонне развитая} личность.

Общественная психология предполагает \emph{постоянное внимание} к ней.

Осознание многими людьми различных моментов общественной жизни играет \emph{большую, если не решающую роль} в развитии общества.

Хотя такое осознание \emph{не может служить} последовательно научным доказательством необходимости тех или иных изменений в жизни общества, оно \emph{является выражением} того факта, что население, люди \emph{настроены} на эти перемены.

Конечно, эта настроенность \emph{должна быть организована}, \emph{оформлена}, для того чтобы могла проявиться в массовом созидании нового.

\emph{Особенности} истории данной страны, нации \emph{накладывают отпечаток} на психологию всего населения, всех ее групп.

\emph{Различия}, которые имеются у подобных социальных групп в различных странах, обусловлены особенностями национальной истории, традициями.

Эти традиции \emph{касаются} некоторых черт и особенностей психического склада.

Психические \emph{черты} нации, \emph{своеобразие} быта и нравов её отдельных слоев проявляются и \emph{в искусстве}, которое выражает некоторые особенности художественного восприятия действительности, исторически выработавшиеся эстетические вкусы народа и т.д.

Если общественная психология представляет собой обыденное сознание, которое \emph{складывается непосредственно} в процессе повседневной жизнедеятельности людей, их взаимного общения, то \emph{идеология} выступает как \emph{более или менее стройная система взглядов}, положений, идей (политических, философских, нравственных, эстетических, религиозных).

Идеология основывается \emph{на более обширном и обобщённом} социальном опыте --- историческом и современном.

Если обыденное сознание складывается \emph{само собой, стихийно}, в процессе жизнедеятельности и взаимодействия людей, то идеология \emph{преимущественно} выступает \emph{как продукт сознательной деятельности}, требующих специальных усилий идеологов.

«\emph{Преимущественно}», ибо, например, религия в первобытном обществе \emph{возникла в неразвитом сознании} первобытных людей, несомненно, стихийно. И \emph{лишь в дальнейшем} религиозные деятели --- жрецы, священнослужители, теологи приводят религиозные взгляды \emph{в определённую систему}.

По своему социальному положению \emph{идеологи} той или иной социальной группы, класса \emph{могут и не принадлежать} к данной группе. Но они \emph{служат этой группе}, являются её интеллектуальной частью.

\emph{Интеллектуалы}, интеллигенция \emph{сознательнее} и \emph{точнее} отражают и выражают развитие групповых интересов и политических отношений в обществе.

Существует точка зрения, рассматривающая \emph{идеологию как ложное}, \emph{иллюзорное сознание}.

При этом \emph{имеются в виду те идеологические концепции}, которые рассматривают мысли, идеи как самостоятельные сущности, якобы обладающие независимым развитием и подчиняющиеся только своим собственным, внутренним законам.

\emph{Речь идёт здесь о тех идеологах}, которые не признают или \emph{не осознают} того факта, что материальные условия жизни людей, в головах которых совершается мыслительный процесс, \emph{в конечном счёте} определяют ход этого процесса.

Такая идеология, которая представляет собой \emph{идеалистическое понимание и истолкование истории}, и есть «\emph{ложное}» сознание, рождающее мистификации и иллюзии.

Что же касается идеологии, которой руководствуются \emph{здравые общественные силы}, то она может быть охарактеризована как такая, которая \emph{не основывается на} идеях, принципах, \emph{выдуманных} тем или иным отдельным человеком, а \emph{является} общим \emph{выражением действительных} отношений исторического движения.

Следует различать сами \emph{типы идеологии}:

\begin{itemize}
\item \emph{научную}, или \emph{ориентированную на науку}, т.е. представляющую собой \emph{адекватное} отражение общественных отношений, и
\item \emph{ненаучную}, отражающую эти же отношения в \emph{иллюзорной}, даже \emph{фантастической} форме.
\end{itemize}

В обществе, состоящем из различных социальных групп, \emph{решающее значение} для определения степени адекватности идеологии \emph{имеет та роль}, которую та или иная социальная группа играет в реализации назревших потребностей жизни общества на данной ступени его развития.

\emph{Лишь постепенно} общественная идеология становится \emph{всё более и более адекватной} своему предмету --- историческому процессу, опираясь в частности на \emph{науки об обществе}, которые стали бурно развиваться сравнительно недавно, начиная с середины XIX в.

\emph{Будучи частью} общественного сознания, \emph{идеология влияет} и на развитие познания природы, т.е. естествознания.

Теоретическое обобщение данных естественных наук \emph{невозможно без} определённого мировоззрения.

\emph{Развитие естественных наук} показывает, что они \emph{не нейтральны} к философий, что \emph{в них} \emph{происходит столкновение} мировоззрений, что достижения естественных наук \emph{служат порой исходным пунктом} прямо противоположным выводов у идеологов различных социальных групп.

Подводя итог сказанному, можно дать \emph{следующее определение} идеологии.

\emph{Идеология есть система воззрений и идей, прямо или опосредованно отражающих экономические и социальные особенности общества, выражающая положение, интересы и цели определённой общественной группы (групп) и направленная на сохранение или изменение существующего общественного устройства}.

Рассмотрим теперь кратко \emph{вопрос о взаимоотношении} общественной психологии и идеологии.

\emph{Общественное сознание} --- и как общественная психология, и как идеология --- \emph{определяется} экономическими и социальными \emph{отношениями}, которые в каждом данном обществе проявляют себя прежде всего \emph{как интересы} определенных общественных групп.

А \emph{что такое} интересы?

\emph{Интерес} --- это \emph{не только} определённое психическое \emph{переживание}, выражающееся в избирательной, целеустремленной направленности человека на приобретение или освоение тех или иных благ--- материальных или духовных.

Интерес \emph{существует и объективно}, будучи связанным с бытием человека, \emph{с условиями} его существования, \emph{с} его \emph{потребностями}, которые лежат в основе его отношения к ценностям, его психических переживаний.

У общества в целом, у социальной группы, у нации имеются свои \emph{объективные интересы}, \emph{не всегда, однако, осознаваемые} людьми.

\emph{Идеология, как осознание своих интересов, вносится} в сознание группы её \emph{теоретиками}, наиболее теоретически подготовленными людьми, \emph{специалистами-идеологами}.

\emph{Идеология} \emph{не может} во всём своем объёме \emph{вырасти} из общественной психологии, \emph{не может рассматриваться}, например, в качестве некоего «\emph{сгустка психологии}», хотя и связана с последней и, несомнененно, испытывает воздействие психологии.

\emph{Психология} группы \emph{и} её \emph{идеология} имеют общие \emph{социальные корни}, в чём и заключена \emph{возможность} приобщения всех членов группы к своей идеологии.

\emph{Личный интерес} представителя той или иной социальной группы по природе своей \emph{не обязательно должен отделять} себя от общего группового интереса --- группы, нации.

\emph{Важно}, чтобы различные \emph{спекуляции} на личных и групповых интересах \emph{не уводили} людей от реального разрешения тех противоречий, которые возникают в связи с реализацией этих интересов.

\subsection{Формы общественного сознания, их социальная функция и особенности}

Общественное сознание выступает в \emph{различных формах}: общественно-политические и правовые взгляды и теории, философия, мораль, искусство, религия.

\emph{Каждая} из этих форм, отражая общественное бытие и активно воздействуя на него через деятельность людей, имеет \emph{свой объект} и \emph{свой способ отражения}, \emph{по-своему влияет} на общественное бытие и сознание людей, \emph{характеризуется своей особой ролью} в идейно-политическом взаимодействии социальных групп в обществе.

\emph{На ранних ступенях} развития общества общественное \emph{сознание не расчленялось} на отдельные формы.

\emph{Примитивному общественному} бытию первобытных людей, крайне низкому уровню материального производства \emph{соответствовало примитивное же}, \emph{недифференцированное} сознание.

Умственный труд \emph{ещё не отделился} от физического, сознание людей было \emph{непосредственно вплетено} в материальную деятельность и в материальное общение людей, в язык реальной жизни.

Однако \emph{уже в доклассовом обществе} на определённых ступенях развития трудовой и познавательной деятельности \emph{возникают зачатки} таких форм общественного сознания, как \emph{искусство}, \emph{мораль}, религия.

\emph{С ростом разделения труда}, деятельности, с появлением различных социальных отношений, дифференциации социальных групп, государства общественная \emph{жизнь стала значительно более сложной}.

Соответственно \emph{усложнилось и} приобрело дифференцированный характер общественное сознание.

\emph{Разделение труда} на материальный и духовный, превращение последнего в \emph{монополию меньшинства} означало \emph{всё большее обособление} сознания от материальной практики людей.

Сознание приобретает \emph{относительную независимость} от общественного бытия.

Сознание способно \emph{представить себя теперь полностью независимым} от бытия \emph{и даже первичным} по отношению к бытию, может перейти к образованию «\emph{чистой}» теории, теологии, философии, морали и т.д.

\emph{На деле же} все эти «\emph{чистые}» формы сознания так или иначе выражали реальные условия и отношения общества, выступали как \emph{идеологическое отражение интересов} определённых социальных групп, сил.

\emph{Рассмотрим кратко отдельные формы} общественного сознания, их особенности и функции в связи с породившими их историческими условиями и социальными потребностями.

\subsubsection{Политическое и правовое сознание}

Вопросы политического и правового сознания в значительной степени \emph{освещались уже} в предшествующих главах: о социальной структуре, о политической организации общества. Здесь мы остановимся лишь \emph{на выяснении специфики этих двух форм сознания} с точки зрения способа отражения в них экономической и социальной структуры.

\emph{Политическая идеология представляет собой систематизиро-ванное, теоретическое выражение взглядов определённой социальной группы, класса на политическую организацию общества, на формы государства, на отношения между различными социальными группами, на их роль в жизни общества, на отношения с другими государствами и нациями и т.п.}

Политическая идеология служит важнейшим \emph{средством влияния} на политическую власть, \emph{защиты}, \emph{обоснования} и \emph{укрепления} определённого политического порядка и его экономических основ.

Тесно связанная с политической идеологией, \emph{правовая идеология является систематизированным теоретическим выражением правосознания социальной группы,} т.е. её \emph{взглядов на природу и назначение правовых отношений, норм и учреждений, на вопросы законодательства, суда, прокуратуры и т.п.}

Правовая идеология ставит своей \emph{целью защиту} и \emph{утверждение} \emph{правового порядка}, соответствующего интересам этой группы.

\emph{Как всякая форма теоретического сознания}, политическая и правовая идеологий выражают свои положения \emph{в логической форме} и опираются на предшествующее развитие данной отрасли знания.

Обе получают своё отражение \emph{в специальных трудах} по теории государства и права.

Однако, обращаясь к политическому сознанию, мы \emph{имеем дело не только с} политическими учениями и теориями, \emph{но и с} политическими программами и платформами, с политической стратегией и тактикой.

\emph{Политическая стратегия} ставит определённые \emph{цели} движения на более или менее длительный период.

\emph{Политическая тактика} помогает выработать определённую \emph{линию поведения} в той или иной конкретной обстановке.

\emph{Цели,} которые выдвигают социальные группы, политические силы в их программах и платформах, \emph{выражают интересы} и \emph{стремления} этих групп, сил, их волю, диктуемую материальными условиями их жизни.

\subsubsection{Мораль}

В формировании сознания и воли людей, в регулировании их поведения \emph{особую роль играет} мораль.

\emph{Зачатки морали} появились ещё в первобытном обществе.

\emph{Освободив индивида} от родовых пут, включив его в более сложный комплекс общественных отношений, классовое общество \emph{стимулировало развитие элементов его самосознания}, поставило перед ним множество новых вопросов.

Эти вопросы \emph{касались отношения} \emph{к новой социальной общности}, к людям определённой социальной группы, к государству и т.д. Их содержание \emph{выходило за рамки прежних традиций} и обычаев рода или племени.

Иначе говоря, \emph{требовались} новые нормы поведения, \emph{рождались} различные взгляды на эти нормы, а также на старые обычаи и традиции.

\emph{С усложнением сферы нравственной жизни} общества появляются своеобразные \emph{моральные кодексы} (основные нормы, правила, заповеди) и \emph{доктрины}, являвшиеся в течение длительного времени преимущественно \emph{частью религиозных учений}.

С развитием философии \emph{мораль становится областью философского знания}, предметом \emph{этики}.

Развитое \emph{моральное сознание} --- сознание связи человека с другими людьми в повседневном общении --- \emph{включается в общее мировоззрение} человека, составляет часть этого мировоззрения, так или иначе связанную с освещением вопросов \emph{о сущности человека}, его положении и роли в окружающем мире, с представлением \emph{о смысле его жизни}, \emph{о добре} и \emph{зле}, \emph{нравственном идеале} и \emph{нравственных ценностях}.

\emph{Выбор} поступков, их \emph{оценка} часто сопровождаются сложными размышлениями и психологическими переживаниями относительно нравственного характере этих поступков.

Человек, воспитанный в духе определённой морали, сам \emph{сознаёт} свой \emph{моральный долг} (т.е. свои личные обязанности по отношению к другим людям и определенной общности), сам \emph{оценивает} свои поступки, морально \emph{осуждает} себя \emph{за неправильный выбор} поступка, \emph{за нарушение} своих обязанностей, своего долга.

Эта \emph{самооценка} поведения, \emph{чувство личной ответственности} за поведение, за выбор поступков носит название \emph{совести}.

\emph{Особенностью морали} \emph{как способа регулирования} человеческого поведения является то, что она \emph{не опирается} непосредственно \emph{ни на какие специальные учреждения}, которые принуждали бы к соблюдению моральных норм (в отличие от права, за которым стоит \emph{сила государства}, способного принуждать к соблюдению норм права).

\emph{За моралью стоит сила} убеждения, примера, общественного мнения, воспитания, традиций, \emph{сила} нравственного авторитета отдельных лиц, организаций или учреждений.

\emph{Моральные нормы} \emph{не являются столь} детально разработанными и строго регламентирующими определённый порядок действий, \emph{как} юридические \emph{или} организационные нормы.

Вместе с тем моральные нормы \emph{распространяются} на такие отношения между людьми, \emph{которые не регулируются} государственными органами \emph{или} общественными организациями (дружба, товарищество, любовь и др.).

\emph{В отличие от} административных, правовых и т.п. норм, имеющих своей непосредственной целью \emph{создание} и \emph{укрепление} определённого социального порядка, мораль \emph{обращена прежде всего к внутреннему миру} и к поведению человека.

\emph{В морали} на первый план выдвигается её \emph{воспитательная роль}.

Воздействуя на индивида, на его психологию, сознание, она осуществляет свою \emph{роль регулятора поведения}, способствует созданию соответствующих \emph{нравственных отношений} между людьми в труде, в быту, в повседневной жизни и общении.

\emph{Это не противоречит} ни тому факту, что мораль имеет общественную природу, ни тому, что моральные нормы и оценки применяются к поведению целых групп, коллективов.

Моральное сознание --- сильный \emph{побудитель действий} больших совокупностей людей.

Исходя из сказанного, \emph{мы можем} определить \emph{мораль} как систему взглядов и представлений, норм и оценок, касающихся регулирования поведения индивидов, согласования поступков отдельных лиц с интересами других людей или определённой общности, способов воспитания людей, создания и укрепления определённых нравственных качеств и отношений.

Осуществляя свою функцию регулирования, мораль превращает свои нормы и оценки \emph{во внутренние побуждения людей}, в их моральные чувства и свойства, в сознание личной обязанности и личной ответственности.

\emph{Этические системы}, разработанные идеологами тех или иных социальных групп, как и мораль этих групп, носят ка себе \emph{печать своего времени}, групповых особенностей.

\emph{Основной вопрос}, которым занималась \emph{этика}, состоял в обосновании нравственного (\emph{добродетельного}) поведения.

\emph{Выводя} требования добродетели из \emph{якобы вечного качества человека}, \emph{прежняя этика} \emph{находила} в этом качестве, в этой «\emph{природе человека}» \emph{альтруистические} (человеколюбивые) или \emph{эгоистические} черты.

Соответственно на первый план выдвигался либо \emph{интерес} (счастье, удовольствие, наслаждение) \emph{отдельного индивида}, как правило идеализированный, либо \emph{всеобщий интерес}, также выступающий в идеализированной \emph{форме универсального морального закона}, которому человек обязан подчинить свои личные стремления и желания.

\emph{Старая этика искала, но не могла найти} пути сочетания личного и общественного интереса, счастья и долга, эгоизма и самопожертвования.

\emph{Мораль зависит} от экономических, социальных отношений.

Как писал американский социолог \emph{Дж. Дэвис}, «контроль капиталистических интересов над религией привёл к тому, что \emph{нравственные нормы христианского общества} в значительной степени \emph{приспособились} к нравственным нормам капитализма». (\emph{Дж. Дэвис}. \emph{Капитализм и его культура}. М., 1949, с. 403).

\emph{Возникает вопрос}: \emph{существуют ли} в обществе, разделённом на различные социальные группы, классы, такие \emph{моральные нормы}, которые необходимы \emph{для всякого человеческого общежития}?

Такие нормы \emph{существуют}.

Это некоторые \emph{простые нормы человеческой нравственности}, формирующиеся в процессе всего исторического развития народов.

Эти простые нормы \emph{призваны охранять} совместную жизнь людей \emph{от} тех или иных \emph{эксцессов}, угрожающих ей (от физического насилия, оскорбления), \emph{требуют} элементарной честности в повседневном общении и т.д.

\emph{К сожалению}, эти простые нормы нравственности и справедливости \emph{часто нарушаются}.

Многими они \emph{признаются на словах}, но попираются на деле.

\emph{Остаётся только надеяться} на прогресс и в области морали и нравственности тоже, \emph{на формирование нового качества} личности, в котором моральные, нравственные нормы будут занимать ещё более важное место, чем сегодня.

Многое зависит \emph{от воспитания}.

Человек, \emph{личность будущего} будет сочетать в себе идейную убежденность и жизненную энергию, культуру, знания и умение их применять.

Этот человек, эта личность \emph{будут жить в подлинно свободном обществе}, в нравственно здоровой атмосфере гуманизма, взаимопомощи и взаимовыручки, подлинного товарищества и дружбы народов и наций.

Требуется, однако, \emph{сознательное отношение каждого человека}, или значительного большинства людей, к общественному долгу, единство слова и дела, которое станет повседневной нормой поведения.

Всё это предполагает \emph{активную борьбу} уже сегодня \emph{всех здоровых сил} общества против отклонения от норм морали, \emph{против рецидивов} узкогрупповой, эгоистической психологии.

\emph{Мораль будущего} будет утверждать подлинно человеческие отношения между людьми.

Она будет включать в себя \emph{все лучшие достижения} морального прогресса человечества, станет высшей ступенью этого прогресса.

Эта мораль будет действительно \emph{стоять выше} групповых, классовых противоречий и противоположностей.

\subsubsection{Искусство}

В духовной жизни современного общества \emph{вместе с наукой} всё большее значение приобретает искусство.

\emph{Искусство --- одна из древнейших форм общественного сознания}; его возникновение относится к доклассовому обществу.

Исследования первобытной культуры свидетельствуют о том, что \emph{начиная с периода палеолита} люди постепенно \emph{научились не только} более целесообразно изготовлять необходимые им орудия, \emph{но и создавать} первые художественные произведения.

Искусство появилось как \emph{форма эстетического} (\emph{по законам красоты}) освоения мира, как деятельность, создающая предметы, которые были предназначены не для обработки земли или охоты на зверя, но \emph{воплощали творческую фантазию} человека, его идеи и переживания.

П\emph{отребность}, которую удовлетворяло искусство, в создании вещей, которые доставляли бы \emph{радость} человеку.

Сама эта потребность развилась в ходе развития \emph{художественной деятельности} как одной из форм творческой деятельности человека.

Эстетическая деятельность требовала от человека \emph{особых эстетических способностей}, формировала \emph{эстетические чувства}, вкусы, оценки, переживания, идеи, являющиеся \emph{специфической формой отражения} мира человеком.

\emph{Позднее} искусство \emph{выделилось} в самостоятельную сферу деятельности, обособленную от материального производства.

«Исключительная концентрация художественного таланта в отдельных индивидах и связанное с этим подавление его в широкой массе есть следствие разделения труда». (\emph{К. Маркс и Ф. Энгельс}. Соч., т. З, с. 393).

Искусство стало преимущественно \emph{делом немногих избранных} --- поэтов, художников, скульпторов, музыкантов и т.д.

Искусство становится особым \emph{предметом теоретического анализа}. Появилась \emph{эстетика} как философская наука о сущности и законах освоения мира по законам красоты. Возникли отдельные \emph{отрасли искусствоведения}, изучающие различные виды искусства.

\emph{Вне} профессиональной сферы искусство развивалось в форме \emph{народного творчества} (мифология, фольклор и др.), уходящего своими корнями в доклассовое общество и создававшего на протяжении веков неувядаемые художественные ценности.

Одним из \emph{коренных вопросов эстетики} и искусствоведения стал вопрос об отношении эстетического сознания (представления о прекрасном, безобразном и др.) \emph{и искусства к действительности}. Этот вопрос философы, теоретики искусства, художники \emph{решали по-разному}.

\emph{Материалистические теории} в эстетике утверждали \emph{определяющую роль} действительности в формировании эстетического сознания.

\emph{Идеалисты}, напротив, полагали, что эстетическое сознание и искусство \emph{не зависят} от общественных отношений.

\emph{Историческое развитие} эстетических представлений свидетельствует, однако, о том, что они \emph{значительно расходились} у людей различных социальных групп, различных эпох.

\emph{В ответ} на утверждение о «\emph{неоспоримости}» (т.е. пригодности для всех эпох) античного идеала красоты, воплощенного в \emph{образе Венеры Милосской}, \emph{Г.В. Плеханов} заметил, что первобытные художники, судя по многим сохранившимся рисункам, решительно \emph{не могли бы найти} в этом образе какой-либо красоты, что искусство средневековья \emph{было очень далеко} от признания этого идеала.

Однако даже в рисунках первобытного человека \emph{мы находим нечто} доставляющее нам эстетическое наслаждение, или по крайней мере удовлетворение.

Произведения великих западноевропейских писателей XIX в. --- \emph{Стендаля}, \emph{Бальзака}, \emph{Гюго}, \emph{Диккенса и др}. могли вырасти \emph{лишь на почве} общественных отношений того времени, \emph{в том числе как выражение} протеста против неограниченной власти денег, волчьей морали, присущих раннекапиталистическому обществу.

Подобный протест мы обнаруживаем и отечественных авторов --- \emph{Толстого}, \emph{Достоевского}, \emph{Тургенева}, \emph{др}.

Но \emph{зависимость искусства} и эстетических взглядов от социальных условий очень сложна, многократно опосредована.

К тому же на художественное воспроизведение действительности большое влияние оказывает \emph{индивидуальность самого художника}, его талант, его мировоззрение и художественная школа, его связь с определёнными традициями и т.д.

\emph{Общая закономерность в развитии искусства} состоит в том, что наиболее значительные его произведения, вошедшие в духовную сокровищницу человечества, были художественным \emph{воплощением правды жизни}, передовых идеалов и стремлений людей определённой эпохи, в том числе в отношении выражения внутреннего мира личности.

При этом \emph{национальная форма искусства} помогала художникам выражать важные идеи своего времени, если только художник, будучи сыном своего народа, \emph{не пренебрегал} достижениями других народов.

Без учёта международного \emph{взаимодействия культур} мы \emph{не сможем понять} многое не только в культуре нашего времени, но и в культуре прошлых эпох.

\emph{Настоящее искусство} служит \emph{одновременно} интересам своего народа и человечества, своему времени и будущему.

Искусство, связанное с жизнью народа, является важным \emph{фактором социального прогресса}. Оно осуществляет эту роль \emph{через художественное освоение мира}, через удовлетворение эстетической потребности человека.

\emph{Отражая} действительность в художественных образах, \emph{искусство действует посредством} этих образов на мысли и чувства людей, на их стремления и поступки, действия.

\emph{Лучшие произведения} искусства передаются от одного поколения к другому и служат как \emph{средством познания} общественной жизни, так и \emph{средством} идейно-эстетического и нравственного \emph{воспитания} нового поколения.

В эстетической мысли существовали и продолжают существовать \emph{взгляды,} \emph{отвергающие общественную роль искусства} и усматривающие его цель исключительно в нём самом.

Подобные взгляды выражают нередко \emph{разлад художника} с окружающей социальной средой, чрезмерную \emph{замкнутость художника} на своём внутреннем мире, \emph{попытки найти} особый художественный язык для выражения явлений этого мира. Нередко это сопровождается \emph{односторонним увлечением} разного рода формалистскими поисками.

Предметом художественного воспроизведения является \emph{преимущественно} жизнь общества, особенно область человеческих отношений.

Искусство изображает также \emph{природу}.

Но художественное изображение природы всегда носит на себе \emph{печать определённых человеческих чувств}, настроений, переживаний и т.д.

Художник \emph{не фотографирует} природу, а \emph{эстетически осваивает} её. Он \emph{находит} в природе \emph{прекрасное}, величественное, безобразное, сообразуясь не просто с собственными качествами предметов, а \emph{применяя свою} «\emph{мерку}».

\emph{Человек} называет величественным тот или иной предмет природы (например, высокие горы), выражая тем самым и объективную природу предмета, \emph{и впечатление}, производимое предметом.

«Задача искусства не в том, чтобы копировать природу, но --- чтобы её выражать... Ни художник, ни поэт, ни скульптор не должны отделять впечатление от причины, которые нераздельны одно в другом». Художник должен \emph{проникать в} «\emph{ум, смысл, облик вещей и существ}». (\emph{Бальзак об искусстве}. М.-Л., 1941, с. 164).

\subsubsection{Религия}

\emph{Религия --- наиболее древняя форма} общественного сознания.

\emph{Ранние формы} религии связаны с \emph{обожествлением природных сил}, растений, животных.

\emph{Остатки} ранних форм религии (\emph{анимизм}, \emph{тотемизм}, \emph{фетишизм}) сохраняются и в более поздних религиях.

Так, древнегреческий \emph{бог Зевс}, наделённый человеческими чертами, \emph{мог превращаться} в быка, орла, лебедя.

Египетский \emph{бог Анубис} имел человеческое тело и голову пса.

От обожествления и почитания явлений природы люди переходили \emph{к обожествлению социальных сил}, а в связи с этим меняются функции богов. В древнегреческой мифологии бог Марс был вначале богом растительности, потом стал богом войны, Гефест был на первых порах богом огня, а затем стал богом и кузнечного ремесла.

\emph{История религии} показывает также, что ни у одного народа религия \emph{не начиналась с монотеизма}, с учения о едином боге, как утверждают некоторые теологи: напротив, монотеизму \emph{предшествовал политеизм}, связанный с культами нескольких богов.

\emph{В войнах} между народами \emph{боги побежденных} уступали место \emph{богам победителей}, которые присваивали себе некоторые черты богов побеждённого народа.

\emph{Объединение} племён и народностей также приводило к \emph{объединению} или даже \emph{слиянию богов} и т.п.

С образованием крупных монархий \emph{вместо многих культов}, характерных для племенных союзов, ранних государств (где из ряда богов уже обычно выделялось верховное божество), создается \emph{культ единого}, всемогущего бога, на которого переносится атрибуты других богов.

\emph{Как бессилие} первобытных людей в борьбе с природой, \emph{так и бессилие} людей перед лицом слепых сил общественного развития порождают \emph{веру в сверхъестественные существа}.

\emph{Наряду с верой} в сверхъестественное и с фантастическими представлениями о мире и человеке во всех религиях значительную, порой громадную роль играет \emph{религиозный культ}.

Культ состоит из определённых \emph{обрядовых действий}, начало которых восходит к \emph{первобытной магии}.

\emph{Подобно тому} как первобытный человек посредством магических действий (заклинаний, жертвоприношений и т.п.) пытался \emph{побудить сверхъестественные силы} исполнить его желания и намерения, \emph{так} посредством определённых обрядов, обрядовых действий и запретов, предписываемых современными религиями, верующие стремятся получить помощь от бога.

В первобытном обществе \emph{посредниками} между людьми и таинственными силами \emph{выступают волхвы}, \emph{шаманы и т.д}.

С развитием общества выделяется особая профессиональная \emph{группа служителей культа} (жрецы, священники).

Церковь приобретает большую \emph{власть над умами людей}. Её идеологическое влияние \emph{усиливается связью с государством} и превращением той или иной религии в государственную.

Религиозный культ получает \emph{новое} развитие.

\emph{Церемониал богослужений}, использующий музыку, пение, играет важную роль в воспитании религиозных чувств, в закреплении религиозных верований.

\emph{Три} \emph{элемента религии} приобретают различное значение в зависимости от социальных условий:

\begin{enumerate}
\item религиозные \emph{представления},
\item религиозные \emph{чувства},
\item \emph{культ} и \emph{обрядность}.
\end{enumerate}

Религия представляет собой \emph{наиболее консервативную}, мало меняющуюся идеологическую форму, увековечивающую свои предписания \emph{именем бога}.

Вместе с тем \emph{история показывает}, что под влиянием крупных социальных переворотов и потрясений \emph{происходила смена религий}.

Древние религии были \emph{побеждены христианством} в период упадка рабовладельческого общества.

При этом \emph{христианство унаследовало} определённые черты старых религий, например признание \emph{Ветхого завета} иудаизма, \emph{мифы} восточных народов о страдающих, умирающих и воскрешающих богах и т.д., \emph{смешав} всё это с \emph{вульгаризированной греческой}, прежде всего стоической, \emph{философией}.

Народившись в условиях Римской империи как \emph{религия низших слоёв}, \emph{христианство} превратилось затем в \emph{официальную идеологию} всего общества, включая господствующие классы.

\emph{С развитием феодализма} христианство принимало вид соответствующей этому строю религии с соответствующей феодальной иерархией.

\emph{В XVI в.} на основе роста и укрепления буржуазных отношений из феодально-христианской католической церкви \emph{выделился протестантизм} с его идеей непосредственного общения человека с богом, с его обращением к отдельной личности.

Развитие капитализма привело к тому, что \emph{католицизм}, сохраняя свою догматику (основы вероучения), \emph{должен был приспособиться} к новым условиям, вырабатывать свою социальную доктрину.

\emph{Ныне} во всех разветвлениях христианства, \emph{включая православие} в России, как и в других религиях, происходит \emph{процесс приспособления} к новым условиям, связанным с развитием науки и техники, с глубокими качественными изменениями в жизни общества.

\emph{Позитивное содержание}, которое в определённые исторические периоды включалось в форму религиозного сознания, \emph{питает религию и сегодня}, хотя и \emph{не изменяет основного качества} религии как в целом сознание, не соответствующее современным требованиям, например, научного подхода, будучи \emph{неадекватным} выражением интересов человека.

\emph{Диалектико-материалистическая философия --- атеистическая} философия, принявшая атеистические подходы прежних материалистических учений. Она \emph{продолжает} последовательный критический анализ религии на основе новейших данных естествознания и общественной науки.

Конечно, \emph{полное преодоление} религиозного сознания, мировоззрения дело отдалённого будущего.

Любые традиции, включая религиозные, обладают значительной \emph{социальной живучестью}, вырастая из традиционного древнего общества.

В этой сфере \emph{категорически недопустимы} какие-либо оскорбительные выпады против религии, что было практически \emph{забыто властью} в известные десятилетия советской истории.

При рассмотрении форм общественного сознания мы, естественно, \emph{не останавливаемся} на философии как специфической форме общественного сознания, так как \emph{в главе I} уже шла речь о предмете философии, о её месте и роли в общественной жизни.

\subsection{Относительная самостоятельность общественного сознания. Связь и взаимное влияние форм общественного сознания}

Общественное сознание определяется общественным бытием, и вместе с тем обо обладает \emph{относительной самостоятельностью}.

Когда происходят коренные изменения в экономической структуре, то это \emph{не значит, что автоматически} следуют за ними и соответствующие изменения в общественном сознании.

Как в общественной психологии, так и в идеологии существует \emph{преемственность развития}, а также \emph{взаимодействие} между различными формами общественного сознания.

Прежде всего следует обратить внимание на большую \emph{роль традиций} и \emph{привычек} в сознании людей, в особенности в обыденном сознании.

Процесс изменений, перестройки общественного сознания происходит \emph{не с одинаковой скоростью} и легкостью у разных социальных групп и даже в одной и той же группе разных подгрупп, отдельных людей.

\emph{Остатки прошлого} в сознании людей продолжают существовать и тогда, когда экономические и социальные основы их существования уже почти исчезли.

Особые черты имеет \emph{относительная самостоятельность в развитии идеологии}.

То обстоятельство, что идеология представляет собой совокупность идей, \emph{приведённых в систему}, накладывает свой отпечаток на её историю.

Хотя развитие идеологии определяется в конечном счёте объективными причинами, у каждого вида идеологии, у каждой формы общественного сознания есть \emph{своя преемственность}.

Например, политическая идеология зависит от базиса, общественного бытия \emph{в большей степени}, чем философия.

Философия отражает общественное бытие \emph{более опосредованным образом} и потому имеет относительно большую самостоятельность развития.

Передовая, продвинутая идеология \emph{ставит назревшие вопросы} общественного развития и в этом смысле \emph{опережает} его объективный ход, но это не должно пониматься так, что сознание перестает определяться опытом.

Речь идёт о том, что \emph{сознание обнаруживает} определённые тенденции развития общественного бытия и более или менее верно отражает их.

Предвидение процессов и тенденций даёт \emph{возможность использовать преобразующую силу} общественных идей, свидетельствует об их активной роли в общественном развитии.

Относительная самостоятельность общественного сознания выражается также \emph{во взаимосвязи и взаимном влиянии форм общественного сознания}.

Это значит, что в истории той или иной идеологической формы, в конечном счёте, определяемой экономическим и социальным развитием, проблемы возникают и решаются \emph{в связи с развитием и других} идеологических форм.

В каждую историческую эпоху \emph{выдвигаются на первый план} определённые формы сознания, в которых \emph{в наибольшей мере} концентрируется сознание данного общества (в первую очередь ведущей социальной группы, класса).

Известно, что \emph{в античной Греции}, \emph{в V в. до н.э}., особенно большую роль в общественном сознании играли \emph{философия} и \emph{искусство} (театр, скульптура, архитектура).

\emph{В средневековой Европе} преобладающее влияние на философию, мораль, искусство, политические и правовые воззрения \emph{оказывала религия}.

Философия в средние века была на положении \emph{служанки богословия}, теологии. Даже проявления материалистической и атеистической мысли могли выступать тогда \emph{лишь в теологической одежде}.

\emph{В условиях капитализма}, современного западного общества религия уже в относительно меньшей степени влияет на умы и сердца людей. В жизни общества значительно \emph{возрастает роль светской идеологии} --- философских, политических и юридических взглядов и теорий, к которым вынуждена приспосабливаться религия.

В определённые периоды различные формы общественного сознания (религия, философия, искусство) \emph{были важным средством пропаганды} политических идей, политической борьбы.

Так, \emph{во Франции} во второй половине XVIII в\textsc{.} , \emph{в Германии} в конце XVIII -- начале XIX и \emph{в России} в 40-60-х годах XIX в. \emph{философия} и \emph{литература} стали \emph{главной ареной политической борьбы} передовых общественных сил за разрешение важнейших вопросов социального развития, в том числе развития самого человека, \emph{освобождения} его от оков средневековых отношений.

\emph{Связь философии и искусства} не ограничивается только их взаимным влиянием --- непосредственным или опосредованным, большие художественные произведения всегда \emph{содержат в себе глубокие философские размышления} о мире и человеке (греческая трагедия, Шекспир, Гёте, Пушкин, Толстой, Достоевский).

\emph{История знает немало примеров} и такого соединения философской мысли и художественного творчества, при котором \emph{философ} выступает \emph{как писатель} или \emph{поэт}.

Таковы философские повести \emph{Вольтера}, некоторые произведения \emph{Дидро} («Племянник Рамо»), \emph{Чернышевского} («Что делать?») и т.д.

Выше мы говорили об отношении \emph{философии и религии}.

Добавим к сказанному, что \emph{идеалистическая философия} нередко не только была \emph{близка религии}, но непосредственно \emph{перерастала} в религиозную философию и \emph{сливалась} с ней (\emph{С. Кьёркегор}, некоторые представители \emph{экзистенциализма} и \emph{персонализма}).

В определённых случаях существенные элементы той или иной философской системы \emph{участвовали в создании} и \emph{обосновании} нового религиозного вероучения.

Издавна привлекал к себе внимание вопрос о взаимодействии между \emph{религией и искусством}, между \emph{религией и моралью}.

И доныне существуют утверждения (выдвигаемые чаще всего самими теологами), что \emph{религия была первоисточником как искусства}, \emph{так и морали}.

Исследования истории первобытной культуры, однако, свидетельствуют, что происхождение и развитие искусства и морали (и самой религии) \emph{связано с определёнными социальными условиями} и особенностями, определёнными потребностями, о которых мы говорили выше.

Религия на протяжении многих веков играла \emph{роль официального морального наставника} человечества, но этот факт не означает, что мораль возникла «\emph{на базе религии}» или что она не может существовать \emph{без религии}.

Эстетическая ценность многих картин и скульптур на религиозные сюжеты \emph{не является производной} от этих сюжетов. Созданные выдающимися мастерами, они, равно как \emph{Парфенон}, \emph{готические соборы} \emph{и т.д}., \emph{были и остаются произведениями искусства}, доставляющими эстетическое наслаждение, независимое от религиозных чувств.

Все рассмотренные выше формы общественного сознания, а также различные области естествознания принимают \emph{участие в формировании мировоззрения людей}.

\emph{Коренные вопросы} мировоззрения на протяжении всей предыдущий истории общества \emph{по-своему решались} с определённых социальных позиций \emph{религией и философией}.

\emph{Немалое влияние} на мировоззрение людей (в некоторых условиях даже более значительное, чем религия и философия) оказывают \emph{политическая и правовая идеология}.

\emph{В отличие от} типов мировоззрения, господствовавших на прежних этапах развития общества, \emph{современное мировоззрение}, формирующееся на базе современной философии, в частности диалектико-материалистической, и современной науки, \emph{является более системным}, преодолевающим ряд суеверий и предрассуд\textsc{ков, в том} числе и связанных с религией.

\subsection{Общественное и индивидуальное сознание}

Хотя общество есть социальный организм и \emph{индивид} \textsc{в} обществе не может рассматриваться \emph{по аналогии с клеткой} в биологическом организме, оно, конечно, немыслимо без составляющих его людей, и, значит, \emph{общественное сознание немыслимо без сознания индивидов}.

\emph{Общие условия} социальной среды, в которой живут те или иные люди, \emph{определяют единство} их воззрений, стремлений, основанное на единстве интересов.

Однако \emph{даже при общности} взглядов, воззрений, мнений это общее выступает у отдельных лиц в \emph{индивидуальном своеобразии}.

По отношению к отдельному индивиду общность социального происхождения и положения представляет \emph{лишь известную возможность}, но отнюдь не абсолютную гарантию соответствующей \emph{социальной ориентации}.

Дело в том, что индивидуальное сознание имеет «\emph{биографию}», отличную от «\emph{биографии}» общественного сознания (как член группы по отношению к группе в целом).

Общественное сознание \emph{управляется} социальными законами. Его история с необходимостью \emph{следует за} историей общественного бытия, и то, какие будут происходить изменения --- эволюционные или революционные --- в общественном сознании, \emph{определяется в конечном счёте} соответствующими изменениями в общественном бытии.

\emph{Индивидуальное сознание} рождается и умирает \emph{вместе} с рождением и смертью данного человека. Оно \emph{выражает неповторимые черты} его жизненного пути, особенности воспитания, разнообразные политические и идеологические влияния.

Для индивидуального сознания объективная среда, под воздействием которой оно формируется, выступает как результат взаимодействия \emph{макросреды} --- общественного бытия и \emph{микросреды} --- условий жизни той или иной малой группы внутри большой социальной группы, а также ближайшего окружения (семьи, друзей, знакомых) и, наконец, условий личной жизни.

Индивидуальное сознание находится под воздействием \emph{и таких факторов}, как уровень развития данного индивида, личный характер и т.п.

Специфические \emph{пути индивидуального развития} личности при прочих равных условиях \emph{определяют отличие} её духовного мира от духовного мира других личностей, создают богатство человеческих индивидуальностей.

Общественное и индивидуальное сознание \emph{постоянно взаимодействуют} между собой, взаимно обогащают друг друга.

\emph{Каждый индивид} на протяжении своей жизни через отношения с другими людьми, путём обучения, воспитания \emph{испытывает влияние} общественного сознания, хотя и относится к этому влиянию не пассивно, \emph{а избирательно}, активно.

Исторически \emph{выработанные обществом нормы} сознания духовно питают индивида; \emph{влияют} на его убеждения, \emph{становятся источником} нравственных предписаний, эстетических представлений и чувств.

Общественное сознание \emph{не только вносится} в индивидуальные умы, оно есть коллективный ум как своеобразный, сложнейший \emph{синтез индивидуальных умов}.

Исходящая от индивида мысль, его убеждения \emph{могут стать и становятся} общественным достоянием, приобретают значение социальной силы, когда они \emph{выходят за пределы личного} существования , \emph{входят в общее} сознание, становятся убеждениями, нормами поведения других людей.

Отсюда необходимость проявления \emph{внимания со стороны общества} к развитию индивида, его творческих возможностей, его талантов и дарований.

\emph{На усвоение индивидом} достижений общественной мысли и на \emph{общественную} «\emph{отдачу}» со стороны индивида \emph{решающее влияние оказывает характер социального строя}.

В современном обществе большинство людей получают вполне реальную \emph{возможность проявлять творческую инициативу}, вносить свой вклад в развитие общества, его знаний, опыта.

Это не значит, что в современном развитом обществе нет борьбы, в том числе довольно острой, различных социальных и идеологических сил по тем или иным вопросам общественного развития.

Современные демократические институты, \emph{при всём их несовершенстве}, все-таки обеспечивают практически каждому человеку \emph{возможность} выразить своё мнение по любому вопросу общественной жизни.

\emph{Вопрос} --- \emph{будет ли он услышан}.

\section{Наука, её место и роль в жизни общества}

Рассмотрев структуру и формы общественного сознания, остановимся на таком \emph{специфическом социальном явлении}, \emph{как наука}, которая тесно связана со всей материальной и духовной жизнью общества и \emph{играет всё возрастающую роль} в его развитии.

\subsection{Наука как особое явление общественной жизни}

Наука зарождается лишь по достижении обществом \emph{определённой ступени зрелости}, и её состояние может служить одним из основных \emph{показателей} общественного \emph{прогресса}.

В наши дни роль науки в развитии общества столь велика, что \emph{XX век} часто называют «\emph{веком науки}».

Конечно, сущность нашей эпохи \emph{не может быть сведена} к этому определению. Но оно \emph{имеет некоторые основания}, если учесть, что современная научно-техническая революция невозможна без естествознания, а преобразование общества на новых демократических и справедливых началах \emph{невозможно без науки} об обществе.

\emph{Что же такое наука?}

На этот вопрос нельзя ответить однозначно, \emph{ибо наука --- многогранное явление} общественной жизни, соединяющее в себе духовные и материальные факторы.

Тем не менее обычное определение \emph{науки как} \emph{системы знаний} о мире может служить \emph{исходным пунктом} в рассмотрении данного вопроса.

\emph{Всякое знание}, в том числе научное, необходимо рассматривать как \emph{отображение} природы и общественного бытия, в том числе бытия личности.

\emph{Объектом} научного познания могут быть \emph{все без исключения процессы} природы, общественной и индивидуальной жизни.

В этом \emph{одно из отличий науки} от таких форм общественного сознания, как политическая, правовая идеология, мораль, отображающее только общественные отношения.

\emph{Наука и религия} --- это явления, \emph{противоположные} по своей сущности.

Если \emph{религия даёт фантастическое}, неадекватное отражение действительности, то \emph{наука}, взятая в целом, --- \emph{истинное отражение} природы и общества.

Возникающие в процессе развития науки \emph{неверные} (\emph{неподтверждающиеся}) гипотезы и теории не меняют сути дела, так как \emph{заблуждение} в науке есть либо результат давления неверной идеологии, либо \emph{побочный продукт} творческих поисков истины.

Наука есть \emph{высший продукт} \emph{человеческого разума} на сегодняшний день, воплощение его силы и могущества, \emph{если разум идет рука об руку} с нравственностью, на единственную дружбу с которой нередко \emph{претендует религия}.

Религия \emph{появляется значительно раньше}, чем наука, при крайне низком развитии практики, при подавляющем господстве над человеком природных и социальных сил, познать и подчинить которые люди \emph{тогда были не в состоянии}.

\emph{Зарождение науки есть прямое следствие} развития общества, его перехода на более высокие ступени развития.

Можно дать определение \emph{науки} как \emph{системы объективно истинного знания, обобщающего практику, из неё полученного и в ней проверяемого}.

Но чтобы пойти дальше, \emph{надо учесть} \emph{также отличие науки} от обыденного сознания и знания и от искусства.

\emph{Обыденные}, повседневные эмпирические \emph{знания}, возникающие непосредственно из практики, могут существовать без науки и вне науки.

Так, \emph{ещё в седой древности} \emph{было замечено}, что день регулярно сменяется ночью или что железо тяжелее дерева.

И в наши дни в мелком производстве человек \emph{зачастую обходится} унаследованными эмпирическими знаниями.

\emph{В быту} подобного рода знаниям принадлежит также немалая роль. Например, \emph{мать по ознобу может заключить}, что у ребенка началось заболевание.

\emph{Отличие науки} от подобного рода донаучных, эмпирических знаний состоит в том, что \emph{она даёт знание} не только отдельных сторон предметов и внешних связей между ними, но прежде всего и главным образом \emph{законов} природы и общества.

Действительно, если знание того факта, что железо тяжелее дерева, \emph{можно приобрести и без науки}, то понятие удельного веса и тем более объяснение причин того, что железо обладает большим удельным весом, чем дерево, принадлежит \emph{физике} и \emph{химии}.

Если \emph{представление о смене дня и ночи} внедряется в сознание на основе эмпирического наблюдения, то \emph{объяснение причин} чередования дня и ночи, периодического возрастания и уменьшения длительности дня на протяжении года \emph{невозможно без астрономии}.

Если болезненное состояние организма, находящее выражение в ознобе, \emph{обнаруживается без помощи науки}, то для того, чтобы можно было поставить правильный диагноз, выписать и изготовить нужные лекарства, \emph{необходима медицинская наука}, опирающаяся на биологию и химию.

Познавательную роль в отношении явлений общественной жизни выполняет, как мы знаем, \emph{не только наука, но и искусство}.

Настоящее \emph{искусство}, подобно науке, может давать и \emph{даёт знание} глубинных социальных процессов, психологии того или иного класса, группы.

Однако в отличие от искусства, которое выражает \emph{общее всегда через индивидуальное}, конкретное, наука представляет \emph{его в абстрактно-логической форме}, через понятия, теории.

Итак, \emph{специфика науки} заключается в том, что она \emph{является высшим обобщением практики, способным охватить все явления действительности, даёт истинное знание сущности происходящих явлений, процессов, законов природы и общества в абстрактно-логической форме}.

\emph{Структура науки} весьма сложна, но может быть охарактеризована на основании \emph{трех основных} взаимодействующих между собой \emph{компонентов}.

\emph{Во-первых}, в науку входят \emph{эмпирические знания}, причём не только заимствованные из обыденного сознания в целях анализа и обобщения, но также специально \emph{добываемые наукой опытным путём} --- через наблюдение и эксперимент.

Зарождение новых областей теории в естествознании обычно \emph{начинается с} \emph{открытия опытным путём} новых фактов, которые «\emph{не умещаются}» в рамки существующих теорий и некоторое время могут не находить удовлетворительного теоретического объяснения.

\emph{Так было} с открытием радиоактивности в конце прошлого века: это \emph{явление было понято} как превращение химических элементов \emph{только через двадцать лет}.

\emph{Во-вторых}, наука --- это область \emph{теоретического знания.}

Теория должна \emph{объяснять} факты, взятые в их совокупности, \emph{открыть} в эмпирическом материале \emph{действие законов}, \emph{свести} эти законы в единую систему.

В каждой области науки \emph{процесс накопления фактов} рано или поздно приводит к \emph{созданию теории} как системы знаний, и это есть верный признак того, что данная область знания \emph{превращается в науку в подлинном смысле} слова.

\emph{Механика} стала наукой благодаря \emph{И. Ньютону}, открывшему \emph{в конце XVIIв}. основные законы движения тел и связавшему эти законы в единую систему.

Во второй половине прошлого века учение о теплоте превратилось в термодинамику \emph{благодаря открытию закона} сохранения и превращения энергии и законов энтропии, а \emph{учение об электричестве} стало такой лишь тогда, когда \emph{Д. Максвелл} создал стройную теорию электромагнитных процессов.

В ту же эпоху совершилось \emph{превращение} политической экономии и социологии \emph{в науку}.

Наука как \emph{теоретическая система} имеет своим \emph{ядром законы} науки, отображающие объективно необходимые, существенные связи явлений.

\emph{К теоретической области} в науке относятся также \emph{гипотезы}, без которых наука не может развиваться и которые в ходе их проверки практикой либо отвергаются, либо очищаются от заблуждений и \emph{перерастают в теории}.

\emph{В-третьих}, неотъемлемым компонентом науки являются её \emph{мировоззренческие, философские основы} и выводы, в которых находит своё прямое продолжение и завершение теория.

\emph{Научная теория} может иметь \emph{различную степень всеобщности}, и, чем выше эта степень, \emph{тем ближе} данная теория \emph{к философии}.

Не удивительно, что \emph{наиболее важные синтетические теории} естествознания отличаются ярко выраженным \emph{философским характером}.

Так, \emph{понимание закона} сохранения и превращения энергии и закона энтропии, положивших начало термодинамике, \emph{невозможно без уяснения философских вопросов} о вечности и бесконечности материи и движения, об их количественной и качественной неуничтожимости.

\emph{Теория относительности} устанавливает связь пространства, времени и материи, \emph{квантовая теория} раскрывает взаимоотношение прерывности и непрерывности в микромире, \emph{а это} не только физические, но \emph{и философские проблемы}.

В \emph{общественные науки} идеологические моменты входят уже при истолковании фактов, т.е. на уровне теории, в то время как \emph{в естествознание}, как правило, на уровне философского истолкования теорий.

Поэтому абсолютное \emph{противопоставление науки и идеологии}, столь характерное для многих подходов, \emph{не соответствует действительности} познания.

Что касается «\emph{очищения}» \emph{науки от идеологии}, то речь может идти только о преодолении только чрезмерной, \emph{ненужной идеологизации}.

Будучи явлением духовной жизни общества, \emph{наука} в то же время \emph{воплощается} и в сфере его материальной жизни.

Наука представляет собой особую область человеческой деятельности, \emph{как теоретической, так и практической.}

Ещё на ранних ступенях развития науки \emph{учёные не только созерцали} природу, \emph{но и действовали}: \emph{изобретали} приборы, вели с их помощью наблюдения, \emph{ставили эксперименты} и добывали таким образом для науки новые факты.

\emph{В древности} был создан, например, такой астрономический прибор, как \emph{гнóмон} --- вертикальный столбик на горизонтальной площадке, с помощью которого греки умели не только \emph{определять высоту солнца над горизонтом}, но и \emph{географическую широту}.

\emph{В новое время} получают более быстрое развитие такие формы научной практики, как \emph{инструментальное наблюдение} и особенно \emph{эксперимент}, а в наши дни ни одна естественная наука невозможна без солидной \emph{экспериментальной базы}.

Во многих областях науки экспериментальная база требует для своего создания \emph{колоссальных затрат}, а в техническом отношении она сложнее любого производства.

Гигантские \emph{синхрофазатроны} (ускорители заряженных частиц), \emph{космические корабли} и ракеты, тончайшие \emph{приборы}, позволяющие измерять промежутки времени и интервалы пространства в микромире, и т.д. --- такова \emph{экспериментальная база современной науки}.

Создание современной научной техники и управление ею является очень \emph{важным видом} практической деятельности.

Разграничение между теорией и практикой во многих областях науки потребовало \emph{разделения труда между учёными}, например \emph{физики-экспериментаторы} ставят опыты, управляют приборами, дают первичное обобщение полученных данных, а \emph{физики-теоретики} целиком посвящают себя обобщению данных эксперимента, развитию теории.

Главная \emph{особенность практической деятельности в науке} в том, что она \emph{подчинена} делу добывания знаний, развития теории.

Конечно, материальный и духовный факторы \emph{переплетены между собой} не только в науке, но и в любой области человеческой деятельности, и поэтому \emph{диалектика взаимодействия} этих факторов должна учитываться при рассмотрении каждой из них.

\emph{Так, если} материальное производство \emph{не существует} без духовного момента, \emph{то любая форма} общественного сознания \emph{не существует без} материального момента.

\emph{Особенно это касается науки}, которая предполагает целый ряд специальных форм практической деятельности (эксперимент, наблюдение), называемых часто «\emph{научной практикой}».

Существование «научной практики», однако, \emph{не может служить аргументом против} того, чтобы \emph{считать науку} прежде всего и главным образом явлением духовной жизни общества, особой формой общественного сознания.

\subsection{Исторические закономерности развития науки. Науки о природе и обществе}

\emph{Важнейшая закономерность развития науки --- возрастание её роли в производстве и управлении обществом, её значения в общественной жизни}.

\emph{Уже на первом этапе} существования науки она \emph{возникает как ответ} на практические, прежде всего производственные, потребности.

Появление астрономии, математики и механики \emph{было вызвано потребностями} ирригации, мореплавания, строительства крупных общественных сооружений --- пирамид, храмов, и др.

Но \emph{в античном мире} Средиземноморья и в других \emph{докапиталистических обществах} наука, по существу, \emph{находилась в пеленках}.

Тогда \emph{рост науки} и её общественного значения шёл \emph{очень медленно}, а подчас \emph{прерывался} на столетия.

Так, \emph{в Западной Европе} раннее средневековье ознаменовалось \emph{утерей} многих научных достижений античного периода.

\emph{Причина} сравнительно медленного развития науки --- \emph{в застойности} производства и общественной жизни в целом.

Основные производственные процессы в земледелии, животноводстве, ремесле, строительстве \emph{велись с помощью примитивных ручных орудий} и на базе традиционных, унаследованных от предшествующих поколений эмпирических знаний.

\emph{В управлении обществом} наука также использовалась в весьма скромных масштабах, \emph{хотя арифметика} была нужна для торговли и сбора налогов, \emph{юридическая наука}, появившаяся вместе с кодификацией обычного права, достигла в Риме весьма высокого уровня, а \emph{политические и философские трактаты} античности были важным средством социальной ориентации и орудием в борьбе различных социальных сил.

\emph{Второй этап} в истории науки начинается \emph{с XV в}., когда \emph{в Европе} \emph{зарождается} современное естествознание и одновременно \emph{происходит бурный рост} общественно-политических учений и философии.

Основная \emph{причина} этого перелома --- \emph{зарождение} в недрах феодализма \emph{нового}, буржуазного общественного \emph{уклада}.

\emph{Возрастание роли} науки в жизни общества \emph{идёт параллельно} её собственному бурному прогрессу, причём во взаимодействии науки и производства последнему принадлежит безусловно \emph{решающая роль}.

Рост научных знаний, \emph{особенно в механике и математике в XVI -- XVIII вв.}\textsc{,} будучи непосредственно связан с нуждами производства, мореплавания и торговли, подготавливал \emph{промышленный переворот в Англии XVIII в}., а \emph{переход к машинному производству}, в свою очередь, \emph{дал науке} новую техническую базу и мощный толчок для дальнейшего развития.

Следовательно, \emph{рост естествознания} и в XIX в. может быть понят прежде всего как \emph{продукт развития} производительных сил буржуазного общества.

Вместе «с распространением капиталистического производства \emph{научный фактор} впервые сознательно и широко развивается, применяется и вызывается к жизни в таких масштабах, о которых предшествующие эпохи не имели никакого понятия». (\emph{К. Маркс} и \emph{Ф. Энгельс}. Соч., т. 47, с. 556).

Происходил также \emph{рост общественных наук}, которые развивались в связи с практикой преодоления феодальных порядков.

\emph{Прогресс общественно-политической мысли} нашёл своё выражение, например, в том, что если ранее имело место \emph{обращение к религии} за обоснованием интересов различных социальных сил, то теперь, в частности во Французской революции, стали обращаться \emph{к общественно-политическим и философским идеям} просветителей \emph{XVIII} \emph{в.}

\emph{В XIX в. возникает диалектико-материалистическая философия} и целая традиция в общественно-политической мысли, ориентированная на неё.

\emph{К сожалению}, этой традиции в известные десятилетия бывшего СССР практикой сталинского режима \emph{был нанесен тяжелейший урон}, преодолеть который будет очень не просто.

Но \emph{применение к этой традиции} всех критериев научной критики, \emph{её собственного диалектико-материалистического метода}, можно быть уверенным, должно \emph{поставить всё на свои места}.

Если эта теоретическая традиция имеет основания жить --- она \emph{будет жить}.

\emph{Отказываться} от неё только в силу изменения всё тех же идеологических установок было бы \emph{весьма и весьма сомнительным делом}.

\emph{Третий этап} в развитии науки и в изменении её общественной роли \emph{начинается в XX в}.

Для данного этапа характерно \emph{необычайное дальнейшее ускорение научного прогресса}, а также существенное \emph{видоизменение соотношения} науки и практики.

Развитие науки становится теперь \emph{исходным пунктом для революционизирования практики}, для создания новых отраслей производства.

\emph{Возрастание социальной роли науки} представляет собой важную закономерность развития общества.

Вместе с тем развитие науки имеет и \emph{свою внутреннюю логику}, свои внутренние закономерности.

\emph{Первоначально} наука появляется как \emph{ещё не расчленённое целое}, \emph{неотделимое}, например, в античной Греции \emph{от философии}.

Но уже тогда начался \emph{процесс дифференциации} научного, или скорее протонаучного, знания.

Единая \emph{наука разветвляется} на науки о природе, науки об обществе, математику и философию, а также \emph{науки о самом познании}, \emph{науке} (\emph{метапознание}, \emph{метанаки}).

\emph{Математика}, занимая особое место в системе наук, \emph{связана с естествознанием более тесно}, чем с общественными науками, и потому во многих случаях её можно рассматривать вместе с ним. Впрочем, сегодня это положение \emph{требует существенных оговорок}, уточнений.

В то же время \emph{философию} можно рассматривать вместе с общественными науками (с теми же оговорками).

\emph{Две основные группы} наук --- \emph{науки о природе} и \emph{науки об обществе}, имея общие черты, существенно различаются по месту в общественной жизни.

\emph{Растущее применение научного знания} породило семью так называемых \emph{прикладных наук}.

Это прежде всего \emph{науки технические} , изучающие действие законов физики и химии в технических устройствах.

Их бурный рост начался \emph{в конце XIX в}., и они являются непосредственным двигателем технического прогресса как в производстве, так и в военном деле.

Это также \emph{науки сельскохозяйственные} и \emph{медицинские}, изучающие действие и использование законов живой природы в сельском хозяйстве и при лечении людей.

Все эти науки непосредственно \emph{примыкают к наукам о природе}.

\emph{Основная функция} естественных и технических наук состоит в том, что они обслуживают общество \emph{знаниями о природе}, о созданных людьми \emph{технических устройствах}, помогают создавать \emph{новые средства техники}.

Основное содержание этих наук \emph{лишено особого социально-группового характера}, здесь \emph{можно говорить лишь} о философских и мировоззренческих особенностях их функционирования в обществе и восприятия их различными социальными группами.

\emph{Несколько иначе} дело обстоит в отношении \emph{общественных наук}. Их предмет прямо и непосредственно \emph{затрагивает интересы} различных социальных групп, и поэтому их основное содержание имеет выраженный \emph{социальный характер}.

Разные социальные силы \emph{могут расходиться} существенным образом в вопросах природы того или иного социального явления, например, деятельности государства и т.п.

\emph{Борьба мнений} в общественных науках имеет определённый политико-идеологический оттенок.

\emph{Интересы различных социальных сил} оказывают на общественные науки весьма заметное влияние, которое, впрочем, \emph{не нужно переоценивать}, тем более, гипертрофировать, что довольно \emph{отчётливо присутствовало} в отечественной науке периода бывшего СССР.

В последние десятилетия \emph{XX -- начала XXI вв}. процесс дифференциации наук идёт \emph{особенно быстро}.

\emph{Фундаментальные науки о природе} (физика, химия, биология, геология, астрономия) \emph{становятся комплексами} всё более \emph{многочисленных ветвей} знания, каждая из которых постепенно вырастает в особую науку.

Возникают \emph{пограничные, стыковые области знания}, которым отводится всё более важная роль (биохимия, геофизика, биофизика, геохимия, физическая химия, \emph{многочисленные информационные варианты дисциплин}).

\emph{Аналогичный процесс} дифференциации наблюдается и \emph{в науках об обществе}, если не в ещё большей мере.

Одновременно действует, однако, и другая, связанная с первой тенденция --- к \emph{интеграции научного знания}

\emph{В естествознании} это проявляется в \emph{растущей роли математики} и \emph{информатики} и их методов, а также \emph{теоретической физики} и в его комплекса физических наук.

\emph{В обществоведении} эта же тенденция находит своё выражение в распространении \emph{синтетических}, \emph{по сути, диалектических подходов}, а также в проникновении и в общественные науки математических методов.

\emph{Тенденция к диалектизации и синтезу}, к объединению наук, в том числе и обеих основных ветвей знания --- наук о природе и наук об обществе, в наше время проявляется со все большей силой.

Формируется и такой \emph{род научного познания, который изучает само познание}, научное, прежде всего, --- \emph{метапознание}.

Эта \emph{тенденция к единству научного знания}, безусловно, не означает возвращения к исходной точке, к нерасчленённой «науке» древности, \emph{первоначальному познавательному синкретизму}.

Эта тенденция знаменует собой становление нового, \emph{диалектического единства} всех наук --- единства в растущем многообразии.

К \emph{общим закономерностям} развития науки следует отнести также \emph{возрастание её относительной самостоятельности}.

Наука \emph{находит внутри себя} всё более мощные стимулы дальнейшего развития уже потому, что, чем больше сумма накопленного знания, тем более ощутимо его «\emph{давление}» при постановке \textsc{новых} познавательных задач.

\emph{Каждый учёный должен освоить всё созданное до него}, а это значит, что он усваивает и \emph{развивает дальше} созданное до него знание, которого становится всё больше и больше.

\emph{Возрастает зависимость учёного от накопленного знания.}

\emph{Возрастание общей суммы знаний} оказывает большое влияние и на структуру науки, поскольку требует \emph{всё большего разделения труда} между учёными.

Разделение труда учёных, в свою очередь, \emph{способствует возрастанию самостоятельности науки}, поскольку в условиях разветвлённого и дробного разделения труда подготовка учёных и замена одних лиц другими становится всё более сложным делом.

\emph{Самостоятельность науки} тем не менее была и \emph{остается относительной}.

\emph{Прогресс науки и в XX в}. \emph{обусловлен} в конечном счёте развитием практики, потребностями производства, управления обществом, военными нуждами, необходимостью охраны здоровья людей, природной среды, воспитанием новых поколений.

Но \emph{чем более обширно поле} научной деятельности и \emph{чем глубже разделение труда} внутри её, \emph{тем большее значение} приобретает \emph{внутренняя логика развития науки}, свойственные ей самой источники прогресса.

\emph{Важнейшим внутренним источником} развития науки является активное \emph{взаимодействие, нередко противостояние различных направлений} в ней, школ, отдельных учёных.

Борьба идей, мнений \emph{всегда двигала} науку вперед.

Без внутренней борьбы, без свободы критики \emph{наука может догматизироваться}, \emph{застыть} на месте, снизить темпы своего развития.

Чем выше уровень науки, тем больше \emph{значение борьбы мнений} при решении стоящих перед наукой задач, хотя сами эти задачи в конечном счёте выдвигаются нуждами практики, научной или вненаучной.

\emph{Возрастание роли науки} в жизни общества \emph{находит выражение} в росте численности научных работников, в увеличении ассигнований на науку, в развитии системы научных учреждений.

Если \emph{всего 100 лет назад} численность учёных во всем мире измерялась от силы \emph{десятками тысяч}, то сегодня она измеряется \emph{миллионами и десятками миллионов}.

Особенно быстро растут ряды работников науки \emph{в наиболее развитых странах}.

\emph{В России до революции 1917} года насчитывалось \emph{около 10 тысяч учёных}.

Перед Великой Отечественной войной их \emph{было уже 98,3 тысячи}, в 1950 г. --- \emph{162,5 тысячи}, в I960 г. --- \emph{354,2 тысячи}, в 1975 г. --- \emph{1 миллион 213 тысяч} человек.

Общественная \emph{роль науки} \emph{измеряется}, конечно, далеко не только численностью учёных, важное значение имеет быстрый \emph{рост расходов на науку}, которые позволяют \emph{оплачивать} не только труд \emph{учёных} и \emph{обслуживающего персонала} научных учреждений, но также \emph{сотен тысяч рабочих}, техников, инженеров, выполняющих заказы науки на приборы и оборудование, занятых печатанием и распространением научных трудов и т.д.

Вместе с тем, разумеется, нужно учитывать \emph{не только количественные показатели}.

Как в области производства, так и в области науки в настоящее время совершается \emph{переход от экстенсивного к интенсивному развитию}.

\emph{Остро встаёт вопрос} о повышении \emph{эффективности} вложений общества в науку, о росте результативности занятых в науке работников.

Развитие науки имеет столь \emph{большое значение} как для настоящего, так и для будущего, что эта область стала важнейшей \emph{ареной конкуренции} между различными странами.

\emph{Мощь отдельных стран} сегодня во многом определяется затратами на науку и \emph{уровнем} её \emph{эффективности}, темпами научно-технической революции и умением быстро использовать её результаты в производстве и других сферах.

Эффективность затрат на науку зависит от \emph{квалификации научных кадров} и от \emph{организации научных исследований}.

Структура научно-исследовательских учреждений и организация труда в них \emph{может разниться} от страны к стране.

Но в развитых странах в настоящее время существуют \emph{три основных типа} научных учреждений.

\emph{Во-первых}, это научные учреждения, \emph{занятые разработкой фундаментальных проблем} в основных областях естественных и общественных наук.

\emph{В России это} прежде всего академические (входящие в систему \emph{академии наук России}) научные институты, в которых занята сравнительно небольшая по численности, но \emph{наиболее квалифицированная} часть учёных.

Значение этих научных учреждений \emph{продолжает расти}, несмотря на сложности современного функционирования Российской академии наук.

Важно отметить, что \emph{разрыв во времени между открытием новых законов природы и их техническим применением сокращается}, а от постановки научного поиска и уровня теоретических исследований в наши дни в первую очередь зависит потенциал страны.

\emph{Во-вторых}, значительная часть отечественных научных сил до недавнего времени была сосредоточена в так называемых \emph{отраслевых, прикладных научно-исследовательских институтах}, прежде всего в технических, медицинских, сельскохозяйственных и т.д., а также в заводских институтах и лабораториях.

Эти научные учреждения и ячейки, \emph{там, где они сохранились}, или даже \emph{возродились}, непосредственно решают задачи по конкретным направлениям научно-технического прогресса, \emph{перебрасывая мосты} от «\emph{чистого}» \emph{естествознания к производству} и другим сферам.

\emph{В-третьих}, большое число учёных сосредоточено \emph{в университетах и других высших учебных заведениях}, где разработка как фундаментальных, так и прикладных задач науки \emph{тесно связана с подготовкой} инженеров, врачей, агрономов и самих учёных.

Формы организации науки \emph{продолжают совершенствоваться}, и это также является одной из закономерностей развития науки.

\subsection{Научно-техническая революция и развитие современного общества}

Научно-техническая революция (\emph{НТР}) развернулась в \emph{середине XX столетия}, революция в физике начала XX в. была её прологом.

В НТР проявляются характерные \emph{черты нового}, третьего этапа развития науки (см. Предыдущие разделы), резко возросшего её значения в общественной жизни.

\emph{Суть НТР} ---\emph{в соединении} научного прогресса с переворотом в технической основе общественной жизни, прежде всего производства.

\emph{НТР} --- это современный \emph{способ развития производительных сил} общества, включая человека, когда наука превращается в \emph{непосредственную производительную силу}.

Если наука, как было указано выше, есть система знаний, духовная сила, то \emph{почему её можно считать} материальной производительной силой?

При ответе на этот вопрос \emph{нужно учитывать диалектику}, т.е. противоречивое взаимодействие материальных и духовных моментов общественной жизни, которая предполагает не только их взаимодействие, но также их теснейшее \emph{взаимопроникновение}, переплетение.

Действительно, \emph{духовный фактор} так или иначе всегда входит в производительные силы, поскольку в процессе труда, любой материальной деятельности \emph{участвуют сознание и воля} человека.

\emph{В современных условиях}, когда производство, другие сферы деятельности все шире используют данные науки, в них \emph{включаются духовные силы} не только непосредственно занятых в этом деле людей, которые непосредственно изменяют вещество природы с помощью орудий труда, других средств, но и техников, инженеров, самих учёных.

На нынешнем этапе развития науки, т.е. в период НТР, характерно как раз то, что во взаимодействии производства и науки \emph{чрезвычайно возрос удельный вес движения от науки к производству}. В этом \emph{принципиальное отличие} положения, сложившегося в наши дни, от положения, существовавшего даже несколько десятилетий назад.

Нет сомнений, конечно, что и в XIX в. наука находила \emph{всё большее применение} в производстве, \emph{но} основные отрасли производства \emph{сохраняли при этом традиционный характер}.

\emph{Металлургия} оставалась производством железа и других металлов из руд, \emph{пищевая промышленность} --- обработкой сельскохозяйственного сырья, хотя открытия \emph{Пастера}, безусловно, сыграли революционную роль в \emph{консервации} продуктов и т.д.

\emph{Но в XX в. возникают} отрасли производства, непосредственно \emph{опирающиеся на новейшие научные открытия}, причём разрыв во времени между этими открытиями и появлением соответствующего производства становится \emph{всё короче}, а связь экспериментальной деятельности учёных с промышленной технологией \emph{всё более непосредственной}.

Такова \emph{радиотехническая промышленность}, при самом своём зарождении исходившая из экспериментов \emph{Герца} и \emph{Попова}, а также из опытов \emph{Ленгмюра и др}., по исследованию газового разряда в лампах, а теперь базирующаяся на достижениях физики полупроводников и т.п.

Такова \emph{химическая промышленность синтетических материалов}, прямо и непосредственно черпающая свои методы в лабораториях ученых.

\emph{Столь же непосредственно} наука определяет технологию в промышленности, изготовляющей антибиотики и гербициды, ракетную и электронно-вычислительную технику.

Во всех этих и многих других областях, отраслях, прибегающих к науке и приобретающих всё большее значение в жизни общества, \emph{прогресс науки выступает в качестве решающего фактора} прогресса техники и производительных сил в целом.

Темпы развития производства сейчас определяются главным образом \emph{темпами прогресса науки}, поскольку «\emph{научные отрасли}» \emph{промышленности} приобретают всё большее значение в сравнении с отраслями традиционными, а те, в свою очередь, \emph{не стоят на месте и модернизируются} на базе современной науки.

Современной науке приходится решать \emph{двуединую задачу} по всем важнейшим направлениям технического прогресса:

--- \emph{совершенствовать существующие методы} производства и иных видов деятельности и

--- \emph{открывать принципиально новые методы}.

Причём \emph{второе} из этих направлений \emph{является главным}.

Научно-технический прогресс \emph{захватывает не только производство}, но и область связи и транспорта, быта и спорта и т.д.

Особо важной областью применения науки является, к сожалению, \emph{сфера военной техники}.

Научно-техническая революция тесно связана с качественными \emph{изменениями в социальной сфере} общества.

Научно-техническая революция \emph{делает возможным} коренное \emph{изменение} природных условий жизни человека, в том числе \emph{климата} и \emph{водного режима} целых географических районов, \emph{отвоевание} у пустынь и болот всё новых больших территорий для расширения сферы обитания и производственного использования их в интересах общества. \emph{При соблюдении}, конечно, всех требований экологической науки.

Особое значение в этом отношении принадлежит \emph{освоению Мирового океана}, который пока в недостаточной мере используется как источник пищевых продуктов и сырья. \emph{Освоение океана} с необходимостью вызовет к жизни новые области техники и новые отрасли индустрии.

\emph{Уже созданные} развитием науки и техники \emph{предпосылки} позволяют ставить вопрос о необходимости \emph{изменения облика городов} и целых индустриальных районов, превратившихся в задымленные и вредные для проживания «\emph{сверхгорода}», а тем самым \emph{о рациональном размещении} промышленности и населения, \emph{о сохранении} природных ресурсов нашей планеты.

Преобразование природы и материальных условий существования человечества предполагает \emph{изменение и самого человека}.

Биология и медицина поставили в повестку дня \emph{вопросы о пересадке органов} человеческого тела, \emph{об искоренении} вирусных и раковых \emph{заболеваний}, \emph{управлении наследственностью} человека, о существенном \emph{продлении сроков} его \emph{жизни}.

Что же касается \emph{социального совершенствования} человека, то эта задача связана \emph{с искоренением} насилия, войн, национального и расового неравенства, справедливой организацией производительных и распределительных отношений.

Решение больших задач, уже поставленных современной наукой и практикой, требует \emph{планомерного и комплексного использования} достижений естественных и общественных наук в масштабах больших регионов \emph{и даже всех стран планеты}, как, например, проблема преодоления \emph{пандемии коронавируса COVID-19 в 2020 г.}, что предполагает создание современных структур как внутри отдельных стран, \emph{так и в международном масштабе} .

\emph{Возрастание роли науки} в жизни общества приведёт со временем к тому, что она \emph{займёт, если уже не заняла, ведущее место} во всей системе общественного сознания и будет оказывать все большее влияние на развитие общественного бытия.

\section{Личность и общество}

Взаимоотношение общества и личности --- один из важнейших, \emph{если не важнейший}, вопросов социального познания.

Этот вопрос имеет большое \emph{значение и для практики совершенствования общественной жизни}, призванной \emph{ликвидировать} любые проявления несправедливости, отчуждения и \emph{создать реальные условия} для свободной жизни личности, для её всестороннего развития.

В этой главе мы \emph{рассмотрим вопрос о природе, сущности личности}, о социальных условиях её жизни и деятельности, \emph{о предпосылках} и \emph{условиях} гармоничного сочетания личных интересов и интересов общественных.

\subsection{Что такое личность?}

Чтобы раскрыть содержание понятия «\emph{личность»}, нужно прежде всего определить \emph{сущность человека} как социального существа, ибо личности существуют только в человеческом обществе.

При этом человек не может быть понят как \emph{некий изолированный индивид}, связанный с другими индивидами «только \emph{природными} узами». (\emph{Л. Фейербах}).

\emph{Понять}, что представляет собой человек определённой эпохи, каковы его характерные особенности, \emph{объяснить}, почему у него складывается такой, а не иной социальный облик, можно, только \emph{исходя из системы общественных отношений} данного общества.

Общественные отношения \emph{обусловливают в немалой степени} биофизические и в ещё большей степени психологические и другие особенности человека.

Из принятого в диалектико-материалистической философии, историческом материализме определения \emph{сущности человека как совокупности всех общественных отношений} отнюдь не следует, что человек \emph{сводится} при этом к этой социальной сущности, что свойства человека не связаны с его физическим, в том числе психо-физическим, бытием.

Когда дело касается \emph{индивида}, то он \emph{предстаёт перед нами как} «совокупность физических и духовных способностей, которыми обладает организм, живая личность человека». (\emph{К. Маркс и Ф. Энгельс}. Соч., т. 23, с. 178).

Итак, человек --- это в известной мере \emph{существо биосоциальное}: \emph{социальное} потому, что оно обладает общественной сущностью; \emph{биологическое} потому, что носителем этой сущности является живой человеческий организм.

\emph{Понятие} «\emph{человек}» есть понятие \emph{конкретное} и \emph{родовое}, выражающее общие черты, свойственные человеческому роду.

\emph{Индивидом} в мире человеческом обычно называют \emph{отдельного человека}.

Ему свойственны \emph{наряду с общими} и индивидуальные черты.

Отдельный человек есть некоторый \emph{особенный индивид}, индивидуальное общественное существо.

\emph{Понятие} «\emph{личность}» неразрывно связано и с понятием «\emph{индивидуальность}».

\emph{Индивидуальность} находит своё выражение в \emph{природных задатках} и \emph{психических свойствах} человека --- \emph{в особенностях} памяти, воображения, темперамента, характера \emph{и во всём многообразии} человеческого облика и его жизнедеятельности.

\emph{Индивидуальную окраску} имеет и всё \emph{содержание сознания}: взгляды, суждения, мнения, которые даже при общности их у разных людей всегда содержат в себе нечто «\emph{своё}».

\emph{Индивидуализированы} потребности и запросы каждого отдельного человека, и на всё, что данный человек делает, он накладывает свою \emph{индивидуальную печать}.

\emph{Личность --- это человек, рассматриваемый} не только с точки зрения его общих свойств и черт, а и \emph{со стороны своеобразия его социальных, духовных, физических качеств}.

Эти \emph{качества могут быть} как положительными, так и отрицательными, а чаще всего \emph{в них сочетаются}, хотя и в разных соотношениях, и достоинства, и недостатки.

\emph{Всеобщим признаком} человека является \emph{социальная деятельность}, выделяющая его из остального мира.

\emph{Человек} --- это, прежде всего, \emph{активно действующий социальный субъект}, изменяющий условия своей жизнедеятельности.

\emph{Человек,} далее, \emph{существо} не только социально деятельное, но и \emph{социально мыслящее и чувствующее}, и все эти качества \emph{неразрывно связаны} между собой.

Так как общественные отношения меняются в ходе исторического развития, то и \emph{социальные типы людей видоизменяются}, следовательно, появляются и исчезают и их индивидуальные воплощения.

Когда в обществе существует \emph{социальный вид} буржуа, то он \emph{индивидуализируется в каждом} \emph{отдельном} буржуа.

Таким образом, при анализе проблемы человека \emph{необходимо исходить} из условий эпохи, социальной структуры данной общественно-экономической формации.

Существуют \emph{концепции человека}, которые носят очевидно неисторический характер.

Это, например, \emph{теории, которые сводят} всю жизнедеятельность человека к проявлению «\emph{естественной}» (физической, биологической) \emph{природы}. Они \emph{игнорируют} человеческую историю и законы общественного развития.

Представители подобных концепций --- \emph{биологических}, \emph{психологических} --- толкуют, например, о «\emph{собственническом инстинкте}», об «\emph{инстинкте собирания}», «\emph{инстинктах эгоизма}», «\emph{драчливости}» и даже «\emph{убивания}».

Такое понимание природы человека \emph{питает легенду} об интеллектуальной и нравственной \emph{неравноценности} рас, полов и т.п.

Есть и \emph{такие теории}, которые хотя и обращаются к социальному фактору, но определяющей основой сознания и поведения человека считают \emph{предрасположения}, коренящиеся в его психике \emph{в виде неудержимых влечений}.

Такова концепция «\emph{реформированного}», «\emph{реконструированного}» психоанализа (\emph{неофрейдизма}).

Один из видных представителей его, \emph{Э. Фромм}, так прямо и заявляет, что мысли, чувства, поступки личности следует рассматривать как \emph{дремлющие} в ней \emph{тенденции}, которые, так сказать, \emph{ждут удобного случая} для своего выражения (\emph{E. Fromm. Escape from Freedom}. NY -- Chicago -- San Francisco, 1964, p. 180).

Такие явления, как \emph{безропотное подчинение} другим, или, наоборот, \emph{стремление к власти}, пассивное, \emph{слепое согласие} с социальными нормами \underline{(}\emph{конформизм}) или стремление к разрушению, объясняются автором действием \emph{особых психических сил}.

По утверждению Фромма, эти силы имеют свои корни в так называемой «\emph{родовой травме}»: которая постигла человека, когда он, выделившись из животного мира, \emph{обрёл сознание} своей \emph{отделённости} от всей совокупности окружающих его явлений и вещей, среды, своей \emph{отчужденности} от природы и других людей.

Хотя \emph{Фромм рассматривает социальный фактор}, показывая, как экономическая и политическая действительность, например, западного мира \emph{обостряет состояние человеческой отчужденности}, он остаётся верен в конечном счете фрейдистскому положению о первичности психических сил.

В действительности же \emph{дело не в извечно заданных} психических предрасположениях, а в \textsc{том, что} \emph{люди суть продукты} исторически меняющихся социальных условий и \emph{обстоятельств} и изменяются с изменением последних.

Вместе с тем \emph{история и делается людьми}.

Поэтому \emph{для изменения людей} в массовом масштабе требуются \emph{не усилия психотерапии}, как это вытекает из концепции психоанализа, \emph{хотя она и нужна}, \emph{а} их \emph{исторические действия}, требуется качественное преобразование социальных условий их жизни.

\subsection{Интересы общества, социальной группы и личности}

Взаимоотношение \emph{общества и личности} --- это прежде всего взаимоотношение их интересов.

Так как со времени возникновения \emph{социальной дифференциации} общества индивид находится в составе той или иной социальной группы, то взаимоотношение интересов личности и интересов общества \emph{преломляется через} отношения интересов социальной группы, в которую входит личность.

Здесь нужен \emph{конкретно-исторический подход} при рассмотрении проблемы личности и общества.

Обращаясь к истории социальной мысли, мы встречаемся с \emph{концепциями}, согласно которым интересы личности и общества \emph{несовместимы}.

Широко известна провозглашённая ещё в древности \emph{формула}: «\emph{Человек человеку --- волк}». Эта формула была повторена в XVII в. английским философом \emph{Т. Гоббсом}. Встречается она \emph{и сегодня}.

Здесь проблемы человеческой жизни объявляются следствием не противоречий и конфликтов социальных групп, а следствием противоречий и конфликтов между личностью и обществом, которые \emph{якобы неустранимы} по самой их «природе».

Отсюда нередко делаются следующие \emph{два вывода}.

\emph{Одни абсолютизируют} личные притязания индивида, требуя \emph{полной} «\emph{свободы}» \emph{личности} от общества (\emph{индивидуалистические} и \emph{анархистские концепции}).

\emph{Другие}, наоборот, \emph{требуют отказа личности} от всякой самостоятельности.

Так, \emph{Гоббс} утверждал, что \emph{государство} есть «\emph{единственная личность}», не признающая никакой личности рядом с собой.

Поэтому люди должны \emph{отказаться от своих прав}, предоставив государству безграничную власть над собой.

Обе эти точки зрения сходятся на признании \emph{враждебности личности и общества}. Различие между ними состоит лишь в том, что первая ищет выходы из конфликта на путях \emph{провозглашения неограниченной свободы} личности, особенно «\emph{сильной личности}», вторая же \emph{требует подавления личности} и её поглощения обществом, государством и т.д.

\emph{В противоположность} подобным концепциям диалектико-материалистическая философия, \emph{исторический материализм} рассматривают конфликт между личностью и обществом как \emph{порождение определённых общественных отношений}, и прежде всего отношений, основанных \emph{на несправедливом} распределении и потреблении общественного богатства, \emph{на несправедливых} отношениях собственности, прежде всего частной, но также и государственной, как показывает советский опыт.

Только в указанных условиях \emph{личный интерес противостоит интересам общества}.

Но \emph{что такое интересы общества?}

Их следует понимать \emph{не просто как сумму} интересов составлявших его людей.

\emph{Интересы общества --- это его потребности}, связанные с функционированием его в качестве социального организма на основе присущих ему объективных законов развития.

\emph{Коренной основой} этого социального процесса является поступательное развитие общественных производительных сил. \emph{А способ}, каким осуществляется производство, исторически изменяется.

\emph{Если некогда} интересам общества соответствовало \emph{утверждение} частной собственности на средства производства во что бы то ни стало, \emph{отделение} умственного труда от физического, \emph{сосредоточение} средств обращения исключительно у «верхних» слоёв общества, располагающих временем для высших форм деятельности (\emph{Аристотель}), т\emph{о теперь} интересам общества соответствует \emph{ограничение} частной собственности на средства производства, \emph{преодоление} противоположности умственного и физического труда.

А это предполагает \emph{качественные изменения} в обществе.

\emph{В современном развитом обществе} очевидна \emph{тенденция сближения} коренных интересов социальных групп, хотя многие моменты противоположности ещё сохраняются.

Современное развитое общество представляет собой \emph{целостное единство}.

Составляющие его \emph{группы} --- социальные общности, разного рода объединения людей --- имеют связующим началом \emph{общность интересов}, которая коренится в материальных потребностях всего общества.

До идеала, конечно, ещё далеко, но \emph{тенденция очевидна}.

Но у каждой группы имеются свои \emph{специфические интересы}, что ставит \emph{проблему их сочетания} с интересами общественными.

Сочетание интересов отдельных групп и общества в целом имеет свою субъективную и объективную \emph{стороны}.

\emph{Объективная сторона} выражается \emph{в достигнутом} \emph{уровне возможностей}, условий, позволяющих удовлетворять потребности той или иной группы.

\emph{Субъективная сторона} проявляется \emph{в действиях людей}, которые могут способствовать сочетанию общественных и групповых интересов, но могут и приносить ему ущерб.

\emph{Известны случаи}, когда под благовидным предлогом заботы об общественных интересах \emph{игнорируются групповые интересы} и, \emph{наоборот}, когда эти последние заслоняют собой общественный интерес.

Процесс сочетания общих и специфических интересов предполагает \emph{правильную организацию} этого процесса, \emph{управление} им, \emph{недопущение} всякого рода субъективистских проявлений.

Но могут встречаться \emph{и такие ситуации}, когда между этими интересами в каком-либо отношении \emph{возникают противоречия}. И тогда неизбежным и единственно правильным становится \emph{принцип приоритета} более широких интересов над более узкими. Тогда мы имеем дело \emph{не только с координацией}, \emph{но и с субординацией} интересов.

Вообще говоря, \emph{даже полная гармония} интересов общества и групповых интересов \emph{не означает} их тождества, их полного совпадения.

Общество \emph{не может стать бесструктурным} относительно составляющих его людей, оно всегда будет связанно с каким-либо групповым членением, пусть даже в виде различных объединений с их специфическими потребностями.

\emph{Некоторые} современные мыслители исходят из идеи, о \emph{неизбежности} «\emph{вечного отчуждения}» от человека его сущности, его сил и потенций.

\emph{Согласно этой концепции}, ставшей весьма модной в западной социальной философии, истинная сущность человека отчуждена от живого индивида и это отчуждение якобы остается \emph{незыблемой судьбой человека}, его нельзя преодолеть ни при каких исторических обстоятельствах.

В действительности \emph{отчуждение не висит над людьми} в качестве вечного проклятия, оно порождено вполне определёнными социально-экономическими условиями, с преодолением которых оно \emph{будет ослаблено} и даже \emph{преодолено} практически полностью.

В частности, \emph{политическое отчуждение} состоит в том, что государство является силой, способной усиливать отчуждение людей от властных механизмов, именно это произошло, в частности, \emph{в бывшем СССР}, несмотря на все лозунги, которые в ту пору выдвигались о единстве партии, государства и народа. Элементы подобного происходят \emph{и в других странах}, что связано прежде всего с \emph{явлением бюрократизации} деятельности государственных и иных институтов.

\emph{Теория} «\emph{вечного отчуждения}» представляет собой \emph{один из вариантов старой идеи} о неизбежности конфликта между личностью и обществом.

Практика современного развитого общества \emph{в известной мере преодолевает} эту идею, демонстрирует возможность существенно большего сочетания общественного, группового и личного интересов.

\subsection{Коллективность и личность}

\emph{Личные интересы} индивида \emph{требуют} удовлетворения его потребностей и развития его задатков, сил и способностей.

\emph{К человеческим потребностям относятся} потребности в пище, одежде, жилище, топливе, предметах быта, в средствах освещения, передвижения и др.

Многие из этих потребностей связаны с существованием человека \emph{в качестве биологического} организма. Но и они имеют \emph{не биологическую, а социальную природу}, поскольку возникают и развиваются \emph{только в социальной среде}.

\emph{Общество создаёт} как сами эти потребности, так и средства для их удовлетоворения.

Но \emph{социальная окрашенность потребностей} выражается не только в этом.

Социальная структура общества нередко \emph{жёстко разрывает} потребности людей, принадлежащих к различным слоям, группам, особенно в старом традиционном обществе.

\emph{Невозможно}, разумеется, \emph{точно очертить} границы потребностей, строго установленные для всех времен. Ведь производство развивается, а создание новых предметов потребления вызывает и потребности в них.

\emph{Удовлетворение разумных потребностей} людей, каждого члена общества происходит, как показывает история, в постоянно \emph{возрастающих размерах}.

Смысл, заключенный, в понятии «\emph{разумные потребности}», достаточно определён: этот смысл не даёт возможности, с одной стороны, превращать потребности, их удовлетворение \emph{в самоцель} и, с другой стороны, устанавливать «\emph{некий минимум}» потребностей, «определённую ограниченную меру».

Современной \emph{развитой личности} чужды как принцип \emph{присвоения ради присвоения}, так и \emph{аскетический принцип} ограничения, отречения от земных благ.

\emph{Цель производства} современного развитого общества, хотя и несколько вуалируется конкуренцией и т.п., заключается \emph{во все более полном удовлетворении} постоянно растущих потребностей людей.

Это предполагает, в частности, \emph{неуклонное сокращение разницы} между высокими и сравнительно низкими доходами, а также все большее удовлетворение личных потребностей \emph{за счёт общественного фонда} потребления, который имеется во всех развитых странах, темпы роста которого будут всё увеличиваться.

Это способствует созданию условий для \emph{постепенного перехода} в будущем \emph{к распределению по потребностям}.

Когда говорят «\emph{каждому --- по потребностям}», то этим утверждают, что у разных людей \emph{потребности различны}.

Также когда говорят «\emph{от каждого по способностям}», то этим утверждают, что \emph{люди неодинаковы}, и по своим способностям.

Что же означает \emph{неравенство способностей}, в чём оно выражается и чем обусловлено?

Способности проявляют себя \emph{в деятельности}, и судить о них можно лишь \emph{по} её \emph{результатам}.

Существует представление о способностях как о \emph{готовом даре природы}.

Однако такое представление нельзя признать правильным, так как оно \emph{игнорирует} роль социальной среды и рассматривает человека \emph{лишь как биологическое}, природное, а не как общественное существо.

Но и противоположный взгляд, совсем \emph{отрицающий роль природы}, \emph{нельзя признавать правильным}, так как он делает непонятным и необъяснимым тот факт, что при одинаковых условиях жизни и воспитания индивидуальные различия в способностях бывают весьма значительными.

\emph{Природные задатки} суть лишь условия для развития способностей, а само их развитие происходит в течение жизни индивида \emph{под влиянием обучения}, воспитания и самовоспитания, в трудовой и общественной деятельности.

На характер способностей индивида \emph{накладывают свою печать} и общие условия социальной жизни, и непосредственное социальное окружение (например, семья, круг соседей, товарищей по работе, коллег, знакомых и т.п.), т.е. \emph{микросреда}, заключающая в себе и разного рода случайные, порой не поддающиеся учету моменты.

Каковы же эти \emph{условия социальной среды}, которые благоприятствуют формированию и развитию способностей?

Деятельность человека, в процессе которой формируются и развиваются его способности, \emph{немыслима вне общения} с другими людьми. Это общение выступает в \emph{двух формах}:

\begin{itemize}
\item \emph{в форме непосредственно коллективного}, совместного действия и
\item \emph{в форме} разделённой пространством и временем \emph{связи, осуществляемой посредством языка} и средств массовой коммуникации (преемственность культуры, традиций).
\end{itemize}

Форма общения посредством языка имеет, несомненно, \emph{огромное значение} для развития способностей людей, для их творческой деятельности.

Вместе с тем \emph{нельзя не видеть}, что с развитием материальной и духовной культуры растут и \emph{потребности в кооперации труда}, в развитии форм непосредственно коллективной деятельности.

\emph{Тенденция к возрастанию совместной деятельности} пробивает себе дорогу в настоящее гремя во всех развитых странах, в частности в различных областях науки, искусства и т.д.

Говоря о коллективных формах работы, \emph{не следует смешивать} понятия «\emph{индивидуальная форма работы}» и «\emph{индивидуальность}».

Рост коллективных форм работы \emph{действительно ограничивает} её индивидуальные формы, отнюдь \emph{не упраздняя} их вовсе.

Что касается \emph{индивидуальности}, то она не только не ограничивается, а, наоборот, обогащается.

\emph{В коллективе и слабый силён} --- не только тем, что коллектив функционирует по \emph{принципу взаимной помощи}, но и потому, что работа в коллективе способствует \emph{активизации индивидом} своих возможностей, приводит в движение его силы и способности.

Развитие способностей предполагает и \emph{силу воли}, направленную на достижение целей.

\emph{Энергия воли}, как и другие способности, тоже \emph{не дана в готовом виде} от природы. Она рождается и закаляется в деятельности.

И как ни значительны побудительные мотивы в виде \emph{личной материальной заинтересованности} или \emph{душевного удовлетворения}, доставляемого общественной похвалой, развивающийся в коллективной работе \emph{дух сплоченности}, взаимной преданности наращивает энергию воли, а через нее умножает силы и способности индивида.

Когда с конвейера сходит очередной автомобиль, то \emph{не найдется никого}, кто мог бы сказать: «\emph{Этот автомобиль сделал я сам}», ибо он есть продукт труда \emph{совокупного работника}.

Степень \emph{слаженности коллектива} во многом зависит от \emph{принципиальности}, предполагающей и требование ответственности, и товарищескую критику, свободную от каких бы то ни было предвзятостей, личных симпатий или антипатий.

Сплоченность коллектива обусловливается также \emph{поощрением} полезной инициативы, \emph{правильным распределением} ролей в коллективе.

\emph{Коллектив} есть по своей природе \emph{арена} для проявления и развития личных сил и способностей, для личной свободы.

\emph{Образ жизни} современного развитого общества предполагает \emph{взаимную заботу} человека об обществе и общества о личности.

\emph{Коллективность} \emph{не есть отрицание свободы}.

\emph{Наоборот}, только в обществе, в коллективе возможна личная свобода.

Но коллективность предъявляет \emph{к личности определённые требования}, возлагает на неё \emph{ответственность} за удовлетворение общих интересов.

Без личной ответственности \emph{невозможна} ни коллективная борьба за общегрупповые цели, ни совместная жизнь людей в современном обществе.

С понятием коллективности тесно связано \emph{понятие гуманизма}.

Поначалу \emph{гуманизм был протестом} против пережитков феодализма.

Он принимал нередко \emph{форму протестантизма} в религии, \emph{либерализма} в политической жизни, \emph{свободного экономического соперничества} в сфере экономики.

Это был, конечно, \emph{ограниченно понимаемый} гуманизм.

\emph{Современный гуманизм} представляет собой \emph{качественно новую ступень} в развитии гуманистических идей.

\emph{Современный гуманизм --- это гуманизм, который рассматривает свободное развитие каждого как условие свободного развития всех.}

Современный гуманизм рассматривает \emph{всестороннее совершенство-вание человека} , его способностей и сил \emph{как высшую цель}.

Впрочем, этому гуманизму приходится пробивать себе дорогу через \emph{множество препятствий} и ограниченностей.

Таким образом, \emph{взаимоотношение общества и личности}, имея в своей основе отношения их интересов, \emph{исторически меняется} в зависимости от изменений материальных потребностей общества и динамики социальной структуры.

\emph{Ликвидация неравенства}, групповых привилегий делает судьбу личности \emph{всё более зависимой} от её индивидуальных качеств, от её отношения к труду, проявлению созидательной активности.

\emph{Современное развитое общество} предполагает \emph{дальнейший рост сочетания} общественных, групповых и личных интересов, по мере продвижения вперёд оно \emph{открывает индивиду} всё больше возможностей для разностороннего развития и применения задатков, сил, способностей и дарований.

\subsection{Роль личности в истории}

\emph{Судьбы истории} в конечном счёте \emph{решает народ}.

Но роль народов \emph{нельзя рассматривать абстрактно}, в отрыве от социальных групп, руководителей, вождей, лидеров, которые их возглавляют.

\emph{Основное направление} развития общества определяется объективными законами, \emph{независящими от воли} и сознания людей, в том числе даже самых выдающихся.

Рассматривая общие причины исторического развития, \emph{можно временно абстрагироваться} от роли личности.

То же можно сказать и \emph{о действии особенных причин} и обстоятельств (например, о влиянии на исторический процесс той или иной страны уровня её развития и особенностей обстановки), которые также не зависят от отдельных личностей.

Но отвлекаться от роли личности \emph{при объяснении конкретных исторических событий}, которые зависят не только от общих и особенных, но и от единичных причин, \emph{недопустимо}.

Так, ход и \emph{исход войн} между государствами \emph{и другие} конкретные исторические события зависят не только от главных, определяющих причин, результаты таких событий \emph{во многом зависят} и от таких факторов, как мудрость, дальновидность или, наоборот, неспособность и близорукость руководителей, вождей, стоящих во главе тех или иных событий.

\emph{Без учёта} личностных факторов, всякого рода исторических случайностей живая, конкретная история приобрела бы фантастический, фаталистический и \emph{мистический характер}.

\emph{История творится людьми, и только людьми.}

\emph{Лишь учёт объективных условий}\textsc{,} определяющих действия людей, даёт возможность \emph{научно объяснить} роль тех или иных групп, движений, а также исторических деятелей в общественной жизни.

Дело заключается в том, \emph{при каких условиях} той или иной личности обеспечен успех в достижении поставленных целей и при каких условиях даже выдающиеся люди неизбежно терпят поражение.

Исторические условия определяют в конечном счёте \emph{рамки} деятельности личности. \emph{Выскочить} за эти рамки никакой деятель, даже самый выдающийся, \emph{не может}.

Так, \emph{если ещё не созрели} необходимые предпосылки, условия для возникновения новой общественно-экономической формации в недрах старого общества, ни один исторический деятель \emph{не в состоянии вызвать} её к жизни. Подобные претензии просто \emph{смешны}.

\emph{Никто}, \emph{никакая личность} не может творить историю по своему произволу, \emph{не может повернуть} социальное развитие вспять.

\emph{Великие деятели}, как и великие общественные идеи, творцами и выразителями которых они являются, \emph{появляются, как правило, в переломные эпохи} мировой \textsc{истории} или \textsc{истории} того или иного народа.

\emph{Именно} исторические условия и исторические потребности \emph{вызывают} появление тех или иных выдающихся деятелей.

Конечно, \emph{нельзя отрицать} того, что выдающаяся личность накладывает свою определённую \emph{печать} на события, во главе которых она стояла.

Возможно, что, \emph{если бы Наполеона} заменил другой генерал, \emph{Франция не имела бы} таких военных успехов, ход конкретных событий, приведших сначала к возвышению, а затем к падению Наполеона, был бы иным.

Но \emph{общее направление} экономического, социального, политического развития Франции в XIX в. \emph{не изменилось бы}.

В \emph{средние века} господствовал \emph{застой} \emph{не потому}, что не было великих людей, политических деятелей. \emph{И тогда рождались} по природе своей очень одарённые, выдающиеся личности, \emph{но время и обстоятельства} не благоприятствовали проявлению их талантов.

В средневековом обществе \emph{господствовал климат}, когда свободная мысль или не проявлялась, или уединялась в монашескую келью.

\emph{Великий Коперник} лишь на смертном одре мог бросить вызов старым, догматическим взглядам о месте Земли в Солнечной системе.

\emph{Вместе с тем} и в период средневековья \emph{шли} медленные, подспудные \emph{процессы}, которые привели в конце концов к эпохе \emph{Возрождения} и к появлению блестящей плеяды великих деятелей в области философии, науки, литературы, искусства.

\emph{Масштабы исторический деятелей} зависят в конечном счёте от степени величия задач, которые ставит перед ними их эпоха.

\section{Исторический прогресс}

\emph{Проблема исторического прогресса} занимает важное место в науке об обществе. Её освещение \emph{связано с оценкой} настоящего и будущего человечества, его ближайших и дальних перспектив, развития.

Идёт ли человечество \emph{к высшим формам} общественной жизни, к более современным и гуманным социальным отношениям, к более высокой культуре и нравственному сознанию, \emph{или}, напротив, оно движется \emph{по нисходящей линии}, навстречу термоядерной или иной какой катастрофе, к гибельному перенаселению земли, к биологическому вырождению человека и т.п.?

Речь идёт \emph{о самом существовании} исторического прогресса.

\emph{В чём сущность прогресса}, каковы его движущие силы, каков критерий, который позволяет различать прогрессивное и регрессивное в общественном развитии, --- таковы некоторые важные стороны рассматриваемой в этой главе проблемы.

\subsection{Сущность исторического прогресса}

\emph{Идея прогресса}, сформулированная такими представителями передовой общественной \emph{мысли XVIII в}., как \emph{А. Тюрго, Ж. Кондорсе, И. Гердер и др}., стала преобладающей \emph{в XIX в}.

Это было \emph{время быстрого} \emph{восходящего развития} капиталистического способа производства, других сфер общества, и идея общественного прогресса казалась \emph{чем-то} \emph{само собой разумеющимся} для многих представителей исторической науки того времени, социологам и философам.

Но тогдашние подходы к обществу \emph{не могли достаточно полно} и последовательно вскрыть действительную сущность, законы и движущие силы исторического прогресса.

\emph{С позиций идеалистического подхода} они искали причины развития человечества по преимуществу или \emph{исключительно в духовном начале}: в бесконечной способности совершенствования человеческого интеллекта (\emph{Тюрго}, \emph{Кондорсе}) или в спонтанном саморазвитии абсолютного духа (\emph{Гегель}).

Соответственно этому \emph{критерий прогресса} они также видели \emph{в явлениях духовного порядка}, в уровне развития той или иной формы общественного сознания: науки, морали, правовых идей, религии и т.д.

Но тогда \emph{оставалось невыясненным}, чем обусловлены \emph{сами эти} формы общественного сознания, \emph{каковы причины} их изменения и развития.

Другим \emph{существенным пробелом} многих тогдашних концепций социального прогресса являлось \emph{недиалектическое рассмотрение} общественной жизни.

Социальный прогресс \emph{понимался большей частью} как \emph{плавное эволюционное развитие}, без революционных скачков, без попятных движений, как непрерывное восхождение по прямой линии.

Так, социологи \emph{О. Конт} и \emph{Г. Спенсер} оставляли в тени противоречивую природу современного им социального процесса, связанного с \emph{т.н. ранним капитализмом}, домонополистическим, а \emph{активные выступления} различных социальных групп, того же пролетариата, расценивали как «\emph{болезненное явление}», даже как \emph{препятствие} на пути развития цивилизации.

Можно подумать, что \emph{цивилизация} может вообще развиваться \emph{без деятельности} конкретных людей, из групп.

Кроме того, \emph{тогдашняя социальная наука} ограничивала горизонт общественного прогресса \emph{рамками именно того} раннекапиталистического общества. Делался вывод о том, что за пределы этого общества \emph{вообще нельзя выйти}, что такой выход будет регрессом.

Диалектико-материалистическая философия, \emph{исторический материализм} делают попытку \emph{преодолеть} отмеченные недостатки, как впрочем и ряд других традиций, но уже гораздо позднее.

Исторический материализм \emph{отказывается от абстрактной постановки вопроса} об историческом прогрессе, как и рассуждений об обществе вообще.

Явления \emph{прогрессивные} в одну историческую эпоху (например, тот же ранний капитализм) \emph{превращаются в реакционные}, регрессивные в другую эпоху.

\emph{Требование конкретного анализа} обязывает \emph{найти специфические особенности} прогресса в различных общественно-экономических формациях.

\emph{История человечества}, несмотря на её противоречивость, временные застои, попятные шаги, \emph{есть в конечном счёте движение по восходящей} линии, движение от старого к новому, от простого к сложному.

\emph{По мере своего развития} человечество создаёт \emph{более} мощные производительные силы, \emph{более} эффективную экономику, \emph{более} совершенные формы политического управления, расширяющие в той или иной степени границы возможностей свободы человека.

Диалектико-материалистический подход к обществу \emph{позволяет обнаружить движущие силы} исторического прогресса \emph{в недрах самого} общества.

\emph{Силы, которые определяют общественное развитие, суть одновременно движущие силы исторического прогресса}.

\emph{Корни исторического прогресса} нужно искать в первую очередь в сфере материального производства, \emph{в области экономики}, т.е. в определяющей сфере общественной жизни.

\emph{Борьба} передовых социальных сил, групп \emph{против отживших} производственных и иных отношений, реакционных социально-политических порядков была и \emph{остаётся решающей силой} восходящего исторического развития.

Исторический прогресс \emph{особенно отчётливо} обнаруживается \emph{в смене общественно-экономических формаций}.

\emph{Все формации}, которые имели место в прошлом, одна за другой с необходимостью \emph{подготовляли своё диалектическое отрицание}, снятие.

\emph{На определённой стадии} развития исторически ограниченный тип производственных отношений, господствующий в той или иной общественно-экономической формации, \emph{начинал тормозить} развитие производительных сия и тем самым \emph{обрекал себя} на замену.

Приходившая ему на смену \emph{новая формация} обеспечивала \emph{более быстрое} развитие производительных сил, включая, теперь можно говорить открыто, производительные силы духовного производства.

Развитие производительных сил осуществлялось \emph{в нарастающих темпах}.

Так, простейшие \emph{каменные орудия} просуществовали без коренных изменений \emph{почти 500 тысяч лет}.

Переход же \emph{от паровых машин} к машинам, работающим на электрической энергии, потребовал \emph{менее 100 лет}.

\emph{Ещё меньше времени} потребовалось для перехода к начальным фазам промышленного \emph{использования атомной энергии}.

Прогресс и развитие производительных сил в конечном счёте \emph{обусловливает и прогресс в области} производственных отношений, в социальных институтах, в духовном развитии общества.

Поступательное развитие общества обнаруживается \emph{не только} в переходе от одной ОЭФ к другой, \emph{но и в пределах каждой формации}.

\emph{Значите ли это}, что на всем протяжении своего существования та или иная \emph{ОЭФ} развивается \emph{по восходящей линии}?

Такое утверждение \emph{без оговорок неприменимо} к прежним формациям, которые на определённых этапах своего развития \emph{утрачивали былую прогрессивность} и начинали тормозить в значительной мере ход истории, необходимых преобразований.

\emph{Так обстояло} с рабовладельческим строем, феодализмом, ранним капитализмом.

\emph{Что касается капитализма}, то, конечно, в нём так или иначе \emph{постоянно происходит развитие} производительных сил, научно-технических знаний, но в тех или иных условиях в нём \emph{проявляются ещё весьма и весьма регрессивные}, даже реакционные явления, порождавшие разрушительные силы, угрожавшие самому существованию народов.

Достаточно вспомнить \emph{две мировые войны} в ХХ в.

На всех этапах истории \emph{защитники} устаревающих режимов противостояли историческому прогрессу \emph{всеми доступными им средствами}.

Но \emph{никому не удалось} приостановить восходящее развитие человечества.

Усилия сопротивляющихся приводили лишь к более или менее длительной \emph{задержке прихода нового} социального порядка.

\emph{По восходящей линии развиваются} \emph{не только} социальные отношения, \emph{но и} культура, нравственное сознание народов.

Развитие человечества в целом есть и \emph{процесс отрицания} отжившего, и \emph{процесс сохранения} ценностей (материальных, духовных, в том числе и моральных), \emph{позволяющих} овладевать силами природы, увеличивать власть людей над стихийными силами общественного развития.

\emph{Имеет место}, следовательно, \emph{диалектическое отрицание} ---отрицание старой системы с удержанием и приумножением всех положительных её приобретений.

Таким образом, исторический прогресс, как и \emph{исторический процесс} в целом, выступает не как \emph{простое чередование формаций}, но как \emph{поступательное движение}, ибо каждая новая формация \emph{превосходит предыдущие} по производительности труда, социальной и политической организации и духовной культуре, по тем условиям, которые предоставляет данное общество для развития человека.

Исторический материализм \emph{нередко критиковали и критикуют} за якобы утверждение о том, что уровень развития производительных сил \emph{будто бы автоматически} определяет уровень прогресса во всех других областях общественной жизни, в том числе и в области духовной культуры.

Такое \emph{допущение противоречит} общественной практике и принципам диалектико-материалистической философии.

Известно, например, что \emph{Россия XIX в}., в развитии производительных сил \emph{значительно отставала} от ряда европейских государств, \emph{что не помешало} ей дать великих мыслителей, писателей, композиторов, живописцев.

В то же время сравнительно высокий уровень производительных сил \emph{западных стран в конце XIX --- первой половине XX в}. \emph{не помешал} заметным \emph{явлениям упадка} нравственного сознания ряда социальных групп и такому \emph{варварству}, как фашизм, расизм, геноцид, мировые войны.

\emph{Развитие духовной культуры} (и вообще \emph{надстроечных явлений}) \emph{нельзя выводить непосредственно} из уровня производительных сил, \emph{не учитывая} производственных и иных отношений, назревших социальных противоречий и целого ряда других факторов.

\emph{Нельзя забывать}, что надстроечные явления обладают \emph{относительной самостоятельностью}, по отношению к экономике, и временами эта самостоятельность может быть очень \emph{резко выражена}.

Так, например, определённые периоды расцвета искусства \emph{не находятся в соответствии} с общим развитием общества, его материальной основы.

\emph{Отмечая прогрессивность} последующей формации в целом по сравнению с предыдущей, исторический материализм оговаривает \emph{необходимость избегания упрощения} этого положения.

\emph{Не по всем} формам культуры новая формация обязательно, тем более в начальный период, \emph{превосходит} старую.

Так, в ряде областей духовной культуры, (например, в философии) \emph{феодальное общество уступало рабовладельческому}, однако нет сомнений, что \emph{в целом} переход от рабовладельческого общества к феодальному означал не попятное движение, а \emph{бесспорный прогресс} в общественном развитии.

Рассматривая \emph{прогресс как объективный закон} исторического развития, диалектико-материалистическая философия предупреждает об \emph{опасности примитивного понимания} социального прогресса как непрерывного, прямолинейного восхождения от низшего к высшему.

\emph{Победа нового строя}, социального качества \emph{не приходит} по точному расписанию.

Стечение обстоятельств порой приводило к \emph{временным откатам} исторически прогрессивных сил, задерживало решение объективно назревших исторических задач.

Так, история утверждения и развития капиталистической формации вплоть до наших дней \emph{показала, насколько извилистым}, противоречивым было становление этой формации в различных странах, \emph{и прежде всего в России}, с её опытом советского времени.

\emph{Опыт СССР} сегодня \emph{требует особого анализа} с точки зрения исторического материализма и диалектико-материалистической философии в целом, как \emph{пример очень противоречивого} развития формации с точки зрения \emph{созревания в ней} особого качества, позволяющего оценивать этот процесс \emph{как выход за рамки} капиталистической формации в некоторую будущую формацию, как её не называй.

Диалектико-материалистическая концепция общества и общественного прогресса выступает не только \emph{против фаталистического}, но и \emph{против субъективно-идеалистического}, волюнтаристского его понимания.

Исторический материализм против \emph{отрицания объективных законов} \emph{истории}, что позволяет с лёгкостью отвергать или, наоборот, признавать прогресс на основе произвольных оценок его сущности.

\emph{В противоположность} такому взгляду исторический материализм обосновывает последовательно \emph{объективный критерий исторического прогресса}.

\subsection{Критерий исторического прогресса}

Что же является \emph{критерием прогресса}, по каким существенным признакам можно \emph{отличить прогрессивные} и \emph{регрессивные} явления в истории?

Отметим прежде всего \emph{многогранность} общественного прогресса. Он выступает как \emph{комплекс} различных социальных процессов.

\emph{Каждая специфическая сфера} общественной жизни (производительные силы, экономика, политика, право, наука, нравственность, художественное творчество и т.д.) \emph{имеет свои} \emph{особые критерии развития}, и их невозможно смешать, не утратив конкретный и специфический подход к оценке того или иного явления.

Возникает вопрос: возможно ли в этих условиях говорить об \emph{общем критерии} социального прогресса?

Многие философы, социологи, и в особенности сторонники \emph{так называемой теории факторов}, отрицающие единую основу общественного прогресса и признающие механическое взаимодействие равноправных социальных факторов, \emph{дают отрицательный} ответ на данный вопрос,

\emph{Диалектико-материалистическая философия}, исторический материализм занимают существенно иную, во многом \emph{противоположную позицию} в решении этой проблемы.

То, что мы называем общественно-экономической формацией, является \emph{целостным, живым социальным организмом}, определённой социальной \emph{системой} со своей структурой, законами развития и функционирования.

Если мы имеем дело не с суммой разрозненных частей, а с их диалектическим единством, целостностью, то очевидно, что для сравнения этих целостностей, выявления их прогрессивности и/или регрессивности \emph{нужно обладать неким общим критерием}.

Поскольку экономические отношения составляют \emph{фундамент всякой ОЭФ} и в конечном счёте обусловливают все стороны общественной жизни, то, следовательно, общий \emph{критерий нужно искать} прежде всего \emph{в сфере экономических отношений}, в сфере производства.

Можно утверждать, что таким \emph{синтетическим критерием} прогресса развития общества являются в первую очередь \emph{производительные силы}.

Такая \emph{оценка обусловлена} тем, что в единстве производительных сил и производственных отношений первые составляют \emph{содержание способа производства}, являются наиболее динамичным его элементом и выражают момент непрерывности в развитии производства.

При этом уровень и характер развития производительных сил может быть \emph{отчётливо определён} и точно измерен.

Надо сказать, что судя по всему сегодня \emph{оправданно распространять понятия} «производительные силы» и «производственные отношения» со всеми оговорками, конечно, \emph{на любые сферы} общественной деятельности как специфические виды общественного производства, в ряду которых материальное производство является лишь одним, хотя и базовым для существования всех иных, из видов производственной деятельности человека.

Так, думается, вполне оправданно \emph{видеть в познании особое производство} --- духовное, --- связанное с \emph{производством знаний}.

То же самое можно сказать и о сферах \emph{искусства}, \emph{религии и т.д}.

\emph{Когда мы сравниваем} первобытнообщинный строй, рабовладельческое общество, феодализм и капиталистический строй, включая его современную стадию, то \emph{степень исторической прогрессивности этих ОЭФ} характеризуется прежде всего тем, \emph{как они обеспечивали} развитие производительных сил.

Возможное превосходство \emph{будущей по сравнению с капитализмом формации} заключается в том, как сегодня это представляется, что она \emph{откроет возможности} \emph{более планомерного}, \emph{быстрого} и \emph{гармоничного развития} производительных сил.

\emph{Высокое развитие} производительных сил, включая человека, и производительности труда \emph{позволит создавать изобилие} материальных и духовных благ, \emph{сделает возможным} их распределение среди людей преимущественно по потребностям, хотя и \emph{не обязательно в безденежном} варианте распределения.

Однако \emph{при оценке} \emph{прогрессивности и/или регрессивности} того или иного общественного строя, его модификации далеко \emph{не достаточно ссылаться} на уровень, характер и темпы развития производительных сил материального производства, взятых сами по себе, \emph{без учёта социальных последствий} их развития, без учёта интересов людей, различных социальных групп.

При этом ни в коем случае \emph{нельзя допускать отрыва} производительных сил от производственных и иных социальных отношений, \emph{как это происходит}, например, у американского социолога и экономиста У. \emph{Ростоу}, предпочитающего при классификации стадий общественного развития оперировать \emph{оторванными от производственных отношений} производительными силами, и прежде всего \emph{в лице техники}.

Если оперировать \emph{лишь уровнем развития} производительных сил \emph{в отрыве} от производственных отношений, от социальной структуры, от реального положения самих производителей, то уже \emph{современный развитой западный капитализм} может предстать как \emph{чуть ли не идеальный вариант социального прогресса}, и даже \emph{как сам идеал} такого прогресса.

В действительности высокое развитие производительных сил пока еще нередко \emph{сочетается с устаревшим} в ряде отношений общественно-политическим строем, требующим существенных качественных изменений, хотя бы усовершенствований.

Кроме того, \emph{нельзя брать факты в статике}, нужно \emph{рассматривать и динамику} их развития.

\emph{Ни одна ОЭФ} не в состоянии сразу, на первой стадии развития \emph{обнаружить} все свои \emph{преимущества} по сравнению с прежней. \emph{Требуется время} для того, чтобы новый общественный строй реализовал все свои возможности.

\emph{Это касается и самого капитализма}, о котором многие представители даже исторического материализма ещё недавно думали, что он уже себя окончательно исчерпал и \emph{должен быть заменён} новой --- коммунистической --- формацией.

Судя по всему \emph{сама такая трактовка} капитализма сегодня выглядит, мягко говоря, \emph{несколько утопической}.

\emph{Современный капитализм}, в том числе научившийся некоторым важным вещам \emph{на примере строительства} так называемого социалистического общества в СССР, других странах, \emph{внёс ряд существенных коррективов} в свое функционирование, в сторону развития, качественных изменений.

\emph{Современное западное развитое общество}, а его по всем формальным признакам надо считать \emph{ещё капитализмом}, открывает неизмеримо \emph{большие возможности}, чем капитализм ранний и даже ещё периода начала XX в., \emph{для развития} трудовой сферы с точки зрения учёта в ней \emph{интересов конкретных людей} --- этой важнейшей производительной силы и высшей ценности среди всех ценностей мира.

\emph{Характер производственных отношений} современного развитого западного общества --- в значительной мере отношений сотрудничества, взаимной поддержки или, во всяком случае, взаимной дополнительности, \emph{создаёт немало объективных условий} для достижения со временем существенного прорыва в сторону действительно совершенного общества.

\emph{Выдвигаемый историческим материализмом критерий прогресса} , с учётом необходимых корректировок, \emph{вытекает из принципа объективности}, преодолевает односторонность попыток выставить в качестве основной меры прогресса факторов, которые сами определяются другими, более фундаментальными первичными явлениями.

\emph{Что касается упрёков}, что этот критерий носит чисто экономический характер, оторван от человека, от места и положения его в обществе, то они \emph{в целом бьют мимо цели.}

Такие упреки не создают основания для выдвижения в качестве таковых критериев \emph{только самих по себе} тех или иных \emph{ценностей абстрактного гуманизма}, морального совершенствования людей и т.п.

Именно потому, что \emph{человек} в диалектико-материалистической философии \emph{является высшей ценностью}, он заслуживает того, чтобы эта ценность была подтверждена и проведена через \emph{все необходимые уровни обоснования} с точки зрения её объективного происхождения, развития в прошлом и будущем.

Апеллируя к производительным силам, характеру и условиям их развития, исторический материализм старается \emph{проникнуть в глубинные основы человеческого существования}, ибо \emph{там таятся силы}, определяющие в конечном счёте реальное положение людей, уровень их жизни, культуры, степень развития свободы, возможности их интеллектуального и нравственного развития и совершенствования.

Высокий уровень производительных сил создаёт \emph{предпосылки} для того, чтобы развитие одних людей и социальных групп в целом \emph{не происходило за счет других}, чтобы развитие производства полностью \emph{совпадало с требованием всестороннего развития человека}.

Это \emph{историческая задача} должна быть разрешена в \emph{качественно ином обществе}, чем сегодняшнее, сохраняющее ещё очень и очень много болезненных \emph{противоречий}, в том числе \emph{антагонистического характера}, в самом социальном прогрессе.

\subsection{Обострённый характер общественного прогресса в традиционном обществе и прогрессивные черты исторического прогресса в современном обществе}

В обществах, имевших место ранее, социальный прогресс неизбежно происходил \emph{при наличии порабощения} большинства населения и подавления личности.

Само \emph{развитие традиционных обществ} осуществлялось в результате острой борьбы, \emph{противостояния} между враждебными социальными силами, группами, классами.

\emph{Использование рабов} было исторически неизбежным \emph{условием} существования античной Греции, этой вершины древнего мира, и Римского государства, их духовной культуры.

\emph{Непреодолимое} \emph{противоречие} между прогрессом и условиями жизни большинства населения продолжало существовать \emph{и при феодализме}, и в значительной мере \emph{на ранних ступенях капитализма}.

За \emph{последние 150 лет} произошли существенные изменения в жизни человеческого рода, включая две мировые войны, попытку построения отличного от раннего капитализма социалистического общества и т.д.

Так \emph{тенденция абсолютного обнищания} основного производителя раннекапиталистического общества \emph{сменилась тенденцией существенного повышения жизненного уровня} и качества жизни в современно развитом западном капиталистическом обществе.

Здесь свою роль сыграла и организованная деятельность трудящихся в условиях \emph{поддержки со стороны бывшего СССР}.

Вынужденные поначалу уступать давлению массы наемных работников владельцы средств производства обнаружили явление, что \emph{при лучшей жизни работник дает большую экономическую отдачу}.

\emph{Началась гуманизация} многих сфер современного производства, в том числе на основе капитально \emph{усовершенствованных технологий} типа фордовского конвейера. Хотя проблем немало \emph{ещё остается}.

\emph{Изменилось государственное отношение} к делам в сфере производства и экономики в целом. Имущие классы \emph{сами оказались заинтересованными} в его реформировании с позиций удовлетворения справедливых требований населения, его основных групп.

Однако \emph{и сегодня сохраняются} многочисленные противоречия в прогрессивном развитии общества.

\emph{Чрезмерная механизированность} современного производства таит в себе \emph{опасность резкого снижения} качества духовной стороны жизни современного человека, что может \emph{опрокинуть все достижения} технического характера, если не позаботиться о \emph{сохранении основного качества личности} как духовного существа.

Последнее обстоятельство \emph{замечено} и многими социальными мыслителями на Западе.

Так, американский публицист \emph{О. Тоффлер} в книге с многозначительным названием «\emph{Столкновение с будущим}» пишет, что при сохранении капиталистической организации общества в прежнем виде \emph{последствия быстрого технического прогресса грозят} стать уничтожающей силой для людей.

\emph{Тоффлер} не отрицает необходимости \emph{замены современного общества} иной социальной системой.

\emph{Э. Калер} в книге, озаглавленной «\emph{Смысл истории}», констатирует \emph{вырождение сверхрациональной цивилизации} в механизированное варварство.

\emph{Переход человечества к будущему} социально-политическому строю также \emph{не будет происходить без проблем} и обострений, возможно даже социальных катаклизмов.

Сегодня \emph{возникает возможность управлять} теми или иными сторонами прогресса в интересах не отдельных групп людей, а всех или явного большинства из них.

\emph{Развитие} производительных сил, науки и техники, \emph{рост} общественного богатства \emph{означает постепенное непрерывное улучшение} материального положения и повышение культурного уровня всех или большинства людей. Уже одно \emph{это обстоятельство стимулирует} материально и морально творческую деятельность современного человека.

\emph{Прогресс} не только в сфере производства, но и в сфере социальных отношений, в политике, в культурном строительстве, в духовной жизни \emph{превращается в дело рук всего населения} развитых стран.

Всем этим определяется \emph{ускоренное общественное развитие} в условиях современного продвинутого общества, призванного оказать соответствующую \emph{помощь народам}, \emph{которые ещё не достигли} достаточного по сегодняшним меркам уровня своего развития.

Похоже, что современное общество вступает в \emph{фазу непрерывного и достаточно всестороннего социального прогресса}, который будет совершаться не в ущерб одним нациям и народам и в интересах других.

Реальные \emph{возможности стабильного прогресса} могут обусловли-ваться известной \emph{плановостью}, базирующейся на научном подходе к обществу.

Но и в будущем скорость и устойчивость общественного прогресса \emph{не будет носить автоматического характера}. Возможны явления, которые не будут сразу поставлены под контроль общества.

Многое зависит от \emph{совершенствования научного управления} общественными процессами, \emph{исключения ошибок} волюнтаристского и иного толка, \emph{преодоления} бюрократизма, некомпетентности руководства теми или иными сферами общественной жизни.

\emph{И в будущем} развитие общества будет с необходимостью носить \emph{противоречивый характер}, будет сталкиваться с необходимостью преодоления серьезных трудностей.

Можно предположить, что \emph{темпы исторического прогресса} будут возрастать и дальше.

\emph{Огромная энергия людей}, которая сегодня расходуется во многом неэффективно или с низкой эффективностью, в том числе на межнациональные войны, на создание орудий уничтожения и т.п., \emph{будет всё более и более переключаться} на осуществление мирных созидательных целей в интересах всего человечества.

\emph{Преодоление анархии} производства, экономических и финансовых кризисов, быстрое \emph{внедрение достижений} науки и техники, \emph{обеспечение пропорционального развития} мировой экономики, её специализация и кооперирование \emph{откроют} широкие \emph{перспективы} для исторического развития.

Но это будет \emph{не завершение, а начало} качественно нового типа социального прогресса.

\chapter{Анализ некоторых основных направлений западной философии хх века}

\section{Западная философия xx века}

\emph{Новые условия} общественного развития, прогресс научного познания природы и общества оказывают существенное \emph{воздействие} \emph{на форму философских теорий}, на способы решения традиционных философских проблем.

Это касается, как отечественного варианта диалектико-материалистической философии, \emph{так и всех видов философствования} в зарубежной философии, прежде всего западной, как наиболее развитой.

В настоящей --- \emph{заключительной} --- главе делается попытка \emph{подвести хотя бы частично итог развитию западной философии}, охарактеризовать \emph{главные} её течения и школы, попытаться увидеть определённое единство в этом процессе.

\subsection{Состояние западного общества и особенности западной философии xx века}

Западная философия \emph{активно реагирует} на особенности современного развития человечества, на процессы происходящие во всём мире и в самих западных странах.

Нередко это принимает характер оценок данного развития \emph{как кризиса} «\emph{современного человека}», «\emph{современной науки}», как «\emph{духовный кризис эпохи}», «\emph{кризис технической цивилизации»}, особенно если учесть, что \emph{добрая половина XX в}. приходится на подготовку двух мировых войн и залечивание ран после них.

Дело доходило до оценки перспектив развития человечества как «\emph{заката мировой культуры»}, «\emph{гибели цивилизации}».

\emph{Западная философия XIX в}. характеризуется \emph{иллюзорным} во многом представлением о том, что капитализм, или «\emph{промышленное общество}», был способен обеспечить устойчивый и длительный общественный прогресс.

Ныне это \emph{также имеет место}, например, в таких концепциях, как концепции «\emph{индустриального}», «\emph{постиндустриального}», «\emph{технотронного}» и т.п. общества, которое должно по их мнению \emph{сменить} прежние социальные системы, в том числе старый капитализм и социализм в лице СССР и его союзников.

\emph{Другой важный признак} современной западной философии --- принципиальное порой \emph{изменение отношения к науке}, разуму.

Здесь нередко подчеркивается «\emph{бессилие разума}», науки, и прежде всего общественной науки.

Так, английский философ \emph{К.Поппер} пишет: «...мы должны отвергнуть возможность \emph{теоретической науки} ... которая служила бы основой для исторического предвидения». (К. Popper\emph{. The Poverty of Historicism}. L., 1957, p. X ).

Нередко имеет меcто \emph{отказ от материалистической гносеологии}, которая была и в целом остается стихийным убеждением естествоиспытателей.

Можно встретить высказывания о том, что, например, \emph{наука не дает нам объективной истины} и «не имеет, по существу, \emph{никакого} значения для приобретения и установления мировоззрения» (\emph{M. Scheler. Gesammlte Werke}. Bd. VI. Bern und München, 1963, s. 17).

\emph{Отсюда делаются} --- при общем убеждении, что \emph{наука и мировоззрение несоизмеримы}, что философское мировоззрение не может ни лежать в основе частных наук, ни быть их результатом, --- \emph{выводы}, ведущие к формированию \emph{позитивистской}, \emph{иррационалистической} и \emph{религиозной} установок в философии.

\emph{Если наука не даёт} нам объективной картины мира, \emph{то не может} на это претендовать \emph{и научная философия}.

Т.н. научная философия может быть лишь «\emph{теорией познания точных наук}», «\emph{философией науки}», в конечном счёте теряя свой самостоятельный предмет и \emph{превращаясь в} «\emph{анализ языка}».

Такова логика становления \emph{неопозитивизма.}

\emph{Современный иррационализм} отвергает идею научной философии потому, что \emph{наука основывается на абстракциях}, сознательно отвлекаясь от всего конкретного, «\emph{жизненного}», не затрагивая «\emph{глубин человеческого бытия}».

Для некоторых западных философов \emph{философия должна отвергнуть ориентацию на науку}, обратившись к «\emph{полноте переживаний жизни}». (\emph{М. Шелер}). Наука должна быть принципиально \emph{не научным}, но «\emph{иным мышлением}» (\emph{К. Ясперс}).

Если \emph{наука --- это совокупность символов}, говорящих неизвестно что неизвестно о чём, то за ней скрывается «\emph{тайна}». \emph{А} отсюда \emph{недалеко и до выводов} и обобщений современной \emph{религиозной философии}, например, \emph{неотомизма.}

Конечно, \emph{к трём названным} тенденциям западной философии XX в. \emph{не сводится} всё многообразие философских школ Запада XX в.

Школ, направлений, течений \emph{гораздо больше}, да и внутри названных трёх \emph{нет особого единства}.

К тому же \emph{к концу ХХ века} из этих трёх \emph{только неотомизм} сохраняет примерно те же позиции, что и ранее.

Уже давно \emph{нет ни чистых позитивистов}, \emph{ни чистых экзистенциалистов}. Что тоже \emph{симптоматично}.

Но тем не менее \emph{к названным трём} тенденциям \emph{тяготеют все} главные течения сегодняшней (\emph{на дворе апрель 2020 г}.) западной философии, включая так разрекламированный \emph{уже в начале XXI века} \emph{постмодернизм}, как течения играющие \emph{второстепенную роль}.

\subsection{Неопозитивизм --- «философия науки»}

\emph{Позитивизм} --- возникшая ещё \emph{в XIX в}. философия, которая ставит во главу угла конкретнонаучное знание, что \emph{само по себе неплохо}, утверждая, что, кроме него, никакого знания вообще быть не может.

\emph{В XIX в}. В лице \emph{О. Конта} (1798 -- 1857) позитивизм провозгласил, что \emph{научная} («\emph{позитивная»}) \emph{философия} может быть лишь некоторой \emph{констатацией} наиболее общих законов, открываемых положительными науками.

Поскольку положительные науки \emph{изучают} \emph{только явления}, \emph{а не непознаваемые} «\emph{сущности}», «\emph{вещи в себе}» и т.д., и наука, и философия могут иметь дело только с этими явлениями.

Таким образом, \emph{по видимости} проявляя величайшее уважение к науке, позитивизм XIX в. соединил его фактически \emph{с положениями старой} субъективно-идеалистической теории познания.

\emph{Как результат}: «Все основные идеи науки --- представительницы реальностей, не могущих быть понятными нам. Пусть будет сделан какой угодно громадный успех в группировании фактов и установлении всё более и более широких обобщений... основная \emph{истина окажется по-прежнему недостижимою}», --- писал английский позитивист \emph{Г. Спенсер} (Г. Спенсер. \emph{Основные начала}. СПб., 1897, с. 55).

Тем не менее позитиизм строил \emph{свою картину мира}, основываясь на результатах современного ему естествознания.

Такова была, например, «\emph{синтетическая философия}» \emph{Спенсера}, описывавшего природу и общество как \emph{процесс постепенной эволюции}, направляемой механическими законами.

\emph{Революция в естествознании}, раскрывшая несводимость природных процессов к механическим, \emph{опрокинула} такого рода обобщения науки, основанные на абсолютизации достигнутых ею относительных истин.

\emph{При отсутствии диалектического подхода} к относительной и абсолютной истине это привело \emph{к отказу} от претензий на создание универсальной картины мира.

В \emph{эмпириокритицизме} \emph{Р. Авенариуса} и \emph{Э. Маха}, заменившем \emph{первую форму позитивизма}, философия была практически \emph{полностью сведена} к теории познания.

\emph{Основным тезисом философии} согласно эмпириоркритикам стало утверждение, что \emph{познание --- это связывание} между собой \emph{ощущений и представлений} , не достигающее никакой иной «реальности», кроме самих ощущений.

Позиция эмпириокритиков объективно не означала \emph{мировоззренческой} «\emph{нейтральности}», как считали они сами.

Субъективизм и агностицизм --- это \emph{тоже мировоззрение}.

Более того, исходя из агностической теории познания, \emph{можно обосновывать и любое иное мировоззрение}, вплоть до религиозного.

Так, возникшее, фактически в рамках позитивизма, учение под названием \emph{прагматизма}, \emph{совмещало} указанные, по видимости взаимно исключающие установки --- \emph{позитивизм}, преклоняющийся перед наукой, и отвергающий «метафизику» , \emph{и религиозную} «\emph{метафизику}».

Происходя от древнегреческого слова \emph{pragma} (\emph{дело}, \emph{действие}), прагматизм имеет свои содержанием убеждение, что всякое \emph{знание --- это не более как} «\emph{прагматическая вера}», т.е. условно принятое положение, \emph{критерием} «\emph{истинности}» которого является не его соответствие действительности, а \emph{успех действия}, осуществляемого на его основе, хотя бы принятое положение и не отвечало реальному состоянию дела или даже противоречило ему.

Отсюда следствие: \emph{задача познания} не формулировка истинных (отвечающих реальности предмета) положений, но «\emph{закрепление веры}», позволяющее действовать уверенно и добиваться успеха.

На этой основе \emph{Ч.С. Пирс} (1839 -- 1914) развивал \emph{прагматистскую теорию познания} и объективно-идеалистическую, \emph{религиозную} «\emph{метафизику}».

Другой основоположник прагматизма, \emph{У. Джемс} (1842 -- 1910), вывел из «\emph{воли верить}» (\emph{the will to believe}), свойственной человеку, и «\emph{полезности}» \emph{религии} \emph{правомерность} «\emph{религиозного опыта}» и религиозной жизни как веры «в существование невидимого порядка вещей и в то, что наше высшее благо состоит в гармоничном приспособлении к нему нашего существа» (\emph{У. Джемс. Многообразие религиозного опыта}. М., 1910, с. 46).

\emph{Махизм} (по имени Э. Маха), \emph{эмпириокрицизм} \emph{в целом} были подвергнуты критике, в том числе и прежде всего \emph{В.И. Лениным} в книге «\emph{Материализм и эмпириокритицизм}».

Несмотря на то, что некоторые положения эмпириокрицизма в теории познания ещё долго имели \emph{хождение среди философов}, занимавшихся проблемами естествознания (известно, например, благожелательное отношение к Маху, прежде всего как физику, со стороны \emph{А. Эйнштейна}), как школа он \emph{быстро прекратил своё существование}, сменившись неопозитивизмом.

\emph{Неопозитивизм возник в начале XX в}\textsc{.} в связи с успехами новой формы логической науки, \emph{математической логики}, в применении к исследованию \emph{оснований математики}.

\emph{Г. Фреге}, \emph{Б. Рассел}, \emph{Л. Кутюра} \emph{и др}. пытались осуществить обоснование математики \emph{посредством логического анализа}, т.е. сведения её исходных понятий к логическим терминам с последующей формулировкой всех её положений \emph{на языке той же логики} и по её правилам.

\emph{Возникло представление}, что \emph{аналогичное применение} логического анализа к философии приведёт к тому, что «ближайшее будущее составит для \emph{чистой философии} такую же эпоху, какою для учения о началах математики являются последние десятилетия». (\emph{А.С. Богомолов. Английская буржуазная философия XX века}, т. 3, М., 1973, с. 169-170).

Но то, что было \emph{в высшей степени ценным} для формальных исследований в области математики, \emph{обернулось утратой} содержательных проблем для философии, имеющих в ней первостепенное значение.

\emph{Б. Рассел} (1872 -- 1970) выдвинул в своей книге «\emph{Наше познание внешнего мира}» (1914) мысль, что все философские проблемы, если их подвергнуть анализу и очищению, оказываются проблемами логическими.

\emph{Л. Витгенштейн} (1899 -- 1951) приходит к выводу, что \emph{философия} --- не доктрина, \emph{не совокупность теоретических положений}, а \emph{деятельность,} состоящая в \emph{логическом анализе языка} науки. А \emph{результат} этой деятельности --- «не некоторое количество «философских предложений», но \emph{прояснение предложений}». (Л. Витгенштейн. \emph{Логико-философский трактат}. М., 1958, с. 50).

Тем самым \emph{ликвидируется мировоззренческая функция} философии --- синтез в ней достижений научного знания, создание философской теории.

Такое ограничение философии анализом языка дополняется специфическим пониманием \emph{сферы применимости} самого этого анализа и его философского смысла.

Неопозитивизм базирует свой анализ науки \emph{на трёх} основных \emph{тезисах}.

\emph{Во-первых}, он \emph{жёстко разграничивает} аналитические (\emph{логико-математические}) и синтетические (\emph{фактические, эмпирические}) высказывания.

\emph{Аналитические} высказывания \emph{служат} элементами формальной структуры теории и \emph{не несут} познавательного содержания.

\emph{Синтетические} высказывания составляют \emph{эмпирический базис} теории.

\emph{Во-вторых}, неопозитизм \emph{основывается на редукционизме}, т.е. утверждении о \emph{сводимости} всех содержательных высказываний теории к непосредственному опыту или высказываниям о нём.

\emph{В-третьих}, неопозитивизм \emph{принимает субъективистскую теорию познания}, восходящую к \emph{Беркли} и \emph{Юму}: наше \emph{знание относится} не к объективному миру, но \emph{к} «\emph{содержанию сознания}», \emph{ощущениям} («наблюдениям», «опыту») \emph{и их фиксациям} в языковых формах.

Неудивительно, что \emph{центральное место} в неопозитивизме так называемого \emph{Венского кружка}, ставшего в 20-е годы ХХ в. ведущей неопозитивистской школой, заняло \emph{учение о} «\emph{верификации}» (проверке).

Венский кружок возник на базе \emph{кафедры философии индуктивных наук Венского университета}.

\emph{Главой} Венского кружка был \emph{М. Шлик} (1882 -- 1936), а \emph{наиболее активными} деятелями --- \emph{Р. Карнап} (189I -- 1970), \emph{О. Нейрат} \emph{и др}.

\emph{Сходные} взгляды развивались «\emph{Обществом эмпирической философии}» в Берлине (\emph{Г. Рейхенбах}, \emph{В. Дубислав}), \emph{Львовско-Варшавской школой} в Польше.

Кружок просуществовал \emph{до 1938 г}., когда, после захвата Австрии фашистской Германией, его участники \emph{вынуждены были эмигрировать} \emph{в США и Англию}.

Эта эмиграция способствовала однако \emph{распространению} и \emph{укреплению} неопозитивизма в англосаксонских странах.

«\emph{Принцип верификации}» играл в неопозитивистском анализе \emph{троякую} роль:

--- \emph{чувственной проверки} эмпирических высказываний,

---\emph{определителя эмпирического значения} терминов и высказываний и, наконец,

---«\emph{демаркационного принципа}», позволяющего отделить эмпирические предложения от неэмпирических, и прежде всего от «\emph{метафизических}» (\emph{=философских}), \emph{с целью устранения} последних из языка науки.

С этой точки зрения если предложение \emph{не может быть сведен}о к конечному числу актов опыта или высказываний о таких актах, («\emph{верифицировано}») и к тому же оно \emph{не есть тавтология}, т.е. логико-математическое высказывание, то оно или \emph{составлено с нарушением} правил синтаксиса и потому \emph{бессмысленно}, или «\emph{метафизично}».

Так, высказывание «\emph{2+2= 4}» или «\emph{Все холостяки не женаты}» суть \emph{тавтологии}. Легко показать, что \emph{первое} предложение означает «\emph{4=4}), а \emph{второе} --- при замене слова «\emph{холостяк}» его словарным значением --- «\emph{Все неженатые люди не женаты}».

Высказывание "\emph{Сейчас на улице --- 4}$^{\circ}$" или «\emph{Все холостяки эксцентричны}» относятся к эмпирическим, так как мы можем их проверить, \emph{посмотрев} на термометр или \emph{произведя эмпирическое исследование} привычек холостяков. Если холостяки \emph{отличаются} от привычек женатых людей (принятых за норму), \emph{то холостяки эксцентричны}. Если \emph{это не так}, то предложение \emph{всё равно эмпирично}, хотя и ошибочно.

В высказывании «\emph{Цезарь есть и}» нарушено правило синтаксиса. \emph{На месте предиката} стоит связка «\emph{и}», т.е. \emph{логический знак}.

Если высказывание \emph{претендует} на научность, \emph{но не сводится} ни к логическим, ни к эмпирическим, то оно «\emph{метафизично}».

\emph{Р. Карнап} так \emph{разъяснял} мысль о метафизичности: «Я буду называть \emph{метафизическими} все те предложения, \emph{которые претендуют} на то, чтобы представлять знание о чём-то, что \emph{находится выше}, или \emph{за пределами всякого опыта}, например, о реальной \emph{Сущности вещей}, о \emph{Вещах в себе}, \emph{Абсолюте} и т.п.».

Таковы, например, утверждения \emph{Фалеса}, что принципом (\emph{началом}) вещей является \emph{вода}, или \emph{Гераклита} --- что их начало \emph{огонь}; утверждение \emph{мониста}, что в мире \emph{одно начало}, так же как утверждение \emph{плюралиста}, что \emph{начал много}; утверждение \emph{материалиста} о том, что \emph{единство мира в его материальности}, и \emph{идеалиста} --- что \emph{мир} по своей природе \emph{духовен}.

Все эти утверждения \emph{эмпирически непроверяемые}, ибо --- «из предложения "\emph{Начало мира есть вода}" нельзя вывести никакого утверждения относительно восприятий, или чувств, или ощущений, или какого бы то ни было опыта, которого следует ожидать». (\emph{R. Carnap. Philosophy and Logical Syntax}. In: ``The Age of Analysis'', N.Y., 1956, p. 213).

Следовательно, неопозитивизм относит к «\emph{метафизике}» и тем самым \emph{исключает} из сферы научного знания все \emph{философские проблемы}. Поскольку они якобы \emph{научно неосмысленны} и их решения не могут быть --- в отличие от научных --- признаны истинными или ложными.

Философские проблемы согласно неопозитивистам \emph{просто лишены смысла}.

Такая точка зрения \emph{неверна}.

Мы знаем, что \emph{философия есть обобщение} научных \emph{знаний} и общественной \emph{практики} людей.

Фалес, \emph{по Аристотелю}, пришёл к выводу \emph{о воде} как начале (\emph{архэ}) всех вещей, «видя, что \emph{пища} всех существ \emph{влажная} и что само \emph{тепло возникает из влаги} и ею живёт (а то, из чего все возникает, --- это и есть начало всего)». (\emph{Аристотель}. \emph{Соч. в четырех томах}, т. Т. М., 1975, с. 71).

\emph{Идея Фалеса} --- это первоначальная, пусть \emph{примитивная}, но \emph{научная гипотеза}, а не просто «метафизическая» фантазия.

\emph{Равным образом}, материалистический монизм --- это учение, \emph{доказывающее} материальность мира «длинным и трудным развитием философии и естествознания». (\emph{Энгельс}).

Но в философии имеет место и \emph{отличие} от естественнонаучных обобщений.

Обобщение философское \emph{рекомендует} определённый мировоз-зренческий и методологический \emph{подход к познанию}, причём \emph{материализм} предписывает искать решение \emph{в сфере исследования природы}, не обращаясь к сверхъестественным, идеальным началам.

«Нейтральный» \emph{в кавычках} неопозитивизм выступает в этой связи «\emph{стыдливым идеализмом}», \emph{протаскивающим} под флагом «\emph{антиметафизической}» философии \emph{субъективистский принцип} в философии.

\emph{Современные исследования научного познания} и соотношение в нём эмпирического и теоретического уровней \emph{разрушают} неопозитивистский подход.

\emph{Принцип верификации} в роли «\emph{интеллектуального полицейского}», представляющего и предписывающего, о чём может и о чём не имеет права говорить наука, \emph{породил больше трудностей}, чем разрешил.

Такие \emph{очевидные научные высказывания}, как формулировки общих законов, общие предложения науки, высказывания о прошлом и будущем, очевидно \emph{непроверяемы непосредственным наблюдением}.

«\emph{Мышьяк ядовит}», «\emph{Тела расширяются при нагревании}», «\emph{Цезарь перешёл Рубикон}» --- ни одно из этих предложений не может быть непосредственно проверено. Значит, их \emph{следует изгнать из науки}?

\emph{Не помогло} делу и «\emph{ослабление}» принципа верификации, а именно: «\emph{Эмпирическое предложение --- то, которое в принципе проверяемо опытом}».

Ведь есть знания, которые мы «\emph{в принципе}» \emph{не можем} проверить сегодня, тогда как завтра сможем это сделать, развив дальше технику наблюдений.

Так, \emph{О. Конт} считал «\emph{метафизическим}» \emph{вопрос о химическом составе} небесных тел, полагая, будто бы его «\emph{принципиально}» нельзя установить. Через два года после его смерти \emph{изобрели спектральный анализ.}

Далее, оказывается, что «\emph{принципиально}» \emph{проверяемо}, к примеру, \emph{всякое суеверие}.

Такова «\emph{эсхатологическая верификация}», согласно которой \emph{логически} допустима \emph{верификация бессмертия души}, ибо если душе бессмертна, то она «в принципе» может \emph{вернуться с того света} и констатировать свой посмертный опыт.

В то же время \emph{опровергнуть бессмертие души нельзя}, ибо столь же «принципиально» нельзя зафиксировать отсутствие «того света».

\emph{Злоключения} «принципа верификации» \emph{завершились тем}, что сам он \emph{оказался} «\emph{метафизическим}», т.е. подлежащим \emph{устранению из науки} на том основании, что \emph{он не относится ни} \emph{к} логико-математическим, \emph{ни к} эмпирическим предложениям.

Все эти \emph{трудности проистекают} из того обстоятельства, что принцип чувственной проверки --- элементарный способ, применимый \emph{только в простейших случаях}, вроде проварки предложения «\emph{Сейчас на улице --4}$^{\circ}$».

\emph{Учение} диалектико-материалистической философии \emph{о практике как критерии истины} (подразумевающее, конечно, что в простейших случаях, например для определения температуры, нет необходимости апеллировать ко всей человеческой практике) \emph{гораздо точнее и полнее} характеризует \emph{проблему истины}, как и проблему соотношения конкретных наук и философии.

\emph{Крушение} «\emph{принципа верификации}» и превращение его в тривиальное положение о необходимости \emph{опытного содержания научного знания} привели к \emph{ещё большему замыканию} философии в \emph{сфере языка}.

«\emph{Лингвистическая философия}», являющаяся ныне наиболее распространённой (по крайней мере в Англии) формой позитивизма, перешла к «\emph{лингвистическому анализу}» \emph{повседневного языка} с целью «\emph{терапии}», т.е. \emph{устранения} из языка «\emph{философских заболеваний}».

С этой точки зрения \emph{философские высказывания} --- это просто неточное, \emph{произвольное истолкование} самых обыденных высказываний.

Во всех таких случаях, когда перед нами обыденные высказывания, мы должны провести «\emph{перефразировку}», которая должна показать нам, о чём идёт речь.

\emph{Лингвистические аналитики упускают} социальную и гносеоло-гическую обусловленность философии, а их «\emph{лингвистический анализ}» в конечном счёте \emph{сводится к тривиальным} с философской точки зрения, хотя \emph{иной раз довольно интересным} с точки зрения самой лингвистики, операциям \emph{уточнения языковых средств} выражения мысли и языковой коммуникации.

Но есть \emph{ещё один момент} в лингвистической философии.

Действительно, эта философия не может делать \emph{ничего, кроме описания языковых средств}, даже вносить улучшения.

Как писал \emph{Л. Витгенштейн}, «философия никоим образом \emph{не может вмешиваться} в действительное употребление языка; она может в конечном счёте \emph{только описывать} его... \emph{\underline{Она оставляет всё, как оно есть}}». (L. Wittgenstein. \emph{Philosophical Investigation}. Oxford, 1963, p. 49).

Последние слова Витгенштейна --- хотел он того или нет --- прекрасно выражают \emph{установки неопозитивзма}.

Фактически неопозитивизм \emph{отрицает возможность} научного мировоззрения, философско-научного, по крайней мере.

Неопозитивизм подвергался \emph{серьёзной критике}.

Подвергнуты сомнению \emph{неопозитивистские догмы} о противопо-ложности аналитических (логических) и синтетических (эмпирических) высказываний, обнаружилась \emph{несостоятельность} «\emph{редукционистской}» \emph{программы} сведения теоретического знания к эмпирическому, к его чувственной основе. (См. книгу \emph{В.С. Швырёва} «\emph{Неопозитивизм и проблемы эмпирического обоснования науки}», М., 1966).

Однако, сохранив основную гносеологическую посылку неопозитивизма, \emph{многие критики неопозитивизма} приходят лишь к некоторым его \emph{отдельным видоизменениям}. Впрочем, в ряде моментов это даже \emph{оправданно}.

Не следует легко отказываться от той или иной традиции в философии, \emph{пока не выявятся} все её возможности и пределы.

Например, известный американский логик и философ \emph{У. Куайн} отверг «\emph{две дегмы}» неопозитивистского эмпиризма --- различение аналитического и эмпирического и редукционизм.

Но в «\emph{эмпиризме без догм}» \emph{Куайна} теория, или так называемая «\emph{концептуальная схема}», оказалась на положении просто напросто мифа. (W. Quine. \emph{From a Logic Point of Wiew}. N.Y., 1963, p. 44).

Действительно, существует \emph{проблема развития познания}, перехода от мифологии к положительным наукам. Её нельзя трактовать иначе, как обращаясь к \emph{углублению} нашего \emph{знания}.

С другой стороны, появились концепции, усиленно, или намеренно \emph{обходящие} эту проблему.

В \emph{англо-американской философии науки} сложилось \emph{два} таких направления, происходящих, в целом, \emph{из неопозитивизма}, но \emph{отвергнувших} обе упомянутые «догмы эмпиризма».

Так. \emph{Т. Кун} в своей известной, можно сказать \emph{даже знаменитой} книге «\emph{Структура научных революций}» (\emph{1962}, \emph{рус. пер. 1975}) представил развитие науки как \emph{смену} «\emph{парадигм}» (своего рода \emph{познавательных образцов}) в период \emph{научной революции} и логическое развёртывание содержания их в период «\emph{нормальной науки}».

\emph{Выразительно показав смену} эволюционных этапов революционными, \emph{Т. Кун,} однако, не обнаружил связи «\emph{парадигм}» с углублением знания объективного мира.

\emph{Источник} парадигм по Куну --- \emph{социально-психологический}.

\emph{Парадигма} --- \emph{это система взглядов}, \emph{принятых «научным сообществом».}

Остаётся открытым \emph{вопрос об объективном содержании} «парадигм».

Со своей стороны, \emph{К. Поппер}, которого критикует Т. Кун, и другие так называемые \emph{постпозитивисты}, видят в развитии знания «\emph{естественный процесс}» \emph{саморазвития} некого «\emph{третьего мира}» --- мира объективного знания.

«Это \emph{мир возможных объектов мысли}: мир теорий в себе и их логических отношений, аргументов в себе и проблемных ситуаций в себе». (\emph{K. Popper. Objective Knowledge}. Oxford, 1974, p. 154).

\emph{Созданный людьми}, в отличие от мира идей \emph{Платона} или абсолютной идеи \emph{Гегеля}, этот «\emph{мир теорий в себе}» тем не менее «\emph{автономен}».

«Знание в объективном смысле есть \emph{знание без познающего субъекта}». (\emph{Там же}, р.109).

Подобный \emph{квазиобъективизм} есть лишь \emph{оборотная сторона неопозитивизма}, ограничивающего познание сферой языка.

Да, \emph{теория рождает} в ходе своего логического \emph{развёртывания} новые проблемы, аргументы, новую теорию.

Но \emph{в основе} этого развёртывания всегда лежит --- опосредованная структурой теории --- \emph{объективная реальность.}

Само развёртывание не может осуществляться \emph{без субъекта познания}, учёного, исследователя, \emph{без} «\emph{сообщества учёных}», научных институтов.

\emph{История неопозитивизма} --- это \emph{история} прежде всего его \emph{отступлений}, неудач, история смены форм анализа языка, заканчивающаяся фактически \emph{распадом} этого влиятельного философского направления.

\subsection{Экзистенциализм}

\emph{Ужасы} и бедствия первой мировой войны \emph{разрушили иллюзии} буржуазного либерализма прошлого века относительно рационального, гармонически развертывающегося в истории бесконечного прогресса.

\emph{На смену} этой иллюзии пришло ощущение многими людьми \emph{бессмысленности} человеческого существования, \emph{бесперспективности} исторического процесса и в то же время \emph{глубокого бессилия}.

\emph{Вдвойне усиливается} это чувство у многих во время второй мировой войны. Довлеет оно над многими \emph{и после} неё.

Описать, а вместе с тем и оправдать это ощущение «\emph{бессмысленности бытия}», это «\emph{самоотчуждение}» человека, состоящее в том, что «сам человек представляется становящимся всё более и более \emph{чуждым своей собственной сущности} в такой мере, что он ставит под сомнение эту свою сущность», лучше всего сумел в современной западной философии \emph{экзистенциализм}. (\emph{G. Marsel. Der Mensch als Problem}. Frankfurt a/M., 1964, s. 18).

Экзистенциализм \emph{возник} в \emph{20-е годы ХХ} столетия.

\emph{Главные} представители экзистенциализма --- \emph{М. Хайдеггер} (1889 -- 1976) и \emph{К. Ясперс} (1883 -- 1969) \emph{в Германии}, \emph{Г. Марсель} (1889 -- 1973), \emph{Ж.-П. Сартр} (1905 -- 1980) \emph{и А. Камю} (1913 -- 1960) \emph{во Франции}.

Экзистеницализм получил значительное распространение \emph{в Италии}, \emph{Испании}, \emph{Латинской Америке}, отчасти \emph{в США}.

Были близки к экзистенциализму \emph{русские} философы-эмигранты \emph{Н.Бердяев} и \emph{Л. Шестов.}

Экзистенциализм имеет по меньшей мере \emph{три главных} источника.

\emph{Во-первых}, это взгляды датского философа \emph{XIX в}. \emph{С. Кьёркегора} (1813--1055), \emph{выдвинувшего} само понятие \emph{экзистенции} (существования).

В отличие от «\emph{абстрактного мыслителя}» традиционной философии и науки, «\emph{экзистирующий мыслитель}» должен рассматривать действительность \emph{субъективно}, т.е. только так, как она преломляется через его \emph{индивидуальное} существование и \emph{эмоциональную} (для Кьёркегора в первую очередь \emph{религиозную}) \emph{жизнь}.

Философская \emph{установка Кьёркегора} существенно отлична от научной объективности.

\emph{Во-вторых}, экзистенциализм \emph{заимствует} у немецкого философа \emph{Э. Гуссерля} (1859 -- 1938) «\emph{феноменологический метод}», основанный на интуиции, имеющей своим объектом внутреннюю структуру (структуры) «\emph{чистого сознания}», которое оказывается в этой связи исходным элементом как познания, так и его объектов.

Феноменологический метод \emph{был применён} к таким элементам сознания, как \emph{забота}, \emph{страх}, \emph{решимость}, \emph{заброшенность}, \emph{вина}, \emph{ответственность} и т.д.

\emph{В-третьих}, экзистенциализм \emph{наследует} ряд моментов «\emph{философии жизни}», в особенности опираясь на идеи \emph{Ф. Ницше}.

\emph{Исходным принципом} экзистенциализма является утверждение, что \emph{существование} (экзистенция) \emph{предшествует сущности}, или, что то же самое, следует \emph{начинать с субъективности}.

Высказанный в работе \emph{Сартра} «\emph{Экзистенциализм --- это гуманизм}» (1946), тезис этот свидетельствует о \emph{субъектоцентристском} характере данного \emph{направления}.

Если в \emph{классическом субъективном идеализме} прошлого \emph{отрицание} объективности внешнего мира, или по крайней мере его познаваемости, \emph{сочеталось с верой в доступность} познанию самого субъекта, то теперь положение меняется. Экзистенциализм утверждает \emph{недоступность} рациональному познанию также и \emph{субъекта}.

«\emph{Экзистенция}, --- писал \emph{Ясперс}, --- это то, что \emph{никогда не становится} объектом. \emph{Первоначало}, исходя из которого я мыслю и действую, о котором я высказываюсь, развивая ход мысли, который ничего не познаёт». (\emph{K. Jaspers. Philosophie}. Bd. I. Berlin, 1956, s. 15).

Но почему экзистенция \emph{не может} быть объектом? Почему она \emph{рационально непознаваема}?

Потому, что она:

--- \emph{во-первых}, \emph{индивидуальная}, тогда как рациональное знание требует общего.

--- \emph{Во-вторых}, \emph{экзистенция --- это я сам}. Но я не могу смотреть на себя «\emph{извне}», как я смотрю на свои объекты, занимаясь наукой.

Отсюда вырастает противопоставление «\emph{проблемы}» науки «\emph{тайне}» экзистенциальной философии.

Это различение принадлежит \emph{Г. Марселю}, который писал: «...между проблемой и тайной есть то существенное различие, что \emph{проблема} --- нечто такое, \emph{на что я натыкаюсь}, что противостоит мне, то, что я, следовательно, \emph{могу} охватить и \emph{подчинить} себе. \emph{В тайну} же \emph{я вовлечён} сам, и, следовательно, она мыслима лишь как сфера, \emph{в которой различение находящегося во мне и передо мной потеряло свое значение и внутреннюю ценность}. В то время как аутентичная проблема \emph{подчинена определённой технике}, с помощью которой она может быть чётко ограничена, \emph{тайна} по определению \emph{превосходит всякую технику}, которую только можно представить». (\emph{G. Marsel. Sein und Haben}. Paderborn, 1954, s.126).

Если я могу, скажем, решить математическое уравнение, используя \emph{известную мне технику}, то \emph{всякий} знающий эту технику \emph{со мной согласится} и примет данное мной решение.

Но \emph{тайны} \emph{бытия} --- смерть и бессмертие, существование или несуществование бога, любовь, \emph{истина}, причём не та истина, которую заучивают, а та, \emph{за которую отдают жизнь}, --- не могут быть отделены от \emph{меня самого}, от моего \emph{личного решения}, они не могут стать общим достоянием.

Экзистенциализмом поднимаются \emph{действительно важные} проблемы человеческой жизни и познания.

Решая эти проблемы, однако, \emph{экзистенциализм} с самого начала \emph{исключает объективный}, в первую очередь \emph{социальный смысл} этих вопросов.

\emph{Научному} решению проблем экзистенциализм \emph{противопоставляет} решение \emph{религиозно-идеалистическое}.

Так, \emph{Г. Марсель} указывает \emph{на бога}, присутствие которого, по Марселю, \emph{постоянно чувствуется} человеком.

А вот \emph{Сартр} приходит к \emph{выводу}, что «\emph{бога нет}», --- причём, выводу, \emph{якобы} принципиально \emph{недоказуемому}.

Экзистенциализм \emph{исходит из иррациональности} человеческого бытия и подтверждение этому усматривает также в том, что «\emph{действительные глубины}» \emph{экзистенции} открываются перед нами только в особых условиях, которые \emph{Ясперс} именует «\emph{пограничными ситуациями}».

\emph{Пограничные ситуации} --- это \emph{смерть}, \emph{страдание}, \emph{страх}, \emph{борьба}, \emph{виновность}, \emph{религиозный экстаз}, \emph{душевное заболевание} и \emph{т.п}.

Лишь в пограничных ситуациях человек \emph{спонтанно осознает} своё «\emph{подлинное существование}» (свою \emph{свободу}), \emph{скрытое} в обычных условиях за «\emph{повседневностью}», «\emph{неподлинностью}» обыденного бытия, или, как говорит \emph{Хайдеггер}, за господством «\emph{das Man}».

\emph{Совместное} обыденное бытие людей полностью \emph{растворяет} подлинное бытие в способе бытия «\emph{другого}».

Термином «\emph{das Man}», производимым от неопределённого местоимения, которое употребляется \emph{в безличных предложениях} немецкого языка (например, \emph{man sagt} -- \emph{говорят}), \emph{Хайдеггер} выражает безличность и невыразимость, невыразительность, \emph{стёртость} человеческого существа в обществе, его растворение в обыденности обстоятельств.

Однако, раскрыв эту характерную черту \emph{положения индивида} в неблагополучном обществе, \emph{Хайдеггер} тут же \emph{уходит от неё}.

«\emph{Безличность}», «\emph{повседневность}» --- это свойство общества как такового, это способ существования человека \emph{в любом обществе}. И от этого никак \emph{нельзя избавиться}.

Понятие «\emph{неподлинное существование}», выраженное различными терминами, свойственно \emph{всем представителям} экзистенциализма.

Порой в представителями экзистенциализма \emph{подмечаются} достаточно глубокие характеристики общественной жизни.

Так, \emph{у Г. Марселя} «\emph{обладание}» противоположно «\emph{бытию}».

\emph{Вещи}, которыми я владею, \emph{владеют мною}, считает Марсель. Наше имущество «\emph{пожирает нас}».

Но \emph{как ликвидировать} эту \emph{подчинённость человека вещам}, которыми он обладает, в том числе созданными им самим?

Исторический материализм видит \emph{выход в изменении социальных условий} жизни человека, его общественных отношений.

\emph{Марсель} усматривает источник противоречия «бытия» (\emph{человечности}) и «обладания» (\emph{собственности}) в \emph{самом человеке}, в двойственной природе его существования и \emph{ищет выхода} в любви и милосердии, в «жертве» и, наконец, в религии, искусстве, философии, способных якобы \emph{возвысить обладание до бытия}.

К сожалению, подобные решения \emph{проблем человека} нередко остаются не более чем \emph{благими пожеланиями}, а то и выливаются в лицемерную \emph{проповедь} собственника, власть придержащего, обращенная к малоимущему, бесправному.

\emph{Экзистенциализм} --- философия, представляющая собой, крути не крути, \emph{выражение индивидуализма}.

Однако экзистенциалистский индивидуализм существенно отличатся от \emph{традиционного индивидуализма}, рассматривавшего человека как «\emph{социальный атом}», самодовлеющую общественную единицу. \emph{В XX в}. такое представление уже \emph{анахронично:} общественная жизнь теперь неизбежно и явно вовлекает в свою орбиту любого человека.

Сегодня \emph{немыслимо} даже \emph{относительно независимое} существование индивида (или «\emph{социальной ячейки}», семьи), которое послужило основой для социального атомизма XIX в.

Экзистенциализм \emph{придаёт} огромную значимость общественному отношению («\emph{другому}» --- как это выражается на языке экзистенциальной философии), но по-своему его трактует.

\emph{Сущность} экзистенциалистского индивидуализма в том, что общественные отношения рассматриваются \emph{как конфликтные отношения}, соединяющие людей только потому, что люди разделяют их.

Так, \emph{по Ясперсу}, связь между людьми («\emph{коммуникация}» как «\emph{жизнь с другими}») есть \emph{общение одиночек}: «Во всяком снятии одиночества коммуникацией, --- пишет Ясперс, --- вырастает \emph{новое одиночество}, которое не может исчезнуть без того, чтобы не прекратил существование я сам как условие коммуникации». (\emph{K. Jaspers. Philosophie}. Bd. II, s. 61).

\emph{Исходной формой} коммуникации оказывается у Ясперса отношение \emph{господства} и \emph{услужения}. В этом случае стремление к коммуникации неизбежно сочетается со \emph{страхом перед коммуникацией}, сомнением в её возможности и т.д.

Экзистенциализм выражает реальную \emph{противоречивость} и даже \emph{конфликтность} общественных отношений, которые \emph{остаются свойственными} и современному обществу, не говоря о прошлых ступенях его развития.

Но противоречивость социальных отношений здесь \emph{чрезмерно увязывается} с природой некоей \emph{абстрактной индивидуальностью} человека.

Противоречия, \emph{рождённые прежде всего} природой социальных отношений, при этом толкуются через обращение к некоей \emph{внутренней природе} человеческого существа,

В западной литературе экзистенциализм нередко именуется «\emph{философией свободы}».

И действительно, \emph{проблема свободы} занимает в экзистенциализме важное место.

Но в чём \emph{сущность свободы}, как её понимают экзистенциалисты?

\emph{Ж.-П. Сартр} выражает свободу так: «Человек не может быть то рабом, то свободным: он полностью и всегда свободен, \emph{или же его нет вообще}». (\emph{J.-P. Sartre - L'Être et le Néant}. Paris, 1943, p. 516).

С точки зрения \emph{философии диалектического материализма} реальная \emph{свобода} --- это \emph{способность действовать} на основе познания необходимости, «\emph{со знанием дела}».

Для \emph{экзистенциализма} свобода --- импульсивный, \emph{эмоциональный выбор}. Она \emph{открывается} человеку \emph{в беспокойстве}, \emph{тревоге}, \emph{заброшенности}.

«Тревога, заброшенность, ответственность..., --- пишет опять же \emph{Сартр}, --- составляют \emph{качество} нашего сознания, поскольку оно представляет собой чистую и простую свободу». (\emph{Там же}, р. 541-542).

Получается, \emph{свобода} --- \emph{бессознательный}, \emph{инстинктивный акт}, не имеющий объективного содержания.

Это, правда, \emph{не произвол}.

Как считает \emph{Сартр}, человек есть некоторое целое и на основе сáмого незначительного поступка, каким бы произвольным он ни казался, можно оценить человека.

Но формирование человека, обусловливающее его возможное поведение, представляется Сартру \emph{глубокой тайной}, а следовательно остаются тайной \emph{и пути воздействия} на человека, пути преобразования его сознания и поведения \emph{именно как свободного} человека.

Свобода \emph{соединяется} у Сартра с \emph{ответственностью} человека.

Собственно, \emph{свобода и ответственность}, по Сартру, \emph{тождественны}.

«Человек, \emph{осуждённый быть свободным}, несёт на своих плечах \emph{тяжесть всего мира}; он ответственен за мир и за себя \emph{как способ бытия}». (\emph{Там же}, р. 639).

Формулируя эту свою мысль, Сартр, исходил из той ответственности \emph{за судьбу Франции} и \emph{всего человечества}, которая легла на плечи \emph{борцов французского Сопротивления} в годы второй мировой войны.

\emph{Эта трагическая ситуация} оказала серьёзное влияние на творчество Сартра, как и практически всех других представителей французского экзистенциализма. Она \emph{легла в основу} целого течения «\emph{литературы ответственности}».

Однако отмеченная абсолютная ответственность \emph{превращается} у экзистенициалистов \emph{по диалектическому закону} перехода в свою противоположность \emph{в абсолютную безответственность}.

Ибо, \emph{во-первых}, \emph{ответственность} общественных групп, классов, индивидов \emph{растворяется} в «\emph{ответственности вообще}», в равной ответственности «\emph{всех}».

\emph{А во-вторых}, Сартр не исходит из \emph{объективных критериев} справедливости или несправедливости поступка, т.е. из объективных критериев ответственности.

Вcё это приводит к тому, что \emph{концепция свободы Сартра} оказывается в конечном счёте \emph{не очень-то содержательной} \emph{абстракцией}.

Экзистенциализм фактически утверждает: «\emph{Невиновных нет\underline{!}}».

Разбирая в своей книге «\emph{Вопрос о вине}» (1946) «\emph{немецкую вину}», ответственность немцев за развязанную мировую войну, \emph{Ясперс} видит «\emph{исходную}» и «\emph{первоначальную}» метафизическую (\emph{философскую}) \emph{вину}.

Т.е. \emph{смысл ответственности за войну} в том, что её истоки лежат в «\emph{человеческом бытии}» \emph{как таковом}.

\emph{Вывод}: хотя отдельные люди, преступники, и не оправдываются, \emph{ответственность} так или иначе \emph{снимается} с непосредственного виновника --- \emph{немецкого милитаризма}, \emph{нацизма} и т.п.

Человек оказывается \emph{бессильным} перед лицом этого «\emph{человеческого бытия}», ведь оно столь же \emph{непостижимо}, сколь \emph{и неподвластно} воздействию и преобразованию.

Если у \emph{экзистенциалиста-атеиста} это вызывает настоящее \emph{отчаяние} («\emph{если бога нет, то всё дозволено}» --- цитируют Достоевского \emph{Сартр} и \emph{Камю}), то \emph{религиозный экзистенциализм} возлагает надежду на бога, ищет спасения \emph{в религии} --- этом исторически многократно испытанном \emph{средстве успокоения} и \emph{надежды}, переносящем удовлетворение желаний и утоление страстей и страхов \emph{в потусторонний мир}.

\subsection{Религиозная философия в xx веке. Неотомизм}

\emph{Современная} религиозная философия \emph{не представляет} собой единого целого. Она \emph{расчленяется} под влиянием различных вероисповеданий, типов философского мышления, её представители примыкают \emph{к различным} школам.

Так, \emph{мы встречаемся} с христианской, иудаистской, мусульманской, буддистской религиозной философией.

\emph{В рамках христианства} существуют католическая, протестантская и православная философия.

\emph{В свою очередь, в рамках католичества} имеются различные неосхоластические школы, течения.

В религиозной философии есть и такие иррационалистические течения, как \emph{теологический экзистенциализм} \emph{К. Барта}, \emph{П. Тиллиха} и др. --- в протестантизме, \emph{экзистенциализм} \emph{М. Бубера} --- в иудаизме, \emph{неоавгустинизм} (последователи «отца церкви» \emph{Авгуатина}) --- в католической философий.

Широко распространён в западных странах \emph{персонализм} --- религиозная философия, признающая мир выражением творческой активности \emph{божественной личности}.

\emph{Наиболее представительная} школа современной религиозной философии --- \emph{неотомизм}, возрождающий схоластическую систему \emph{философии XIII в. Фомы} (Thomas) \emph{Аквинского}.

Возрождение томизма было провозглашено в \emph{энциклике папы Льва XI} в 1879 г. и \emph{закреплено созданием} ряда кафедр томистской философии, философского института в \emph{Лувене} (\emph{Бельгия}) и развёртыванием пропаганды томизма всеми идеологическими органами католической церкви.

В настоящее время \emph{центрами} разработки неотомизма \emph{являются} Академия св. Фомы \emph{в Ватикане}, Католический университет \emph{во Фрейбурге} (\emph{Швейцария}), Академия Альберта Великего \emph{в Кёльне}, католические университеты \emph{в Вашингтоне}, \emph{Оттаве}, католические институты \emph{в Париже}, \emph{Инсбруке}, \emph{Мадриде}, \emph{Торонто} и др.

\emph{В области философии} деятельность центров неотомизма объединяется «\emph{Всемирным союзом католических философских обществ}».

\emph{Ведущее место} среди философов-неотомистов занимали \emph{М. Грабман}, \emph{М. де Вульф}, \emph{Э. Жильсон}, известные главным образом своими работами по истории схоластической философии и историческому обоснованию неотомизма; \emph{Г. Манзер}, \emph{М. Гейзер}, \emph{Ж. Маритен} и др. разрабатывают систему неотомизма; \emph{Г. Гундлах}, \emph{О. Нейлл-Брейнинг} и др. излагают «социальное учение католицизма»; \emph{И. Бохеньский}, \emph{Г. Веттер} и др. посвящают свои усилия «\emph{критике}» философии диалектического материализма.

В неотомизма можно различать \emph{два течения}.

\emph{Одно} из них, «\emph{палеотомизм}», представители которого именуют себя «\emph{строгими томистами}», сохраняет в неприкосновенности учение \emph{Фомы}, считая, что в его трудах содержатся ответы на все философские проблемы.

\emph{Другое} --- \emph{собственно неотомизм} придерживается сформулирован-ного папой \emph{Львом XIII} лозунга: «\emph{Старое обогащать новым}» и стремится «\emph{развить}» застарелые, даже с его собственной точки зрения, \emph{тезисы Фомы} за счёт \emph{заимствований} некоторых идей из философии нового времени (у \emph{Канта} в особенности) или современной философии (\emph{феноменология}, \emph{неопозитивизм} и т.д.).

Но \emph{основное содержание} томизма остаётся \emph{незыблемым}.

Это --- \emph{религиозные положения} о существовании бога, бессмертии души, свободе воли, а также \emph{некоторые чисто философские} положения (учение о реальности сотворенного богом материального мира, учение о сверхопытном, трансцендентном мире, «рационализм» и т.д.).

Но \emph{почему именно томизм} оказался столь удобной системой философии, что его удается приспосабливать к нуждам современной 319.практики в западном обществе?

Чтобы ответить на этот вопрос, нужно \emph{вернуться к системе Фомы Аквинского}.

Фома \emph{соединил} философию \emph{Аристотеля} с католической теологией.

При этом он \emph{сумел избежать} крайностей \emph{отрицания} научного \emph{знания} в пользу \emph{веры} и рационалистического противопоставления разума и веры как двух независимых друг от друга источников истины («\emph{теория двойственной истины}»).

С заметной для \emph{XIII} \emph{в}. смелостью \emph{Фома заявил}, что «в области человеческого знания \emph{аргумент от авторитета --- наислабейший}». (\emph{S. Thomas Aquinatis. Summa Theologiae}, Ia, 1.8, ad. 2).

В то же время «\emph{человеческое знание}», с точки зрения Фомы, подчинено знанию «\emph{божественному}» --- не \emph{противоразумному}, а \emph{сверхразумному} знанию.

«Христианская \emph{теология}, --- \emph{писал Фома}, --- проистекает из \emph{света веры}, \emph{философия} же из естественного \emph{света разума}. Философские истины не могут противоречить истинам веры. Конечно, они недостаточны, но они допускают также общие аналогии; некоторые из них, более того, предвосхищают («\emph{истины веры}» -- \emph{Ред}.), ибо природа --- предвестие благодати». (\emph{S. Thomas Aquinatis. De Trinitate}, II, 3).

Приведенные мысли Фомы и \emph{легли в основу} неотомистского \emph{тезиса} о «\emph{гармонии веры и разума}», имеющего целью доказать, будто разум (\emph{научное мышление}) свободен в своих рассуждениях \emph{до тех пор, пока} он не противоречит вере.

\emph{Философия}, согласно неотомизму, оказывается независимой от науки, но зависимой от положений теологии. \emph{Наука де не вправе} выдвигать и решать мировоззренческие проблемы, а тем самым оказывать обратное влияние на философию.

А что же даёт \emph{право вере} и религиозной философии вмешиваться в дела науки?

Здесь неотомизм пытается \emph{использовать особенности} реальных отношений между отдельными науками.

\emph{Известно}, например, что различные отрасли науки \emph{тесно связаны} между собой, так что учёный, работающий в одной области знания, должен считаться с тем, что говорит смежная наука.

Так, не может быть противоречия \emph{между физикой и химией} в трактовке свойств молекулы; физика должна \emph{подчинять} свой математический аппарат авторитету математики и т.д.

\emph{Аналогично}, утверждают неотомисты, обстоит дело и в отношениях \emph{между верой и разумом}, теологией и философией, философией (религиозной, конечно) и наукой.

\emph{Формальная независимость} отдельных наук от философии, а философии от теологии сочетается с их материальной (\emph{по содержанию}) зависимостью.

«...Философ столь же мало может пытаться опровергнуть \emph{достоверные} данные теологии, как и \emph{достоверные} заключения частных наук... Многообразные формы научной деятельности регулируются и ограничиваются взаимным подчинением её ветвей... Отрицать такие взаимные ограничения значило бы отрицать принцип противоречия и впасть в релятивизм, пагубный для всего познания». (\emph{M. de Wulf. An Introduction to Scholastic Philosophy}. N.Y. 1956, p. 192).

Но само данное рассуждение представляется \emph{софистическим}.

Ведь достоверность научного знания просто \emph{несопоставима} с «\emph{достоверностью}» \emph{теологических} положений, почерпнутых из выработанных в условиях отсутствия научного знания религиозных \emph{мифов}.

\emph{Мифы} и \emph{догмы религии} не только не могут регулировать мировоззренческие выводы науки, но, наоборот, \emph{преодолеваются наукой} и научной материалистической философией.

Религиозные философы, стремясь избежать \emph{прямого сопоставления} науки и религии, склонны, вслед за неопозитивистами утверждать, что религиозные положения относятся к \emph{совсем} \emph{иной области}, чем эмпирическая наука и не могут находиться с ней в противоречии.

\emph{Обе} указанные установки (\emph{подчинение} философии и конкретных наук по содержанию теологии, с одной стороны, \emph{противопоставление} их --- с другой) \emph{лишь внешне} отличаются друг от друга. Результат же в обоих случаях один: \emph{сохранение} религии, \emph{отстаивание} её мировоззренческой значимости в отношении науки и научной философии.

В то же время \emph{различие этих установок} порождает различное отношение к трудностям, возникающим в ходе научного прогресса.

Некоторые религиозные философы \emph{пользуются наукой} для того, чтобы, опираясь на отдельные научные гипотезы, «\emph{вознестись к богу}».

В этих целях \emph{отвергается} всё, что противоречит догме, и \emph{подхватывается} всё, что представляется согласующимся с ней.

Так, для «обоснования» \emph{существования} бога и \emph{сотворения} мира используется известная «\emph{теория тепловой смерти Вселенной}», гипотеза о «\emph{конечности}» Вселенной, \emph{трудности} научного объяснения жизни и психики и т.д.

Однако \emph{открытия} современного естествознания свидетельствуют \emph{не о} творении мира богом, \emph{а о} внутренней активности, самодвижении материи, \emph{о том, что} материальный мир может быть объяснен \emph{без гипотезы} «творца», «двигателя» --- бога.

Неудивительно, что \emph{многие неотомисты}, справедливо сомневаясь в значимости доказательств конечности мира, исходящих, например, из идеи «тепловой смерти Вселенной», предпочитают \emph{более осторожное} отношение к научным фактам, и в особенности гипотезам, лишённое теологической \emph{прямолинейности}. Они \emph{истолковывают} научные гипотезы и факты \emph{в духе учения} Фомы Аквинского и принципов католицизма.

Эти истолкования осуществляются в \emph{неотомистской натурфилософии}, которая основывается на \emph{гилеморфизме} (от греческого \emph{hyle} -- материя, \emph{morphe} -- форма), т.е. на утверждении, что все явления природы состоят из материи и формы, причём форма определяет материю.

В соответствии со \emph{степенью совершенства} формы природные объекты образуют \emph{иерархию}, на низшем уровне которой стоят неорганические тела, затем органические, растения, животные и, наконец, человек.

В свою очередь, в обществе существует \emph{иерархия мирская} и церковная, а выше --- «\emph{небесная иерархия}».

Согласно взгляду Фомы, принятому неотомизмом, в этой иерархии \emph{определяющим} является именно высшее и поэтому нельзя представлять её эволюционно, как порождение высшего низшим.

Своё объяснение натурфилософия неотомизма находит в «\emph{метафизике}», оперирующей такими понятиями, как «\emph{бытие}», «\emph{акт и потенция}», «\emph{сущность и существование}». Анализ этих понятий показывает \emph{бедность} и \emph{схоластичность} неотомизма.

Попробуем, например, познакомиться с рассуждением томиста о «\emph{бытии}».

\emph{Исходное понятие} метафизики неотомизма --- «\emph{бытие}», получено в результате \emph{отвлечения от всего конкретного} содержания мира.

Вот рассуждение, которое взято \emph{не из} \emph{трактата XII} или \emph{XIII в}., а из статьи профессора Лувенского университета \emph{Н. Бальтазара}, написанной в \emph{1914 г.} специально для сборника «\emph{Новые идеи в философии}». (СПб., 1914).

«За пределами законов, свойственных движению, развитию, которыми занимаются естественные науки, за пределами качества, протяжения, пространства, которые изучает математика, остается осадок. Это --- сущее с самыми первыми его началами, общими всякому разумному познанию.

Будучи самым общим понятием, бытие может быть определено только тавтологично: «сущее есть сущее», «сущее не есть не сущее», «между сущим и не сущим нет третьего».

Так выводятся законы формальной логики (тождества, непротиворечия, исключенного третьего) в качестве определений бытия. Затем вводятся шесть «трансценденталий», т.е. понятий, характеризующих всё сущее: \emph{ens} (существо), \emph{res} (вещь), \emph{unum} (единое), \emph{aliquid} (иное), \emph{verum} (истинное), \emph{bonum} (благое). Иногда, впрочем, употребляются только четыре трансценденталии: единство, истина, добро и красота. Принадлежа всему сущему, они могут быть по аналогии перенесены на «всереальнейшее существо» --- бога, служа его определениями. Поскольку они перенесены на бога лишь по аналогии, мы знаем, что они присущи богу, но не знаем, как они ему присущи. Таким образом, томизм избегает, с одной стороны, агностицизма, отрицающего возможность постичь божественные атрибуты, а с другой --- антропоморфизма, приписывающего богу человеческие свойства».

Мы \emph{имеем здесь} дело со специфически \emph{схоластическим философствованием}, исходящим из «\emph{самоочевидных понятий}», из которых выводятся с помощью формальной логики все возможные следствия.

Неотомизм не желает замечать \emph{содержательные пустоты} этих рассуждений, по известному замечанию \emph{Ф. Бэкона}, подобных тому, как \emph{паук} вытягивает из себя паутину.

Конкретизация понятия бытия (сущего) состоит во введении \emph{заимствованных} у \emph{Аристотеля} понятий \emph{потенции} (возможности) и \emph{акта} (действительности).

\emph{В отличие} от Аристотеля, считавшего эти понятия выражением \emph{реальных процессов} перехода возможности в действительность, неотомизм рассматривает возможность как выражение \emph{несовершенства} любого конечного бытия.

Кроме бога, представляющего собой «\emph{чистый акт}», всякое сущее есть \emph{сочетание акта} и \emph{потенции}.

Так, ребенок --- \emph{актуально} ребенок, но \emph{потенциально} взрослый.

Исходя из соотношения акта и потенции, рассматриваются \emph{другие категории}: сущность и существование (\emph{сущность --- потенция, реализующаяся в существовании}), субстанция и её свойства --- акциденции (\emph{акциденции относятся к субстанции, как потенция к акту}), становление (\emph{переход потенции в акт}; всякое становящееся поэтому несовершенное бытие).

Снова бросается в глаза \emph{малая содержательность} томистских представлений.

Неотомизм \emph{явно не учитывает диалектики} возможности и действительности, в нём эти категории становятся основой своеобразного \emph{религиозного дуализма}, противопоставляющего божество с его «совершенствами» материальному миру как «несовершенному».

\emph{Особенно ярко} выступает религиозный дуализм в теории познания неотомизма.

Представители неотомизма обычно характеризуют свою гносеологию \emph{двумя} понятиями: «\emph{реализм}» и «\emph{рационализм}».

«\emph{Реализм}» означает, что неотомизм исходит из реальности, независимой от \emph{человеческого разума}.

Однако этот «реализм» сразу \emph{сводится на нет} утверждением, что действительный мир зависим от \emph{божественного разума}, являясь творением бога, т.е. идеальным по своей сущности образованием.

Неотомисты признают также «\emph{реальность}», т.е. независимость от человеческого ума, «\emph{универсалий}», или общих понятий.

Они стоят на точке зрения так называемого \emph{умеренного реализма}, признавая, что универсалии существуют \emph{до вещей} (в божественном разуме), \emph{в вещах} (в качестве умопостигаемой сущности\} и \emph{после вещей} (в виде понятий человеческого ума).

«Реалистическое» учение о\textsuperscript{::} универсалиях (общих понятиях) сложилось \emph{в средние века}.

\emph{Крайние реалисты} утверждали, что универсалии --- общие понятия --- реально существуют \emph{только в божественном уме}, тогда как материальный \emph{мир} представляет собой их несовершенную \emph{копию}.

В результате \emph{чувственное познание}, определённую роль которого неотомизм признаёт, имеет дело лишь с «\emph{материальной оболочкой}», тогда

как \emph{сущность} вещей постигается \emph{интеллектом}. Таким образом, «реализм» неотомистов оказывается объективным идеализмом.

Не основательнее является подход неотомизма к \emph{рациональному}. Современные томисты утверждают, что они \emph{отвергают иррационализм}, \emph{признают достоинства разума}, интеллекта.

Но этот «рационализм» оказывается на поверку \emph{псевдорацио-нализмом}, поскольку он исходит из примата «\emph{сверхразумных истин}» \emph{откровения} перед человеческим разумом, наукой.

Более того, \emph{вслед за Фомой} томисты пытаются использовать сам разум, в меру его ограниченных возможностей, для «\emph{доказательства}» \emph{бытия бога}.

Томистский и неотомистский «рационализм» очевидно берёт \emph{сторону религии}, что от него, наверное, и требуется.

Принуждая разум \emph{защищать} религиозную веру, такой «рационализм» ставит \emph{под вопрос} саму философию.

Об этом остроумно говорит \emph{Б. Рассел}: «Прежде чем Аквинский начинает философствовать, он уже знает истину: она возвещена в католическом вероучении... Но отыскание аргументов для вывода, данного заранее, --- \emph{это не философия, а система предвзятой аргументации}». (\emph{Б. Рассел. История западной философии}, с. 481).

Этим, пожалуй, \emph{сказано главное} о теоретическом значении неотомизма.

Средневековый томизм для своего времени был \emph{всё-таки} философским «обоснованием» идеологии феодального общества, т.е. современного ему общества.

Неотомизм сегодня оказался в состоянии выполнять \emph{лишь отчасти} подобную функцию по отношению к большому числу людей современного западного общества.

\subsection{Западная философия истории xx в}

Западная философия истории \emph{в наше время}, как и раньше, стремится иметь дело с универсальными принципами и методами, предназначенными для охвата всей истории.

Само собой разумеется, что стремление обнаружить наиболее общие черты общественного развития, движущие силы истории, взятой в целом, --- это \emph{реальная проблема}, над разрешением которой, в частности, работает социальная философия диалектического материализма --- исторический материализм.

В XVIII в. и в первые десятилетия XIX в. такие виднейшие представители философии истории, как \emph{Монтескье}, \emph{Вольтер}, \emph{Кондорсе}, \emph{Гердер}, \emph{Гегель}, отразили умонастроения наиболее активных сил своего времени, прежде всего класса восходящей буржуазии, заинтересованной в познании исторических явлений.

В целом в умозрительных концепциях и системах этих мыслителей (\emph{социальная натурфилософия}) содержались \emph{выдающиеся} мысли, догадки об исторической необходимости, о закономерном характере общественного развития, о социальном прогрессе и т.д.

В \emph{конце первой трети XIX в}. в старой философии истории произошли изменения, она \emph{была атакована} позитивистской социологией, основатели которой (\emph{О. Конт}, а затем и \emph{Г. Спенсер}) и их последователи ратовали за конкретные, позитивные знания об обществе, отвергали отвлечённые, оторванные от эмпирических фактов исторические схемы.

Хотя сами представители позитивиской философии и социологии \emph{и не думали отказываться} от идеалистических общесоциологических концепций, тем не менее \emph{позитивистский поворот} в сторону эмпирической социологии не мог не ослабить позиции философии истории.

Свременная западная философия истории \emph{многое утратила}, по сравнению с прошлым. Она \emph{отвлекается}, как правило, от обобщения многих важных сторон исторического процесса и создаёт свои концепции, преувеличивая, \emph{абсолютизируя} те или иные стороны действительности.

Многие представители западной философии истории призывают \emph{стать} «\emph{выше}» и \emph{материализма} и \emph{идеализма}, отвергают монистическое объяснение истории и отстаивают большей частью \emph{плюралистический} взгляд на общественную жизнь, т.е. рассматривают различные \emph{факторы}, взаимодействующие в историческом процессе, как \emph{равноценные} и абсолютно независимые начала.

Например, западные философы истории видят в экономических отношениях \emph{лишь один из факторов} среди многих других.

В то же время все рассуждения о равноценности, независимости факторов \emph{не мешают} им при этом в той или иной степени \emph{подчинять} возникновение и развитие объективных экономических отношений духовному фактору, началу.

Хapaктерны в этом отношении рассуждения известного французского социолога \emph{Р. Арона}, который \emph{готов} признать важную роль производительных сил, техники. \emph{Но} вслед за этим он называет движущей силой исторического развития познавательные способности человеческого разума.

Наряду с идеализмом \emph{другой} важной чертой современных западных учений об обществе является \emph{принижение} или \emph{отрицание} закономерного характера социального развития.

Идее исторического материализма о необходимом, закономерном переходе от одной ОЭФ к другой они \emph{противопоставляют} большей частью идею \emph{индетерминизма} в историческом процессе, видя в нём лишь случайное сцепление индивидуальных, неповторимых фактов и ситуаций.

Старое неокантианское \emph{противопоставление природы} и \emph{общества}, провозглашение первой царством слепой необходимости, а второго --- сферой свободы является чем-то \emph{само собой разумеющимся} для многих современных философов, а вслед за ними и социологов и историков субъективистской ориентации.

На \emph{методологические позиции} западной философии истории значительное влияние \emph{оказывает иррационализм}.

Не последнее место здесь занимают взгляды немецкого философа \emph{В. Дильтея}, который рассматривал \emph{историю как иррациональный поток}, лишенный структурного оформления, законосообразности.

Согласно Дильтею, \emph{бессмысленно} искать несуществующие общественные \emph{законы}, объяснять, исходя из них, исторические факты.

Историю нужно \emph{не объяснять}, \emph{а понимать}.

Процесс \emph{понимания} при этом толкуется как процесс \emph{переживания}.

\emph{Задача историка} заключается в том, чтобы \emph{с максимальной} «\emph{адекватностью}» \emph{пережить} то, что переживали люди, творившие историю, \emph{описать} эти переживания.

Освобождённая от объективных законов, история, таким образом, становится преимущественно \emph{предметом описательной психологии}.

Весьма популярна в западной философии история и мысль о том, что история является преимущественно \emph{делом ума и воображения} тех, кто пишет историю.

Многие современные западные философы и историки \emph{абсолютизируют} специфические \emph{особенности} и \emph{трудности} познания исторического процесса.

\emph{Отвергая} возможность объективного исторического познания, многие современные социальные мыслители выступают \emph{против возможности научного предвидения} в социальном познании. Что не мешает им \emph{предсказывать} то, \emph{что им бы хотелось} увидеть в будущем.

\emph{Существенная черта} современных западных учений об обществе --- \emph{метафизический}, нередко даже откровенно \emph{антидиалектический} подход к обществу, отказ от самого принципа историзма.

При исследовании общественных явлений \emph{остаются невыясненными} внутренние противоположности, их борьба, диалектическое отрицание старого новым, в том числе быстрое качественное преобразование старого в новое в любых его проявлениях.

Никто, конечно, \emph{не может поставить под сомнение} плодотворность и необходимость изучения той или иной социальной системы, той или иной её сферы, части, аспекта в их относительном постоянстве и устойчивости.

Но когда постоянство и устойчивость превращают в абсолютные, когда социальный организм вырывается из потока времени, закономерных изменений, --- тогда \emph{утрачивается верность} и \emph{глубина} социального познания.

Значительной популярностью на Западе и сегодня пользуется философия истории английского историка и социального мыслителя \emph{А. Тойнби}, автора работы «\emph{Исследование истории}».

Одно из исходных положений Тойнби --- \emph{отрицание единства} исторического процесса.

В своеобразной форме \emph{Тойнби} развивает \emph{идею О. Шпенглера}, который в духе средневекового номинализма, отрицая объективное существование общих понятий, утверждал, что «\emph{человечество}» --- \emph{пустое} «\emph{слово}», реальностью же обладают только отдельные \emph{этническо-культурные общности}.

По схеме английского мыслителя, \emph{история есть история} различных \emph{замкнутых цивилизаций}, которые возникали, развивались и исчезали, не соприкасаясь друг с другом.

\emph{Какая же сила} обусловливает движение цивилизации, её возникновение и развитие?

Такой силой является \emph{духовная элита}, мыслящее и творческое меньшинство, которое ведёт за собой «\emph{инертное большинство}», лишённое собственного разума и воли к самостоятельному историческому творчеству.

У \emph{Тойнби}, как и многих других, \emph{вместо} понятия \emph{ОЭФ} используется категория \emph{цивилизации}, основным содержанием которой считается \emph{духовное начало}, \emph{творческий порыв}, совокупность специфических \emph{культурных ценностей}.

\emph{Материальное производство} в этих системах фигурирует в лучшем случае в качестве \emph{одного из факторов} общественной жизни, одного из компонентов цивилизации.

\emph{Стремление уловить} специфические качественные особенности различных цивилизаций вполне \emph{правомерно}.

Но \emph{Тойнби абсолютизирует} эти особенности и фактически \emph{отбрасывает} связи между цивилизациями.

Конечно, в истории были и относительно \emph{изолированные цивилизации}, которые исчезли, не оставив сколько-нибудь заметного следа в дальнейшем развитии материальной и духовной культуры.

Но не эти общества характеризуют процесс \emph{становления всемирной}, внутренне связанной, поступательно развивающейся \emph{истории} человечества и человеческой культуры.

Концепция Тойнби \emph{отрицает} \emph{наличие всеобщей истории}, \emph{атомизирует} её, \emph{расчленяет} на не общающиеся между собой цивилизации, \emph{не видит} идеи исторического прогресса как всеобщего социального закона.

\emph{Другим} крупным представителем современной западной социальной мысли является \emph{американский социолог русского происхождения П.А. Сорокин}.

Если Тойнби оперировал понятием «\emph{цивилизация}», то в учении Сорокина центральное место занимает учение о \emph{социокультурных системах}, под которыми понимаются \emph{сверхорганические} социальные, культурные явления, которые в отличие от физико-химических и физико-биологических явлений \emph{наделены смыслом}, \emph{обладают ценностью} или \emph{нормой}.

Если лишить смысловой значимости, ценности \emph{Венеру Милосскую}, пишет Сорокин, то она окажется \emph{всего-навсего глыбой мрамора} определённой геометрической формы.

Сорокин оперирует также понятием «\emph{суперсистема}», которая включает в себя \emph{множество культур}, основанных на общих принципах.

Человеческая история есть \emph{смена одной суперсистемы другой} (очевидная аналогия со сменой ОЭФ).

\emph{Различие суперсистем}, по мнению Сорокина, определялось соответственно тому, \emph{что} они \emph{считали ценностью}, как они определяли её природу и какой \emph{тип мировоззрения} в связи с этим лежал в их основе (спиритуалистическое, интегрально-идеалистическое , сенсуалистическое).

\emph{Понятие суперсистемы} призвано фактически заменить, \emph{вытеснить понятие общественно-экономической формации.}

Но при этом \emph{остаётся невыясненным}, какие \emph{реальные силы} порождают эти суперсистемы, каковы \emph{причины} их смены, каковы \emph{закономерности} истории человечества.

Современная западная философия истории \emph{по-разному определяет} этапы, эпохи исторического развития человечества, что также \emph{не способствует} целостному его пониманию.

\chapter{Заключение}

Мы \emph{рассмотрели основные проблемы} диалектико-материалистической философии, её социальной составляющей --- исторического материализма.

Сделаем некоторые \emph{выводы}.

\begin{enumerate}
\item Диалектико-материалистическая философия появляется \emph{как результат закономерности} развития философии.

Она является важным \emph{средством теоретического познания} и \emph{практического преобразования} мира.

Диалектико-материалистическая философия в переработанном виде \emph{вобрала в себя то ценное}, что было достигнуто в результате более чем двухтысячелетнего развития философии, общественного сознания.

\item Материалистическая диалектика является важным \emph{средством анализа} развития природы и общества, а также человеческого познания.

\item Диалектико-материалистическая философия является специфической \emph{формой научного освоения} мира.

Опираясь на достижения науки, \emph{подытоживая} развитие общественной практики человечества, она \emph{стремится воспроизвести} динамично развивающуюся картину мира, \emph{выступает} как обобщённое рационалистическое философское мировоззрение.

Она \emph{дополняет} множество специальных методов частных наук общим философским подходом, пригодным для всех отраслей знания в качестве общей методологии.

Она \emph{ориентирует науку} на всё более глубокое познание мира, не прибегая к допущениям, противоречащим сущности научного познания.

Она \emph{не ставит границ} человеческому познанию, \emph{не абсолютизирует} те или иные достижения науки, а \emph{старается обосновывать} необходимость раскрытия все новых и новых сторон, закономерностей окружающего мира.

Наконец, диалектико-материалистическая философия, исторический материализм \emph{даёт возможность} глубже понять существо и значение происходящих в XX в. революционных изменений в обществе, науке и технике, \emph{показывает} их последствия.

\item В качестве социальной, социально-нравственной основы философия диалектического материализма имеет \emph{объективные  интересы} основных социальных групп современного общества и   основанные на них, вытекающие из них \emph{общечеловеческие ценности}, неискаженные никакими вторичными различиями.

Она \emph{продолжает традиции} материализма прошлого, активно борется со всякого рода предрассудками.

\item   Важнейшим принципом диалектико-материалистической философии является   \emph{связь теории и практики}, проверка теоретических   положении на практике, развитие теории на основе обобщения   практического опыта.

Это позволяет ей быть \emph{творческой традицией}, \emph{преодолевать застой} мысли, проявления духовного консерватизма, догматизма.

Она \emph{не даёт готовых решений} ни общественной практике в целом, ни какой-нибудь отдельной отрасли научного знания.

Она \emph{требует конкретного применения} своих принципов с учётом специфических особенностей изучаемых и осваиваемых практически явлений.

\item Диалектико-материалистическая философия --- важная \emph{составная часть современной мировой философская культуры} и культуры в целом.
\end{enumerate}

\chapter{Список цитируемых и упоминаемых источников}

\begin{enumerate} \def\labelenumi{\arabic{enumi}.} \item   К. Маркс и Ф. Энгельс, Соч., тт. 3, 4, 13, 20, 23, 25, 27, 32, 37, 42,   47. \item   В.И. Ленин. ПСС, тт. 18, 29. \item   Аристотель. Соч. в четырех томах, т. Т. М., 1975. \item   Бальзак об искусстве. М.-Л., 1941. \item   Бернал Дж.Д.. Наука в истории общества. М., 1956. \item   Богомолов А.С. Английская буржуазная философия XX века, т. 3, М.,   1973. \item   Витгенштейн Л. Логико-философский трактат. М., 1958. \item   Гегель. Сочинения, тт. 1, 17. \item   Гегель. Соч. Т. VI. М, 1939. \item   Джемс У. Многообразие религиозного опыта. М., 1910. \item   Дорст Ж. До того как умрёт природа. М., I960. \item   Дэвис Дж. Капитализм и его культура. М., 1949. \item   Кант И. Соч. в шести томах, т. 4, ч. I, М., 1965. \item   Кассирер Э. Познание и действительность. Спб., 1912. \item   Кун Т. Структура научных революций. М., 1975. \item   Лаплас П. Опыт философии теории вероятностей. М., 1908. \item   Материалисты древней Греции. М., 1955. \item   Новые идеи в философии. СПб., 1914. \item   Природа и общество. М., 1968. \item   Рассел Б. История западной философии. М., 1959. \item   Рассел Б. Человеческое познание. М., 1957. \item   Спенсер Г. Основные начала. СПб., 1897. \item   Тоффлер О. Столкновение с будущим. М., 1972. \item   Швырёва В.С. Неопозитивизм и проблемы эмпирического обоснования науки,   М., 1966. \item   Юм Д. Соч. в двух томах, т. 2, М., 1965. \item   Carnap R. Philosophy and Logical Syntax. In: ``The Age of Analysis'',   N.Y., 1956. \item   Fromm E. Escape from Freedom. NY -- Chicago -- San Francisco, 1964 \item   Jaspers K. Philosophie. Bd. I, II. Berlin, 1956. \item   Lewis J. Man and Tvolution. L., 1962. \item   Marsel G. Der Mensch als Problem. Frankfurt a/M., 1964. \item   Marsel G. Sein und Haben. Paderborn, 1954. \item   Popper K. Objective Knowledge. Oxford, 1974. \item   Popper К. The Poverty of Historicism. L., 1957. \item   Quine W. From a Logic Point of Wiew. N.Y., 1963. \item   S. Thomas Aquinatis. De Trinitate, II. \item   S. Thomas Aquinatis. Summa Theologiae, Ia, 1.8, ad. 2. \item   Sartre J.-P. L'Être et le Néant. Paris, 1943. \item   Scheler M. Gesammlte Werke. Bd. VI. Bern und München, 1963. \item   Wittgenstein L. Philosophical Investigation. Oxford, 1963. \item   Wulf M. de. An Introduction to Scholastic Philosophy. N.Y. 1956. \end{enumerate}

\chapter{Указатель имён}

\emph{А}

Агустин --- 318.

Авенариус Р. --- 17, 18, 304.

Анаксимандр --- 12, 84.

Анаксимен --- 12, 84.

Анохин П.К. --- 108.

Аристотель --- 78, 125, 129, 130, 143, 159, 282, 308, 319, 323.

Арон Р. --- 325.

\emph{Б}

Бальзак О. де --- 254, 255.

Бальтазар Н. --- 322.

Барт К. --- 318.

Бенеке Ф. --- 120.

Бердяев Н.А. --- 312.

Беркли Дж. --- 17, 18, 147, 306.

Бернал Дж. Д. --- 203.

Богданов А.А. --- 180.

Богомолов А.С. --- 305.

Бойль Р. --- 85.

Бокль Г. --- 190.

Борн М. --- 26.

Бернштейн Н.А. --- 108.

Бохеньский И. --- 319.

Бройль Луи де. --- 26.

Бубер М. --- 318.

Бунге М. --- 164.

Бэкон Ф. --- 16, 103, 125, 159, 322.

\emph{В}

Введенский Н.Е. --- 108.

Вернадский В.И. --- 192, 194.

Веттер Г. --- 319.

Вико Дж. Б. --- 54.

Виндельбанд В. --- 175.

Витгенштейн Л. (Wittgenstein L.) --- 71, 306.

Вольтер --- 260, 325.

Вульф М. де (Wulf M. de) --- 319, 320.

\emph{Г}

Галилей Г. --- 130.

Гассенди П. --- 24, 85.

Гегель Г.В.Ф. --- 10, 16, 18, 21, 22, 30, 31, 34, 43, 45, 56, 57, 69, 76, 78, 85, 103, 115, 125, 127, 154, 290, 311, 325.

Гейзенберг В. --- 154.

Гейзер М. --- 319.

Геккель Э. --- 116.

Гексли Т. --- 18, 127.

Гельвеций К. --- 16, 125.

Гельмгольц Г. --- 32.

Гераклит --- 12, 84, 307.

Гердер И. --- 290, 325.

Герц Г. --- 276.

Гёте И.В. --- 260.

Гоббс Т. --- 16, 85, 281.

Гольбах П. --- 16, 71, 85.

Грабман М. --- 319.

Гундлах Г. --- 319.

Гуссерль Э. --- 312.

Гюго В. --- 254.

\emph{Д}

Дарвин Ч. --- 33, 85.

Декарт Р. --- 10, 22, 116, 125, 127, 148, 149.

Демокрит --- 16, 22, 27, 70, 84, 103, 125, 130.

Джемс У. ---305.

Джоуль Дж. --- 32.

Дидро Д. --- 16, 33, 85, 103, 125, 260.

Диккенс Ч. --- 254.

Дильтей В. --- 326.

Дирак П. --- 152.

Дорст Ж. --- 193.

Достоевский Ф.М. --- 254, 260, 317.

Дубислав В. --- 306.

Дэвис Дж. --- 251.

\emph{Ж}

Жильсон Э. --- 319.

\emph{К}

Калер Э. --- 299.

Калесник С.В. --- 192.

Камю А. --- 312.

Кант И. --- 10, 18, 22, 66, 125, 126, 127, 139, 149, 319.

Карнап Р. (Carnap R.) --- 306, 307.

Карнеад --- 126.

Кассирер Э. --- 135.

Ковалевский М.М. --- 194.

Кондорсе Ж. --- 290, 325.

Конт О. --- 290, 303, 309, 325.

Коперник Н. --- 289.

Куайн У. (Quine W.) --- 310.

Кун Т. --- 311.

Кутюра Л. --- 305.

Кьёркегор С. --- 260, 312.

\emph{Л}

Ламетри Ж. --- 16.

Лаплас П. --- 67, 71, 86.

Лев XI --- 318.

Лев XIII --- 319.

Левкипп --- 24, 84.

Лейбниц Г.В. --- 10, 125, 127.

Ленгмюр --- 276.

Ленин В.И. --- 8, 27, 36, 81, 87, 93, 115, 127, 128, 145, 305.

Ленц Э. --- 32.

Лобачевский Н.В. --- 143, 152.

Локк Д. --- 125.

Ломоносов М.В. --- 24, 85.

Лукреций Кар --- 84, 103.

Льюис Дж. (Lewis J.) --- 188.

\emph{М}

Майер Ю. --- 32.

Максвелл Д. --- 86, 266.

Мальтус Т. --- 194.

Манзер Г. --- 31.

Маритен Ж --- 319.

Маркс К. --- 8, 34, 110, 151, 173, 174, 177, 178, 187, 193, 197, 198, 209, 226, 253, 269, 278.

Марсель Г. (Marsel G.) --- 312, 313, 314, 315.

Мах Э. --- 17, 18, 147, 304, 305.

Менделеев Д.И. --- 81, 164.

Монтескье Ш. --- 190, 325.

\emph{Н}

Наполеон --- 289.

Нейл-Брейнинг О. --- 319.

Нейрат О. --- 306.

Ницше Ф. --- 18, 313.

Ньютон И. --- 85, 97, 129, 130, 165, 266.

\emph{О}

Орбели Л.А. --- 108.

Оствальд В. --- 93.

\emph{П}

Павлов И.П. --- 108, 109.

Парменид --- 12.

Пастер Л. --- 276.

Пиррон --- 126.

Пирс Ч.С. ---305.

Пифагор --- 11, 129.

Планк М. --- 26, 152.

Платон --- 16, 18, 27, 125, 311.

Плеханов Г.В. --- 8, 253.

Попов А.С. --- 276.

Поппер К. (Popper K.) --- 144, 302, 311.

Пушкин А.С. --- 260.

\emph{Р}

Рассел Б. --- 15, 22, 66, 141, 305, 324.

Резерфорд Э. --- 163.

Рейхенбах Г. --- 25, 306.

Реклю Э. --- 190.

Риккерт Г. --- 175.

Риман Б. --- 143.

Робине Ж. --- 116.

\emph{С}

Сартр Ж.-П. (Sartre J.-P.) --- 312, 313, 314, 316, 317.

Сеченов И.М. --- 108.

Содди Ф. --- 163.

Сократ --- 67.

Сорокин П.А. --- 328.

Спенсер Г. --- 290, 304, 325.

Спиноза Б. --- 16, 70, 125, 148.

Стендаль (Бейль Анри) --- 254.

\emph{Т}

Тиллих П. --- 318.

Тойнби А. --- 327, 328.

Толстой Л.Н. --- 141, 254, 260.

Тоффлер О. --- 299.

Тургенев И.С. --- 254.

Тюрго А. --- 290.

\emph{У}

Ухтомский А.А. --- 108.

\emph{Ф}

Фалес --- 12, 83, 307, 308.

Фарадей М. --- 32, 86.

Фейербах Л. --- 14, 30, 31, 103, 125, 129, 278.

Ферсман А.Е. --- 192.

Фихте И.Г. --- 10, 22.

Фома Аквинский (S. Thomas Aquinatis) --- 318, 319, 320, 324.

Франклин Б. --- 187.

Фреге Г. --- 305.

Фрейд З. --- 111.

Фромм Э. (Fromm E.) --- 280.

\emph{Х}

Хайдеггер М. --- 312, 314.

\emph{Ц}

Цезарь --- 308.

\emph{Ч}

Чернышевский Н.Г. --- 260.

\emph{Ш}

Шванн Т. --- 33.

Швырёв В.С. --- 310.

Шекспир У. --- 260.

Шелер М. (Scheler M.) --- 302, 303.

Шеллинг Ф. --- 22, 34.

Шестов Л. --- 312.

Шлейден М. --- 33.

Шлейермахер Ф. --- 120.

Шлик М. --- 306.

Шпенглер О. --- 54, 327.

\emph{Э}

Эйнштейн А. --- 25, 129, 137, 152, 154, 164, 305.

Энгельс Ф. --- 8, 32, 33, 34, 94, 110, 144, 173, 209, 221, 226, 253, 269, 278, 308.

Энесидем --- 126.

Эпикур --- 16, 84, 103.

Эрлих П. --- 196.

\emph{Ю}

Юм Д. --- 65, 66, 126, 127, 147, 306.

\emph{Я}

Ясперс К. (Jaspers K.) --- 303, 312, 313, 315, 316, 317.

\end{document}